%_ **************************************************************************
%_ * The TeX source for AMS journal articles is the publishers TeX code     *
%_ * which may contain special commands defined for the AMS production      *
%_ * environment.  Therefore, it may not be possible to process these files *
%_ * through TeX without errors.  To display a typeset version of a journal *
%_ * article easily, we suggest that you retrieve the article in DVI,       *
%_ * PostScript, or PDF format.                                             *
%_ **************************************************************************
%  Author Package
%% Translation via Omnimark script a2l, September 25, 2002 (all in one day!)
\controldates{4-JUN-2003,4-JUN-2003,4-JUN-2003,5-JUN-2003}
 
\RequirePackage[warning,log]{snapshot}
\documentclass{tran-l}
\issueinfo{355}{11}{November}{2003}
\pagespan{4451}{4473}
\dateposted{June 10, 2003}
\PII{S 0002-9947(03)03198-2}
\copyrightinfo{2003}{American Mathematical Society}


\newcount{\cols}{\catcode`,=\active\catcode`|=\active \gdef\Young(#1){\hbox{$\vcenter{\mathcode`,="8000\mathcode`|="8000 \def,{\global\advance\cols by 1 &}\def|{\cr \multispan{\the\cols}\hrulefill\cr &\global\cols=2 }\offinterlineskip\everycr{}\tabskip=0pt \dimen0=\ht\strutbox \advance\dimen0 by\dp\strutbox \halign {\vrule height \ht\strutbox depth \dp\strutbox## &&\hbox to \dimen0{\hss$##$\hss}\vrule\cr \noalign{\hrule}&\global\cols=2 #1\crcr\multispan{\the\cols}\hrulefill\cr }}$}} }





\theoremstyle{plain}
\newtheorem*{theorem1}{Theorem 1}
\newtheorem*{theorem2}{Corollary 2}
\newtheorem*{theorem3}{Proposition 1.1}
\newtheorem*{theorem4}{Corollary 1.2}
\newtheorem*{theorem5}{Corollary 1.3}
\newtheorem*{theorem6}{Proposition 1.4}
\newtheorem*{theorem7}{Proposition 1.5}
\newtheorem*{theorem8}{Proposition 1.6}
\newtheorem*{theorem9}{Proposition 1.7}
\newtheorem*{theorem10}{Theorem 1(a)}
\newtheorem*{theorem11}{Proposition 2.1}
\newtheorem*{theorem12}{Corollary 2.2}
\newtheorem*{theorem13}{Lemma 2.3}
\newtheorem*{theorem14}{Lemma 3.1}
\newtheorem*{theorem15}{Theorem 3.2}
\newtheorem*{theorem16}{Corollary 3.3}
\newtheorem*{theorem17}{Conjecture 4.1}
\newtheorem*{theorem18}{Proposition A1.1}
\newtheorem*{theorem19}{Proposition A3.1}
\newtheorem*{theorem20}{Lemma A3.2}
\newtheorem*{theorem21}{Proposition A3.3}
\newtheorem*{theorem22}{Lemma A3.4}
\newtheorem*{theorem23}{Corollary A3.5}
\newtheorem*{theorem24}{Proposition A3.1$'$}
\newtheorem*{theorem25}{Lemma A3.2$'$}
\newtheorem*{theorem26}{Corollary A3.5$'$}

\theoremstyle{remark}
\newtheorem*{remark1}{Remarks}

\theoremstyle{definition}
\newtheorem*{definition1}{Example 1}
\newtheorem*{definition2}{Example 2}
\newtheorem*{definition3}{Example 3}
\newtheorem*{definition4}{Definition}
\newtheorem*{definition5}{Definition}

\newcommand{\GL}{\operatorname{GL}}
\newcommand{\SL}{\operatorname{SL}}
\newcommand{\Ann}{\operatorname{Ann}}
\newcommand{\End}{\operatorname{End}}
\newcommand{\Hom}{\operatorname{Hom}}
\newcommand{\g}{{\mathfrak g}}
\newcommand{\gl}{{\mathfrak {gl}}}
\newcommand{\Sym}{\operatorname {Sym}}
\newcommand{\Tor}{\operatorname {Tor}}
\newcommand{\Ext}{\operatorname {Ext}}
\newcommand{\myS}{{\mathcal S}}
\newcommand{\F}{{\mathcal F}}
\newcommand{\FF}{{\mathbf F}}
\newcommand{\myth}{\operatorname{th}}
\newcommand{\coker}{\operatorname{coker}}
\newcommand{\ann}{\operatorname{ann}}
\newcommand{\NN}{\mathbb {N}}
\newcommand{\ZZ}{\mathbb {Z}}
\newcommand{\PP}{{\mathbf P}}


\begin{document}

\title{Fitting's Lemma for $\mathbb{Z}/2$-graded modules}
\author{David Eisenbud}
\address{Department of Mathematics, University of California, 
Berkeley, California 94720}
\email{de@msri.org}
\author{Jerzy Weyman}
\address{Department of Mathematics, Northeastern  University,
Boston, Massachusetts 02115}
\email{j.weyman@neu.edu}
\thanks{The second named author is grateful to the
Mathematical Sciences Research Institute for support in the period this
work was completed. Both authors are grateful
for the partial support of the National Science Foundation.}
\date{March 20, 2002 and, in revised form, May 29, 2002}
\subjclass[2000]{Primary 13C99, 13C05, 13D02, 16D70, 17B70}
\begin{abstract}Let $\phi :\; R^{m}\to R^{d}$
be a map of free modules over a commutative ring $R$.
Fitting's Lemma shows that the ``Fitting ideal,'' the
ideal of $d\times d$ minors of 
$\phi $, annihilates the cokernel of $\phi $ and 
is a good approximation to the whole annihilator in a certain
sense. In characteristic 0 we define a Fitting ideal
in the more general case of a map of graded
free modules over a $\mathbb{Z}/2$-graded skew-commutative algebra 
and prove corresponding theorems about the annihilator; for
example, the Fitting ideal and the annihilator of the cokernel are equal
in the generic case.
Our results generalize the
classical Fitting Lemma in the commutative case
and extend a key result of Green (1999) in the exterior algebra case.
They depend on the Berele-Regev theory of representations
of general linear Lie superalgebras. In the purely even and
purely odd cases we also offer a standard basis approach to
the module $\operatorname{coker}\phi $ when $\phi $ is a generic matrix.
\end{abstract}
\maketitle

\section*{Introduction}

The classical Fitting Lemma (Fitting \cite{Fit}) gives information
about the annihilator of a module over a commutative ring in terms
of a presentation of the module by generators and relations.
More precisely, let
\begin{equation*}\phi : R^{m} \to R^{d}
\end{equation*}
be a map of finitely generated free modules over a commutative ring
$R$, and for any integer $t\geq 0$ 
let $I_{t}(\phi )$ denote the ideal in $R$ generated by the 
$t\times t$ minors of $\phi $.
Fitting's result says that the module $\coker \phi $
is annihilated by $I_{d}(\phi )$, and that if $\phi $ is the 
generic map---represented by a matrix whose entries 
are distinct indeterminates---then the
annihilator is equal to $I_{d}(\phi )$. 
Thus $I_{d}(\phi )$ is the best approximation to the
annihilator that is compatible with base change. Moreover, 
$I_{d}(\phi )$ is not too bad an approximation to $\ann \coker \phi $
in the sense that 
$I_{d}(\phi )\supset (\ann \coker \phi )^{d}$, or more precisely
$(\ann \coker \phi )I_{t}(\phi )\subset I_{t+1}(\phi )$ for all $0\leq t<d$.
In this paper we will
prove corresponding results in the case 
of $\ZZ /2$-graded modules over a
skew-commutative $\ZZ /2$-graded algebra containing a field $K$
of characteristic 0.
Let $R$ be a 
$\ZZ /2$-graded skew-commutative $K$-algebra: that is, $R=R_{0}\oplus R_{1}$ as vector spaces, $R_{0}$ is a commutative central subalgebra,
$R_{i}R_{j}\subset R_{i+j\ (\operatorname{mod}\ 2)}$, and every element of
$R_{1}$ squares to 0. Any homogeneous map $\phi $ of $\ZZ /2$-graded free
$R$-modules may be written in the form
\begin{equation*}\phi :\quad R^{m}\oplus R^{n}(1)
\xrightarrow{\left(\begin{smallmatrix} X &A\\ B&Y \end{smallmatrix}
\right)}
R^{d}\oplus R^{e}(1),
\end{equation*}
where $X, Y$ are matrices of even elements of $R$, and $A, B$ are
matrices of odd elements. We will define an ideal 
$I_{\Lambda (d,e)}$ and
show that it is contained in the
annihilator of
the cokernel of $\phi $, with equality 
in the {\em generic case\/} where the entries of the matrices
$X,Y,A,B$ are
indeterminates (that is,
$R$ is a polynomial ring on the entries of $X$ and $Y$
tensored with an exterior algebra on the entries of $A$ and $B)$. 
In the purely even and purely odd cases our results are
characteristic-free, but in general we 
give examples to show that the annihilator can look 
quite different---for example, it might be generated by
forms of different degrees---in positive characteristic.
\goodbreak Now let $K$ be a field, and let 
$U=U_{0}\oplus U_{1}$ and $V=V_{0}\oplus V_{1}$ be $\ZZ /2$-graded vector
spaces of dimensions
$(d,e)$ and $(m,n)$ respectively.
We consider the generic ring
\begin{equation*}\myS =
\myS (V\otimes U) := S(V_{0}\otimes U_{0} )\otimes S(V_{1}\otimes U_{1}
)\otimes \wedge (V_{0}\otimes U_{1}
)\otimes \wedge (V_{1}\otimes U_{0} ),
\end{equation*}
where $S$ denotes the symmetric algebra and $\wedge $ denotes
the exterior algebra, and the {\em generic\/}, or
{\em tautological\/}, map
\begin{equation*}\Phi : \myS \otimes V\to \myS \otimes U^{*}.
\end{equation*}
This map $\Phi $ is defined
by the condition that
$\Phi |_{V}=1\otimes \eta : V\to V\otimes _{K} U\otimes _{K} U^{*}\subset R\otimes _{K} U^{*}$,
where $\eta : K\to U\otimes _{K} U^{*}$ is the dual of the contraction
$U^{*}\otimes _{K} U\to K$.
We will make use of this notation throughout the paper. 
We will
compute the annihilator of the cokernel of $\Phi $. Of course if
we specialize $\Phi $ to any map of free modules $\phi $ over a
$\ZZ /2$-graded ring,
preserving the grading, then we can derive elements in the
annihilator of the cokernel of $\phi $ by specializing the
annihilator of the cokernel of $\Phi $.

In the classical case, where $V$ and $U$ have only even parts
($e=n=0$), the annihilator is an invariant ideal for the action of the
product of general linear groups $\operatorname{GL}(V)\times \operatorname{GL}(U)$.
Such invariant ideals have been studied by DeConcini, Eisenbud, and
Procesi in \cite{DEP} and have a very simple arithmetic.  In the general
case, Berele and Regev \cite{BR} have developed a highly parallel theory,
using the $\ZZ /2$-graded Lie algebra $\g ={\gl }(V)\times {\gl }(U)$ in
place of $\operatorname{GL}(V)\times \operatorname{GL}(U)$.  They show that the generic
ring $\myS $ is a semisimple representation of $\g $ (even though not all
the representations of $\g $ are semisimple)  and that the irreducible
summands of $\myS $ of total degree $t$ are parametrized by certain
partitions of the integer $t$, just as in the commutative case.
The Berele-Regev theory is described in detail below, in Section 1 of this
paper.  

If $\Lambda $ is a partition, we write $I_{\Lambda }$ for the
ideal of $\myS $ generated by the irreducible representation
corresponding to $\Lambda $. If $\phi $ is a matrix representing any map
of $\ZZ /2$-graded free modules over a $\ZZ /2$-graded skew-commutative
$K$ algebra $R$, then there is a unique ring homomorphism $\alpha :
\myS \to R$ such that $\phi = \alpha (\Phi )$, and we write
$I_{\Lambda }(\phi ):=\alpha (I_{\Lambda }(\Phi ))R$ for the ideal generated by
the image of $I_{\Lambda }=I_{\Lambda }(\Phi )$.
If $e=n=0$, so that the ring $\myS $ is a polynomial ring,  
the classical Fitting
Lemma (see for example Eisenbud \cite[Prop.~20.7]{Eis}) 
shows that the annihilator of 
$\coker \Phi $ is the ideal of $d\times d$
minors $I_{d}(\Phi )$. In representation-theoretic terms,
this is the ideal 
generated by the representation 
\begin{equation*}\wedge ^{d}V_{0}\otimes \wedge ^{d}U_{0}\subset \Sym _{d}(V_{0}\otimes U_{0}),
\end{equation*}
the irreducible representation associated to the partition with
one term $(d)$. In our notation, 
$I_{d}(\Phi )= I_{(d)} = I_{(d)}(\Phi )$. On the other
hand, if $d=n=0$, so that the ring $\myS $ is an exterior algebra,
Green \cite[Proposition 1.3]{Gre} shows that the
representation 
\begin{equation*}S_{e}(V_{0})\otimes \wedge ^{e}U_{1} \subset \wedge ^{e}(V_{0}\otimes U_{1})
\end{equation*} 
is at least contained in the annihilator of $\coker \Phi $.
This is the representation associated to the partition
$(1^{e})=(1,1,\dots ,1)$ with $e$ parts (see Appendix 1 for a
characteristic-free treatment).
Here is the common
generalization of these results,
which is the main result of this paper:

\begin{theorem1} Suppose that $K$ is a field of characteristic $0$,
and let  
\begin{equation*}\phi :\quad R^{m}\oplus R^{n}(1) 
\overset{\left(\begin{smallmatrix}X&A\\B&Y\end{smallmatrix}\right)}{\longrightarrow}
R^{d}\oplus R^{e}(1)
\end{equation*}
be a $\ZZ /2$-graded map of free modules over a 
$\ZZ /2$-graded skew-commutative
$K$-algebra $R$.

{\bf a) } When $R=\myS $ and $\phi =\Phi $, the generic map
defined above,
the annihilator of the cokernel of $\Phi $, is
$I_{\Lambda (d,e)}(\Phi ),
$
where
$\Lambda (d,e)$ is the partition $(d+1,d+1,\dots ,d+1, d)$ of $(d+1)(e+1)-1$
into $e+1$ parts. In general we have
$I_{\Lambda (d,e)}(\phi )\subset \ann \ \coker (\phi )$.

{\bf b)} 
If $x_{1} ,\ldots ,x_{e} \in \ann \ \coker (\phi )$, then
$x_{1}\ldots x_{e}\in I_{\Lambda (0,e)}(\phi )$.
Moreover, if  $0\le s\le d-1$, and 
$x_{1} ,\ldots ,x_{e+1}\in \ann \ \coker (\phi )$, 
then 
$x_{1}\ldots x_{e+1}I_{\Lambda (s,e)}(\phi )\subset I_{\Lambda (s+1,e)}(\phi ).
$
\end{theorem1}


The proof is given in sections 2 and 3 below.
An alternate approach through standard bases
is given in Appendix 3 in the purely even and purely
odd cases, and this approach is characteristic-free. 
In the classical case ($e=n=0$) we can also describe the
annihilator of $\coker \Phi $ by saying that it is nonzero only if
$m\geq d$, and then it is generated, as a
${\gl }(V)\times {\gl }(U)$-ideal, by an
$m\times m$ minor of $\Phi $. To simplify the general statement,
we note that a shift of degree
by 1 does not change the annihilator of the cokernel of $\Phi $,
but has the effect of interchanging $m$ with $n$ and $d$ with $e$.
\begin{theorem2} With notation as above, 
the annihilator of the cokernel of $\Phi $ is
nonzero only if

{a)} $m> d$ (or symmetrically $n> e$) or

{b)} $m=d$ and $n=e$.


\noindent In each of these cases the annihilator is generated as
a $\g $-ideal by one element $Z$ of degree $de+d+e$ defined as follows:


In case $a)$ when $m>d$,
\begin{equation*}Z=Z_{1} \cdot X(1,\ldots ,d\ |\ 1,\ldots ,d ),\end{equation*}
where $X(1,\ldots ,d\ |\ 1,\ldots ,d)$ is the
$d\times d$ minor of $X$
corresponding to the first $d$ columns and
$Z_{1} =\prod _{j\leq e,k\leq d+1}b_{j,k}$ is the product
of all the elements in the first $d+1$ columns of $B$
(and symmetrically if $n>e$).

In case $b)$,
\begin{equation*}Z= W_{1}\cdots W_{e}\cdot \det(X), \end{equation*}
where $W_{s}$ is the $(d+1)\times (d+1)$ minor of
$\Phi $ containing $X$ and the entry $y_{s,s}$, that is,
\begin{equation*}W_{s} = \det(X)y_{s,s}+\sum _{1\le i,\le d}\pm 
\det(X({\hat i},{\hat j}))a_{i,s}b_{s,j}.\end{equation*}
\end{theorem2}


Corollary 2 follows from intermediate results in the proof of Theorem
1(a). We next give some examples of Theorem 1(a) and Corollary 2.
\begin{definition1} Suppose that $d=n=0$, so that the presentation
matrix $B$ has only odd degree entries.
A central observation of Green \cite{Gre}
is that the ``exterior minors'' of $\Phi $ are in the annihilator of
$\coker \Phi $ (see Appendix 1 for a direct proof of
Green's result that is different from the one given by Green,
and can serve as an introduction to the proof we give for Theorem 1(a)
in general).
The element $Z$ of Corollary 2 is the product of the elements
in the first column of $\Phi $. Quite generally, it is not hard to see that the product
of all elements
in a $K$-linear combination of the columns of $B$ is an exterior
minor in Green's sense. 
The representation corresponding to the
partition $(1,\dots ,1)$ of $e$ is generated by
${\binom{m+e-1}{e}}$ such products; so 
 ${\binom{m+e-1}{e}}$ exterior
minors generate the annihilator in the generic case.
For
example, taking $m=2$ and $e=2$, the annihilator of the
cokernel of the generic matrix
\begin{equation*}
\left(\begin{matrix}
b_{1,1}&b_{1,2}\\
b_{2,1}&b_{2,2}\end{matrix}\right)
\end{equation*}
where the variables all have odd degree, 
is minimally generated by the three
exterior minors
\begin{equation*}
b_{1,1}b_{2,1},\quad b_{1,2}b_{2,2},\quad (b_{1,1}+b_{1,2})(b_{2,1}+b_{2,2}).
\end{equation*}
\end{definition1}
\begin{definition2} Now suppose that
our generic matrix has size $2\times 2$
with the first row even and the second row odd ($m=2,\; n=0,\; d=e=1$):
\begin{equation*}
\left(\begin{matrix}
x_{1,1}&x_{1,2}\\
b_{1,1}&b_{1,2}\end{matrix}\right).
\end{equation*}
In this case our result shows that the cokernel has annihilator equal
to the product
\begin{equation*}(x_{1,1},\quad x_{1,2})(b_{1,1}b_{1,2},\quad x_{1,1}b_{1,2}-x_{1,2}b_{1,1}),
\end{equation*}
which is minimally generated by 4 elements. The element $Z$ is
$x_{1,1}b_{1,1}b_{1,2}$.
\end{definition2}
\begin{definition3} As a final $2\times 2$ example, consider the case
$m=n=d=e=1$, which for simplicity we write as
\begin{equation*}
\left(\begin{matrix}
x&a\\
b&y\end{matrix}\right).
\end{equation*}
Here the annihilator of the cokernel is again minimally generated
by 4 elements, namely
\begin{equation*}axy,\quad bxy,\quad (xy-ab)x,\quad (xy+ab)y.
\end{equation*}
The element $Z$ is $(xy+ab)x$.
In Appendix 2 we will explain the action of 
$\g $ on these elements.
\end{definition3}
\subsection*{{Positive characteristics}}As we have remarked, Theorem 1 is
characteristic-free in
the purely even or odd cases (and in these cases $\coker \Phi $
is free over $\ZZ $). But
already with $m=n=d=e=1$ as in Example 3, the annihilator is different 
in characteristic 2: in characteristic zero the annihilator is generated by forms of
degree 3, but in characteristic 2 the
algebra $R$ is commutative, and so
the determinant $xy-ab$ is in the annihilator as well.
The annihilator can differ in other characteristics as well. 
Macaulay2 computations show that the case
$d=1,\ e=p-1,\ m=2,\ n=0$ is exceptional in characteristic $p$
for $p=3, 5$  and 7. Perhaps the same holds
for all primes $p$.
The cokernel of the generic matrix over the integers
can also have $\ZZ $-torsion. For example, 
Macaulay2 computation shows that if $d=1,\ e=2,\ m=3,\ n=1$,
then the cokernel of $\Phi _{\ZZ }$ has 2-torsion.

Our interest in extending the Fitting Lemma
was inspired by Mark Green's paper
\cite{Gre}. Green's striking use of
his result on annihilators to prove one of the
Eisenbud-Koh-Stillman conjectures on
linear syzygies turns on the fact that if $N$ is a module
over a polynomial ring $S=K[X_{1},\dots ,X_{m}]$, then $T:=\Tor _{*}^{S}(K,M)$
is a module over the ring $R=\Ext _{S}^{*}(K,K)$, which is an
exterior algebra. Green in effect
translated the hypothesis of the linear syzygy
conjecture into a statement about the degree 1 part of the $R$-free
presentation
matrix of the submodule of $T$ representing the linear part of the
resolution of
$N$, and then showed that the exterior minors generated a certain
power of the maximal ideal of the exterior algebra, 
which was sufficient
to prove the conjecture.
Green's result only gives information on the annihilator in the
case where the elements of the presentation matrix are all odd.
Elements of even degree in an exterior algebra can behave (if the
number of variables is large) very much like variables in a polynomial
ring, at least as far as expressions of bounded degree are concerned.
Thus to extend Green's work it seemed natural to deal with the
case of $\ZZ /2$-graded algebras.

This work is part of a program to study modules and resolutions over
exterior algebras; see Eisenbud-Fl\o ystad-Schreyer \cite{ES},
and Eisenbud-Popescu-Yuzvinsky \cite{EPY} for further information.

We would never have undertaken the project reported in this paper
if we had not had the program Macaulay2 (www.math.uiuc.edu/Macaulay2) of
Grayson and Stillman as a tool; its ability to compute in skew commutative
algebras was invaluable in figuring out the pattern that the results should
have and in assuring us that we were on the right track.

\section*{1. Berele-Regev theory}

For the proof of Theorem 1 we will use the beautiful results of Berele and
Regev \cite{BR} giving the structure of $R$ as a module over $\g $. For the
convenience of the reader we give a brief sketch of what is needed. {We make
use of the notation introduced above: $U=U_{0}\oplus U_{1}$ and $V=V_{0}\oplus
V_{1}$ are $\ZZ /2$-graded vector spaces over the field  $K$ of characteristic
0 with $\dim U=(d,e)$ and $\dim V=(m,n)$.}

The
$\ZZ /2$-graded Lie algebra
$\gl (V)$ is the vector space of $\ZZ /2$-graded
 endomorphisms of $V=V_{0}\oplus V_{1}$. Thus
\begin{equation*}\gl (V)= \gl (V)_{0}\oplus \gl (V)_{1},\end{equation*}
where $\gl (V)_{0}$ is the set of endomorphisms preserving the grading of $V$
and $\gl (V)_{1}$ is the set of
endomorphisms of $V$ shifting the grading by 1. Additively,
\begin{gather*}\gl (V)_{0} = \End_{K} (V_{0} )\oplus 
\End_{K} (V_{1} ),\\
\gl (V)_{1} =\Hom_{K} (V_{0} ,V_{1} )\oplus \Hom_{K} (V_{1} ,V_{0} ).
\end{gather*}
The commutator of the pair of homogeneous elements $x,y\in \gl (V)$ is
defined
by the formula
\begin{equation*}[x,y]= xy-(-1)^{\deg(x)\deg(y)}yx.\end{equation*}

By
a $\gl (V)$-module we mean a $\ZZ /2$-graded vector space $M=M_{0}\oplus M_{1}$
with a bilinear map of $\ZZ /2$-graded vector spaces
$\circ :\gl (V)\times M\rightarrow M$   satisfying the identity
\begin{equation*}[x,y]\circ m = x\circ (y\circ m)-(-1)^{\deg(x)\deg(y)} y\circ (x\circ m))
\end{equation*}
for homogeneous elements $x,y\in \gl (V), m\in M$.

In contrast to the classical theory, not every representation of the
$\ZZ /2$-graded Lie algebra $\gl (V)$ is
semisimple. For example, its natural action on mixed tensors $V^{\otimes k}\otimes V^{*}{}^{\otimes l}$ is in
general not completely reducible. However, its action on
$V^{\otimes t}$ decomposes just as in the ungraded case:

\begin{theorem3} The action of $\gl (V)$ on $V^{\otimes t}$ is
completely reducible for each $t$.
More precisely, the analogue of Schur's double centralizer theorem holds,
and the irreducible $\gl (V)$-modules
occurring in the decomposition of $V^{\otimes t}$  are in \textup{1-1}
correspondance with irreducible representations of
the symmetric group $\Sigma _{t}$ on $t$ letters. These irreducibles are the Schur
functors
\begin{equation*}\myS _{\lambda }(V) = e(\lambda )V^{\otimes t}\end{equation*}
where $e(\lambda )$ is a Young idempotent corresponding to a partition
$\lambda $ in the group ring of the
symmetric group $\Sigma _{t}$.
\end{theorem3}


This notation is consistent with the notation above in the sense that the $d$-th
homogeneous component of the ring $\myS (V)$ is $\myS _{d} (V)$ where $d$ represents the
partition $(d)$ with one part.

Here we use the symbol $\myS _{\lambda }$ to denote the $\ZZ /2$-graded
version of the Schur functor $S_{\lambda }$; the latter acts on ungraded vector
spaces. Recall that the functor $\wedge ^{\lambda }$ is by definition the same
as the functor $S_{\lambda '}$, where $\lambda '$ denotes the partition
that is conjugate to $\lambda $. (For example, the
conjugate partition to $(2)$ is $(1,1)$.)
We will extend this by writing
$\bigwedge ^{\lambda }:= \myS _{\lambda '}$ for the $\ZZ /2$-graded version.
The partition $(d)$ with only one part will be denoted simply $d$; so,
for example,
$\myS _{2}(V)=\bigwedge ^{(1,1)}V=S_{2}(V_{0})\oplus (V_{0}\otimes V_{1})\oplus \wedge ^{2}V_{1}$ and similarly
$\bigwedge ^{2}V=\myS _{(1,1)}V=\wedge ^{2}V_{0}\oplus V_{0}\otimes V_{1}\oplus S_{2}(V_{1})$.
In each case the decomposition is as representations of the subalgebra
$\gl (V_{0})\times \gl (V_{1})\subset \gl (V)$. Similar decompositions hold for
all $\myS _{d}$ and $\bigwedge ^{d}V$. (If we were not working in
characteristic zero, we would use divided powers in place of
symmetric powers in the description
of $\bigwedge V$.)

Proposition 1.1 implies that the parts of the representation theory of $\gl
(V)\times \gl (U)$ that involve only tensor products of $V$ and $U$ and their
summands are parallel to the representation theory in the case $V_{1} =U_{1}
=0$, which is the classical representation theory of a product of the two
general linear Lie algebras $\mathfrak{gl}(V_{0})\times
\mathfrak{gl}(U_{0})$.

The proposition also implies that the decompositions into irreducible
representations of tensor products of the $\myS _{\lambda }(V)$, as well as the
decompositions of their symmetric
and exterior powers,  correspond to the
decompositions in the even case: we just have
to replace the ordinary Schur functors $S, \wedge $ by their
$\ZZ /2$-graded analogues $\myS , \bigwedge $.

The
formulas giving equivariant embeddings or
equivariant projections may also be derived from the corresponding formulas
in the even case by
applying the principle of signs: The formulas in the even case involve many
terms where the basis elements are
permuted in a prescribed way. The basis elements have degree $0$. To write
down a $\ZZ /2$-graded analogue of
such a formula, we simply allow the basis elements to have even or odd degree
and we adjust the signs of terms in
such a way that changing the order of two
homogeneous elements $x$ and $y$ of
$V$  in the $\ZZ /2$-graded analogue of the formula
will cost the additional factor
$(-1)^{\deg(x)\deg(y)}$.

There is a $\ZZ /2$-graded analogue of the Cauchy decomposition,
which follows as just described from Proposition 1.1 together with
the corresponding result in the even case (proven in Macdonald \cite{Mac},
Chapter 1, and in DeConcini, Eisenbud, and Procesi \cite{DEP}).
Recall that $\g =\gl (V)\times \gl (U)$.

\begin{theorem4} The $t$-th component
$\myS _{t} (V\otimes U)$ of $\myS (V\otimes U)$
decomposes as a $\g $-module as
\begin{equation*}\myS _{t} (V\otimes U)=\bigoplus 
_{\lambda ,|\lambda |=t} \myS _{\lambda }(V)\otimes \myS _{\lambda }(U).
\end{equation*}
\end{theorem4}


Another application of the same principle shows that to
describe the annihilator of the cokernel of $\Phi $, and what
generates it, it suffices to describe which
representations $\myS _{\lambda }V\otimes \myS _{\lambda }U$ it contains:

\begin{theorem5} If $I\subset \myS (V\otimes U)$ is a $\g $-invariant
ideal, then $I$ is a sum of subrepresentations $\myS _{\lambda }V\otimes \myS _{\lambda }U$.
Moreover, the ideal generated by $\myS _{\lambda }V\otimes \myS _{\lambda }U$ contains
$\myS _{\mu }V\otimes \myS _{\mu }U$ if and only if
$\mu \supset \lambda $.
\end{theorem5}


Although it is not so simple to describe the vectors in $\myS _{t}(V\otimes U)$
that lie in a given irreducible summand, we can, as
in the commutative case, define a filtration that has these
irreducible representations as successive factors.
We start by defining a map
$\rho _{t} : \bigwedge ^{t} V\otimes \bigwedge ^{t}
U\hookrightarrow \myS _{t} (V\otimes U)$ as the composite
\begin{equation*}\bigwedge ^{t} V\otimes \bigwedge ^{t}U\to \otimes ^{t}V\ \otimes \ \otimes ^{t}U\to \myS _{t}(V\otimes U)
\end{equation*}
where the first map is the tensor product of the two diagonal maps
(here we use the sign conventions for $\ZZ /2$-graded vector spaces)
and the second map simply pairs corresponding factors. Thus
\begin{equation*}\rho _{t} (v_{1}\wedge \ldots \wedge v_{t}\otimes u_{1}\wedge \ldots \wedge u_{t}
)=\sum _{\sigma \in \Sigma _{t}} \pm (v_{1}\otimes u_{\sigma (1)})\cdot \ldots \cdot (v_{t}\otimes u_{\sigma (t)})
\end{equation*}
 where the sign $\pm $ is the sign of the
permutation $\sigma $ adjusted by
the rule that switching homogeneous elements
$x, y$ from either $V$ or $U$ means we multiply by $(-1)^{\deg(x)\deg(y)}$.
For example, if $V$ and $U$
were both even, the image of this map would be
the span of the $t\times t$ minors of the generic matrix;
when $V$ is even and $U$ is odd, the image is the span of the
space of ``exterior minors" as in Green \cite{Gre}.

For any partition $\lambda =(\lambda _{1},\dots ,\lambda _{s})$
we define  $\F _{\lambda }$ to be
the image of the composite map
\begin{equation*}m\circ (\rho _{\lambda _{1}}\otimes \ldots \otimes \rho _{\lambda _{s}}):\bigwedge ^{\lambda _{1}}
V\otimes \bigwedge ^{\lambda _{1}}U
\otimes \ldots \otimes \bigwedge ^{\lambda _{s}}V\otimes \bigwedge ^{\lambda _{s}}U\rightarrow \myS _{|\lambda | }(V\otimes U)
\end{equation*}
where $m$ denotes the multiplication map in $\myS (V\otimes U)$.

As in the even case, we order partitions of $t$ by saying $\lambda < \mu $ if
and only if $\lambda ^{\prime }_{i}
>\mu ^{\prime }_{i}$ for the smallest number $i$ 
such that $\lambda ^{\prime }_{i}
\ne \mu ^{\prime }_{i}$.
Finally, we define the subspaces
\begin{equation*}\F _{<\lambda }=\sum _{|\mu |=|\lambda |,\ \mu <\lambda }\F _{\mu }\qquad \subset \qquad \F _{\le \lambda }=\sum _{|\mu |=|\lambda |,\ \mu \le \lambda } \F _{\mu }.
\end{equation*}
In the classical case, $\F _{\leq \lambda }$ is spanned by certain products of
minors of the generic matrix. The {\em straightening law\/} of
Dubillet, Rota, and Stein
\cite{DRS} shows that we get a basis if we choose only ``standard'' products of
these
types, and the successive quotients in the filtration
are the irreducible representations of $\operatorname{GL}(V)\times \operatorname{GL}(U)$.
The analogue
in our $\ZZ /2$-graded case is

\begin{theorem6} The subspaces $\F _{\le \lambda }$ define a
$\g $-invariant filtration on
$\myS _{|\lambda |}(V\otimes U)$. The quotient $\F _{\le \lambda }/ \F _{<\lambda }$ is
isomorphic to $\bigwedge ^{\lambda }V\otimes \bigwedge ^{\lambda }U=\myS _{\lambda ^{\prime }} V\otimes \myS _{\lambda ^{\prime }} U$.\end{theorem6}


There is also one element of each irreducible representation that is easy to
describe: the {\em highest weight vector.} To speak of highest weight vectors
we must choose ordered bases $\{u_{1},\dots ,u_{d}\}$ and $\{u'_{1},\dots
,u'_{e}\}$ of $U_{0}$ and $U_{1}$, and ordered bases  $\{v_{1},\dots ,v_{m}\}$
and $\{v'_{1},\dots ,v'_{n}\}$ of $V_{0}$ and $V_{1}$.

\begin{theorem7}
Let
$\lambda = (\lambda _{1} ,\ldots ,\lambda _{s})$
be a partition, and let
$w^{1}_{i}\in \bigwedge ^{\lambda _{i}}(V)$ and
$w^{2}_{i}\in \bigwedge ^{\lambda _{i}}(U)$ be the elements
\begin{gather*}
w_{i}^{1} = \begin{cases}
v_1\wedge \dotsb \wedge v_{\lambda_i},&\text{if}\ \lambda_i\le m,\\
v_1\wedge \dotsb \wedge v_m\wedge v_i^{\prime(\lambda_i-m)},&\text{otherwise},
\end{cases}\\
w_{i}^{2} = 
\begin{cases}
u_1\wedge \dotsb \wedge u_{\lambda_i},&\text{if}\ \lambda_i\le d,\\
u_1\wedge \dotsb \wedge u_d\wedge u_i^{\prime(\lambda_i-d)},&\text{otherwise}.
\end{cases}
\end{gather*}
The element
\begin{equation*}c_{\lambda }= \prod _{i=1}^{s} \rho _{\lambda _{i}} (w_{i}^{1}\otimes w_{i}^{2} )
\quad \in \quad \myS (V\otimes U)
\end{equation*}
is the  highest weight vector from the irreducible component
$\bigwedge ^{\lambda }V\otimes \bigwedge ^{\lambda }U
= \myS _{\lambda '} V\otimes \myS _{\lambda '} U$,
where
$\lambda '$
is the conjugate partition to $\lambda $.
\end{theorem7}


The Berele-Regev theory allows us to give explicit
generators for the annihilator, generalizing the 
ordinary minors of $\Phi $ in the commutative case.
More generally, we can give generators for arbitrary
$J_{\lambda }$. 

We start with a double tableau $(S,T)$,
 that is, two sequences of tensors
$v_{i,1}\wedge \ldots \wedge v_{i,\lambda _{i} }\in \bigwedge ^{\lambda _{i}} V$ 
and 
$u_{i,1}\wedge \ldots \wedge u_{i,\lambda _{i}}\in \bigwedge ^{\lambda _{i}}U$ 
($1\le i\le s$). We imagine that the elements
$v_{i,j}\in V$ correspond to the $i$-th row of the tableau 
$S$ of shape $\lambda $, 
and the elements $u_{i,j}\in U$ correspond to the $i$-th row of
another tableau $T$ of shape $\lambda $. 
We define
\begin{equation*}\rho (S\otimes T)= 
\prod _{1\le i\le s} \rho _{\lambda _{i}}(v_{i,1}
\wedge \ldots \wedge v_{i,\lambda _{i}}\otimes u_{i,1}\wedge \ldots \wedge u_{i,\lambda _{i}}).
\end{equation*}

We think of $\lambda $ as a Ferrers diagram. If $S$ is a tableau of
shape $\lambda $ and $\sigma $ is a permutation of the boxes in $\lambda $,
then $\sigma (S)$ is another tableau of shape $\lambda $ (here we write
$\sigma $ as a product of transpositions, and introduce a minus sign
whenever we interchange two elements of odd degree).
Let $P(\lambda )$ be the group of permutations of the boxes in 
$\lambda $ that preserve the columns of $\lambda $.

\begin{theorem8}
The representation
$\bigwedge ^{\lambda }V\otimes \bigwedge ^{\lambda }U\subset R$
is generated by elements 
\begin{equation*}\pi (S, T)=\sum _{\sigma \in P(\lambda )} \rho (\sigma S\otimes T ),
\end{equation*} 
or,
equivalently, by 
\begin{equation*}\pi ^{\prime }(S, T)=\sum _{\sigma \in P(\lambda )} \rho (S\otimes \sigma T)
\end{equation*}
where $S$ and $T$ range over all tableaux of shape $\lambda $.
\end{theorem8}
\begin{proof} We show that the $\pi (S,T)$ generate $\bigwedge ^{\lambda
}V\otimes \bigwedge ^{\lambda }U$; the proof for $\pi '$ is
similar. 
Since $\rho (S,T)$ is antisymmetric in the elements appearing
in each row of $S$, the element 
$\sum _{\sigma \in P(\lambda )}\rho (\sigma (S), T)$ is 
the $\gl (V)$-linear projection of $\rho (S,T)$ to the 
$\bigwedge ^{\lambda }V$-isotypic component of $R$.
By Corollary 1.2 we have 
$R=\bigoplus _{\lambda }\bigwedge ^{\lambda }V\otimes \bigwedge ^{\lambda }U$;
so this isotypic component is 
$\bigwedge ^{\lambda }V\otimes \bigwedge ^{\lambda }U\subset R$.
\end{proof}


To find a minimal set of generators for
 $J_{\lambda }$ inside this generating set, choose bases 
$\{u_{i}\}$,$\{u'_{i}\}$, $\{v_{i}\}$,$\{v'_{i}\}$ as above, and 
order the bases of $U$ and $V$ by
$u_{1}<\cdots <u_{d}<u'_{1}<\cdots <u'_{e}$ and
$v_{1}<\cdots <v_{m}<v'_{1}<\cdots <v'_{n}$.
A double tableau $(S,T)$ whose entries $u_{i,j}$ and $v_{i,j}$
come from these bases is called {\em standard\/} if the
following conditions are satisfied:
 \begin{gather*}
v_{i,j}<v_{i,j+1}\ 
\operatorname{if}\ v_{i,j}\ \operatorname{is\ even},\ 
v_{i,j}\le v_{i,j+1}\ \operatorname{if}\ v_{i,j}\ \operatorname{is\ odd},\\
v_{i,j}\le v_{i+1,j}\ \operatorname{if}\ v_{i,j}\ \operatorname{is\ even},\ 
v_{i,j}< v_{i+1,j}\ \operatorname{if}\ v_{i,j}\ \operatorname{is\ odd},
\end{gather*}
and similarly for the $u_{i,j}$.

\begin{theorem9} The ideal $J_{\lambda }$ is
minimally generated by the elements $\pi (S,T)$ where
$(S,T)$ ranges over the set of double standard tableaux
of shape $\lambda $.
\end{theorem9}
\begin{proof} Berele and Regev proved that 
the standard tableaux form a basis of $\bigwedge ^{\lambda }U$.
\end{proof}
\section*{2. Proof of Theorem 1(\textup{a})}

In this section
$U$ and
$V$ are $\ZZ /2$-graded vector spaces of dimensions $(d,e)$
and $(m,n)$ respectively, and $\Phi $ is the generic map,
defined tautologically over $R=\myS (V\otimes U)$.

We write $\Lambda (d,e)$ for the partition with
$e+1$ parts $((d+1)^{e},d)=(d+1,\dots ,d+1,d)$;
that is,
$\Lambda (d,e)$ corresponds to the Ferrers diagram
that is a
$(d+1)\times (e+1)$ rectangle minus the box in the lower
right-hand corner.
For example, $\Lambda (2,3)$ may be represented by the Ferrers
 diagram
\begin{equation*}\Lambda (2,3) =\ \Young (,,|,,|,,|,).
\end{equation*}

For any
partition $\lambda $ we denote by $I_{\lambda }$ the
ideal in $R$ generated
by the representation $\bigwedge ^{\lambda }V\otimes \bigwedge ^{\lambda }U$.
With this notation, Theorem 1(a) takes the form

\begin{theorem10} The annihilator of the cokernel of $\Phi $
is equal to $I_{\Lambda (d,e)}$.
\end{theorem10}


Theorem 1(a) implies that the representations
appearing in the annihilator of $\coker \Phi $ depend
only on the dimension of $U$, not the
dimension of $V$, as long as the dimension of $V_{0}$ is large
(Corollary 2.2), and we begin by proving this.
For the precise statement, we will use the following notation:
Let $V^{\prime }$ be another $\ZZ /2$-graded
vector space, and let
$\Phi ^{\prime }$ be the generic map $R^{\prime }\otimes V^{\prime }\to R^{\prime }\otimes U^{*}$
where $R^{\prime }= \myS (V^{\prime }\otimes U )$.
If $V$ is a summand of $V'$, so that $V'=V\oplus W$,
then the ring $R=\myS (V\otimes U)$ can be identified with a subring of
$R^{\prime }$. We want to compare the
annihilators of the modules $\coker \Phi $ and
$\coker \Phi ^{\prime }$.

\begin{theorem11}
If $V$ is a $\ZZ /2$-graded
summand of $V'$,
then
$\ann \coker \Phi =R\cap \ann \coker \Phi '$. More precisely,

\textup{a)} $\coker \Phi $ is an $R$-submodule of $\coker \Phi ^{\prime }$, and 

\textup{b)} $\coker \Phi '$ is a quotient of
$(\coker \Phi )\otimes R'$.
\end{theorem11}
\begin{proof} The first statement follows
easily from a) and b).

For the proof of a) and b) we may write
$V'=V\oplus W$, and we  make
use of the $\NN $-grading of $R'$ for which $V\otimes U$
has degree 0 and $W\otimes U$ has degree 1. (This grading
has nothing to do with the $\ZZ /2$-grading used elsewhere in
this paper!)
The map $\Phi ^{\prime }$ is homogeneous of degree 0 if
we twist the summands of its source appropriately,
\begin{equation*}\Phi ^{\prime }: R^{\prime }\otimes W(-1)\oplus R^{\prime }\otimes V
\to ^{(\Phi '_{1},\Phi '_{0})}
R^{\prime }\otimes U^{*}.
\end{equation*}
So we have an induced $\NN $-grading on $\coker \Phi ^{\prime }$.
Since $\Phi '_{0}=\Phi $, we see that $(\coker \Phi ')_{0}\linebreak
=\coker \Phi $.
Since the elements of $R$ have degree 0, this is an $R$-submodule,
as required for a).

For b) it suffices to note that $\coker \Phi '$ is obtained
from $(\coker \Phi )\otimes R'$ by factoring out the relations
corresponding to $W\otimes R'$.\end{proof}


\begin{theorem12} With $U,V,V',\Phi ,\Phi '$
as above, suppose that $V$ is such that $I_{\lambda }\neq 0$
in $\myS (V\otimes U)$. If $I_{\lambda }\subset \ann \coker \Phi $, then
$I_{\lambda }\subset \ann \coker \Phi ^{\prime }$.
\end{theorem12}
\begin{proof} 
The inclusion $R=\myS (V\otimes U)\subset R'=\myS (V^{\prime }\otimes U)$ carries
$\bigwedge ^{\lambda }V\otimes \bigwedge ^{\lambda }U$
into
$\bigwedge ^{\lambda }V^{\prime }\otimes \bigwedge ^{\lambda }U$.
The conclusion now follows from
Proposition 2.1 a) and b).
\end{proof}
\begin{proof}[Proof of Theorem \textup{1(a)}]
We first show that
$I_{\Lambda (d,e)}$ is contained in the annihilator of
$M:=\coker \Phi $. By Corollary 2.2, it is enough, given $U$, to
produce one nonzero element from
$\bigwedge ^{\Lambda (d,e)}V\otimes \bigwedge ^{\Lambda (d,e)}
U\subset \myS (V\otimes U)
$
that annihilates the cokernel of $\Phi $ for some space $V$.
By Corollary 2.2 it suffices to prove this result
in the case $m=d+1, n=0$, that is,  $\dim V=(d+1,0)$.

Let $u_{1} ,\ldots ,u_{d}$  be a basis of $U_{0}$, let
$u^{\prime }_{1} ,\ldots ,u^{\prime }_{e}$ be
a basis of $U_{1}$,  and let $v_{1} ,\ldots ,v_{d+1}$ be a basis
of $V=V_{0}$.
We denote the variables from the
$U_{0}\otimes V_{0}$ block by $x_{i,k}$ ($1\le i\le d,\ 1\le k\le d+1$),
and the variables from the $U_{1}\otimes V_{0}$ block by
$b_{j,k}$ ($1\le j\le e,\ 1\le k\le d+1$). Thus:
\begin{equation*}
\Phi = \left(\begin{matrix} X\\B\end{matrix}\right),
\quad X=(x_{i,k}), \quad B=(b_{j,k}).
\end{equation*}

Let
\begin{equation*}Z=Z_{1} \cdot X(1,\ldots ,d\ |\ 1,\ldots ,d )\end{equation*}
where $Z_{1} =\prod _{j,k} b_{j,k}$
is the product of all the entries of $B$ and
$X(1,\ldots ,d\ |\ 1,\ldots ,d )$
is the $d\times d$ minor
of the matrix $X$ corresponding to the
first $d$ columns.

We now show that $Z$ annihilates $M$.
Indeed, by the classical
Fitting Lemma we know that $X(1,\ldots ,d\ |\ 1,\ldots ,d )$ annihilates
the even generic module $M/(V\otimes U_{1})M$. Thus
every basis element $u_{k}$ multiplied by
$X(1,\ldots ,d\ |\ 1,\ldots ,d )$
can be expressed modulo the image of $\Phi $
as a linear combination of $u^{\prime }_{1} ,\ldots ,u^{\prime }_{e}$ with
coefficients of positive degee in the
variables $b_{j,k}$. Since these variables are odd,
$Zu_{k} =0$ in $M$.

To see that $Z u_{l}^{\prime }$ is also 0 in $M$,
we use the classical 
Fitting Lemma  again on the
first $d$ columns of the matrix of $\Phi $ to see that for any
$1\leq i\leq d$ the element
$X(1,\ldots ,d\ |\ 1,\ldots ,d )u_{i}$
can be expressed,
modulo the image of $\Phi $, as
a linear combination of the $b_{s,t}u_{j}^{\prime }$.
On the other hand, if we multiply the last
column of the matrix of $\Phi $ by the product $Z_{1}'$ of all
the $b_{j,k}$ except for the $b_{l,d+1}$, we get
an expression for $Z_{1} u_{l}^{\prime }$, modulo the
image of $\Phi $, as a linear
combination of $Z_{1}'u_{1} \ldots ,Z_{1}'u_{d}$.
Thus $Zu'_{l}=X(1,\ldots ,d\ |\ 1,\ldots ,d )Z_{1}u'_{l}=0$ in $M$,
as required.

Next we prove that the element $Z$ is a weight vector (not generally
a highest weight vector) and lies in
$\bigwedge ^{\Lambda (d,e)} V\otimes \bigwedge ^{\Lambda (d,e)} U$.
Indeed, $X(1,\ldots ,d\ |\ 1,\ldots ,d)$ is a weight vector in
$\bigwedge ^{d} V\otimes \bigwedge ^{d} U$.
The element $Z_{1}$ is a weight vector in the representation
$\bigwedge ^{(d+1)^{e}}V\otimes \bigwedge ^{(d+1)^{e}}U$.
The product is thus
contained in the ideal
$\F _{\le \Lambda (d,e)}$.
The element $Z$ has degree $(e+1)(d+1)-1$, but it involves only $d+1$
elements from $V=V_{0}$. By Proposition 1.5 its
weight can occur only in representations 
$\bigwedge ^{\lambda }V\otimes \bigwedge ^{\lambda }U\subset \myS _{\lambda }(V\otimes U)$ 
with
$\lambda $ having all parts $\le d+1$. Since
$\Lambda (d,e)$ is the only partition $\lambda $ with at most
$e+1$ parts having
$|\lambda |=(d+1)(e+1)-1$
and each $\lambda _{i}\le d+1$, we are done.
This argument shows that
$I_{\Lambda (d,e)}$ is contained in the annihilator of the cokernel of
$\Phi $.

Now let $\mu $ be a partition not containing $\Lambda (d,e)$.
To complete the proof of Theorem 1(a), we must show that
the ideal $I_{\mu }$ does not
annihilate $M=\coker \Phi $ or, equivalently, that the highest
weight vector $c_{\mu }$ does not annihilate $M$.

Since $\mu $ does not contain
$\Lambda (d,e)$, it does not contain one of the extremal boxes of
$\Lambda (d,e)$.
By shifting the gradings of $V, U$ by $1$ we do not alter the
annihilator of the generic map, but we
change the notation so that all partitions are changed to their
conjugates. Thus we may assume that $\mu _{e}\le d$.
By Corollary 2.2 we may further assume that $n=0$, so
that $V=V_{0}$, and that $m>>0$.
To prove the theorem, we will carry out an induction on $d$.

If $d=0$, we must show that the annihilator
of the cokernel of $\Phi $ is contained in
$I_{(1^{e} )}$; or, equivalently, that it contains
no $I_{\lambda }$ where $\lambda $ has fewer than $e$ parts.
Set
\begin{equation*}Z_{1}=\prod _{1\le j\le e-1 ,1\le k\le m} b_{j,k}.
\end{equation*}
By Proposition 1.5,
$Z_{1}$ is the highest weight vector in
$\bigwedge ^{m^{e-1}}V\otimes \bigwedge ^{m^{e-1}}U$.
The element
$Z_{1} u_{e}^{\prime }$ is not
in the image of $\Phi $,
because the coefficient of
$u_{e}^{\prime }$ in any element from the image of $\Phi $ is in the
ideal generated by $b_{e,1},\ldots ,b_{e,m}$,
while $Z_{1}$ is not in this ideal.
Since $Z_{1}$ does not annihilate $M$, no $I_{\lambda }$
such that $\lambda $ has
$<e$ parts can annihilate $M$.

In case $d>0$ the matrix of $\Phi $ will contain an
even variable $x_{1,1}$. To complete the induction we will
invert this variable and use
\end{proof}
\begin{theorem13} \textup{a)}
 Over the ring $R_{1} =R[x_{1,1}^{-1}]$ the map $\Phi $ can be
reduced by row and column operations to the form
\begin{equation*}\Phi ^{\prime }\oplus \ id :\
(V^{\prime }\otimes _{K} R_{1}) \oplus R_{1} \rightarrow U^{\prime }{}^{*}\otimes _{K} \oplus R_{1}
\end{equation*}
where $V$ is a $\ZZ /2$-graded vector space of dimension $(m-1,n)$ and
$U$ is a $\ZZ /2$-graded vector space
of dimension $(d-1,e)$. Moreover, the ring $R^{\prime }$ generated over $K$ by
the entries of $\Phi ^{\prime }$ is
isomorphic to $\myS (V^{\prime }\otimes U^{\prime })$, and $R_{1}$ is a flat
extension of $R^{\prime }$.

\textup{b)} The localization of the ideal $I_{\mu }$ at $x_{1,1}$ is isomorphic
to the extension of the ideal
$J^{\prime }_{\nu }$ from $R^{\prime }$ where $\nu $ is the
partition obtained from $\mu $ by subtracting $1$ from each nonzero part.
\end{theorem13}
\begin{proof}[Proof of Lemma \textup{2.3}]
Column and row reduction give the following formulas
for the entries of $\Phi ^{\prime }$:
\begin{equation*}x_{i,k}^{\prime }=x_{i,k}-{{\frac{x_{1,k} x_{i,1}}{x_{1,1}}}},
\qquad a_{i,l}^{\prime }=a_{i,l}-{{\frac{a_{1,l} x_{i,1}}{x_{1,1}}}},
\end{equation*}
\begin{equation*}b_{j,k}^{\prime }=b_{j,k}-{{\frac{x_{1,k} b_{j,1}}{x_{1,1}}}},
\qquad y_{j,l}^{\prime }=y_{j,l}-{{\frac{a_{1,l} b_{j,1}}{x_{1,1}}}}.
\end{equation*}
Consequently,
\begin{equation*}R_{1} = R^{\prime }[x_{1,1}, x_{1,1}^{-1}][x_{1,2},\ldots ,x_{1,m},\
a_{1,1},\ldots ,a_{1,n},\  x_{2,1},\ldots ,x_{d,1},\
b_{1,1},\ldots ,b_{e,1}]
\end{equation*}
in the sense of $\ZZ /2$-graded algebras. This proves part a).

To prove part b), we first observe that the localization of the ideal $I_{(t
)}(\Phi )$ gives the ideal
$I_{(t-1)}(\Phi ^{\prime })$. Indeed, the ideal $I_{(t )}(\Phi )$
is generated by $\ZZ /2$-graded analogues of $t\times t$ minors of $\Phi $. After localization it becomes the ideal $I_{(t
)}(\Phi ^{\prime }\oplus id_{R_{1}} )$ generated by the $\ZZ /2$-graded analogues
of $t\times t$ minors of $\Phi ^{\prime }\oplus id_{R_{1}}$. Let us call the row and column of the matrix
 $\Phi ^{\prime }\oplus id_{R_{1}}$ corresponding to the summand $R_{1}$ the
distinguished row and column. 
Every $\ZZ /2$-graded analogue of a $t\times t$ minor of
$\Phi ^{\prime }\oplus id_{R_{1}}$ is either a $(t-1)\times (t-1)$ minor of
$\Phi ^{\prime }$ (in case it contains the distinguished row
and column), zero (if it contains the distinguished row but not the
distinguished column, or vice versa), or a $t\times t$
minor of $\Phi ^{\prime }$ if it does not contain the distinguished row or
column.

To show that the result generalizes to an arbitrary partition
$\mu $, we order the bases so that the
distinguished row and column come first. We saw
in Proposition 6 that the highest weight vectors in $\bigwedge ^{\mu }V\otimes \bigwedge ^{\mu }U$ are the products of minors of the matrix
$\Phi $ on
some initial subsets of rows and columns of $\Phi $. So
after localization each factor 
will contain both the distinguished row and
the distinguished column of $\Phi ^{\prime }\oplus id_{R_{1}}$.
\end{proof}


\begin{proof}[Completion of the Proof of Theorem \textup{1(a)}]
Now suppose that $d>0$ and $n=0$. 
We may of course assume that $m\neq 0$,
so that the matrix of 
$\Phi $ contains the even variable $x_{1,1}$.
It is enough to prove that 
$I_{\mu }M\ne 0$ after inverting $x_{1,1}$.
The ideal
$I_{\mu }$ will localize to the ideal
$J^{\prime }_{\nu }$ where $\nu $ is equal to 
$\mu $ with all parts decreased by
$1$.  The graded vector space
$U$ of dimension $(d,e)$ will change to the 
$\ZZ /2$-graded vector space
$U^{\prime }$ of dimension
$(d-1,e)$. The desired conclusion 
follows by induction on $d$.\end{proof}


\section*{3. Proof of Theorem 1(\textup{b})}

If $\phi : R^{m}\to R^{d}$ is a matrix representing a map of free modules
over a commutative ring, then, as we noted in the introduction, there
are 
inclusions  
$\ann(M)\cdot I_{i}(\phi )\subset I_{i+1}(\phi )$ for $0\leq i<d$, and thus, by induction,
$\ann(M)^{d}\subset I_{d}(\phi )$;
see for example Eisenbud \cite{Eis}. 
To
prove these inclusions, one first notes that the cokernel of 
$\phi $ is the same as the cokernel of 
\begin{equation*}\psi :V_{0}\otimes R\oplus U_{0}^{*}\otimes R\rightarrow U_{0}^{*}\otimes R
\end{equation*}
where $\psi =(\phi , \, a\cdot Id )$. Thus $I_{j} (\phi )=I_{j} (\psi )$
and
$I_{j+1}(\phi )=I_{j+1}(\psi )\supset a\cdot I_{j} (\phi )$.
We will carry out the same
approach in the $\ZZ /2$-graded case. 

In this section we work  with an arbitrary map
\begin{equation*}\phi :V\otimes R\rightarrow U^{*}\otimes R
\end{equation*}
of $\ZZ /2$-graded free modules over a $\ZZ /2$-graded commutative ring $R$.
The first step is to show that, just as in the classical case, 
the ideal $I_{\lambda }(\phi )$ depends only on the cokernel of $\phi $ and
on the number and degrees of the generators chosen.

\begin{theorem14}
If $\alpha :V'\otimes R\rightarrow V\otimes R$, then 
\begin{equation*}I_{\lambda }(\phi \alpha )\subset I_{\lambda }(\phi ).
\end{equation*}
In particular, if
$\psi :V'\otimes R\rightarrow U^{*}\otimes R$ has the same cokernel
as $\phi $, then $I_{\lambda }(\phi )=I_{\lambda }(\psi )$.
\end{theorem14}
\begin{proof} The second statement follows from the first, because
each of the maps $\phi $ and $\psi $ factors through the other.

To prove the first statement, we use the notation of Proposition 1.6.
For any map $W\otimes R\to U^{*}\otimes R$, and any tableaux $S$ and $T$
of elements in $W$ and $U$, both of shape $\lambda $, 
we let
$\pi '_{\psi }(S,T)$ be the result of specializing the element $\pi '(S,T)$ defined
for the generic map $\Phi $ when $\Phi $ is specialized to $\psi $.
By Proposition 1.6,
it is enough to show that when $W=V'$ the element $\pi '_{\phi \alpha }(S,T)$
is in $I_{\lambda }(\phi )$. We have
\begin{align*}
&\rho _{l}(v_{1}'\wedge \dots v_{l}'\otimes u_{1}\wedge \dots u_{l})\\
&\qquad =
\sum _{i_{1}<\cdots <i_{l}}
\rho _{l}(v_{1}'\wedge \dots v_{l}'\otimes v_{i_{1}}^{*}\wedge \dots v_{i_{l}}^{*})
\rho _{l}(v_{i_{1}}\wedge \dots v_{i_{l}}\otimes u_{1}\wedge \dots u_{l})
\end{align*}
where $v_{1},\dots , v_{m+n}$ and 
$v_{1}^{*},\dots , v_{m+n}^{*}$ are dual bases of $V$ and $V^{*}$.
Using this identity to rewrite the formula for
$\pi '_{\phi \alpha }(S',T)$, 
where $S'$ is a tableau of shape $\lambda $ with entries in $V'$
and $T$ is a tableau of shape $\lambda $ with entries in $U$, we see that
$\pi '_{\phi \alpha }(S',T)$ is a linear combination of elements of the form
$\pi '_{\phi }(S,T)$, where 
$S$ is a tableau of shape $\lambda $ with entries in $V$.
\end{proof}


Lemma 3.1 implies, in particular, that two presentations of the same module
with the same numbers of even and odd generators have the same 
ideals $I_{\lambda }(\phi )$. Similar arguments show that we can allow
for presentations with different numbers of generators as long 
as we change the partitions suitably: if we add $d'$ even and $e'$ odd 
generators, then we have to expand $\lambda $ by adding $d'$ columns
of length equal to the length of the first column  and 
$e'$ rows of length equal to the length of the first row of the
resulting partition (or vice versa). In this sense 
the ideals $I_{\lambda }(\phi )$ depend only on the cokernel of 
$\phi $.

The main result of this section is the following. 

\begin{theorem15} Let $R, U, V, \phi $ be as in 
the beginning of the introduction, and let $M=\coker \phi $.

\textup{a)} Let $s$ be an integer, $0\le s\le d-1$. 
If $x_{1} ,\ldots ,x_{e+1}\in \Ann_{R} M$, 
then 
\begin{equation*}x_{1}\ldots x_{e+1}I_{\Lambda (s,e)}(\phi )\subset I_{\Lambda
(s+1,e)}(\phi ). 
\end{equation*}

\textup{b)} If $x_{1} ,\ldots ,x_{e} \in \Ann_{R} M$, then
$x_{1}\ldots x_{e}\in I_{\Lambda (0,e)}(\phi )$.
\end{theorem15}


As in the classical case, we derive

\begin{theorem16} Let $M$ be a 
$\ZZ /2$-graded module over a $\ZZ /2$-graded ring $R$,
with the presentation
$\phi :V\otimes R\rightarrow U^{*}\otimes R$. 
Assume that $\dim U = (d,e), \dim V=(m,n)$. Let
$x_{1} ,\ldots ,x_{(d+1)(e+1)-1}$ be  homogeneous elements from
$\Ann_{R} M$. Then $x_{1} \ldots x_{(d+1)(e+1)-1}\in I_{\Lambda (d,e)}(\phi )$.
\end{theorem16}

\begin{proof}[Proof of Theorem \textup{3.2}]  We begin with part b).  We work
with a presentation  $(\phi ,\psi ):V\otimes R\oplus W\otimes R\rightarrow
U^{*}\otimes R$  where $W$ is a $\ZZ /2$-graded vector space of dimension $e$
with the $i$-th generator $w_{i}$ going to $x_{i}$ times the $i$-th generator
$u_{i}$ of $U^{*}$. The parity of the generators of $W$ is adjusted so that
$\psi $ is of degree $0$. Now, taking a double tableau $ (S, T)$ of the shape
$(1^{e})$ with $w_{i}$ and $u_{i}$ in the $i$-th row, and applying the
definition above, we see that the generator $\pi (S, T)$ is just $x_{1}\ldots
x_{e}$.

To prove part a) we distinguish two cases. In the case $s<d-1$ we use the
presentation $(\phi ,\psi ) :V\otimes R\oplus W\otimes R\rightarrow
U^{*}\otimes R$ where $W$ is a $\ZZ /2$-graded vector space of dimension $e+1$
with the $i$-th generator $w_{i}$ going to $x_{i}$ times the $(s+i+2)$-th
generator $u_{i}$ of $U^{*}$.  The parity of the generators of $W$ is adjusted
so that $\psi $ is of degree $0$. We can assume without loss of generality that
$\phi :=\Phi $ is generic. Then it is enough to prove that $c_{\Lambda
(s,e)}x_{1} \ldots x_{e+1}\in I_{\Lambda (s+1,e)}$, where $c_{\Lambda (s,e)}$
is the highest weight vector defined as in Proposition 1.5.

We pick a tableau $(S, T)$ of shape $\Lambda (s+1,e )$ as follows. The
entries $v_{i,j}$, $u_{i,j}$ in the $i$-th row are the same as in the
canonical tableau, except the last ones. The last entry in the tableau
$v$ in the $i$-th row is $w_{i}$, and the last entry in the $i$-th row
is $u_{s+2}$. The element $\pi ^{\prime }(S, T)$ is easily seen to be
$c_{\Lambda (s,e)}x_{1} \ldots x_{e+1}$.

In the case $s=d-1$ we use the presentation $(\phi ,\psi ) :V\otimes R\oplus
W\otimes R\rightarrow U^{*}\otimes R$ where $W$ is a $\ZZ /2$-graded vector
space of dimension $e+1$ with the the $i$-th generator $w_{i}$ going to $x_{i}$
times the $(d+i)$-th generator $u_{i}$ of $U^{*}$ for $1\le i\le e$ and
$w_{e+1}$ going to $x_{e+1}$ times $u_{d}$. The parity of the generators of $W$
is adjusted so that $\psi $ is of degree $0$. We can assume without loss of
generality that $\phi :=\Phi $ is generic. Then it is enough to prove that
$c_{\Lambda (d-1,e)}x_{1} \ldots x_{e+1}\in I_{\Lambda (d,e)}$, where
$c_{\Lambda (d-1,e)}$ is the canonical tableau.

We pick a tableau $(S, T)$ of shape $\Lambda (s+1,e )$ as follows. The entries
$v_{i,j}$, $u_{i,j}$ in the $i$-th row are the same as in the canonical
tableau, except the last ones. The last entry in the tableau $v$ in the $i$-th
row is $w_{i}$, and the last entry in the $i$-th row is $u_{d+i}$ for $1\le
i\le e$, and $u_{d}$ for the $(e+1)$-st row. The element $\pi ^{\prime }(S, T)$
is easily seen to be $c_{\Lambda (d-1,e)}x_{1} \ldots x_{e+1}$. 
\end{proof}


\section*{4. The resolution of a generic $\mathbb{Z}/2$-graded module}

In this section we work over the generic ring $R=\myS $ as in the
introduction, and we conjecture the form of a
minimal free resolution over $R$ of the cokernel $C$ of the
generic map $\Phi $. This
resolution is a natural generalization of the one constructed in \cite{BE} in
the commutative case. We work over a field $K$
of characteristic $0$. We define some
$\ZZ /2$-graded free $R$-modules $\FF _{i}$ as follows:
\begin{gather*}
{\FF }_{0} = U^{*}\otimes R,\qquad {\FF }_{1} = V\otimes R,\\
{\FF }_{i} =\bigoplus _{|\alpha |+|\beta | = i-2}  \myS
_{\Theta (d,e,\alpha ,\beta )}V\otimes \myS _{\Lambda (d,e,\alpha ,\beta
)}U\otimes R 
\end{gather*}
where $\Lambda (d,e,\alpha ,\beta ) = (d+1+\beta _{1} ,d+1+\beta _{2},\ldots ,
d+1+\beta _{e} ,e,\alpha ^{\prime }_{1} ,\ldots ,\alpha ^{\prime }_{s} )$,
$\Theta (d,e,\alpha ,\beta )= (d+1+\alpha _{1} ,d+1+\alpha _{2},\ldots,
d+1+\alpha _{e} ,e+1,\beta ^{\prime }_{1} ,\ldots ,\beta ^{\prime }_{s} )$, and
we sum over all pairs of partitions $\alpha ,\beta $ with at most $e$ parts.

\begin{theorem17} There exists an equivariant differential $d_{i} :{\FF
}_{i}\rightarrow {\FF }_{i-1}$, linear for $i\ge 3$ or $i=1$ and of degree
$|\Lambda (d,e)|$ for $i=2$, that makes ${\FF }_{\bullet }$ into a minimal
$R$-free $\ZZ /2$-graded resolution of $C$. \end{theorem17}

In the even case ($U_{1}=V_{1}=0$) the desired complex is the Buchsbaum-Rim
resolution (see, for example, Buchsbaum and Eisenbud \cite{BE}). We have
checked the conjecture computationally, using Macaulay2, in a few more cases.

\section*{Appendix 1. Elementary proof of a special case}

In this appendix we consider the case where the entries of  $\Phi $ are all of
odd degree, and  give the most transparent proof we know of Theorem 1(a), the
fact that a certain ideal $I$ annihilates $\coker \Phi $. The result and its
proof in this case are characteristic-free. The ideal $I$ has a simple
description as the ``ideal of exterior minors" in the sense of Green
\cite{Gre}, and Proposition A1.1 was originally proved by him in that paper in
a different way. We give a simple way of generating the ideal of exterior
minors, which works whenever the ground field $K$ is infinite. This appendix
will also serve as an introduction to the proof of Theorem 1(a). A different
method of treating the purely odd and purely even cases, which we do not know
how to generalize, is given in Appendix 3.

More precisely, we do the
special case of Theorem 1(a)
where $d=n=0$ (the other case where $\Phi $
has odd matrices, namely where $e=m=0$, differs just
by shifting degrees). Here $U=U_{1}$ and
$V=V_{0}$, so that the matrix
$\Phi $ consists of a single block, with odd degree entries,
and we wish to show that the ideal $I=I_{\Lambda (0,e)}=I_{(1^{e})}$
annihilates the cokernel of $\Phi $. 

In this special $d=n=0$ case the ring $\myS =\myS (V\otimes U)$ is the ordinary
exterior algebra $\wedge (V\otimes U)$.  For simplicity, we work throughout
this appendix with the ordinary Schur functors $\wedge ^{t}$ and $S_{t}$ in
place of their super analogues $\bigwedge ^{t}$ and $\myS _{t}$. 

Over the integers the representation 
$S_{e}(V)\otimes \wedge ^{e} U$ is not a summand of $\wedge (V\otimes U)$,
but its saturation is easy to identify as $D_{e}(V)\otimes \wedge ^{e}(U)$,
where $D_{e}(V)$ is the $e$-th homogeneous component of the 
divided power algebra. It is perhaps simplest to think of
$D_{e}(V)$ as the dual representation of $S_{e}(V^{*})$, and with this
as definition we can define the embedding 
$\iota :\ D_{e}(V)\otimes \wedge ^{e}(U)\subset \wedge ^{e}(V\otimes U)
$
as the dual of the surjection 
$\pi :\ \wedge ^{e}(V^{*}\otimes U^{*}) \to S_{e}(V^{*})\otimes \wedge ^{e}(U^{*}),
$
which comes in turn by extending the identity map
$\wedge (V^{*}\otimes U^{*})_{1} = (S(V^{*})\otimes \wedge (U^{*}))_{1,1}
$
to an algebra homomorphism
$\wedge (V^{*}\otimes U^{*})\to S(V^{*})\otimes \wedge (U^{*}),
$
using the fact that the elements of 
$(S(V^{*})\otimes \wedge (U^{*}))_{1,1}$
square to 0 and anticommute. Explicitly, the map $\pi $ 
can be specified
by its action on pure vectors, which is
\begin{equation*}\pi : \ 
(\hat v_{1} \otimes \hat u_{1}) \wedge \cdots \wedge (\hat v_{e} \otimes \hat u_{e})
\mapsto (\hat v_{1} \cdots \hat v_{e})\otimes (\hat u_{1}\wedge \cdots \wedge \hat u_{e}).
\end{equation*}
We write
$a_{i,j}=v_{j}\otimes u_{i}$ for the exterior variables.
Theorem 1(a) may be stated 
for the case $d=n=0$ as follows.

\begin{theorem18}
Let $E$ be the exterior algebra
on variables $a_{i,j}$, $ i=1,\dots ,e,\ j=1,\dots , m$, over
a field $K$ of arbitrary characteristic, and let 
\begin{equation*}\Phi =
\left(\begin{matrix}
a_{1,1}&\cdots&a_{1,m}\\
\vdots&\ddots&\vdots\\
a_{e,1}&\cdots&a_{e,m}
\end{matrix}\right)
\end{equation*}
be the $e\times m$ generic matrix over $E$. The cokernel
of $\Phi $ is annihilated by the element
$b= b_{1}\wedge b_{2}\wedge \cdots \wedge b_{e}$ where
\begin{equation*} \beta =
\left(\begin{matrix}
 b_1\\
\vdots\\
 b_e
\end{matrix}\right)
 \end{equation*}
is any $K$-linear combination of the columns of $\Phi $. If
$K$ is infinite, then these
elements generate the representation
\begin{equation*}D_{e}(V)\otimes \wedge ^{e}(U)\subset \wedge ^{e}(V\otimes U).
\end{equation*}
\end{theorem18}

\begin{proof}[Proof]
First consider the case $m=\dim V=1$, 
where $\Phi $ has
just one column, and let $\{v_{1}\}$ be a basis for $V$. 
Let $u_{1},\dots ,u_{e}$ be a basis of 
$U$ and let $\overline{u}_{i}$ be the image of $u_{i}$ in
$\coker \Phi $. In this case $V\otimes U$ has
dimension $e$, and so $\wedge ^{e}(V\otimes U)$ is 1-dimensional,
equal to $D_{e}(V)\otimes \wedge ^{e}(U)$, and generated by the 
product $a_{1,1}\wedge \cdots \wedge a_{e,1}$.

We have $\sum a_{i,1}\overline{u}_{i}=0$.
Thus
\begin{align*}
(a_{1,1}\wedge \cdots \wedge a_{e,1}) \overline{u}_{e}
&=
(a_{1,1}\wedge \cdots \wedge a_{e-1,1})\wedge a_{e,1}\overline{u}_{e}\\
&=
-\sum _{i\neq e}
(a_{1,1}\wedge \cdots \wedge a_{e-1,1}) \wedge a_{i,1}\overline{u}_{i}
=0,
\end{align*}
since $a_{i,1}\wedge a_{i,1}=0.$

In the general case $\dim V= m$, we note that the cokernel of
the augmented matrix
\begin{equation*}
\left(\begin{matrix}
a_{1,1}&\cdots&a_{1,m}& b_1\\
\vdots&\ddots&\vdots&\vdots\\
a_{e,1}&\cdots&a_{e,m}& b_e
\end{matrix}\right)
\end{equation*}
is the same as the cokernel of $\Phi $, and thus
is a quotient of the cokernel of the matrix $\beta $.
By the case $m=1$, the module $\coker \beta $ is annihilated 
by $ b_{1}\wedge \cdots \wedge b_{e}$;
so its quotient $\coker \Phi $ is too.

Finally, we must show that, if $K$ is infinite, the products 
$b$ corresponding to columns $\beta $ generate the representation
$\iota (D_{e}(V)\otimes \wedge ^{e} U)$. Let $v\in V$ be the
element corresponding to the linear combination of the 
columns $b$, and let $V'\subset V$ be the 1-dimensional
space spanned by $v$. The element $b$ generates the
image of the composite map 
\begin{equation*}D_{e}(V')\otimes \wedge ^{e} U\to D_{e}(V)\otimes \wedge ^{e}
U\to ^{\iota }\wedge ^{e}(V\otimes U). 
\end{equation*} 
We must 
show that the images of all such maps span 
$\iota (D_{e}(V)\otimes \wedge ^{e} U)$.
The space $\wedge ^{e}U$ is 1-dimensional. Thus it suffices
to show that the images of all $D_{e}(V')$ in $D_{e}(V)$ span 
$D_{e}(V)$. (Note that this would fail for $S_{e}(V)$ in place of
$D_{e}(V)$ in characteristic $p$ if $e=p$.)

Dually, it suffices to show that the intersection of the  kernels of the maps
$S_{e}(V^{*})\to S_{e}(V^{'*})$ induced by all 1-dimensional projections
$V^{*}\to V^{'*}$ is zero. Such a projection is a point in the projective space
$\PP (V)$, and the kernel is the set of polynomials of degree $e$ vanishing at
the point. The desired assertion follows because only the zero polynomial
vanishes at all the points of $\PP (V)$ when $K$ is infinite. \end{proof}

\section*{Appendix 2. Comments on the action of $\mathfrak{g}$}

It may at first be surprising that the generators given in 
Examples 1-3 of the Introduction are permuted by the action of $\g $.
So we will make the action explicit in one case, Example 3 (the
other cases are similar and simpler). 

When we think of $R=\myS (V\otimes U)$ as a $\g $-module, we think of
$\g $ acting on the left. But we may identify $U\otimes V$
with $\operatorname{Hom}(V,U^{*})^{*}=\operatorname{Hom}(U^{*},V)$, and thus identify $R$ with the
coordinate ring of the space $\operatorname{Hom}(V,U^{*})$. In this identification
it is natural to think of the Lie algebra $\g =\gl (V)\times \gl (U)$
as $\gl (V)\times \gl (U^{*})$, with the $\gl (U^{*})$ acting on the right.
To make this identification, we use the {\em supertranspose\/}, which
is the anti-isomorphism 
\begin{equation*}
\gl (U)\to \gl (U^{*});\quad 
\left(
\begin{matrix}
U_{0,0}&U_{0,1}\\
U_{1,0}&U_{1,1}\end{matrix}
\right)
\mapsto
\left(\begin{matrix}
U_{0,0}^t&U_{1,0}^t\\
-U_{0,1}^t&U_{1,1}^t
\end{matrix}\right).
\end{equation*}

Now consider the case presented in Example 3 of the Introduction,  whose
notation we use. To see the action of $\g $, let us act by two elements of the
Lie algebra on the element $axy$. First we act with the element $v_{0,1}$ from
$\gl (V)$, changing an odd element to an even one. We get a sum of terms, each
of which is $axy$ with one changed factor; we can replace $a$ by $x$ or $y$ by
$b$. Thus we get terms $xxy$ and $axb$. The first comes with positive sign
($v_{0,1}$ acts from the left and we replaced the first factor, and so there
are no switches), the second term comes with negative sign (we replaced $y$ by
$b$, and so had to switch $v_{0,1}$ with $a$). Thus we get $x(xy-ab)$.

Let us also act on $axy$ by the Lie algebra element $u_{1,0}$ from $\gl
(U^{*})$ exchanging an even element with an odd one. We get a sum of terms
where each term is $axy$ with one factor changed; we can change $x$ to $b$ and
$a$ to $y$. We get terms $yxy$ and $aby$. Both come without sign, since
$u_{1,0}$ acts from the right and $x,y$ have even degree. Thus we get
$(xy+ab)y$. 

\section*{Appendix 3. A standard basis approach to the generic module}

In this appendix we describe a general method for analyzing the cokernel of the
generic matrix by giving a {\em standard basis\/} in the case where all the
variables of $S$ are even or all are odd. It would be very interesting to give
a generalization to the situation where there are both even and odd variables.
In the cases treated here, the method, and the results, are
characteristic-free.

To explain the method in the most familiar setting, we begin with the even
case---the classical case of the cokernel of a matrix whose entries are
distinct variables of a commutative polynomial ring. The standard basis
approach to the ring $S$ itself in this case is due to Doubillet, Rota, and
Stein \cite{DRS}; see also DeConcini, Eisenbud, and Procesi \cite{DEP}. The
standard basis for the generic module in the even case was obtained by Bruns
and Vetter \cite{BV} and by Bruns \cite{Bru}. This standard basis was shown to
be a Gr\"{o}bner basis by Onn \cite{Onn}. We give the proof based on
representation theory because it exends to the odd case. Then we  sketch the
odd case, which is quite parallel. Throughout this appendix we think of
tableaux as being filled with numbers, not with elements of the numbered basis
as before.

\medskip \noindent {\bf The even case.} Consider vector spaces $V=K^{m},
U=K^{d}$ and the generic map $\Phi :V\otimes _{K} S\rightarrow U^{*}\otimes
_{K} S$  of free modules over the polynomial ring $S = \Sym_{K} (V\otimes U)$.
We will give a good basis for the module $\coker \Phi $. Let $\lbrace u_{1}
,\ldots ,u_{d}\rbrace $, $\lbrace v_{1} ,\ldots ,v_{m}\rbrace $ be bases of $U,
V$ respectively. Let $\lbrace u_{1}^{*} ,\ldots ,u_{d}^{*} \rbrace $ be the
dual basis of $U^{*}$. Write $\Phi (v_{j} )=\sum _{i=1}^{d} x_{i,j}u_{i}^{*}$.
The ring $S$ can be identified with the commutative  polynomial ring 
$K[x_{i,j}]_{1\le i\le d,1\le j\le m}$. Recall that a {\em double  tableau of
shape $\lambda $,\/} where  $\lambda = (\lambda _{1} \geq \cdots \geq \lambda
_{r}\geq 0)$,  is a pair of tableaux
\begin{equation*}P=(P_{1} ,\ldots ,P_{r} ),\quad Q=(Q_{1} ,\ldots ,Q_{r} )
\end{equation*} 
with
$P_{j} = (p_{j,1},\ldots ,p_{j,\lambda _{j}} )$ and
$Q_{j} = (q_{j,1},\ldots ,q_{j,\lambda _{j}} )$ 
satisfying 
$1\le p_{i,j}\le d$ and
$1\le q_{i,j}\le m$
(and thus $\lambda _{1}\leq \min (m,d))$. 
Corresponding to the double tableau $(P,Q)$
is the
product $M_{1}\cdots M_{r}\in S$, 
where $M_{j}$ is the signed minor of the generic matrix
$\Phi = (x_{i,j})_{1\le i\le d, 1\le j\le m}$ 
involving the rows indexed by $P$
and the columns indexed by $Q$. We will henceforward think of 
the double tableau as elements of $S$. Thus, for example,
$(p_{1},\dots ,p_{r})\mid (q_{1},\dots ,q_{r})$ denotes an $r\times r$
minor of $\Phi $.

The tableau $P$ is {\em standard\/} if
$p_{i,j}<p_{i,j+1}$ for all $i,j$ and $p_{i,j}\le p_{i+1,j}$ for all
$i, j$. Similarly we define the standardness of $Q$, and a double
tableau $(P, Q)$ is standard if both $P$ and $Q$ are standard.
The algebra $S$ has a basis consisting of (the elements corresponding
to) the {\em double standard tableaux.\/} 
A $K$-basis of the free module $U^{*}\otimes S$ is given by 
the products
$(P, Q)u_{i}^{*}$ where $(P, Q)$ is a standard double tableau and 
$1\le i\le d$. Thus we can construct a basis for $\coker \Phi $
by taking an appropriate subset.

\begin{definition4} A product $(P, Q)u_{i}^{*}$ is {\em admissible\/}
 if the
double tableau $(P, Q)$ is standard and the first row of $P$ does not
contain the interval $[1,i]$. Notice that the additional condition
is really a condition on $P$ and $u_{i}^{*}$,
and if it is satisfied, we will say that $P u_{i}^{*}$ is admissible.
\end{definition4}
\begin{theorem19} The images of the
admissible products in $\coker \Phi $ form a $K$-basis.
\end{theorem19}
\begin{proof}  
We first show that the
admissible products span
$\coker \Phi $.
Order the bases by setting $u_{1} <\ldots <u_{d}$, $u_{d}^{*}<\ldots <u_{1}^{*}$, $v_{1} <\ldots <v_{m}$. 
Order the products $(P, Q)u_{i}^{*}$ (where $(P, Q)$ is a double
tableau, not nessesarily standard) by reading lexicographically,
first the element
$u_{i}^{*}$, then $P$ by rows and then $Q$ by rows. It
suffices to show that if a
product is not admissible, it is a combination of earlier
products.
\renewcommand{\qed}{} \end{proof}


The $(i,j)$-th entry of $\Phi $ may be written
as $(i|j)$. Thus the relations
$\sum _{i=1}^{d} (i | j)u_{i}^{*} =0$
hold in $\coker \Phi $. We will use a generalization:

\begin{theorem20} In $\coker \Phi $, 
\begin{equation*}\sum _{i=1}^{d} 
(a_{1} ,\ldots ,a_{n-1} ,i, a_{n+1},\ldots ,a_{r} \ |\ b_{1},\ldots ,b_{r} )
u_{i}^{*} =0.
\end{equation*}
\end{theorem20}
\begin{proof} Use the Laplace expansion with respect to the
$n$-th row of each minor.

Now assume that the product $(P, Q)u_{i}^{*}$ is not admissible. Use the
relation from Lemma A3.2 for $n=i$, taking $a_{1} ,\ldots ,a_{i-1},
a_{i+1},\ldots ,a_{r}$  and  $b_{1},\ldots ,b_{r}$  to be the first row of $P$
minus $i$ and the first row of $Q$, respectively.  We may assume that the rows
of $P$ and $Q$ are ordered, so that $a_{1} =1,\ldots , a_{i-1}=i-1$. The
relation shows that in $\coker \Phi $ the element $(P_{1}|Q_{1})u_{i}^{*}$ is a
linear combination of $u_{i+1}^{*}, \ldots ,u_{d}^{*}$ with polynomial
coefficients, because the other terms involve minors with repeated rows. These
are earlier terms in our order.  We multiply this  relation by the other rows
of $(P, Q)$, and we see that $(P, Q)u_{i}^{*}$ is a linear combination of
earlier terms in the same way.
  
It remains to show that the admissible products are linearly independent over
$K$. It is enough to prove this over $\mathbf{Z}$, and thus it is enough to
prove the linear independence over $\mathbf{Q}$. We will show that in a given
degree the dimension of the module $\coker \Phi $ is at least equal to the
number of admissible products.  To do this we use the representation theory of
$\GL(U)$. 

We label the irreducible
representation $S_{\lambda }U$ of $\GL(U)$ by its highest weight. This
means they are labeled by the sequences $\lambda = (\lambda _{1} ,\ldots
,\lambda _{d} )$ of integers such that $\lambda _{1}\ge \ldots \ge \lambda
_{d}$,
but we do not assume $\lambda _{d}\ge 0$. Irreducible representations of
$\SL(U)$ correspond to weights $\lambda $ with $\lambda _{d} =0$ and are
called Schur functors. We have
\begin{equation*}S_{(\lambda _{1} +1,\ldots ,\lambda _{d} +1 )}U
= 
S_{\lambda }U\otimes \wedge ^{d} U.
\end{equation*}

With this notation $\wedge ^{i} U = S_{(1^{i} ,0^{d-i})}U$ and
$\wedge ^{i} U^{*} = S_{(0^{d-i}, (-1)^{i} )}U$; in particular,
the naturally isomorphic representations $\wedge ^{i} U=\wedge ^{d} $ and
$U\otimes \wedge ^{d-i}U^{*}$ both have highest weight $(1^{i} ,0^{d-i})$.
The representation with general $\lambda _{n}$
can be expressed as a Schur functor tensored with an
integer
power of $\wedge ^{d} U$. We can now express
$\coker \Phi $ as a representation:

\begin{theorem21}
If $K$ is a field of
characteristic zero, then
the decomposition of $\coker \Phi $ into irreducible
representations of $\GL(V)\times \GL(U)$ is
\begin{equation*}\coker \Phi = \bigoplus _{\lambda = (\lambda _{1} ,\ldots
,\lambda _{d-1})} S_{\lambda }V\otimes S_{(\lambda _{1} ,\ldots ,\lambda
_{d-1}, -1 )}U. 
\end{equation*}
\end{theorem21}

\begin{proof}[Proof of Proposition \textup{A3.3}]
The free module 
$U^{*}\otimes S$ 
decomposes as 
\begin{equation*}U^{*}\otimes S =\bigoplus _{\lambda = (\lambda _{1} ,\ldots ,\lambda _{d}),\ \lambda _{d}\ge 0}
\ S_{\lambda }V\otimes S_{\lambda }U\otimes U^{*} .
\end{equation*}
Let us look at the isotypic component of $S_{\lambda }V$. 
Using Pieri's formula, we can decompose 
$S_{\lambda }U\otimes U^{*}$. 
Using the isomorphism $U^{*} =\wedge ^{d-1} U\otimes \wedge ^{d} U^{*}$, we get
\begin{equation*}S_{\lambda }U\otimes U^{*} =
\bigoplus _{i{\text{ such\ that }} \lambda _{i}>\lambda _{i-1}}
S_{\lambda _{1} ,\ldots ,\lambda _{i+1}\lambda _{i} -1,\lambda _{i-1}\ldots ,
\lambda _{d} }U.
\end{equation*}
The representation $V\otimes S$
contains only representations $S_{\lambda }U$ of $\GL(U)$
with all entries of $\lambda $ nonnegative.
On the other hand, if $\lambda _{d} =0$, then 
$S_{\lambda }U\otimes U^{*}$ contains the representation $S_{\lambda _{1}
,\ldots ,\lambda _{d-1}, -1}U$. Thus this representation must occur in
$\coker \Phi $.

The following
lemma concludes the proof of Proposition A3.3 and also gives
the linear independence necessary to finish
the proof of Proposition A3.1.
\renewcommand{\qed}{} \end{proof}

\begin{theorem22}
The dimension of 
$S_{(\lambda _{1} ,\ldots ,\lambda _{d-1},-1)}U$ 
is equal to the number of the products $P u_{i}^{*}$ where
$P$ is a standard tableau of shape $\lambda $ and the first row of
$P$ does not contain the interval $[1,i]$.
\end{theorem22}

\begin{proof} 
The representation
$S_{(\lambda _{1} ,\ldots ,\lambda _{d-1},-1)}U$ 
has the same dimension as 
\[S_{(\lambda _{1},\ldots ,\lambda _{d-1} ,-1)}U\otimes \bigwedge ^{d} U 
= S_{(\lambda _{1} +1,\ldots ,\lambda _{d-1}+1,0)}U.\]
This is equal to the number of
standard tableaux of shape 
$(\lambda _{1} +1,\ldots ,\lambda _{d-1}+1,\allowbreak 0)$. 

On the other hand, the admissibility condition on the product $P u_{i}^{*}$ is
the same as the standardness condition on the tableau of shape  $\lambda $
starting with the row $1,2,\ldots ,i-1,i+1,\ldots ,d$ and continuing with the
rows of $P$. Thus the number  of  admissible products $P u_{i}^{*}$ is equal to
the dimension of the representation  $S_{\lambda _{1},\ldots ,\lambda _{d-1},
-1}U$  in $S_{\lambda }U\otimes U^{*}$.   
\end{proof}

This also completes the proofs of Lemma A3.2 and Proposition A3.1.
\end{proof}


\begin{theorem23}
\textup{a)} The ideal $I_{d}$ generated by $d\times d$ minors
of $\Phi $ annihilates $\coker \Phi $.

\textup{b)} $\coker \Phi $ is a torsion free module over the determinantal
ring $S/I_{d}$.
\end{theorem23}
\begin{proof} To prove a), consider the relation from 
Lemma A3.2
with $r=d$,  
\begin{equation*}\lbrace a_{1} ,\ldots , a_{n-1}, a_{n+1},\ldots ,a_{d}\rbrace
= \lbrace 1,2,\ldots , n-1,n+1,\ldots ,d\rbrace .\end{equation*} 
 It says that
\begin{equation*}(1,2,\ldots ,d\ |\ 
b_{1} ,\ldots ,b_{d} )u_{i}^{*}=0 .\end{equation*}

To prove b) we define an embedding of $\coker \Phi $ into a free
$S/I_{d}$-module. We define the homomorphism
\begin{equation*}\Psi :\coker \Phi \rightarrow \wedge ^{d-1}V^{*}\otimes \wedge ^{d}
U^{*}\otimes S/I_{d}
\end{equation*}
by setting 
\begin{equation*}\Psi (u_{i}^{*} ) =\sum _{i=1}^{d} (a_{1} ,\ldots , a_{d-1}\ |b_{1} ,\ldots ,b_{d-1} ) u_{a_{1}}^{*}\wedge \ldots \wedge u_{a_{d-1}}^{*}\wedge u_{i}^{*}\otimes v_{b_{1}}^{*}\wedge \ldots \wedge v_{b_{d-1}}^{*} .
\end{equation*}
Consider the complex
\begin{equation*}V\otimes S/I_{d} 
\xrightarrow{\Phi \otimes 1} U^{*}\otimes S/I_{d} 
\xrightarrow {\Psi }\wedge ^{d-1}V^{*}\otimes \wedge ^{d}
U^{*}\otimes S/I_{d}.
\end{equation*}
One shows easily that 
$\Psi (\Phi \otimes 1)=0$ (the coefficient of 
every image is a combination of
$d\times d$ minors, so is zero in $S/I_{d}$).
To prove the exactness of our complex, we notice 
that it is enough to prove the exactness over 
$\mathbf{Q}$. Indeed,
our straightening law showed that the cokernel of 
$\Phi \otimes 1$ is a free module
over $\mathbf{Z}$. 

To prove the exactness over $\mathbf{Q}$ we use representation theory
methods by decomposing to irreducibles. Let us look at our complex,
and for a partition $\nu $ let us look at the $\GL(V)$-isotypic
component of $S_{\nu }V$, with $\nu = (\nu _{1} ,\ldots ,\nu _{d-1})$. It is
enough to look at such components, since no other representations occur
in the middle term of the complex. We get the following complex:
\begin{equation*}S_{\nu /(1)}U\rightarrow U^{*}\otimes S_{\nu }
U\rightarrow \bigwedge ^{d} U^{*}\otimes S_{(\nu _{1}+1,\ldots ,\nu _{d-1}+1 )}.
\end{equation*}
The cokernel of the first map consists of the sole irreducible
representation 
\[S_{(\nu _{1} ,\ldots ,\nu _{d-1},-1)}U\]
which occurs in
the right-hand-side term. It is therefore enough to show that the map
$\Psi $ is nonzero on our isotypic component. The best way to do it is
to see that the image of $\Psi $ on the corresponding highest weight
vector is nonzero. This highest weight vector is, however, easy to
calculate, since it equals $(C_{\nu },C_{\nu })u_{d}^{*}$ where $C_{\nu }$ is the
canonical tableau of shape $\nu $ having $(1,2,\ldots ,\nu ^{\prime }_{i} )$
in the $i$-th row.
\end{proof}
\begin{remark1}
Bruns and Vetter (\cite{BV}, chapter 13) show that $\coker \Phi $ is
torsion free over $S/I_{d}$ by a different argument. 
What is more, they show that $\coker \Phi $ is
reflexive over $S/I_{d}$. 
This can also be proved by our method, by extending
our sequence by one term:
\begin{align*}
V\otimes S/I_{d} \,&{\overset{{\Phi \otimes 1}}{\longrightarrow }} U^{*}\otimes
S/I_{d} \\
&{\overset{\Psi }{\longrightarrow }}\wedge ^{d-1}V^{*}\otimes \wedge
^{d} U^{*}\otimes S/I_{d} {\overset{\Theta }{\longrightarrow }} 
\wedge ^{d}V^{*}\otimes \wedge ^{d} U^{*}\otimes U^{*}\otimes S/I_{d}
\end{align*}
where $\Theta $ is defined on generators by the map 
\begin{align*}
\wedge ^{d-1}V^{*}\otimes \wedge ^{d} U^{*}&\rightarrow \wedge 
^{d-1}V^{*}\otimes \wedge ^{d} U^{*}\otimes V^{*}\otimes V\otimes 
U^{*}\otimes U\\
&\rightarrow \wedge ^{d} V^{*}\otimes \wedge ^{d} U^{*}\otimes U^{*}\otimes
(V\otimes U) ,
\end{align*}
and then identifying $V\otimes U$ with $S_{1}$. We prove the exactness
of the extended complex in the same way as above.  This shows 
that $\coker \Phi $ is a second syzygy over $S/I_{d}$, and therefore reflexive. 
The dual module of $\coker \Phi $ turns out to be 
\[\wedge ^{m-d+1}(\coker \Phi ^{*}).\]
\end{remark1}


\noindent {\bf The odd case.}
The results in this case are very similar; so we only sketch the proofs.

Again, let $K$ be a field  of arbitrary
characteristic. Consider the vector spaces $V=K^{n}, U=K^{d}$. We
consider the odd generic map $\Phi :V\otimes _{K} E\rightarrow U^{*}\otimes _{K} E$  of free modules over the exterior algebra $E =
\wedge ^{\bullet }_{K} (V\otimes U)$. We are interested in the generic
module $\coker \Phi $.

Let $\lbrace u_{1} ,\ldots ,u_{d}\rbrace $, $\lbrace v_{1} ,\ldots
,v_{n}\rbrace $ be bases of $U, V$ respectively. Let $\lbrace u_{1}^{*} ,\ldots
,u_{d}^{*} \rbrace $ be the dual basis of $U^{*}$.

Let us write $\Phi (v_{j} )=\sum _{i=1}^{d} a_{i,j}u_{i}^{*}$. The ring $E$
can be identified with the exterior algebra
$\wedge ^{\bullet }[a_{i,j}]_{1\le i\le d,1\le j\le n}$.

It is shown in Akin, Buchsbaum, and Weyman \cite{ABW} that  $E$ has a basis
consisting of double standard tableaux, where now the notion of double standard
tableaux is interpreted as follows: Suppose that $(P, Q)$ is a double  tableau
of shape $\lambda $, where  $\lambda = (\lambda _{1} ,\ldots , \lambda _{r})$, 
and  $P=(P_{1} ,\ldots ,P_{r} ), Q=(Q_{1} ,\ldots ,Q_{r} )$  with $P_{j} =
(p_{j,1},\ldots ,p_{j,\lambda _{j}} )$,  $Q_{j} = (q_{j,1},\ldots ,
q_{j,\lambda _{j}} )$,  for $1\le p_{i,j}\le d$, $1\le q_{i,j}\le m$. The
corresponding element of $E$ is the product $M_{1}\ldots M_{r}$ where $M_{j}$
is an exterior  minor of the generic matrix  ${\tilde X} = (a_{i,j})_{1\le i\le
d, 1\le j\le n}$ corresponding to the rows  $P_{j} = (p_{j,1},\ldots
,p_{j,\lambda _{j}} )$ and (maybe repeated) columns  $Q_{j} = (q_{j,1},\ldots
,q_{j,\lambda _{j}} )$.  Recall that $P$ is standard if $p_{i,j}<p_{i,j+1}$ for
all $i,j$ and $p_{i,j}\le p_{i+1,j}$ for all $i, j$. The standardness of $Q$ is
defined by the conditions  $q_{i,j}\le q_{i,j+1}$ for all $i,j$ and $q_{i,j}<
q_{i+1,j}$ for all $i, j$. $(P, Q)$ is standard if both $P$  and $Q$ are
standard.

From the representation-theoretic point of view, the map given by
exterior minors of size $s$ is the embedding
\begin{equation*}D_{s} V\otimes \wedge ^{s} U\rightarrow \wedge ^{s} (V\otimes U),
\end{equation*}
which is the specialization of our $\rho _{s}$ to the odd case. Here $D_{s} V$
is the divided power of $V$.

The $K$-basis of the free module $U^{*}\otimes E$ is given by products
$(P, Q)u_{i}^{*}$ where $(P, Q)$ is a standard double tableau and $1\le i\le d$. 

\begin{definition5} A product $(P, Q)u_{i}^{*}$ is admissible if the double
tableau $(P, Q)$ is standard and the first row of $P$ does not contain the
interval $[1,i]$. Notice that the additional condition is really a condition on
$P$ and $u_{i}^{*}$. So we can also talk about admissible products $P
u_{i}^{*}$.

In the sequel we denote the exterior minor of $X$ corresponding to
the rows $a_{1} ,\ldots ,a_{r}$ and columns $b_{1} ,\ldots ,b_{r}$ by
$(a_{1} ,\ldots ,a_{r} |b_{1} ,\ldots ,b_{r} )$.
\end{definition5}
\begin{theorem24} The admissible products form a $K$-basis of
$\coker \Phi $.
\end{theorem24}
\begin{proof} The proof proceeds as in the even case.
The representation-theoretic 
content is identical; only the representation on the $V$
side changes from $S_{\lambda }V$ to $S_{\lambda ^{\prime }}V$. 
The essential point is that in $\coker \Phi $ we have relations on
the $u_{i}^{*}$ that are analogous to those in Lemma A3.2:
\end{proof}
\begin{theorem25} In $\coker \Phi $ we have the relations
\begin{equation*}\sum _{j=1}^{d} 
(a_{1} ,\ldots ,a_{n-1} ,j, a_{n+1},\ldots ,a_{r} |
 b_{1},\ldots ,b_{r} ) u_{j}^{*} =0.
\end{equation*}
\end{theorem25}
\begin{proof} By linearity and the argument at the
end of the proof of Proposition A1.1, it suffices to treat the case where
$b_{1}=b_{2}=\cdots =b_{r}$. But, again as in A1.1, we have
\begin{equation*}(a_{1} ,\ldots ,a_{n-1} ,j, a_{n+1},\ldots ,a_{r} |
 b_{1},\ldots ,b_{1} )
=
(a_{1}|b_{1})\wedge (a_{2}|b_{1})\wedge \cdots \wedge (a_{r}|b_{1}).
\end{equation*}
So the given relation is a multiple of the relation
$\sum _{j=1}^{d} (j|b_{1} ) u_{j}^{*} =0
$
given by column number $b_{1}$ of $\Phi $.
\end{proof}


\begin{theorem26}
\textup{a)} The ideal $I_{(d)}$ generated by $d\times d$ exterior minors
of $\Phi $ annihilates $\coker \Phi $. Moreover, $u_{d}^{*}$ generates
a free $E/I_{(d)}$-module in $\coker \Phi $. So the annihilator
of $\coker \Phi $ is exactly $I_{(d)}$.

\textup{b)} $\coker \Phi $ is a second syzygy over the
ring $E/I_{(d)}$.
\end{theorem26}
\begin{proof} To prove a), consider the relation from 
Lemma A3.2$'$
with $r=d$,  
\begin{equation*}\lbrace a_{1} ,\ldots , a_{u-1}, a_{u+1},\ldots ,a_{d}\rbrace = \lbrace 1,2,\ldots , i-1,i+1,\ldots ,d\rbrace .
\end{equation*}
 It says that
\begin{equation*}(1,2,\ldots ,d\ |b_{1} ,\ldots ,b_{d} )u_{i}^{*}=0 .
\end{equation*}
The last statement of a) follows because 
$(P|Q)u_{d}^{*}$ is admissible
for every standard double tableau of shape $\lambda $
with $\lambda _{1}<d$, while
$E/I_{(d)}$ is spanned by those with $\lambda _{1}<d$.

To prove b) we define an embedding of $\coker \Phi $ into a free
$E/I_{(d)}$-module. We define the homomorphism
\begin{equation*}\Psi :\coker \Phi \rightarrow S_{d-1}V^{*}\otimes \wedge ^{d}
U^{*}\otimes E/I_{(d)}\end{equation*}
by setting 
\begin{equation*}\Psi (u_{i}^{*} ) =\sum (a_{1} ,\ldots , a_{d-1}\ |b_{1} ,\ldots ,b_{d-1} ) u_{a_{1}}^{*}\wedge \ldots \wedge u_{a_{d-1}}^{*}\wedge u_{i}^{*}\otimes v_{b_{1}}^{*}\ldots v_{b_{d-1}}^{*} .\end{equation*}
Consider the complex
\begin{equation*}V\otimes E/I_{(d)} {\overset{{\Phi \otimes 1}}{\longrightarrow }}
U^{*}\otimes E/I_{(d)} {\overset{\Psi }{\longrightarrow }}S_{d-1}V^{*}\otimes \wedge ^{d}
U^{*}\otimes E/I_{(d)}. \end{equation*}
 One shows easily that $\Psi (\Phi \otimes 1)=0$ (the coefficient of every image is a combination of
$d\times d$ minors, hence zero in $E/I_{(d)}$).
To prove the exactness of our complex, we notice that this exactness is clear over
$\mathbf{Q}$ by counting representations in an isotypic component of $S_{\lambda }V$. But
our straightening law showed that the image of $\Phi \otimes 1$ is a free module
over $\mathbf{Z}$ and that $(U^{*}\otimes E/I_{(d)} )/
\operatorname{Im}(\Phi \otimes 1)$ is also free over
$\mathbf{Z}$. This proves the exactness of our complex, 
and so $\coker \Phi $ is torsion free.

We can extend our sequence to a longer sequence
\begin{align*}
V\otimes E/I_{(d)} &{\overset{{\Phi \otimes 1}}{\longrightarrow }} 
U^{*}\otimes E/I_{(d)} {\overset{\Psi }{\longrightarrow }}S_{d-1}V^{*}\otimes
\wedge ^{d} 
U^{*}\otimes E/I_{(d)} \\
&{\overset{\Theta }{\longrightarrow }}S_{d}V^{*}\otimes
\wedge ^{d} U^{*}\otimes U^{*}\otimes E/I_{(d)}\end{align*} 
where $\Theta $ is defined on generators by the map 
\begin{align*}S_{d-1}V^{*}\otimes \wedge ^{d} U^{*}&\rightarrow
S_{d-1}V^{*}\otimes \wedge ^{d} U^{*}\otimes V^{*}\otimes V\otimes U^{*}\otimes
U\\
&\rightarrow S_{d} V^{*}\otimes \wedge ^{d} U^{*}\otimes U^{*}\otimes (V\otimes
U)\end{align*} 
and then identifying $V\otimes U$ with $E_{1}$. 
The exactness of the longer complex
is proven in the same way as the analogous 
result in the even case.\end{proof}



\bibliographystyle{amsalpha}
\begin{thebibliography}{ABW} 

\bibitem[ABW]{ABW}{K.~Akin, D.~A.~Buchsbaum, and J.~Weyman}: Schur functors and
Schur complexes, Adv. Math. 44 (1982), 207-278. \MR{84c:20021}

\bibitem[BR]{BR}{A. Berele and A. Regev}: Hook Young diagrams with applications
to combinatorics and to representations of Lie superalgebras. Adv. in Math. 64
(1987) 118--175.
\MR{88i:20006}

\bibitem[Bru]{Bru}{Bruns, Winfried}: Generic maps and modules, Compositio 
Math. 47 (1982), 171-193. \MR{84j:13011}

\bibitem[BV]{BV}{Bruns, Winfried} and Udo Vetter: Determinantal rings.  Spinger
Lecture Notes in Math.~1327. Springer-Verlag, Berlin/Heidelberg/NY 1988.
\MR{89i:13001}

\bibitem[BE]{BE}{D. A. Buchsbaum and D. Eisenbud}: Generic free
resolutions and a family of generically perfect
ideals. Advances in Math. 18  (1975), no. 3, 245-301. \MR{53:391}

\bibitem[DEP]{DEP}{C. DeConcini, D. Eisenbud and C. Procesi}: Young
diagrams and determinantal varieties, Invent.
Math. 56  (1980), 129--165.  \MR{81m:14034}

\bibitem[DRS]{DRS}{P. Doubilet, G.-C. Rota and Joel Stein}:
On the foundations of combinatorial theory. IX.
Combinatorial methods in invariant theory.
Studies in Appl. Math. 53 (1974), 185--216. \MR{58:16736}

\bibitem[Eis]{Eis}{D. Eisenbud}: {\em Commutative Algebra with a View Toward
Algebraic Geometry}. Springer-Verlag, New York, 1995. \MR{97a:13001}

\bibitem[EPY]{EPY}{D. Eisenbud, S. Popescu and S. Yuzvinsky}: Hyperplane
arrangement cohomology and monomials in the exterior algebra. Preprint, 2000.

\bibitem[ES]{ES}{D. Eisenbud, G. Fl\o ystad and F.-O. Schreyer}: Free resolutions and
sheaf cohomology over exterior algebras. Preprint, 2000.

\bibitem[Fit]{Fit} {H.~Fitting}: Die Determinantenideale eines Moduls. Jahresbericht
der Deutschen Math.-Vereinigung 46 (1936) 195-229.

\bibitem[GS]{GS}{D.~Grayson and M.~Stillman}: {\em Macaulay2}, a
software system devoted to supporting research in algebraic geometry
and commutative algebra.  Contact 
the authors, or download 
from\hfill \linebreak {\tt http://www.math.uiuc.edu/Macaulay2}.

\bibitem[Gre]{Gre}{M. Green}: The Eisenbud-Koh-Stillman conjecture on linear
syzygies. Invent. Math. 136 (1999) 411--418. \MR{2000j:13024}

\bibitem[Mac]{Mac}{I. G. MacDonald}: {\em Symmetric functions and Hall
polynomials.} Second edition. With contributions by A. Zelevinsky. Oxford
Mathematical Monographs, Oxford University Press, New York, 1995. \MR{96h:05207}

\bibitem[Onn]{Onn}{S. Onn}: Hilbert Series of 
Group Representations and Gr\"{o}bner Bases for Generic Modules,
Journal of Algebraic Combinatorics, vol. 3, pp. 187-206, 1994. \MR{95b:20010}

\end{thebibliography}

\end{document}

