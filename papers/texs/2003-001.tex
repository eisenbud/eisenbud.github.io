%_ **************************************************************************
%_ * The TeX source for AMS journal articles is the publishers TeX code     *
%_ * which may contain special commands defined for the AMS production      *
%_ * environment.  Therefore, it may not be possible to process these files *
%_ * through TeX without errors.  To display a typeset version of a journal *
%_ * article easily, we suggest that you retrieve the article in DVI,       *
%_ * PostScript, or PDF format.                                             *
%_ **************************************************************************
%  Author Package
%% Translation via Omnimark script a2l, November 29, 2001 (all in one day!)
\controldates{29-AUG-2002,29-AUG-2002,29-AUG-2002,30-AUG-2002}
 
\RequirePackage[warning,log]{snapshot}
\documentclass{proc-l}
\issueinfo{131}{03}{March}{2003}
\pagespan{701}{708}
\dateposted{September 17, 2002}
\PII{S 0002-9939(02)06575-9}
\copyrightinfo{2002}{American Mathematical Society}

\renewcommand{\labelenumi}{(\theenumi)}


\theoremstyle{plain}
\newtheorem*{theorem1}{Theorem 0.2}
\newtheorem*{theorem2}{Proposition 1.3}
\newtheorem*{theorem3}{Theorem 1.4}
\newtheorem*{theorem4}{Proposition 1.5}
\newtheorem*{theorem5}{Theorem 1.6}
\newtheorem*{theorem6}{Lemma 1.7}
\newtheorem*{theorem7}{Proposition 1.8}
\newtheorem*{theorem8}{Theorem 2.2}
\newtheorem*{theorem9}{Proposition 2.4}

\theoremstyle{definition}
\newtheorem*{definition1}{Definition 0.1}
\newtheorem*{definition2}{Example 1.1}
\newtheorem*{definition3}{Definition 1.2}
\newtheorem*{definition4}{Example 1.1 continued}
\newtheorem*{definition5}{Definition 2.1}
\newtheorem*{definition6}{Definition 2.3}


\newcommand{\R}{\mathcal R}
\newcommand{\ZZ}{\mathbb Z}
\newcommand{\m}{\mathbf m}
\newcommand{\ra}{\rightarrow }
\newcommand{\Hom}{\hbox {\rm Hom}}
\newcommand{\Sym}{\hbox {\rm Sym}}
\renewcommand{\refname}{Bibliography}

\begin{document}

\title[Rees algebra of a module]{What is the Rees algebra of a module?}
\author{David Eisenbud}
\address{Mathematical Sciences Research Institute, 1000 Centennial Dr.,
Berkeley, California 94720}
\email{de@msri.org}
\author{Craig Huneke}
\address{Department of Mathematics, University of Kansas, Lawrence, Kansas
66045}
\email{huneke@math.ukans.edu}
\author{Bernd Ulrich}
\address{Department of Mathematics, Purdue University, West Lafayette, Indiana
47907}
\email{ulrich@math.purdue.edu}
\thanks{All three authors were partially supported by the NSF}
\subjclass[2000]{Primary 13A30, 13B21; Secondary 13C12}
\keywords{Rees algebra, module, integral dependence}
\date{May 2, 2001 and, in revised form, October 19, 2001}
\commby{Wolmer V. Vasconcelos}
\begin{abstract}
In this paper we show that the Rees algebra can be made
into a functor on modules over a ring in a way that extends its
classical definition for ideals.  The Rees algebra of a module $M$
may be computed in terms of a ``maximal'' map $f$ from $M$ to a
free module as the image of the map induced by $f$ on symmetric
algebras.  We show that the analytic spread and reductions of $M$
can be determined from any embedding of $M$ into a free module,
and in characteristic 0---but not in positive
characteristic!---the Rees algebra itself can be computed from any
such embedding.
\end{abstract}
\maketitle



The Rees algebra of an ideal $I$ in a ring $R$, namely $\R =\bigoplus _{n=0}^{\infty }I^{n}=R[It]\subset R[t], $ plays a major role
in commutative algebra and in algebraic geometry since
$\text{Proj}(\R )$ is the blowup of $\text{Spec}(R)$ along the
subscheme defined by $I$. Several authors have found it useful to
generalize this construction from ideals to modules; see for
instance Gaffney and Kleiman \cite{GK}, Katz \cite{K}, Katz and
Kodiyalam \cite{KK}, Kleiman and Thorup \cite{KT}, Kodiyalam \cite{Ko},
 Liu \cite{L}, Rees \cite{R}, Simis, Ulrich, and Vasconcelos 
\cite{SUV1}, \cite{SUV2},
and Vasconcelos \cite{V},
who define the Rees algebra of a module satisfying one or another
hypothesis. Usually this hypothesis was tailored to approach the
problem(s) the authors were interested in solving.

The goal of this paper is to clarify the definition for arbitrary
finitely generated modules over a Noetherian ring $R$. Our
interest in this clarification arose through our work on
generalized prinicipal ideal theorems and the heights of ideals of
minors, where we heavily use Rees algebras (see
Eisenbud-Huneke-Ulrich \cite{EHU1}, \cite{EHU2}). It seems worthwhile to
understand the differences and similarities of the various
approaches from the papers above, and to make the definition as
functorial as possible. Even for ideals there is a problem: in the
grade 0 case it is not clear from the definition above whether the
Rees algebra depends on the embedding of $I$ in $R$.

A natural approach is to define the Rees algebra of a module as
the symmetric algebra modulo  $R$-torsion (that is, modulo elements
killed by non-zerodivisors of $R)$. This does  not provide a
satisfactory definition in all cases in the sense that it may give
the wrong answer even for an ideal, if the ring is not a domain.
In general, as was well-known, it is a good definition when the
module $M$ ``has a rank'', i.e., when $M$ is free of constant rank
locally at the associated primes of $R$. This hypothesis is
sufficient for many applications; however, for example, it is not
necessarily preserved replacing $M$ by $M/xM$ and $R$ by $R/xR$
even if $x$ is a non-zerodivisor on $M$ and $R$.

Another alternative is to consider a module $M$ together with an
embedding into a free module $G$, and define the Rees algebra of
$M$ to be the subalgebra of the symmetric algebra of $G$ generated
by $M$. More generally, for any homomorphism $g \colon M\to N$ we
define $\R (g)$, the Rees algebra of $g$, to be the graded
$R$-algebra which is the image of the map $\Sym (g) \colon \Sym (M)\to \Sym (N)$. One may then try to define the Rees algebra
of $M$ as $\R (g)$ for an embedding $g$ of $M$ into a free module
$G$.

However, the result may depend on the chosen embedding  $g$. In Section 1 we
give as example a principal ideal $I$ in an Artinian ring of characteristic
$p>0$, and an embedding $g \colon I\to R^{2}$ such that $\R (g)$ is not
isomorphic to $\R (I)=\bigoplus _{n= 0}^{\infty }I^{n}$. Of course $\R (I)$ may
also be expressed as $\R (i)$, where $i$ is the inclusion of $I$ as an ideal of
$R$.  Thus $\R (g)$ depends on $g$, not only on $M$.

Here we take a third alternative:

\begin{definition1} If $R$ is a ring and $M$ is an
$R$-module, we define the {\em Rees algebra\/} of $M$ to be
\begin{equation*}\R (M)=\Sym (M)/(\bigcap _{g}L_{g})
\end{equation*}
where the intersection is taken over all homomorphisms $g$ from
$M$ to free $R$-modules, and $L_{g}$ denotes the kernel of
$\Sym (g)$.
\end{definition1}


Although the definition may at first appear somewhat complicated,
it is at least obviously functorial: if $h \colon M\to N$ is a
homomorphism of $R$-modules, then for every homomorphism from $N$
to a free module $g \colon N\to G$ the map $gh$ is a homomorphism
from $M$ to a free module, so $\Sym (h) \colon \Sym (M)\to \Sym (N)$
induces an $R$-algebra homomorphism $\R (M)\to \R (N)$.  As the
symmetric algebra functor preserves epimorphisms, so does the Rees
algebra functor.

In Section 1 we solve the problem of computing $\R (M)$ (if $M$ is
finitely generated) by showing that $\R (M)=\R (f)$ for any
homomorphism from $M$ to a free module $F$ such that the dual map
$F^{*}\to M^{*}$ is surjective. This implies that forming the Rees
algebra of a finitely generated module over a Noetherian 
ring commutes with flat base
change.  We also show that if $g \colon M\to R$ is an embedding in
a free module of rank 1, then $\R (M)=\R (g)$, so that $\R (M)$
agrees with the classical definition for ideals (in particular,
this shows that the classical definition is independent of the
choice of representation of $M$ as an ideal). Moreover, we show
that in many cases $\R (M)$ can be computed from any embedding.
The following is a special case of what we prove:

\begin{theorem1} Let $R$ be a Noetherian ring and let $M$ be
a finitely generated $R$-module. If $R$ is torsion free over
$\ZZ $, or $R$ is unmixed and generically Gorenstein, or $M$ is
free locally at each associated prime of $R$, then
$\R (M)\cong \R (g)$ for any embedding $g \colon M\to G$ of $M$ into
a free module $G$.
\end{theorem1}


In Section 2 we use our Rees algebra construction to introduce
analytic spread and integral dependence for arbitrary modules. We
prove that for any embedding $g$ of $M$ into a free module, the
natural map $\R (M)\to \R (g)$  has nilpotent kernel. It follows
that analytic spread and integral dependence can be determined in
$\R (g)$.

In two subsequent papers \cite{EHU1}, \cite{EHU2} we apply the notions
developed here to obtain new generalized principal ideal theorems
and results on heights of ideals of minors of a matrix.

\section*{1. Rees algebras}

We begin with an example showing that the Rees algebra of a module
cannot be defined from an arbitrary embedding into a free module,
even when the module is an ideal:

\begin{definition2} Let $k$ be a field of characteristic
$p$, let
\begin{equation*}R= k[X,Y,Z]/((X^{p},Y^{p}) + (X,Y,Z)^{p+1}),
\end{equation*}
write $x,y,z$ for the images of $X,Y,Z$ in $R$, and take $M$ to be
the ideal $M = Rz \cong R/(x,y,z)^{p}$. Write $g_{1} \colon M\to R$
for the inclusion $M=Rz\subset R$, and let $g_{2} \colon M\to R^{2}=Rt_{1}\oplus Rt_{2}$ be the homomorphism sending $z$ to
$xt_{1}+yt_{2}$. It is easy to see that $g_{2}$ is also an embedding.
The algebra $\R (g_{1})$ is the same as the classical Rees algebra
$\R =\bigoplus _{n=0}^{\infty }(z^{n})$, and has $p^{\text{th}}$ graded
component $(z^{p})\neq 0$. On the other hand
\begin{equation*}\R (g_{2})_{p} = R(x^{p} t^{p}_{1} + y^{p} t^{p}_{2}) = 0,
\end{equation*}
and it follows that $\R (g_{2})$ cannot surject onto the classical
$\R =\R (g_{1})$ by any graded homomorphism, so $\R (g_{2})\not \cong \R (M)$
as graded  rings. 
\end{definition2}


To compute the Rees algebra of a module, we use the following
notion.

\begin{definition3} Let $R$ be a ring and let  $M$ be
an $R$-module. We say that $f \colon M \to F$ is a {\em versal map
to a free module}, if $F$ is a free
$R$-module, $f$ is a homomorphism, and every homomorphism 
from $M$ to a free module factors through $f$.
\end{definition3}


It follows at once from the definition that 
if $f \colon M\to F$ is a versal map to a free $R$-module
$F$, then $\R (M)=\R (f)$. With a finiteness assumption it is easy
to find such a map:

\begin{theorem2} Let $R$ be a ring and let $M$ be a
finitely generated $R$-module. Let $f \colon M\to F$ be a
homomorphism to a free $R$-module $F$ so that the dual map 
$F^{*}\to M^{*}$ is surjective (such $f$ can be obtained by 
composing the natural
map $M\to M^{**}$ with the dual of any epimorphism from a finitely
generated free module onto $M^{*}$). One has that $f$ is versal and
$\R (M)=\R (f)$. In particular, formation of
the Rees algebra of a finitely generated module over a Noetherian
ring commutes with flat base change.
\end{theorem2}


\begin{proof} Let $g \colon M\to G$ be a homomorphism to a free
$R$-module. We must show that $g$ factors through $f$. Since $M$
is finitely generated, $g$ factors through a finitely generated
free summand of $G$, and we may assume that $G$ is finitely
generated. It follows that the dual $G^{*}$ is free. Consequently we
may write $g^{*}=f^{*}h$. We may also suppose that $F$ is finitely
generated. Since $F$ and $G$ are reflexive, the desired
factorization is $g=h^{*}f$.

The equality $\R (M)=\R (f)$ has been observed just
before the Proposition. Finally
assume that $R$ is Noetherian and let $f$ be any homomorphism
satisfying our hypothesis. If $S$ is a flat $R$-algebra, then
$\Hom _{S} (S\otimes _{R} M,S)=S\otimes _{R} \Hom _{R}(M,R)$ because $M$ is
finitely presented. Thus $S\otimes _{R} f$ has a surjective $S$-dual,
and it follows that $\R (S\otimes _{R} M)=\R (S\otimes _{R} f)=S\otimes _{R}
\R (f)=S\otimes _{R} \R (M)$, as required. \end{proof}


\begin{definition4} By Proposition 1.3 the map
$g_{2}$ is not versal. It is easy to check that $M^{*}$ requires 3
generators and a versal map from $M$ to $R^{3}$ may be written as
\begin{equation*}f \colon M=Rz \ra R^{3} = Rt_{1} \oplus Rt_{2} \oplus Rt_{3},
\quad z\mapsto xt_{1}+yt_{2}+zt_{3}.
\end{equation*}
We have
\begin{equation*}\R (f)_{p} = R(x^{p} t_{1}^{p} + y^{p} t_{2}^{p}  + z^{p} t_{3}^{p}) = Rz^{p}t^{p}_{3} =
\R (M)_{p},
\end{equation*}
as implied by Proposition 1.3.
\end{definition4}


Next we show that our definition of the Rees algebra agrees with
the classical notion for ideals, which is thus independent of the
embedding of the ideal into $R$.

\begin{theorem3} Let $R$ be a ring, let $M$ be a finitely
generated ideal of $R$, and let $g \colon M\to R$ be the inclusion
map. The natural map  from $\R (M)$ to the classical Rees algebra
$\R =\bigoplus _{n=0}^{\infty }I^{n}=\R (g)$ is an isomorphism.
\end{theorem3}


\begin{proof} Let $f \colon M\to F$ be a versal map to a free
module $F$, of rank $n$, say. Since $f$ is versal we may find a
homomorphism $h \colon F\to R$ so that $g = hf$. We must show that
$\phi \colon =\Sym (h)$ is a monomorphism on the subring of
$\Sym (F)$ generated by $f(M)$.

Write $\Sym (F)=R[t_{1},\dots ,t_{n}]$, and $\Sym (R)=R[z]$. Let
$m_{1},\dots ,m_{s}$ be generators of $M$, and write $a_{i}=g(m_{i})\in R$,
so that $\phi(f(m_{i}))=a_{i}z$. Let $S=R[x_{1},\dots ,x_{s}]$ be a polynomial
ring, and consider the homomorphism $\psi \colon S\to \Sym (F)$
sending $x_{i}$ to $f(m_{i})$. We must show that the kernel of $\psi $
is the same as the kernel of $\phi \psi $. Giving each $x_{i}$ degree
1, the kernel of $\phi \psi $ is homogeneous, so it suffices to show
that if $u\in S$ is a form of degree $d$ such that $\phi \psi (u)=0$,
then $\psi (u) =0$. We do induction on $d$, the case $d=0$ being
obvious.

We may write ${u = \sum ^{s}_{i=1} x_{i}u_{i}}$ where the $u_{i}$ are forms
of degree $d-1$. We see that
\begin{equation*}0 =\phi \psi (u) =\phi \psi (\sum _{i}x_{i}u_{i}) =\phi \psi (\sum _{i}a_{i}u_{i})z,
\end{equation*}
so $\phi \psi (\sum _{i}a_{i}u_{i})=0$. By our induction hypothesis,
$\psi (\sum _{i}a_{i}u_{i})=0$, too.

We may expand each $\psi (u_{i})$ in the form $\psi (u_{i})=\sum _{\alpha }r_{i,\alpha }t^{\alpha }$, where the sum runs over all multi-indices
$\alpha $ of weight 
$d-1$, and thus $\sum _{i}\sum _{\alpha }a_{i}r_{i,\alpha }t^{\alpha }= 0$. Since the distinct monomials
$t^{\alpha }$ are linearly independent, we have
$\sum _{i}a_{i}r_{i,\alpha }=0$ for each $\alpha $. By our hypothesis
that $g$ is an embedding, the $f(m_{i})$ satisfy the same linear
relations as the $a_{i}$, so we get $\sum _{i}f(m_{i})r_{i,\alpha }=0$ for
each $\alpha $, and finally $\psi (u)=\sum _{i}\sum _{\alpha }f(m_{i})r_{i,\alpha }t^{\alpha }= 0$, as required.  \end{proof}


For the proofs that follow we need to identify the minimal and the
associated primes of $\R (M)$.

\begin{theorem4} Let $R$ be a Noetherian ring and let
$M$ be a finitely generated $R$-module. There is a one-to-one
correspondence between the associated primes of $\R (M)$ and the
associated primes of $R$ given by $P\mapsto R\cap P$, and likewise
for minimal primes.
\end{theorem4}


\begin{proof} Notice that $\R (M)$ is an $R$-subalgebra of a
polynomial ring $S$ in finitely many variables over $R$. Every
associated prime of $R$ is the contraction of an associated prime
of $\R (M)$, and every associated prime of $\R (M)$ is the
contraction of an associated prime of the polynomial ring $S$,
which in turn is extended from an associated prime of $R$. Since
the resulting one-to-one correspondence between $\text{Ass}(\R (M))$ and
$\text{Ass}(R)$ is order preserving, we have also proved
the claim about minimal primes.   \end{proof}


\smallskip Note that a versal map from a finitely generated module 
$M$ to a free module has the same
image as the natural map from $M$ to its double dual. This image
is called the {\em torsionless quotient\/} of $M$. Any homomorphism from
$M$ to a free module factors uniquely through the torsionless
quotient of $M$. The following result gives conditions under which the Rees
algebra of a torsionless module can be deduced from any inclusion
into a free module. For convenience in applications we will state
it without the torsionless hypothesis. 

\begin{theorem5}  Let $R$ be a Noetherian ring, let $M$ be a
finitely generated $R$-module, and let $g \colon M \to G$ be a 
homomorphism to
a free $R$-module $G$ inducing an inclusion on the torsionless
quotient of $M$. If for
each associated prime $Q$ of $R$ either $R_{Q}$ is Gorenstein,  or 
$M_{Q}$ is free, or $R_{Q}$ is ${\mathbb{Z}}$-torsion free, then the
natural epimorphism $\R (M) \to \R (g)$ is an isomorphism.
\end{theorem5}


\begin{proof} Replacing $M$ by its torsionless quotient,
we may assume that $g \colon M\to G$ is an inclusion.

Let $f \colon M \to F$ be a versal map from $M$ to a free module, and let
$\R =\R (M)=\R (f)$ be the Rees algebra of $M$. Let $h \colon F\to G$ be a
homomorphism with $g=hf$ and let $\phi \colon \R \to \R (g)$ be the induced
epimorphism. By Proposition 1.5 every associated prime of $\R $ contracts
to an associated prime of $R$. Hence to prove the injectivity 
of $\phi $ we may replace $R$
by $R_{Q}$, where $Q$ is an associated prime of $R$.

If $R$ is Gorenstein (and hence Artinian), then free modules are
injective, and therefore any monomorphism from a module to a free
module is versal.  Similarly, if $M$ is free, then since $R$ has
depth 0, any monomorphism from $M$ to a free module splits, and
thus again is versal. In either case we see that $f$ and $g$ are
both versal, and the injectivity of $\phi $ follows from the
functoriality of the Rees algebra.

Finally, we treat the case where $R$ is ${\mathbb{Z}}$-torsion free.
It suffices to consider the case where $F = G \oplus H$ is
finitely generated with $H$ free, and $h$ is the natural
projection. Set $J=H\cdot \Sym (F)$, the kernel of $\Sym (h)$. Since
$g$ is an inclusion,  $\phi $ is an injection in degree 1, and we
must show that $\phi $ is an injection in every degree, or
equivalently $J\cap \R =0$. This follows from the next Lemma:


\begin{theorem6}  Let $R$ be a ring  and let $F$ be a finitely
generated free $R$-module. Let $M$ be a submodule of $F$, and let
$\R $ be the subalgebra of $\Sym (F)$ generated by $M$. Let $H$ be a
summand of $F$ with $H\cap M=0$. If $d!$ is a non-zerodivisor in
$R$, then $(H\cdot \Sym (F))\cap \R _{d}=0$.
\end{theorem6}


\begin{proof} It is enough to prove that the Lemma holds after
localizing at each maximal ideal of $R$, so we may assume that $R$
is local.

Let $t_{1},\dots ,t_{n}$ be a basis of $F$. Since $R$ is local we may
suppose that $H$ is generated by $t_{m+1},\dots ,t_{n}$.  Set
$J=H\cdot \Sym (F)$.  We may assume that $d > 1$.  Since $(d-1)!$
is a non-zerodivisor as well, we know by induction that $J \cap \R _{d-1} = 0$.  Writing $\partial _{i} = \frac{\partial }{\partial t_{i}}$, one has $\partial _{i} (\R ) \subset \R $ for every $i$ because
the $R$-algebra $\R $ is generated by linear forms, $\partial _{i}(J)
\subset J$ for every $i \le m$, and $\partial _{i}(J^{2}) \subset J$
for every $i$.

Now if $u \in J \cap \R _{d}$, then $\partial _{i} (u) \in J \cap \R _{d-1}
= 0$ for every $i \le m$.  Since $d!$ is a non-zerodivisor on $R$,
it follows that $u \in R[t_{m+1}, \dots , t_{n}]_{d} \subset J^{2}$. Thus
$\partial _{i}(u) \in J \cap \R _{d-1} = 0$ for every $i$, and hence
$u = 0$ because $d!$ is a non-zerodivisor. \end{proof}


\smallskip To connect our definition of the Rees algebra with the torsion in
the symmetric algebra, suppose that $R$ is a Noetherian ring, $M$
is a finitely generated $R$-module, and $\mathcal{A}$ is the
$R$-torsion of Sym$(M)$.  If $M_{Q}$ is free for every associated
prime $Q$ of $R$ and if $f \colon M\to F$ is a versal map to a
free module, then Sym$(f)$ induces an isomorphism Sym$(M)/{\mathcal{A}}
\tilde {\ra } \R (M)$. This follows from Proposition 1.5 and the fact
that Sym$(f)_{Q}$ is injective for every associated prime $Q$ of
$R$.

If we are only interested in the reduced structure of the Rees
algebra of $M$, then we can compute it from any embedding of the
torsionless quotient of $M$.

\begin{theorem7} Let $R$ be a Noetherian ring, let $M$
be a finitely generated $R$-module, and let $g \colon M \to G$ be
a homomorphism to a free $R$-module $G$ inducing an inclusion on
the torsionless quotient of $M$.
The kernel of the natural epimorphism $\R (M)\to \R (g)$ is
nilpotent.
\end{theorem7}


\begin{proof} By Proposition 1.5 every minimal prime of $\R (M)$
contracts to a minimal prime of $R$, so  it is enough to prove the
result after localizing at a minimal prime of $R$. Thus we may
assume that $(R,\m )$ is Artinian and local. We may also suppose
that $G$ is finitely generated and we may replace $M$ by its
torsionless quotient to assume that $g$ is a monomorphism.

Let $f \colon M\to F$ be a versal map from $M$ to a free module,
and let $K$ be the kernel of the natural epimorphism
$\R (M)=\R (f)\to \R (g)$. Suppose first that $M$ has no free
summand. Because $R$ is local, it follows that $\operatorname{Im} f\subset \m F$. The kernel $K$ must be contained in the positive degree part
of $\R (M)$, which is contained in $\m \Sym (F)$. As $\m $ is
nilpotent, $K$ is nilpotent as well.

In the general case, let $H$ be a maximal free submodule of $M$.
Since any inclusion of finitely generated free modules over an
Artinian ring splits, we may write  $G=G'\oplus H$ and $M=M'\oplus H$ in such a way that $g=g'\oplus 1_{H}$. The map $\R (M)\to \Sym (G)$
is obtained from the map $\R (M')\to \Sym (G')$ induced by $g'$ by
adjoining polynomial variables. As the kernel of the latter map is
nilpotent, the desired result follows. \end{proof}
\renewcommand{\qed}{} \end{proof} 

\section*{2. Integral dependence}

In this section we introduce general definitions of integral
dependence and of analytic spread for modules that we will apply
elsewhere.

\begin{definition5}  Let $R$ be a ring, $M$
an $R$-module, and $U \subset L$ submodules of $M$. \begin{enumerate}

\item Let $U',L'$ be the images of $U,L$ in ${\R }(M)$ and consider
the subalgebras $R[U'] \subset R[L']\subset {\R }(M)$.  We say $L$
is {\em integral} over $U$ {\em in} $M$ if the ring extension
$R[U'] \subset R[L']$ is integral.
\item We say $M$ is {\em integral} over $U$ or $U$ is a {\em reduction} of
$M$, if $M$ is integral over $U$ in $M$.
\end{enumerate}\end{definition5}

In the situation of Definition 2.1(1) the Rees algebra of $L$ maps
to the Rees algebra of $M$, so if $L$ is integral over $U$, 
then $L$ is integral over $U$ in $M$.

\begin{theorem8}  Let $R$ be a Noetherian ring, $M$ a
finitely generated $R$-module, $U \subset L$ submodules of $M$,
and $f \colon M \ra F$ a versal map from $M$ to a free
$R$-module.  The following are equivalent: \begin{enumerate}
\item $L$ is integral over $U$ in $M$.
\item For every minimal prime $Q$ of $R$, the module
$L'$ is integral over $U'$ in $M'$, where $'$ denotes images in
$F/QF$.
\item For every homomorphism
$M\to G$ to a free $R$-module and for every homomorphism $R \ra S$
to a domain $S$, the module $L'$ is integral over $U'$ in $M'$,
where $'$ denotes tensoring with $S$ and taking images in
$S\otimes _{R}G$.
\item For every homomorphism $R \ra V$ to a rank one discrete valuation
ring $V$ whose kernel is a minimal prime of $R$, we have $U' =
L'$, where $'$
denotes tensoring with $V$ and taking images in $V\otimes _{R}F$.
\item (Valuative Criterion of Integrality)
For every homomorphism $M\to G$ to a free $R$-module and every
homomorphism $R \ra V$ to a rank one discrete valuation ring $V$,
we have $U' = L'$, where $'$ denotes tensoring with $V$ and taking
images in $V\otimes _{R}G$.
\end{enumerate}\end{theorem8}


\begin{proof} By the functoriality of the Rees algebra, we may
assume $G=F$ in parts (3) and (5). As ${\R }(M)$ embeds into
${\R }(F) = \text{Sym}(F)$, we may replace $M$ by $F$ in (1).  In
parts (2)--(5), the rings $R/Q, S, V$ are domains, and hence by
Theorem 1.6 the embedding of $L'$ and $M'$ into the free modules $F/QF,
S\otimes _{R}F, V\otimes _{R}F$, respectively, can be used to define $\R(L')$
and 
${\R }(M')$.  In particular we may replace $M$ by $F$ in these parts as
well.

Now it is obvious that (1) implies (3).  Part (1) follows from (2)
since ${\R }(F)/\sqrt {0} \subset \prod _{Q} {\R }(F/QF)$, where $Q$
ranges over all minimal primes of $R$ and $F/QF$ is considered as
a module over $R/Q$. Finally, the equivalence of (2) and (4) and
of (3) and (5) has been shown in Rees \cite[1.5(ii)]{R}. \end{proof}


We see from Theorem 2.2 that our definition of integrality differs
from that of Rees \cite[p. 435]{R} when $R$ is not a domain. Rees'
definition amounts to saying that for every minimal prime $Q$ of
$R$, the module $L'$ is integral (in our sense or his) over $U'$
in $M'$, where now $'$ denotes images in $M/QM$. If, for example,
$k$ is a field, $R = k[x]/(x^{2})$, and $U = 0 \subset L = M =
(x)/(x^{2})$, then $M$ is integral over $U$ in our sense but not in
the sense of Rees.

\begin{definition6} Let $R$ be a local ring
with residue field $k$ and let $M$ be a finitely generated
$R$-module.  The {\em analytic spread} $\ell (M)$ of $M$ is the
Krull dimension of $k \otimes _{R} \R (M)$.
\end{definition6}


By way of illustration, we remark that the analytic spread of a
finitely generated module $M$ over an Artinian local ring $(R,\m ,
k)$ is equal to the rank $r$ of a maximal free summand $H$ of $M$
(and is also equal to the dimension of the Rees algebra of the
module). Indeed, writing $M=M'\oplus H$  we have that any
homomorphism from $M$ to a free module $F$ carries $M'$ into $\m F$, which generates a nilpotent ideal of $\Sym (F)$. Thus
$(\R (M)/\m \R (M))_{\text{red}}=\R (M)_{\text{red}}$ is a
polynomial ring over $k$ in $r$ variables.

If $k$ is infinite one can show as in the case of ideals, using a
homogeneous Noether normalization of $k \otimes _{R} {\R }(M)$ and
Nakayama's Lemma, that
\begin{equation*}\ell (M) = \min \{\mu (U) \mid U \hbox { is a reduction of }M\}.
\end{equation*}
Furthermore, $\ell (M) \leq \mu (M)$ and equality holds if and only
if $M$ has no proper reduction.

\begin{theorem9} Let $R$ be a Noetherian local ring with
residue field $k$, let $M$ be a finitely generated $R$-module,
and let $g \colon M\to G$ be a homomorphism to a free $R$-module
$G$. If $g$ induces an inclusion on the torsionless quotient of
$M$, then $\ell (M)=\dim k\otimes _{R}\R (g)$.
\end{theorem9}


\begin{proof} Proposition 1.8 shows that $\R (M)$ differs from
$\R (g)$ only by a nilpotent ideal, and thus the same holds after
tensoring with $k$. \end{proof}


 
\bibliographystyle{amsalpha}
\begin{thebibliography}{EHU1}




\bibitem[EHU1]{EHU1}
D.~Eisenbud, C.~Huneke and B.~Ulrich, {\em Order
ideals and a generalized Krull height theorem}, to appear in Math. Ann. 


\bibitem[EHU2]{EHU2}
\bysame , {\em Heights of ideals of minors}, preprint, 2001.



\bibitem[GK]{GK}
T.~Gaffney and S.~Kleiman, {\em Specialization of
integral dependence for modules}, Invent. Math. {\bf 137} (1999), 541--574.
\MR{2000k:32025}



\bibitem[K]{K}
D.~Katz, {\em Reduction criteria for modules}, Comm.~in Algebra {\bf 23}
(1995), 4543--4548. \MR{96j:13022}



\bibitem[KK]{KK}
D.~Katz and V.~Kodiyalam, {\em Symmetric powers of
complete modules over a two-dimensional regular local ring},
Trans.~Amer.~Math.~Soc. {\bf 349} (1997), 747--762. \MR{97g:13041}



\bibitem[KT]{KT} S.~Kleiman and A.~Thorup, {\em Conormal geometry of maximal
minors}, J. Algebra {\bf 230} (2000), 204--221. \MR{2001h:13006}



\bibitem[Ko]{Ko} V.~Kodiyalam, {\em Integrally closed modules over
two-dimensional regular local rings}, Trans.~Amer.~Math.~Soc. {\bf 347} (1995),
3551--3573. \MR{95m:13015}



\bibitem[L]{L} J.-C.~Liu, {\em Rees algebras of finitely generated torsion-free
modules over a two-dimensional regular local ring}, Comm.~in Algebra {\bf 26}
(1998), 4015--4039. \MR{99k:13003}



\bibitem[R]{R}
D. Rees, {\em Reduction of modules}, Math. Proc.
Camb. Phil. Soc. {\bf 101} (1987), 431--449. \MR{88a:13001}


\bibitem[SUV1]{SUV1}
A. Simis, B. Ulrich and W. V. Vasconcelos, {\em Codimension, multiplicity and integral
extensions}, Math. Proc. Camb. Phil.~Soc. {\bf 130} (2001), 237--257.
\MR{2002c:13017}

\bibitem[SUV2]{SUV2}
\bysame, \emph{Rees algebras of modules}, to appear in Proc. London Math. Soc. 



\bibitem[V]{V} W. V. Vasconcelos, {\em Arithmetic of Blowup Algebras}, London
Math. Soc. Lect. Notes, vol.~195, Cambridge University Press, Cambridge, 1994.
\MR{95g:13005}


\end{thebibliography}

\end{document}
