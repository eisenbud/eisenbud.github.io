%_ **************************************************************************
%_ * The TeX source for AMS journal articles is the publisher's TeX code    *
%_ * which may contain special commands defined for the AMS production      *
%_ * environment.  Therefore, it may not be possible to process these files *
%_ * through TeX without errors.  To display a typeset version of a journal *
%_ * article easily, we suggest that you either view the HTML version or    *
%_ * retrieve the article in DVI, PostScript, or PDF format.                *
%_ **************************************************************************


%\controldates{17-NOV-1997,17-NOV-1997,17-NOV-1997,8-DEC-1997}
 
\documentclass{proc-l}
%\issueinfo{126}{03}{March}{1998}
\pagespan{687}{691}
\PII{S 0002-9939(98)04229-4}
\def\copyrightyear{1998}




\theoremstyle{plain}
\newtheorem*{theorem1}{Corollary 1.1}
\newtheorem*{theorem2}{Theorem 2.1}
\newtheorem*{theorem3}{Theorem 3.1}
\newtheorem*{theorem4}{Proposition 3.2}

\theoremstyle{definition}
\newtheorem*{definition1}{Example 4.1}
\newtheorem*{definition2}{Example 4.2}




\newcommand{\precc}{{\prec \kern -5pt\prec }}


\begin{document}

\title[Non-commutative Gr\"{o}bner bases for Commutative Algebras]
{Non-commutative Gr\"{o}bner bases \\ for commutative algebras}
\author{David Eisenbud}
\address{MSRI, 1000 Centennial Dr., Berkeley, California 94720}
\email{de@msri.org}
\author{Irena Peeva}
\address{Department of Mathematics, Massachusetts Institute of Technology,
Cambridge, Massachusetts 02139}
\email{irena@math.mit.edu}
\author{Bernd Sturmfels}
\address{Department of Mathematics, University of California, Berkeley,
California 94720}
\email{bernd@math.berkeley.edu}
\date{September 6, 1996}
\thanks{The first and third authors are grateful to the NSF and
 the second and third authors
are grateful to the David and Lucille Packard Foundation for 
partial support in preparing this paper. }
\begin{abstract} An ideal
$I$ in the free associative algebra  $k\langle X_{1},\dots ,X_{n}\rangle $
over a field $k$
is shown to have a finite Gr\"{o}bner basis if the algebra defined by $I$ is
commutative; in characteristic 0 and generic coordinates the 
Gr\"{o}bner basis may even be constructed by lifting a commutative Gr\"{o}bner
basis and adding commutators.
\end{abstract}
\subjclass{{Primary 13P10, 16S15}}
\commby{Wolmer V. Vasconcelos}
\maketitle



\section*{1. Introduction}

Let $k$ be a field and let 
$k[x] = k[x_{1},\dots ,x_{n}]$ be the polynomial ring in
$n$ variables and $k \langle X \rangle = k\langle X_{1},\dots ,X_{n}\rangle $ 
the free associative algebra in $n$ variables. Consider the
natural map $\gamma :\, k \langle X \rangle \to k[x] $ taking $X_{i}$ to $x_{i}$.
It is sometimes useful to regard a commutative algebra
$k[x]/I$ through its non-commutative presentation 
$k[x]/I \cong k \langle X \rangle /J$, 
where $J = \gamma ^{-1}(I)$. This is especially
true in the construction of free resolutions as in \cite{An}. 
Non-commutative presentations have been exploited in 
\cite{AR} and \cite{PRS} to study homology of 
coordinate rings of Grassmannians and toric varieties.
These applications all make use of Gr\"{o}bner bases for $J$
(see \cite{Mo}  for non-commutative  Gr\"{o}bner bases).
In this note we give an explicit description (Theorem 2.1) of the minimal
Gr\"{o}bner bases for $J$ with respect to monomial orders on
$k \langle X \rangle $ that are lexicographic extensions of
monomial orders on $k [x] $. 

 Non-commutative  Gr\"{o}bner bases are usually infinite; for
example, if $n=3$ and  $\,I = (x_{1}x_{2}x_{3}) \,$ then $\gamma ^{-1}(I)$
does not have a finite Gr\"{o}bner basis
for any monomial order on $k \langle X \rangle $.  (There are only two ways
of choosing leading terms for the three commutators,
and both cases are easy to analyze by hand.)
However, after a linear change of variables 
the ideal becomes $\,I'=(X_{1}(X_{1}+X_{2})(X_{1}+X_{3}))$, and we shall 
see in Theorem 2.1 that
$\,X_{1}(X_{1}+X_{2})(X_{1}+X_{3})$ and the three commutators 
$X_{i} X_{j} - X_{j} X_{i}$ are a Gr\"{o}bner basis for $\gamma ^{-1}(I')$
with respect to a suitable order. This situation is rather general: 
Theorems 2.1 and 3.1  imply the following  result:

\begin{theorem1} Let $k$ be an infinite field and
$I\subset k[x]$ be an ideal.
After a general linear change of variables,
the ideal  $\gamma ^{-1}(I)$ in $k\langle X \rangle $
has a finite Gr\"{o}bner basis.  
In characteristic 0, if $I$ is homogeneous,
such a basis
can be found with degree at most 
$\,max \{2,regularity (I) \}$.
\end{theorem1}


In characteristic 0 the Gr\"{o}bner basis of $\gamma ^{-1}(I)$ in Corollary 1.1
may be obtained by lifting the Gr\"{o}bner basis
of $I$, but this is not so in characteristic $p$; see Example 4.2.
Furthermore, $\gamma ^{-1}(I)$ might have no finite Gr\"{o}bner basis at all
if the field is finite; see Example 4.1. 

The behavior of $\gamma ^{-1}(I)$
is in sharp contrast to what happens for arbitrary ideals in 
$k \langle X \rangle $.
For example, the defining ideal
in $k \langle X \rangle $ of the group algebra
of a group with undecidable word problem has no 
finite Gr\"{o}bner basis. Another example is Shearer's algebra
$k\langle a,b\rangle /(ac-ca,aba-bc,b^{2}a)\rangle$,
which has irrational Hilbert series \cite{Sh}. As any finitely
generated monomial ideal defines an algebra with rational Hilbert
series, the ideal $(ac-ca,aba-bc,b^{2}a)$ can have no finite 
Gr\"{o}bner basis. (Other consequences of having a finite
Gr\"{o}bner basis are deducible from \cite{An} and \cite{Ba};
these are well-known in the case of commutative algebras!)

In the next section we present the basic computation of the
initial ideal and Gr\"{o}bner basis for $J = \gamma ^{-1}(I)$.
In \S 3 we give the application to finiteness and liftability
of Gr\"{o}bner bases.

\section*{2. The Gr\"{o}bner basis of $\gamma ^{-1}(I)$}

Throughout this paper we fix an ideal 
$I\subset k[x] $ and  
$J := \gamma ^{-1}(I)\subset k \langle X \rangle $.
We shall make use of the {\em lexicographic splitting} of 
$\gamma $, which is defined as the $k$-linear map
\begin{equation*}\delta \,\,: \,\, k[x] \to k \langle X \rangle \, , \quad x_{i_{1}} x_{i_{2}} \cdots x_{i_{r}}  \mapsto X_{i_{1}} X_{i_{2}} \cdots X_{i_{r}}\ \ \
   \hbox {if}\ \ \ i_{1} \leq i_{2} \leq \cdots \leq i_{r}.  \end{equation*}
Fix a monomial order $\prec $ on $k[x]$.
The {\em lexicographic extension\/} $\precc $ of $\prec $
to $k \langle X \rangle $ is
defined for monomials
$M, N \in k \langle X \rangle $ by
\begin{equation*}M \,\precc \, N \ \ \ \hbox {if} \ \ \ \begin{cases}\gamma (M)\prec \gamma (N) &\text{or}\\
\gamma (M) = \gamma (N) &\text{and 
$M$ is lexicographically smaller than $N$.}
\end{cases}
\end{equation*}
Thus, for example, $\,X_{i}X_{j}\precc X_{j}X_{i} \,$ if $\, i<j$.

To describe the $\precc $-initial ideal of $J$ we use the
following construction:
Let $L$ be any monomial ideal in $k[x]$.
If $\, m = x_{i_{1}} \cdots x_{i_{r}} \,\in \, L \,$
and $\, i_{1} \leq \cdots \leq i_{r} \,$, denote by 
$\, {\mathcal{U}}_{L} (m)  \,$  the set of all monomials
$ \, u \in k[x_{i_{1}+1},\ldots , x_{i_{r}-1}]\,$ such that
neither $\, u\frac{m}{x_{i_{1}}}\,$ nor
$\, u\frac{m}{x_{i_{r}}}\,$ lies in $L$. 
For instance,
if $L = (x_{1} x_{2} x_{3},\, x_{2}^{d})$ then 
${\mathcal{U}}_{L} (x_{1} x_{2} x_{3}) = \{\,x_{2}^{j} \mid j < d  \,\}.$

\begin{theorem2}
The non-commutative initial ideal  $\,in_{\precc }(J) \,$
is minimally generated by the set
$\, \{\, X_{i} X_{j} \,\mid \, j<i \,\}$ 
together with the set
\begin{equation*}\{\delta ( u \cdot m) \,\mid \,
m \ \hbox{is a generator of} \
in_{\prec }(I) 
\
\hbox {and} 
\
u \in {\mathcal{U}}_{in_{\prec }(I)}(m)\}.
\end{equation*}
In particular, a 
minimal $\precc $-Gr\"{o}bner basis for $J$ consists of
$\{ X_{i} X_{j} - X_{j} X_{i} : j < i \} $ together
with the elements $\delta ( u \cdot f) $
for each polynomial $f$ in a minimal $\prec $-Gr\"{o}bner basis for $I$
and each monomial $\,u \, \in \,
{\mathcal{U}}_{in_{\prec }(I)}(in_{\prec }(f))$.
\end{theorem2}


\begin{proof} 
We first argue that a non-commutative monomial 
$\,M \,= \, X_{i_{1}}  X_{i_{2}} \cdots X_{i_{r}} \, $
lies  in $\, in_{\precc }(J)$ if and only if
its commutative image
$\, \gamma (M) $ is in $\, in_{\prec }(I) \,$ or $i_{j}>i_{j+1}$ for some $j$.
Indeed, if $i_{j}>i_{j+1}$ then
$ M \in in_{\precc }(J)$ because
$X_{s}X_{t}-X_{t}X_{s}\in J$ has initial term $X_{s}X_{t}$ with $s>t$.
If on the contrary $i_{1}\leq \dots \leq i_{r}$ but 
$\,\gamma (M) \in in_{\prec } (I) \,$, then there exists
$\,f \in I \,$ with $in_{\prec }(f) = \gamma (M)$.
The non-commutative polynomial $\,F = \delta (f)\,$
satisfies $\,in_{\precc }(F) = M \,$.
The opposite implication follows because $\gamma $ induces an isomorphism
$\,k[x]/I \,\cong \, k \langle X \rangle / \gamma ^{-1}(I)$.

Now let
$m' = u \cdot m$, where $\,m = x_{i_{1}} \cdots x_{i_{r}}\,$
is a minimal generator of $in_{\prec }(I)$ with
$i_{1} \leq \cdots \leq i_{r} $.
We must show that $\, \delta (u \cdot m)\,$ is a minimal generator 
of $\,in_{\precc }(J) \,$ if and only if $\, u \in {\mathcal{U}}_{in_{\prec }(I)}(m) $.

For the ``only if'' direction, suppose that $\,\delta (u \cdot m)\,$ 
is a minimal 
generator of $\,in_{\precc }(J) $. 
Suppose that $u$ contains the variable $x_{j}$.  We must have 
$j>i_{1}$, since else, taking $j$ minimal, we would have
$\,\delta (u \cdot m)\,=\, X_{j} \cdot \delta (\frac{u}{x_{j}}m) \,$. 
Similarly $j<i_{r}$.
Thus $ \, u \in k[x_{i_{1}+1},\ldots , x_{i_{r}-1}]$.
This implies $\,\delta (u \cdot m)\,=\,  
X_{i_{1}} \cdot \delta ( u \frac{ m}{x_{i_{1}}} ) \,= \,
\delta ( u\cdot \frac{ m}{x_{i_{r}}} )  \cdot X_{i_{r}} $.
Therefore neither
$\,\delta ( u\frac{m}{x_{i_{1}}} ) \,$ nor
$\,\delta ( u\frac{m}{x_{i_{r}}} ) \,$ 
lies in $\,in_{\precc }(J)$, and hence neither
$\, u\frac{m}{x_{i_{1}}}\,$ nor
$\,u\frac{m}{x_{i_{r}}} \,$ lies in $\, in_{\prec }(I)$.

For the ``if'' direction we reverse the last few implications.
If $\, u \in {\mathcal{U}}_{in_{\prec }(I)} (m) \,$ then neither
$\,\delta ( u \frac{ m}{x_{i_{1}}} ) \,$ nor
$\,\delta ( u \frac{ m}{x_{i_{r}}} ) \,$ 
lies in $\,in_{\precc }(J)$, and therefore
 $\,\delta (u \cdot m)\,$ is a minimal generator of $\,in_{\precc }(J)$.
\end{proof}


\section*{3. Finiteness and lifting of non-commutative Gr\"{o}bner bases}

We maintain the notation described above. Recall
 that for 
a prime number $p$ the {\em Gauss order\/} on the natural numbers is
described by 
\begin{equation*}s\leq _{p}t \quad \text{if} \quad {\binom{t}{s}}\ \not \equiv \ 0 \ 
\, (\text{mod } p). \end{equation*}
We write $\,\leq _{0} \,= \,\leq \, $ for the usual order
on the natural numbers. A monomial 
ideal $L$ is called {\em $p$-Borel-fixed\/} if it satisfies the
following condition:
For each monomial generator $m$ of $L$,
if $m$ is divisible by $x_{j}^{t}$ but no higher power of 
$x_{j}$,
then $(x_{i}/x_{j})^{s} m\in L$ for all $i<j$ and $s\leq _{p} t$. 

\begin{theorem3} With notation as in Section 2:
\par
(a)  If $\,in_{\prec }(I)\, $ is 0-Borel fixed, then a minimal 
$\precc $-Gr\"{o}bner basis of $J$ is obtained by applying 
$\delta $ to a minimal $\prec $-Gr\"{o}bner basis of $I$
and adding commutators.

(b) If $\,in_{\prec }(I)\, $ is $p$-Borel-fixed for any $p$,
then $J$ has a finite $\precc $-Gr\"{o}bner basis.
\end{theorem3}


\begin{proof}
Suppose that the monomial ideal 
$\,L := in_{\prec }(I)$ is $p$-Borel-fixed for some $p$.
Let $m = x_{i_{1}} \cdots x_{i_{r}}$ be any generator of
$L$, where $i_{1} \leq \cdots \leq i_{r}$, and let
$x_{i_{r}}^{t}$ be the highest power of $x_{i_{r}}$ dividing $m$.
Since $t\leq _{p} t$  we have $x_{l}^{t} m/x_{i_{r}}^{t}  \in L$
for each $l<i_{r}$. This implies
$\,x_{l}^{t} m/x_{i_{r}} \in L$ for $l < i_{r}$, and
hence every monomial  $u \in {\mathcal{U}}_{L}(m)$
satisfies $deg_{x_{l}} (u) < t$ for $i_{1} < l < i_{r}$.
We conclude that ${\mathcal{U}}_{L}(m)$ is a finite set.
If $p=0$ then   ${\mathcal{U}}_{L}(m)$ consists of $1$ alone,
since $\,x_{l} m/x_{i_{r}}  \in L \,$ for all $l < i_{r} $.
Theorem 3.1 now follows from Theorem 2.1.
\end{proof}


\begin{proof}[Proof of Corollary 1.1]
We apply Theorem 3.1 together with
the following results, due to Galligo, Bayer-Stillman and Pardue, 
which can be found in \cite[Section 15.9]{Ei}:
if the field $k$ is infinite, then after
a generic change of variables, the initial ideal of $I$ with 
respect to any order $\prec $ on $k[x]$ is fixed under the
Borel group of upper triangular matrices.
This implies that $in_{\prec }(I)$ is $p$-Borel-fixed in characteristic $p\geq 0$
in the sense above. If the characteristic of $k$ is $0$ and $I$ is homogeneous
then, taking the reverse lexicographic
order in generic coordinates, we get a Gr\"{o}bner basis whose
maximal degree equals the regularity of $I$.  \end{proof}

We call the monomial ideal  $L$ {\em squeezed} if
$\, {\mathcal{U}}_{L}(m)\,= \{ 1 \} \,$ for all generators $m $ of $ L$
or if, equivalently, 
$m=x_{i_{1}}\cdots x_{i_{r}}\in L $ and $i_{1}\le \dots \le i_{r}$
imply $\, x_{l}\frac{m}{x_{i_{1}}}\in L \, $
or $\,x_{l}\frac{m}{x_{i_{r}}}\in L \, $
for every index $l$ with $i_{1} < l < i_{r}$.
Thus Theorem 2.1 implies that a minimal $\prec $-Gr\"{o}bner basis 
of $I$ lifts to a Gr\"{o}bner basis 
of $J$ if and only if the initial ideal $in_{\prec }(I)$ is squeezed. 
Monomial ideals that are 0-Borel-fixed, and more generally
stable ideals (in the sense of \cite{EK}), are squeezed. 
Squeezed ideals appear naturally in algebraic combinatorics:

\begin{theorem4}
A square-free monomial ideal $L$ is squeezed if and
only if the simplicial complex
associated with $L$ is the complex 
of chains in a poset.
\end{theorem4}


\begin{proof}
This follows from Lemma 3.1 in \cite{PRS}.
\end{proof}


\section*{4. Examples in characteristic $p$}

Over a finite field Corollary 1.1 fails even for very simple ideals:

\begin{definition1} Let $k$ be a finite field and $n=3$.
If $I$ is the principal ideal generated by the product of all
linear forms in $k[x_{1},x_{2},x_{3}]$, then
$\gamma ^{-1}(I)$ has no finite Gr\"{o}bner basis, even
after a linear change of variables. 
\end{definition1}


\begin{proof}
The ideal $I$ is invariant under all linear changes of variables.
The $\precc $-Gr\"{o}bner basis for $J$ is computed by Theorem 2.1,
and is infinite. That no other monomial
order on $k \langle X \rangle $ yields a finite
Gr\"{o}bner basis can be shown by direct computation as in the example
in the second paragraph of the introduction.
\end{proof}


Sometimes in characteristic $p>0$ no Gr\"{o}bner basis for a commutative algebra
can be lifted to a
non-commutative  Gr\"{o}bner basis, even after a change of variables:

\begin{definition2} Let $k$ be an infinite field of 
characteristic $p>0$, and consider the Frobenius power 
\begin{equation*}L \,\,:= \,\,\bigl ( \,(x_{1},\ x_{2},\ x_{3})^{3} \bigr )^{[p]}
\quad \subset \quad k[x_{1},x_{2},x_{3}]
\end{equation*} 
of the cube of the maximal ideal in $3$ variables.
No minimal Gr\"{o}bner basis of $L$ lifts to a Gr\"{o}bner basis
of $\gamma ^{-1}(L)$, and this is true even after any linear
change of variables.
\end{definition2}


\begin{proof} The ideal $L$ is invariant under linear changes of 
variable, so it suffices to consider $L$ itself.
Since $L$ is a monomial ideal, it is its own initial ideal,
so by Corollary 3.2 it suffices to show that $L$ is not squeezed,
that is, that neither
$\, x_{1}^{p-1} x_{2}^{p+1} x_{3}^{p}\,$ nor $\,x_{1}^{p} x_{2}^{p+1} x_{3}^{p-1} \,$ 
is in $L$. This is obvious, since the power of each variable 
occurring in a generator of $L$ is divisible by $p$ and has total
degree $3p$.
\end{proof}


\bibliographystyle{amsalpha}
\begin{thebibliography}{PRS}

\bibitem[An]{An} D.~Anick, {\em On the homology of associative algebras},
Transactions Amer.~Math.~Soc. {\bf 296} (1986), 641-659. \MR{87i:16046}



\bibitem[AR]{AR} D.~Anick, G.-C.~Rota, {\em Higher-order syzygies for the
bracket algebra and for the ring of coordinates of the Grassmannian},
Proc.~Nat.~Acad.~Sci.~U.S.A. {\bf 88} (1991), 8087--8090. \MR{92k:15058}



\bibitem[Ba]{Ba}
J.~Backelin, {\em On the rates of growth of the homologies of Veronese subrings.}, Algebra, algebraic topology
and their interactions (Stockholm, 1983),  
Lecture Notes in Math. 1183, Springer-Verlag, NY, 1986, p.~79--100.
\MR{87k:13042}



\bibitem[Ei]{Ei} D.~Eisenbud, {\em Commutative Algebra With a View Toward
Algebraic Geometry}, Springer-Verlag, NY, 1995. \MR{97a:13001}



\bibitem[EK]{EK} S.~Eliahou, M. Kervaire, {\em Minimal resolutions of some
monomial ideals}, {\sl Journal of Algebra} {\bf 129} (1990) 1--25.
\MR{91b:13019}



\bibitem[Mo]{Mo} T. Mora, {\em An introduction to commutative and
non-commutative Gr\"{o}bner bases}, Theoretical Computer Science {\bf 134}
(1994), 131--173. \MR{95i:13027}\pagebreak



\bibitem[PRS]{PRS} I. Peeva, V. Reiner and B. Sturmfels, {\em How to shell a
monoid}, preprint, 1996.



\bibitem[Sh]{Sh}
J. B. Shearer, {\em A graded algebra with a nonrational Hilbert series}, J.
Alg. 62 (1980), 228--231. \MR{81b:16002}
\end{thebibliography}

\end{document}
