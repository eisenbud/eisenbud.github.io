% **************************************************************************
% * The TeX source for AMS journal articles is the publisher's TeX code    *
% * which may contain special commands defined for the AMS production      *
% * environment.  Therefore, it may not be possible to process these files *
% * through TeX without errors.  To display a typeset version of a journal *
% * article easily, we suggest that you either view the HTML version or    *
% * retrieve the article in DVI, PostScript, or PDF format.                *
% **************************************************************************
% Author Package file for use with AMS-LaTeX 1.2
%\controldates{10-APR-1996,10-APR-1996,10-APR-1996,23-APR-1996}
\documentclass{bull-l}
\usepackage{graphicx} 
%\issueinfo{33}{3}{JUL}{1996}

\usepackage{amssymb}
\usepackage{upref}
\newcommand{\Ext}{\operatorname{Ext}}
\newcommand{\Ann}{\operatorname{Ann}}
\newcommand{\Hom}{\operatorname{Hom}}
\newcommand{\MM}{\operatorname{m}}
\newcommand{\codim}{\operatorname{codim}}
\newcommand{\supp}{\operatorname{supp}}
\newcommand{\Sym}{\operatorname{Sym}}

\newtheorem{prop}{Proposition}
\newtheorem{thm}[prop]{Theorem}
\newtheorem{cor}[prop]{Corollary}

\newtheoremstyle{pplain}%
{}%                                   leave usual amount of space above
{}%                                   leave usual amount of space below
{\itshape}%                                   text is italic font
{}%                                   indent the head 0pt
{\bfseries}%                          print the heading bold
{.}%                                  print  .  at the end of the head
{.5em}%                               leave .5em space after the head
{%
\thmname{#1}%                         print the theorem name first
\thmnumber{ CB#2}%                     print theorem number second 
                                      % Note, space before open parens; no space
\thmnote{ {\mdseries(#3)}}%   print any theorem note medium series, in parens
}

\theoremstyle{pplain}
\newtheorem{thmcb}{Theorem}
\newtheorem{concb}[thmcb]{Conjecture}

\theoremstyle{definition}
\newtheorem*{dfn}{Definition}

%\newtheorem{thmcb}{Theorem CB}

\begin{document}

\title{Cayley-Bacharach theorems and conjectures}
\author[D. Eisenbud]{David Eisenbud}
\address {Department of Mathematics, Brandeis University, Waltham, Massachusetts
02254-9110} 
\email{eisenbud@math.brandeis.edu}
\author[M. Green]{Mark Green}
\address{Department of Mathematics, University of California, Los Angeles, Los
Angeles, California 90095-1555}
\email{mlg@math.ucla.edu}
\author[J. Harris]{Joe Harris}
\address{Department of Mathematics, Harvard University, Cambridge,
Massachusetts 02138-2901}
\email{harris@abel.math.harvard.edu}
\subjclass{Primary 14N05, 14H05, 14-02; Secondary 13-03, 13H10}
\date{March 24, 1995, and, in revised form, November 3, 1995}
\begin{abstract}
A theorem of Pappus of Alexandria, proved in the fourth century A.D., began a
long development in algebraic geometry.  In its changing expressions one can
see reflected the changing concerns of the field, from synthetic geometry to
projective plane curves to Riemann surfaces to the modern development of
schemes and duality.  We survey this development historically and use it to
motivate a brief treatment of a part of duality theory.  We then explain one of
the modern developments arising from it, a series of conjectures about the 
linear conditions imposed by a set of points in projective space on the forms
that vanish on them. We give a proof of the conjectures in a new special case.
\end{abstract}

\maketitle

\tableofcontents
%\contentsline{section}
% {\numberline{}{Introduction}}{}
% {\numberline {Part I:}The past}{}
%\contentsline{subsection} 
% {\numberline{1.1}Pappus, Pascal, and Chasles}{}
% {\numberline{1.2}Cayley and Bacharach}{}
% {\numberline{1.3}The twentieth century}{}
% {\numberline{1.4}Proof of the final Cayley-Bacharach Theorem}{}
%\contentsline{section}
% {\numberline{Part II:}The future?}{}
%\contentsline{subsection} 
% {\numberline{2.1}Cayley-Bacharach Conjectures}{} 
% {\numberline{2.2}A proof of the conjecture for $r\le 7$}{}
%\contentsline{section}
% {\numberline{}References}{}



\section*{Introduction}
Suppose that $\Gamma$ is a set of $\gamma$ distinct points in $\mathbb{R}^n$
(or $\mathbb{C}^n)$.  In fields ranging from applied mathematics (splines and 
interpolation) to transcendental numbers, and of course also in algebraic
geometry, it is interesting to ask about the polynomial functions that vanish
on $\Gamma$.  If we substitute the coordinates of a point $p$ of $\Gamma$ for
the variables, then the condition that a polynomial $f$ vanishes at $p$ becomes
a nontrivial linear condition on the coefficients of $f$.  Thus the vanishing
of $f$ on $\Gamma$ is ensured by $\gamma$ linear conditions on the coefficients
of $f$.  These $\gamma$ conditions are linearly independent when applied to the
space of $\mathbb{R}[x_1,\dots,x_n]$ of all polynomials because, as is easily
seen, there is a polynomial vanishing at all but any given point of $\Gamma$.

However, we are usually interested in some finite-dimensional subspace of 
\linebreak  
$\mathbb{R}[x_1,\dots,x_n]$, typically the space of polynomials of degree at
most a given number $d$.  In this case the $\gamma$ conditions are generally
not independent.  As a trivial example, consider the three conditions imposed by
three collinear points in $\mathbb{R}^2$, and take $m=1$.  A linear polynomial
vanishing on any two of the points vanishes on the line joining them and thus
automatically vanishes on the third point.  Thus the three points impose only
two independent conditions on the polynomials of degree $\le 1$.  In general,
if $\lambda$ of the $\gamma$ conditions imposed by $\Gamma$ suffice to imply
all of them, and $\lambda$ is the least such number, then we say that $\Gamma$
imposes $\lambda$ independent conditions on polynomials of degree $\le m$. 
Since $\lambda\le \gamma$, and the natural first estimate for $\lambda$
is that it should be ``near'' $\gamma$, we concentrate on the
difference $\gamma-\lambda$, the ``failure of $\Gamma$ to impose independent 
conditions on polynomials of degree $\le m$''.  There are of course many
variants of the question.  Perhaps the most basic (and useful) is to take the 
points in projective space and to ask about homogeneous forms of degree $m$
instead of polynomials of degree at most $m$.

The Cayley-Bacharach Theorem, in its classical form, may be seen as a result
about the number of independent  conditions imposed on polynomials of given
degree by certain sets of points in the plane.  The idea is that if the points
lie on some algebraic plane curve $X$ of low degree, then the number of 
conditions imposed by the points can be related to the geometry of $X$.  The
result has a long and interesting history, starting with a famous result by
Pappus of Alexandria, proved in the fourth century A.D.    As the methods and
substance of algebraic geometry have changed over the years, the result has
been successively generalized, improved, and reinterpreted, and this
development continues today.

The first part of this paper is purely expository: in it, we shall trace the
evolution of the Cayley-Bacharach Theorem.  We give altogether nine versions of
the result beginning with Pappus's Theorem and continuing with results of
Pascal, Chasles, Cayley, and Bacharach.  The modern versions of the
Cayley-Bacharach Theorem are many; the ideas used have to do with Gorenstein
rings and are due to Macaulay, Gorenstein, Serre, and Bass.  We shall give a
proof of Chasles's version of the theorem using the methods of projective
geometry from the middle of the nineteenth century similar to those used 
by Chasles,
a  proof of Bacharach's version using the classical algebraic geometry of 
linear series on curves in the style of the late nineteenth century, and the
current proof by commutative algebra.   This last connects the geometric
statement of the theorem to the Gorenstein property of the polynomial ring; the
Cayley-Bacharach Theorem is finally part of duality theory, and Gorenstein
rings seem to intervene in all modern treatments of duality in algebraic
geometry.  (A little warning about our style of history is in order:  We have
NOT preserved the terminology and ``look'' of the old works; rather, we try to
interpret what the mathematicians of the last century were doing in modern
terms.  We realize that such ``interpretations'' are often more accurately
described as ``fictions''.  Our excuse is that our purpose is didactic rather
than truly historical.)

Our goal in this part of the paper is to show, in this example, how
the ideas used in the modern treatment of the Cayley-Bacharach Theorem flow
from classical techniques.  In particular we introduce some of the basic
developments of modern commutative algebra---the Cohen-Macaulay and Gorenstein 
properties, the Ext functor and the like---in the context of a concrete
geometric problem, which may serve to motivate and explain them.

One difficulty we have had in writing this report is that while the beginning
of our story is accessible with only the most elementary considerations of
curves in the plane, the latter part requires much more advanced ideas; in
short, the level is uneven.  There are really three distinct stages in the
proofs that we describe.  The first, as we have said, requires almost nothing
but the definitions of projective space and polynomials.  The second uses the
machinery of linear systems on curves, and ultimately the Riemann-Roch theorem.
We have given a fairly complete but very quick description of what we need, so
that the exposition is almost self-contained.  The third part uses the language
of commutative algebra and  schemes.   
We have tried to provide elementary definitions and have included
proofs of results that are certainly well known to experts and accessible in
advanced textbooks rather than referring the reader elsewhere.  In particular
we have tried to define and motivate the ideas of commutative algebra that are
necessary to understand the first properties of Gorenstein rings.  The second
and third levels are treated independently---the reader can skip either of them
without losing logical continuity.

When reviewing the past, it is always tempting to try to predict the future. 
In the second part of this paper, we propose a possible next step in
Cayley-Bacharach theory, in the form of a conjecture.  This conjecture is
itself a special case of a more general conjecture on the Hilbert functions of 
points (or, more generally, zero-dimensional schemes) in projective space, which
is described in \cite{EGH1} and \cite{EGH2}.

To understand what this conjecture says, observe that the earliest forms of the
Cayley-Bacharach Theorem may be interpreted as saying that certain sets of 
points $\Gamma$ impose independent conditions on forms of certain degrees $d$. 
Later forms compute the number of independent conditions in terms of the number
of conditions imposed by some related set of points $\Lambda$.  One problem in
using these statements is that one may know no more about $\Lambda$ than about
$\Gamma$!   Thus it is interesting to return and ask what sets of points 
actually impose independent conditions on forms of degree $d$.  Our conjectures
are related to this question.  To understand the form they take, note first
that if a set of points $\Gamma$ fails to impose independent conditions on
forms of degree $d$, then any larger set of points fails too.  Thus it is
interesting to ask about the minimal sets of points that fail to impose
independent conditions on forms of degree $m$.  In the classical 
Cayley-Bacharach setting the points in question are always given as a subset of
the intersection of two curves of given degrees without common components.  Our 
conjectures correspondingly concern subsets of sets of points that are
intersections of $n$ hypersurfaces of given degrees $d_1,\dots,d_n$ in
$\mathbb{P}^n$, whose intersection contains no curves.  Conjecturally they give
the smallest number of points in such an intersection that can fail to impose
independent conditions on forms of given degree $m$.

We have succeeded in verifying this conjecture in a number of cases; we
conclude the paper by giving the proof in one of these cases.

There is another direction in which current activity is extending the 
Cayley-Bacharach Theorem: Davis, Geramita, Robbiano, Kreuzer, and others have
defined ``Cayley-Bacharach schemes'' and studied their properties.  For an idea
of what is going on, the reader may consult Geramita, Kreuzer, and Robbiano
\cite{GKR}.  We have  excluded this material only because nine versions of the 
theorem seemed enough, and our purpose of introducing some of the commutative 
algebra associated with Gorenstein rings was already served by what is here.

In what follows we have used the term \emph{plane curve} always for algebraic
plane curve.  The field of definition is actually unimportant.  Of course it
was taken as $\mathbb{R}$ originally, and later as $\mathbb{C}$, and the naive
reader would do well to think of these cases, but the expert will have no
difficulty in adapting our arguments to the general case.

\section*{Part I: The past} 
\subsection*{{\rm 1.1.} Pappas, Pascal, and Chasles}
The first case of the Cayley-Bacharach Theorem to have been discovered is
literally ancient: it appears as Proposition 139 of Book VII of the
\emph{Mathematical Collection} of Pappus of Alexandria in the fourth century
A.D. (see Coxeter \cite{Co}).

\begin{thmcb}[Pappus's Theorem] \label{thmcb1}
Let $L$ and $M$ be two lines in the plane.  Let $p_1,p_2$, and $p_3$ be
distinct points of $L$, and let $q_1,q_2$, and $q_3$ be distinct points of $M$,
all distinct from the point $L\cap M$.  If for each $j\ne k\in\{1,2,3\}$ we let
$r_{jk}$ be the point of intersection of the lines $\overline{p_jq_k}$ and $
\overline{p_kq_j}$, then the three points $r_{jk} $ are collinear \textup{(}see
Figure \textup{1)}.
\end{thmcb}

\begin{figure}
\includegraphics{bull666l-fig-1}\label{fig:one}
\caption{}
\end{figure}

Pappus's original proof used, in Coxeter's words, ``a laborious development of
Euclid's methods.''  It is interesting to note that some notion of the
projective plane (not formally introduced until very much later) is necessary if
the statement we have given of Pappus's Theorem is to be true in all cases: It
is perfectly possible to choose the points so that the lines
$\overline{p_1q_2}$ and $\overline{p_2q_1}$ are parallel, in which case we
would say that the ``point'' $r_{12}$ is on the line at infinity; the
appropriate statement can be made in nonprojective terms by saying that in this
case the line through the points $r_{13}$ and $r_{23}$ is parallel to the lines
$\overline{p_1q_2}$ and $\overline{p_2q_1} $.  A still more degenerate case
occurs when both the pairs of lines $\overline{p_1q_2}, \overline{p_2q_1}$ and
$\overline{p_1q_3},\overline{p_3q_1}$ are parallel; then the statement is that
all three of the points $r_{jk}$ lie on the line at infinity or, in elementary
terms, that the last pair of lines---$\overline{p_2q_3},
\overline{p_3q_2}$---is parallel too.  It may be seen from this simple example
what a simplification the projective plane introduced!  We shall not give a
separate proof of Pappus's Theorem, since it is an immediate special case of
the following theorem.

In the early part of the seventeenth century there was a new interest in
geometry.  It included the geometry of perspective, which in the hands of
Desargues became the geometry of projective space: Desargues's fundamental work
of 1639 formally introduced the points at infinity and the line made from them.
There was also a new reason to be interested in the geometry of conics, coming
from Kepler's amazing discoveries about the orbits of planets in 1609.  The
second avatar of the Cayley-Bacharach Theorem that appeared is a famous result
of Pascal (again, see Coxeter \cite{Co}) that built on both these developments. 
The statement was for Pascal a fundamental tool, by means of which he proved
many properties of conics.  It was published under the title ``Essay Pour Les
Coniques'' in a \emph{petit placard en forme d'affiche}---that is, a one page
handbill---in 1640 and is reproduced in Struik \cite{S}.  The proof was
not included.  Leibniz reports in a letter in 1676 that he studied it when he
visited Paris, so we know that Pascal had a proof, but the manuscript has
disappeared.

\begin{figure}[t]
\includegraphics{bull666l-fig-2}
\caption{}\label{fig:two}
\end{figure}

\begin{thmcb}[Pascal's Theorem] \label{thmcb2}
If a hexagon is inscribed in a conic in the projective plane, then the opposite
sides of the hexagon meet in three collinear points.
\end{thmcb}

The line at infinity of the projective plane is again required, as the
reader may see by considering the case of a regular hexagon inscribed in a
circle.  Pascal was immediately inspired by Desargues's work and seems to have
been quite aware of this situation.\footnote{Pascal wrote in his \emph{Essai}: 
``$\dots$ M.\ Desargues of Lyons [is] one of the great minds of this time, and
one of the best versed in mathematics, particularly in conics, whose writings
on this subject, although few in number, give abundant proof of his knowledge to
those who seek for information.  I should like to say that I owe the little
that I have found on this subject to his writings, and that I have tried
to imitate his method, as far as possible, in which he has treated the 
subject.\ $\dots$'' (translation from Struik \cite{S}).}

A little notation will make the relation of Pascal's Theorem to Pappus's 
Theorem more transparent: Let $C$ be the plane conic, and let $p_1,p_2,p_3$ and
$q_1,q_2$, and $q_3$ be six distinct points on $C$.  These points determine an
inscribed hexagon if we join adjacent points in the order
$p_1,q_2,p_3,q_1,p_2,q_3,p_1$.  This order is chosen so that the ``opposite
sides'' are the pairs of lines $\overline{p_jq_k}$ and $\overline{p_kq_j}$
where $i\ne j\in\{1,2,3\}$.  If we denote by $r_{jk}$ the point of intersection
of these opposite sides, then the assertion is that the three points $r_{jk}$
are collinear, as in Figure \ref{fig:two}.


Pappus's Theorem becomes a special case of Pascal's Theorem if we allow the
word ``conic'' to refer not only to an irreducible conic such as an ellipse or
hyperbola but also to the union of two straight lines $L$ and $M$, which is a
kind of limiting position of a hyperbola; the six points $p_i$ and $q_i$ must
be taken with all the $p_i$ on one line and all the $q_i$ on the other, and all
distinct from the intersection of the two lines, or the statement of Pascal's 
Theorem becomes indeterminate (two of the lines $\overline{p_jq_k}$ and $
\overline{p_kq_j}$ coincide)  or trivial (all the points lie on one of $L$ and
$M)$.  From the algebraic point of view, this is natural because then a
``conic'' is simply the zero locus of any nonzero quadratic polynomial,
including products of linear forms.  The proof we shall give below applies to
all such cases, though we could also deduce Pappus's Theorem from Pascal's by
``continuity'', an important if slightly imprecise method of the early
projective geometers, arguing that the six lines $\overline{p_jq_k}$ and $
\overline{p_kq_j}$ appearing in Pappus's Theorem are the limiting positions of
the sides of a hexagon inscribed in a conic as in Pascal's Theorem.

\begin{figure}
\includegraphics{bull666l-fig-3}
\caption{Pascal's construction of the tangent line (at $p_1=q_1)$ to a conic.}
\label{fig:three}
\end{figure}

One of Pascal's many applications of his theorem concerns the ``degenerate''
case $p_1=q_2$: In this case he interprets the line $\overline{p_1q_2}$ of the
``hexagon'' to be tangent to the conic at $p_1=q_2$; see Figure
\ref{fig:three}.
Since the theorem constructs two further lines that meet at a point of the line
$\overline{p_1q_2}$, namely, $\overline{p_2q_1}$, and the line through $r_{13}$
and $r_{23}$, it can be used to construct the tangent to the conic at $p_1$. 
This application reflects a modern concern for what happens in the degenerate
cases of geometric theorems; the justification for treating the degenerate
chord as a tangent line would have been phrased by Pascal in terms of
continuity, but would now be justified in the language of schemes.  In all the
subsequent Cayley-Bacharach theorems that we shall discuss, such degenerate
cases are allowed, and the proofs that we shall give in the most general
cases explicitly include them.

A third development in the first half of the seventeenth century put the work
on the geometry of projective conics into the shade: this was the introduction
by Descartes of coordinate geometry.  Perhaps because of the excitement
generated by this development, along with the infinitesimal calculus, geometry
moved in other directions, and it was about two hundred years before Pascal's 
Theorem was fundamentally reexamined and extended.

To understand the nature of the extension, note that in the transition from
Pappus's Theorem to Pascal's Theorem the two lines $L$ and $M$ are confounded
into a single conic and the two separate groups of points $\{p_1,p_2,p_3\}$ and
$\{q_1,q_2,q_3\}$ are seen to be an arbitrary subdivision of a single group of
six points on the conic, the vertices of the hexagon.  In both theorems we
construct three further points $r_{ij}$ and a line on which they lie.  It was
the beautiful insight of Michel Chasles, published in his \emph{Trait\'e des
Sections Coniques} \cite{Ch},  that this last line could be combined with the
conic in Pascal's Theorem and confounded into a cubic plane curve and that the
group of six points and the remaining three could be taken to be any nine
points on the cubic, with none distinguished from another.  It seems plausible
that coordinates for the projective plane, which had recently been defined by 
M\"obius and (in a way closer to the modern one) by Pl\"ucker, were necessary
before arbitrary plane curves of higher degree, such as cubics, became
attractive.  In any case, Chasles's result is both simpler and more powerful
than Pascal's.  It is the result that commonly (though incorrectly) goes under
the name Cayley-Bacharach.

\begin{thmcb}[Chasles] \label{thmcb3}
Let $X_1,X_2\subset \mathbb{P}^2$ be cubic plane curves meeting in nine points
$p_1,\dots,p_9$.  If $X\subset \mathbb{P}^2$ is any cubic containing
$p_1,\dots,p_8$, then $X$ contains $p_9$ as well.
\end{thmcb}

Pascal's Theorem follows if we take the cubics $X_1$ and $X_2$ to be the
triangles formed by alternate edges of the hexagon,  $X_1=\overline{p_1q_2}
\cup\overline{p_2q_3}\cup \overline{p_3q_1}$ and $X_2=\overline{p_1q_3}\cup 
\overline{p_2q_1}\cup \overline{p_3q_2}$ in Figure \ref{fig:three}, and take
$X$ to be the union of $C$ with the line $\overline{r_{12}r_{13}}$. The point
$r_{23}$ also lies on $X_1\cap X_2$, so Chasles's Theorem says it must lie on
the union of $C$ and $\overline{r_{12}r_{13}}$.  Since it does not lie on $C$,
it must lie on $\overline{r_{12}r_{13}}$; that is, $r_{23}$ is collinear with
$r_{12}$ and $r_{13}$; as required.  The theorem of Pappus is the degenerate
case in which $X$ is the union of $L,M$, and $\overline{r_1r_2}$.

Chasles's Theorem is the last in the sequence of theorems stated here that can
be proved by purely elementary means, i.e., without invoking the residue 
theorem or some other form of duality theory.  We shall prove it without making
any assumptions about the smoothness or irreducibility of $X_1$ and
$X_2$---that is, they can be the zero loci of any homogeneous cubic polynomials
$F_1,F_2$ on $\mathbb{P}^2$.  All we shall need is the classical theorem of
B\'ezout: Plane curves of degrees $d$ and $e$ cannot meet in more than $d\cdot
e$  points unless they have a component in common (that is, unless the
equations defining them have a common factor).  A refined form of B\'ezout's
Theorem says that if the curves have no common component and meet in $d\cdot e$
distinct  points, then the curves must each be nonsingular at these points and
must meet transversely.  In particular the hypotheses of Theorem CB\ref{thmcb3} 
rule out the possibility of $X_1$ or $X_2$ having multiple components, that is,
$F_1$ or $F_2$ having repeated factors.  (In fact, the later versions of
Cayley-Bacharach will apply to give very interesting and nontrivial statements
even in these cases.)

\begin{proof}[Proof of Chasles's Theorem]
We introduce a piece of classical terminology: If $\Gamma=\{p_1,\dots,p_m\}
\subset \mathbb{P}^2$ is any collection of $m$ distinct points, we shall say
that $\Gamma$ \emph{imposes} $l$ \emph{conditions} on polynomials of degree $d$
if in the vector space of polynomials of degree $d$ on $\mathbb{P}^2$ the
subspace of those vanishing at $p_1,\dots,p_m$ has codimension $l$, or 
equivalently if in the projective space of curves of degree $d$ the subspace of
those containing $\Gamma$ has codimension $l$. We denote the number of 
conditions imposed by $\Gamma$ on forms of degree $d$ by $h_\Gamma(d)$, and
call $h_\Gamma$ the \emph{Hilbert function} of $\Gamma$.

With this language, Chasles's Theorem is equivalent to the following statement:
If $\Gamma=\{p_1,\dots,p_9\}$ is the intersection of two plane cubics and $
\Gamma'=\{p_1,\dots,p_8\}$ is any subset omitting one point, then $\Gamma$ and
$\Gamma'$ impose the same number of conditions on cubics.  We shall actually
prove the stronger statement that $\Gamma$ and $\Gamma'$ each impose exactly
eight conditions on cubics; that is, that the eight points of $\Gamma'$ impose
independent conditions on cubics and the ninth point in $\Gamma$ imposes a 
condition dependent on these eight.

Part of this is obvious: the nine points of $\Gamma$ visibly fail to impose
independent conditions on cubics, since the 10-dimensional vector space of
cubic polynomials contains at least a two-dimensional subspace of polynomials
vanishing on $\Gamma$, namely, that spanned by the defining equations $F_1,F_2$
of $X_1$ and $X_2$.  The remaining ingredient of the proof is the ``only if''
part of the $d=3$ case of the following proposition, which for clarity we shall
state and prove for arbitrary $d$.
\renewcommand{\qed}{}
\end{proof}

\begin{prop} \label{prop:one}
Let $\Omega=\{p_1,\dots,p_n\}\subset \mathbb{P}^2$ be any
collection of $n\le 2d+2$ distinct points.  The points of $\Omega$ fail to
impose independent conditions on curves of degree $d$ if and only if either
$d+2$ of the points of $\Omega$ are collinear or $n=2d+2$ and $\Omega$ is
contained in a  conic.
\end{prop}

This line of thought leads to the numerical character of Gruson-Peskine and to
a theorem of Ellia-Peskine \cite{EP} giving a general criterion forcing a
subset of a set of points in the projective plane to lie on a curve of low
degree.

To complete the proof of Chasles's Theorem using Proposition \ref{prop:one}, it
suffices by what we have already said to show that if $\Gamma'$ consists of
eight of the nine points in which two plane cubics intersect, then $\Gamma'$
imposes independent conditions on cubics.  We take $\Omega=\Gamma'$, so $n=8$,
and $d=3$ in Proposition \ref{prop:one}.  By B\'ezout's Theorem, if $\Gamma'$ lay
on a conic $C$, then both $X_1$ and $X_2$ would have to contain (a component
of) $C$; and similarly if four or more points of $\Gamma$ lay on a line $L$,
then $X_1$ and $X_2$ would necessarily contain $L$.  Since we have supposed
that $X_1$ and $X_2$ have no component in common, this is impossible.

\begin{proof}[Proof of Proposition \ref{prop:one}]
The ``if'' direction of the proposition is easy: If $d+2$ of the points of $
\Omega$ lie on a line $L$, then by B\'ezout's Theorem any curve of degree $d$
containing $\Omega$ must contain $L$.  The subset of curves of degree $d$
containing $L$ has the same dimension as the set of curves of degree $d-1$, so
the codimension of the set of curves containing $L$ is only
$\binom{d+2}{2}-\binom{d+1}{2}=d+1$. The remaining $n-(d+2) $ points of
$\Omega$ impose at most $n-(d+2)$ conditions on curves of degree $d$, so we see
that $\Omega$ imposes at most $n-1$ conditions.  A similar argument, using the
fact that it is only $2d+1$ conditions to contain an irreducible conic, works
in the second case.

For the more serious ``only if'' direction, we do induction first on the degree
$d$ and second on the number $n$ of points.  By the induction hypothesis on
the number $n$ of points we may assume that any proper subset of $\Omega$ does
impose independent conditions on curves of degree $d$.  Thus the statement that
$\Omega$ itself fails to impose independent conditions amounts to saying that
\emph{any plane curve of degree $d$ containing all but one of the points of $
\Omega$ contains} $\Omega$.

To start the inductions, we note first that Proposition \ref{prop:one} is
trivial for $d=1$: any set $\Omega$ of $n\le 4$ points in the plane fails to
impose independent conditions on lines if and only if either $n=3=d+2$ and the
points of $\Omega$ are all collinear, or $n=4$.  Second, for arbitrary $d$, the
result is easy for $n\le d+1$: it suffices by the remark above to exhibit a
curve of degree $d$ containing all but one given point $p_n$ of $\Omega$, and
we may do this by taking the union of general lines  $L_i$ through $p_i$ for
$i=1,\dots,n-1$ and an arbitrary plane curve of degree $d-n+1$ not passing
through $p_n$.

We now take $d$ arbitrary and suppose that $n>d+1$.  Suppose first that
$\Omega$ contains $d+1$ points lying on a line $L$.  Assume that no further 
point of $\Omega$ lies on $L$, and let $\Omega'\subset \Omega$ be the
complementary set of $n-d-1\le d+1$ points of $\Omega$.  We claim that
$\Omega'$ must fail to impose independent conditions on curves of degree $d-1$;
otherwise, we could find a curve of $X$ degree $d-1$ containing all but any one 
point of $\Omega'$, and then the union $L\cup X$ would be a curve of degree $d$
containing all but one point of $\Omega$.  By induction $\Omega'$ must consist
of exactly $d+1$ points on a line $M$.  Thus either $L$ contains $d+2$ points
of $\Omega$ or $n=2d+2$ and $\Omega$ lies on the conic $L\cup M$.

Next suppose only that some line $L$ contains $l\ge 3$ points of $\Omega$.  By
the same argument as in the last paragraph, the remaining $n-l$ points of $
\Omega$ must fail to impose independent conditions on curves of degree $d-1$
and so must include at least $d+1$ collinear points.  We are thus back in the
case considered in the preceding paragraph.

We are now done except in the case where $\Omega$ contains no three collinear 
points.  Choose any three points $p_1,p_2,p_3\in \Omega$, and let $\Omega'$ be
the complement of these three.  If for any $i$ the points of
$\Omega'\cup\{p_i\}$ impose independent conditions on curves of degree $d-1$,
we are done: for then we can find a curve $C$ of degree $d-1$ containing $
\Omega'$ but not $p_i$, and the union of this curve and the line joining $p_j$
and $p_k$ is a curve of degree $d$ containing all but exactly one point of $
\Omega$.  Thus we may assume that $\Omega'\cup\{p_i\}$ fails to impose
independent conditions on curves of degree $d-1$.  Since it cannot contain
$d+1$ collinear points, we have by induction $n=2d+2$, and for each $i$ the set
$\Omega'\cup\{p_i\}$ lies on a conic $C_i$.  Note that in case $d=2$ we are
done, since six points fail to impose independent conditions on conics if and
only if they lie on a conic.  On the other hand, if $d\ge 3$, then $\Omega'$
contains at least five points, no three collinear, and so there can be at most
one conic containing $\Omega$.  Thus all the conics $C_i$ must be equal to a
single conic curve $C$, which then contains all of $\Omega$.
\end{proof}

There is a much quicker proof of Theorem CB\ref{thmcb3}, which exploits the
relation of the theorem to residue theory; since this relation belongs to the
next part of our story, we shall give it there.

\subsection*{{\rm 1.2.} Cayley and Bacharach}
Arthur Cayley, probably the most distinguished mathematician in this story,
does not play a glorious role in it.  Chasles's book appeared when he was
sixteen.  In 1843, when he was twenty-two, he published a note stating an
extension of Chasles's result to the case of intersections of curves of degree
higher than 3.  The basis of this extension was again the idea of counting 
conditions imposed by sets of points.  His first observation was that if curves
$X_1$ and $X_2$ of degrees $d$ and $e$ meet in a collection $\Gamma$ of $d\cdot
e$ points, then for any $k$ the number $h_\Gamma(k)$ of conditions imposed by $
\Gamma$ on forms of degree $k$ is independent of the choice of curves $X_1$ and
$X_2$; it can be written down explicitly as
\[h_\Gamma(k)=\binom{k+2}{2} -\binom{k-d+2}{2}-\binom{k-e+2}{2}+\binom
{k-d-e+2}{2}\]
where the binomial coefficient $\binom{a}{2}$ is to be interpreted as 0 if
$a<2$.  He went on to conclude that if $\Gamma'$ is any subset of $h_\Gamma(k)$ 
points of $\Gamma$, then a form of degree $k$ vanishing on the points of $
\Gamma'$ must vanish on $\Gamma$.  (The case $d=e=k=3$ is Chasles' Theorem.)

Both parts of Cayley's statement are rather remarkable.  The first part is
equivalent to a special case of the whole: any curve $X$ passing through all
the points of the intersection of $X_1$ and $X_2$ is defined by a polynomial
that is a linear combination (with polynomial coefficients) of the polynomials
defining $X_1$ and $X_2$.  That is, if $F=0,G=0$, and $H=0$ are the equations
of $X_1,X_2$, and $X$ respectively, then there exist polynomials $A$ and $B$
such that $H=AF+BG$.  This assertion was used by Cayley in the case where the
two curves meet transversely.  He apparently regarded it as evident; he applies
it without comment or reference.  But the statement is rather subtle: in modern
language, it says that the polynomial ring in three variables is
Cohen-Macaulay.  It was finally given a proof by Max Noether, first in the
special case Cayley used \cite[p.\ 314]{Ca} and then in general \cite{No} in a
paper devoted to filling this gap\footnote{Noether writes, ``In einer Reihe von
geometrischen und functionentheoretischen Arbeiten findet sich eine L\"ucke, die
das Folgende auszuf\"ullen bestimmt ist.''  (``In a series of geometric and 
function-theoretic papers there is a gap, which the following is intended to
fill.'')  It would be interesting to know just when and how, in between 1843 and
1873, it came to be considered a gap.}, under the name of the ``Fundamental
Theorem of Algebraic Functions'' (now often---and perhaps
unfortunately---called ``Max Noether's $AF+BG$ Theorem'').

Given Noether's Theorem and the fact that the dimension of the space of all
forms of degree $k$ in three variables is $\binom{k+2}{2}$, the formula for $h_
\Gamma$ given above is easy: it just says that the vector space of forms of
degree $k$ that can be written as $AF+BG$ is the sum of the dimension of the
space of possible $A$'s (of degree $k-d)$ and the space of possible $B$'s (of
degree $k-e)$, minus the dimension of the space of pairs $(A,B)$ such that
$AF+BG=0$.  Since $F$ and $G$ have no factors in common, $AF+BG=0$ can hold
only if $A$ is a multiple of $G$, and $B$ is minus the corresponding multiple
of $F$, and the multiplier has degree $k-d-e$.   Thus the dimension of the
space of these pairs is the dimension of the space of forms of degree $k-d-e$.

By the argument given in the proof of Chasles's Theorem above, the second part
of Cayley's statement amounts to a broad generalization of 
Proposition~\ref{prop:one}---but without any hypotheses!  What is necessary is
precisely the assertion that the points of the subset $\Gamma'$ impose
independent conditions on forms of degree $k$.  This would of course be true if
the points were general points.  Cayley does comment on this point.  He says, 
``$\dots$and though the $[\dots]$ points are not perfectly arbitrary, 
there appears
to be no reason why the relation between the positions of these points should
be such as to prevent'' the conditions from being independent.

Despite this ``appearance'', points may have their own reasons and can indeed
fail to impose independent conditions.  The conclusion of Cayley's Theorem is
simply false!  Perhaps the first example is the following: Let $L$ be a line in
the plane, and let $\Gamma'=\{p_1,p_2,p_3\}$ be three points on $L$.  Let $X_1$
and $X_2$ be two nonsingular cubic curves containing $\Gamma'$.  It is easy to
arrange that $X_1$ and $X_2$ meet in a collection of nine distinct points $
\Gamma$. We have $h_\Gamma(1)=3$ (that is, $\Gamma$ does not lie on any lines),
and thus if the points of $\Gamma'$ imposed independent conditions on lines,
every line containing them would contain all of $\Gamma$ (that is, there would
be no lines containing them).  This is of course nonsense: The points of $
\Gamma'$ do lie on the line $L$, and none of the other points of
$X_i$ lie on $L$, or $L$ would be a component of $X_i$.

What is needed is a way of measuring the dependence of the conditions imposed
by the points of $\Gamma'$. More powerful tools were soon to be available.  
Riemann's remaking of complex function theory led Brill and Noether to an
extensive theory of linear series on algebraic curves; Noether's Fundamental 
Theorem of Algebraic Functions was in fact one of the tools they developed for
 the purpose.  This theory was powerful enough to supply the missing term in 
Cayley's equation; the fundamental paper of Brill and Noether appeared in 1874,
and its tools were used by Bacharach to correct Cayley's error.  Bacharach's
work seems to have earned him a professorship in Erlangen.  He presented
his findings in 1881 in his inaugural dissertation.  Perhaps the problem with
Cayley's ``Theorem'' had been recognized for some time.  Bacharach published a
somewhat extended version of his results in \cite{Ba}: this paper contains the
subtle missing term, and we shall give his statement below.  In the case just
given it amounts to the observation that the fact that the points of $\Gamma'$
impose dependent conditions on lines is equivalent to the statement that the
``residual'' set $\Gamma'':=\Gamma-\Gamma'$ imposes dependent conditions on
conics; that is, the other six points of $\Gamma$ actually lie on a conic.

It may be illuminating to see the ideas of linear series applied first in the
simple case studied by Chasles; in fact, without any analysis or struggle, the
method shows that if two curves of any degree $d$ intersect a curve of degree
$e\ge 3$ in sets $\Gamma$ and $\Gamma'$ differing by at most one point, then
in fact $\Gamma=\Gamma'$.  We shall give the easy proof after developing the
ideas from the theory of curves on which it is based.

For simplicity, the reader may assume at this point that the ground field is
the complex numbers, although the expert will have no trouble adapting the
argument to the general case.  If $X\subset \mathbb{P}^2$ is a nonsingular 
algebraic plane curve of degree $d$, then by a \emph{divisor} $D$ on $X$ we
mean a formal linear combination, with integer coefficients, of the points of
$X$; by the \emph{degree} of $D$ we will mean the sum of the coefficients.  The
divisor is called effective if all the coefficients are nonnegative.  Any
rational function $f$ on $X$ determines a divisor (f), the divisor of its zeros
minus the divisor of its poles.  Divisors $D$ and $D'$ are said to be \emph{
linearly equivalent}, written $D\sim D'$, if they differ by the divisor of a
rational function.

The same notions can be defined for an arbitrary plane curve $X$ (and even for
more general curves), but some care is necessary.  First, if $X$ is irreducible
and singular and the divisors in question involve only smooth points, the most
interesting situation for us, we need only change the definition of linear
equivalence to require the rational function $f$ to be regular and nonzero at
the singular points of the curve, so that the linear equivalence takes place
entirely inside the smooth locus.  If $X$ is reducible, then in addition we
must require that the rational functions have only isolated zeros and poles on
the curve, not along a whole component.  If we wish to include the singular 
points as divisors in some way, there is an additional complication, one that
was not faced squarely in the nineteenth century: divisors must be interpreted
as what are now called \emph{Cartier} divisors.  For this final refinement see
any modern book on algebraic geometry, for example Hartshorne \cite{H}.

For example, if $C$ is any plane curve intersecting $X$ only in isolated smooth 
points of $X$, we define the \emph{divisor cut on} $X$ \emph{by} $C$ to be the
divisor $C\cdot X:=\Sigma a_ip_i$, where the $p_i$ are the points of
intersection of $C$ with $X$ and $a_i$ is the multiplicity of intersection of
$C$ with $X$ at $p_i$.  If $C$ and $C'$ are plane curves of the same degree not
containing $X$, defined by homogeneous polynomials $F$ and $F'$, the divisors
$C\cdot X$ and $C'\cdot X$ are linearly equivalent: their difference is the
divisor of the rational function $F/F'$ restricted to $X$.  In this context we
may give a refined version of the B\'ezout Theorem:

\begin{thm}[B\'ezout]\label{thm:two}
If $C$ is a plane curve of degree $e$ not containing $X$, the degree of
the divisor cut on $X$ by $C$ is $d\cdot e$.
\end{thm}

One of the central results proved by Brill and Noether was the following.  The
original version takes the curve $X$ to be irreducible, but the difference is
mostly a matter of how the definitions are formulated.

\begin{thm}[Restsatz: Brill-Noether \cite{BN}] \label{thm:three}
If $X$ is a plane curve, then the linear series cut on $X$ by plane curves of
any degree $d$ is complete\textup{:} that is, given a plane curve $C$ of degree
$d$ not containing any component of $X$ and a divisor $D$ linearly equivalent
to $C\cdot X$, there is a plane curve $C'$ of degree $d$ not containing any 
component of $X$ such that $C'\cdot X=D$.
\end{thm}

\begin{proof}  %[Proof of Proposition~\ref{prop:five}]
We need the following form of Max Noether's $AF+BG$ Theorem (which itself is a
consequence of Lasker's Unmixedness Theorem; see Theorem~\ref{thm:eight}
below): If a curve $Y$ contains $C\cdot X$, then the equation $H$ of $Y$ may be
written as $AF+BG$, where $F$ is the equation of $C$ and $G$ is the equation of
$X$.  It follows that the curve $Y'$ with equation $H-BG$ meets $X$ in the same
way $Y$ does; that is, $Y'\cdot X=Y\cdot X$, and $Y'$ contains $C$ as a 
component.

The hypothesis of Corollary~\ref{cor:five} asserts that there is a rational 
function $\varphi$ on $X$ whose divisor $(\varphi)$ satisfies $C\cdot
X-\Gamma=D+(\varphi)$. Adding $\Gamma$ to both sides, we reduce at once to the
case where $\Gamma=0$.  The rational function $\varphi$ may be expressed as
$P/Q$ where $P$ and $Q$ are homogeneous polynomials in three variables of the
same degree.  Let $Y$ and $Z$ be the curves defined in the plane by $P$ and $Q$
respectively.  We have $C\cdot X+Z\cdot X=D+Y\cdot X$.  Multiplying together
the equations of $C$ and $Z$, we get the equation of a curve $Z'$ such that
$Z'\cdot  X=D+Y\cdot X$.  In particular, $Z'$ contains $Y\cdot X$.  By Max
Noether's Theorem, we may replace $Z'$ by a curve $Z''$ containing $Y$ and such
that $Z''\cdot X=Z'\cdot X=D+Y\cdot X$.  If $S$ is the equation of $Z''$, then
since $Z''$ contains $Y$ we see that $P$ divides $S$.  If we write $C'$ for the
curve defined by $S/P$, which we may represent as $C'=Z''-Y$, then $C'\cdot
X=D$.
\end{proof}

Noether's Fundamental Theorem of Algebraic Functions mentioned above was a
fundamental tool in the proof of Theorem~\ref{thm:three} given by Brill and
Noether.  From a modern point of view, this is not surprising. 
Theorem~\ref{thm:three} is the statement that the homogeneous coordinate ring
of the curve $X$ has the Cohen-Macaulay property, which is equivalent to saying
that the polynomial ring itself has this property, and Noether's Fundamental 
Theorem is essentially equivalent to this assertion too!  See the discussion
after Theorem~\ref{thm:eight} for more information.

Given this, the proof of Chasles's Theorem CB\ref{thmcb3}, and even its
generalization given above, becomes trivial: If $X$ is a plane curve of degree
$e\ge 3$ and $C$ and $C'$ are plane curves of some degree $d$ meeting $X$
in divisors $D$ and $D'$ that differ by at most one smooth point, say,
$D-D'=p-q$, then $p$ and $q$ are linearly equivalent.  Choose a general line
$L$ through $p$; we will show that $L$ passes through $q$, and thus $p=q$.  By
the Restsatz there is some line $L'$ that cuts out the divisor  $L\cdot X-p+q$. 
Because $e\ge 3$, the divisor $L\cdot X-p$ already contains at least two
points, and thus spans $L$, so $L=L'$ as required.  (Chasles's Theorem is the
special case $d=e=3$.)

In fact a much stronger statement, still short of the full version of
Bacharach, can be derived by related methods and will play a role in the
sequel, so we pause to examine it.  

\begin{thmcb}  \label{thmcb4}
Let $X_1,X_2\subset \mathbb{P}^2$ be plane curves of degrees $d$ and $e$
respectively, meeting in a collection of $d\cdot e$ distinct points
$\Gamma=\{p_1,\dots,p_{de}\}$.  If $C\subset \mathbb{P}^2$ is any plane curve
of degree $d+e-3$ containing all but one point of $\Gamma$, then $C$ contains
all of $\Gamma$.
\end{thmcb}

In the terms of Bacharach's Theorem CB\ref{thmcb5}, this result is true because
any set $\Gamma'$ consisting of all but one point of $\Gamma$ really does
impose independent conditions on forms of degree $d+e-3$.

To apply the Brill-Noether theory in a naive form to this result, we shall have
to assume that one of the curves  $X_i$, say, $X_1$, is nonsingular.  (The
expert could avoid this hypothesis, either by using a more subtle version of 
Riemann-Roch or by using Bertini's Theorem: Since we are assuming  that $\Gamma$
consists of $d\cdot e$ distinct points, the refined form of B\'ezout's Theorem
shows that $X_1$ and $X_2$ meet transversely, and then a classic theorem  of
Bertini says that (if $d>e)$ we can find a nonsingular curve $X'_1$ that meets
$X_2$ in the same set of points $\Gamma$ by adding to the equation of $X_1$ an
appropriate multiple of the polynomial defining $X_2$.  But since later
versions of the theorem will be more general anyway, we shall not go further
into this here.)  This is in fact not the approach we shall adopt ultimately to
prove the final and most general version of the Cayley-Bacharach theorem, and
the reader who wishes to can certainly skip the proofs of statements
CB\ref{thmcb4}--CB\ref{thmcb7} or just glance at them to get a sense of the
ingredients without suffering any logical gaps.

Using the language of the theory of curves and denoting by $H$ the divisor cut
on a plane curve $X$ by a line $L\subset \mathbb{P}^2$, we can state a
surprising property of the divisor $(d-3)H$, from which Theorem CB\ref{thmcb4}
will follow immediately.

\begin{prop}\label{prop:four}
Let $X\subset \mathbb{P}^2$ be a nonsingular plane curve of degree $d$, and let
$p$ be a point of $X$.  Every effective divisor linearly equivalent to
$(d-3)H+p$ actually contains $p$.
\end{prop}

This baffling statement loses (some of) its mystery when viewed as a very
special case of the Riemann-Roch Theorem (in fact, of Riemann's Theorem itself,
though we shall not worry about this distinction).  To explain this, we first
note that any two rational differential forms on $X$ have a ratio that is a
rational function on $X$, and thus the divisor of zeros minus the divisor of
poles of any rational differential form is a well-defined divisor class on $X$,
called the \emph{canonical divisor class}, and denoted $K_X$.   Abusing
notation in a traditional way, we shall also denote by $K_X$ any divisor in
this class.  Next, we recall that if the ground field is $\mathbb{C}$, then $X$
may be viewed as a Riemann surface, and as such it has a genus $g$ (the
``number  of handles''); this invariant and the fact that it is a compact
orientable surface specifies its topology completely.  There is a miraculous
connection between the topology of $X$ and the geometry on $X$: the degree of
$K_X$ is $2g-2$.  (In the case of a general ground field, this can be taken as
the definition of the genus.) Now for a plane curve it is not hard to compute
$K_X$ by writing down a specific rational differential form.  The result is a
special case of what algebraic geometers call the \emph{adjunction formula},
and it tells us that
\[K_X\sim (d-3)\cdot H.\]
The divisor $(d-3)H$ that appears in Proposition~\ref{prop:four} is ``really''
$K_X$.  It is worth noting another consequence of the adjunction formula in
passing.  By B\'ezout's Theorem the degree of the divisor $(d-3)H$ is $d(d-3)$,
so we get $2g-2=(d-3)d$, or $g=(d-1)(d-2)/2$.

To exploit the identification of $(d-3)H$ with $K_X$, we must use the 
Riemann-Roch Theorem, which we now introduce.  For any divisor $D$ on $X$, we
shall denote by $L(D)$ the vector space of rational functions $f$ such that $D$
plus the divisor of $f$ is effective.  We denote the vector space dimension of
$L(D)$ by $l(D)$.  For example, Liouville's Theorem asserts that the only
rational functions with no zeros or poles are the constant functions, so 
$l(0)=1$.

With these definitions, the Riemann-Roch Theorem asserts that for any
divisor $D$ of degree $n$ on $X$,
\[l(D)=n-g+1+l(K_X-D)\]
where $g=(d-1)(d-2)/2$ is the genus of  $X$.  For example,
$l(K_X)=(2g-2)-g+1+l(0)=g-1+1=g$.

\begin{proof}[Proof of Proposition~\ref{prop:four}]
Let $p$ be a point of $X$.  The degree of the divisor $K_X+p$ is $2g-1$, so by 
Riemann-Roch we have 
\[l(K_X+p)=(2g-1)-g+1+l(-p)=g+l(-p).\]
But the degree of $-p$ is $-1$, so no effective divisor can be equivalent to
$-p$; thus $l(-p)=0$.  It follows that $l(K_X+p)=g=l(K_X)$.  But given any
effective divisor $D$ linearly equivalent to $K_X$, we may make a divisor $D+p$ 
linearly equivalent to $K_X+p$, the map $D\mapsto D+p$, as a map  $L(D)\to
L(D+p)$, is actually a vector space homomorphism.  It is obviously injective,
and since the two spaces have the same dimension, they must be equal.
\end{proof}

\begin{proof}[Proof of Theorem \rm CB\ref{thmcb4} 
(under the assumption  that
$X_1$ is nonsingular)]\hspace{\fill}Suppose\linebreak  
$C\subset \mathbb{P}^2$ is a plane curve of degree $d+e-3$ containing
all of $X_1\cap  X_2$ except for the point $p=p_{de}$.  We can write the
divisor cut on $X_1$ by $C$ as
\[C\cdot X_1=p_1+\dots+p_{de-1}+q_1+\dots+q_{d(d-3)+1}.\]
We have $K_{X_1}\sim (d-3)\cdot H$, while $p_1+\dots+p_{de}\sim e\cdot H$ and
$C\cdot X_1\sim (d+e-3)\cdot H$.  Thus we can rewrite the equation as
\[(d+e-3)\cdot H\sim e\cdot H-p+q_1+\dots+q_{d(d-3)+1}\]
or, equivalently,
\begin{equation*} 
\begin{split}
q_1+\dots +q_{d(d-3)+1}&\sim (d-3)\cdot H+p\\
&\sim K_{X_1}+p.
\end{split}
\end{equation*} 
By Proposition~\ref{prop:four}, we get $p\in\{q_1,\dots,q_{d(d-3)+1}\}$.  In
particular, $p\in C$.
\end{proof}

Another way to express this last argument, more explicitly invoking  the
Residue Theorem, is to choose $(x,y)$ affine coordinates on the plane; let
$f(x,y),g(x,y)$, and $h(x,y)$ be the equations of the curves $X_1,X_2$, and $C$
in these coordinates; and consider the differentials
\[\omega=\frac {dx}{\partial f/\partial y}\]
and
\[\varphi=\frac{h(x,y)} {g(x,y)}\omega\]
on $X_1$. The form 
$\omega$ itself is regular and nonzero on the affine part of $X_1$,
having zeros of order $d-3$ along the divisor cut on $X_1$ by the line at
$\infty$.  It follows that $\varphi$ is regular at the points at $\infty$, with
simple poles exactly at the points of $X_1$ contained in $X_2$ but not in $C$. 
There is at most one such point, namely, $p$.  By the Residue Theorem, the sum
of the residues of any differential form must be 0, so a differential on a
curve cannot have exactly one simple pole.  We conclude that $p\in C$.

A more careful application of Riemann-Roch yields the version of
Cayley-\linebreak  
Bacharach that was actually proved in Bacharach \cite{Ba}.  It relates
the number of conditions imposed on curves of various degrees by two
complementary subsets of the intersection of two plane curves.  We need one
more fact from the general theory of plane curves: Let $X$ be a nonsingular
plane curve of degree $d$, and let $C$ be a plane curve of degree $k$.  The
refined form of B\'ezout's Theorem says that by counting each point with an
appropriate multiplicity, we may regard the intersection of $C$ and $X$ as a
divisor on $X$ of degree $d\cdot k$, which we will denote by $C\cdot X$.  The
fact we want is a corollary of Theorem \ref{thm:three}. We may express it in
classical language as 
%expressed in classical language by saying that the family of
%plane curves containing $\Gamma$ ``cuts out the complete linear series of
%divisors equivalent to $C\cdot X-\Gamma$''.  In our terms this says:

\begin{cor}\label{cor:five}
Let $X$ be a nonsingular plane curve, and let $C$ be any plane curve not
containing any component of $X$.  Let $\Gamma$ be any effective divisor on $X$. 
The family of plane curves containing $\Gamma$ cuts out on $X$ the complete 
linear series of divisors linearly equivalent to 
%If an effective divisor $D$ on $X$ is linearly equivalent to 
$C\cdot X-\Gamma$. %,
%then $D=C'\cdot X-\Gamma$ for some other plane curve $C'$.
\end{cor}


We shall exploit Corollary~\ref{cor:five} to express the number of
conditions imposed on forms of degree $m$ by a set of points of $X$ in
terms that are accessible to the Riemann-Roch Theorem.  The precise result is:

\begin{cor} \label{cor:six}
Let $X$ be a smooth plane curve of degree $d$, and let $\Lambda\subset X$ be a
set of $\lambda$ points, regarded as a divisor on $X$.  The number of linear 
conditions imposed by $\Lambda$ on forms of degree $m$ is equal to $l(mH)-l(mH-
\Lambda)$.  In particular, the \textup{``}failure of $\Lambda$ to impose
independent conditions on forms of degree $m$\textup{''} is 
$\lambda-(l(mH)-l(mH-\Lambda))$.
\end{cor}

\begin{proof} The ``number of linear conditions imposed by $\Lambda$ on forms
of degree $m$'' is the dimension $t$ of the vector space of forms of degree $m$
modulo those vanishing on $\Lambda$.  The number $l(mH)$ is the dimension of
the vector space $L(mH)$, which by Corollary~\ref{cor:five} (applied in the
case when $\Gamma$ is the empty set) is the dimension of the space of forms of
degree $m$ modulo those vanishing on $X$.  Similarly, the number
$l(mH-\Lambda)$ is by Corollary~\ref{cor:five} the dimension of the space of
forms of degree $m$ vanishing on $\Lambda$ modulo those vanishing on all of
$X$.  Thus $t=l(mH)-l(mH-\Lambda)$.  The ``failure of $\Lambda$ to impose
independent conditions is simply the number of points $\lambda$ of $\Lambda$
(the maximal number of conditions that $\Lambda$ could impose) minus the number
of conditions actually imposed, or $\lambda-(l(mH)-l(mH-\Lambda))$.
\end{proof}

\begin{thmcb}[Bacharach] \label{thmcb5}
Let $X_1,X_2\subset \mathbb{P}^2$ be plane curves of degrees $d$ and $e$
respectively, intersecting in $d\cdot e$ points  $\Gamma=X_1\cap
X_2=\{p_1,\dots,p_{de}\}$, and suppose that $\Gamma$ is the
disjoint union of subsets $\Gamma'$ and $\Gamma''$.  Set $s=d+e-3$.  If $k\le
s$ is a nonnegative integer, then the dimension of the vector space of forms of
degree $k$ vanishing on $\Gamma'$ \textup{(}modulo those containing all 
of $\Gamma)$ is equal to the failure of $\Gamma''$ to impose independent 
conditions on forms of degree $s-k$.
\end{thmcb}

This theorem is one possible departure point for a beautiful series of 
mathematical developments known as liaison theory.   Liaison is the equivalence
relation gotten by allowing a variety $X'$ to be replaced by the variety $X''$
residual to it for some complete intersection $X$ containing $X'$.

Note that Theorem CB\ref{thmcb4} is just the case where $k=s$ and $\Gamma''$ is
a single point $p=p_{de}$: Since any point $p\in \Gamma$ imposes one 
independent condition on polynomials of degree 0, the conclusion is that there
are no hypersurfaces of degree $k$ containing $\Gamma-\{p\}$ except those
containing $\Gamma$.  Theorem CB\ref{thmcb5} says further that any curve of
degree $s-1=d+e-4$ containing all but two points of $\Gamma$ contains $\Gamma$
and there exists a curve of degree $s-1$ containing all but three points
$p,q,r\in \Gamma$ but not containing $\Gamma$ if and only if $p,q$, and $r$ are 
collinear, and so on.

Once again we give the proof under the assumption that $X_1$ is nonsingular;
the more general case may be treated using Bertini's Theorem, as indicated
before.

\begin{proof}[Proof \rm (under the assumption that $X_1$ is nonsingular)]
As before, denote by $H$ the hyperplane divisor on $X_1$; we shall consider $
\Gamma,\ \Gamma'$, and $\Gamma''$ as divisors on $X_1$ as well.   The dimension
of the family of curves of degree $k$ containing $\Gamma'$, modulo those
containing all of $\Gamma$, is the same as the number of conditions imposed by
$\Gamma$ on curves of degree $k$ minus the number of conditions imposed by $
\Gamma'$, that is, $h_\Gamma(k)-h_{\Gamma'}(k)$.  By
Corollary~\ref{cor:five}, the families of plane curves of degree $k$
containing $\Gamma$ and containing  $\Gamma'$ cut on $X_1$ the linear series of
divisors equivalent to $k\cdot H-\Gamma'$ and $k\cdot H-\Gamma=(k-e)\cdot H$
respectively.   By Riemann-Roch we have
\begin{equation*} 
\begin{split}
&l(k\cdot H-\Gamma')-l((k-e)\cdot H)\\
&\qquad =kd-\gamma'-g+l((d-3-k)\cdot H+\Gamma')\\
&\qquad \quad -[(k-e)d-g+l((d-3-k+e)\cdot H)]\\
&\qquad =d\cdot e-\gamma'-[l((d-3-k)\cdot H+\Gamma')-l((d-3-k+e)\cdot H)].
\end{split}
\end{equation*}
Now, since the divisors $\Gamma'$ and $\Gamma''$ add up to $e\cdot D$ and their
degrees  correspondingly add up to $e\cdot d$, we can rewrite this as
\[=\gamma-[l((s-k)\cdot H)-l((s-k)H-\Gamma'')]\]
which is the failure of $\Gamma''$ to impose independent conditions on curves
of degree $s-k$.
\end{proof}

There is an immediate generalization of Theorem CB\ref{thmcb5} to a statement
about the transverse intersection of $n$ hypersurfaces $X_i$ of degrees $d_i$
in $\mathbb{P}^n$.  Here the role of the number $d+e-3$ in the preceding
statement is played by $s=\Sigma d_i-n-1$.

\begin{thmcb} \label{thmcb6}
Let $X_1,\dots, X_n$ be hypersurfaces in $\mathbb{P}^n$ of degrees
$d_1,\dots,d_n$ respectively, meeting transversely, and suppose that the
intersection $\Gamma=X_1\cap \dots \cap X_n$ is the disjoint union of subsets $
\Gamma'$ and $\Gamma''$.  Set $s=\Sigma d_i-n-1$.  If $k\le s$ is a nonnegative
integer, then the dimension of the family of curves of degree $k$ containing $
\Gamma'$ \textup{(}modulo those containing all of $\Gamma)$ is equal to the
failure of $\Gamma''$ to impose independent conditions of curves of
\textup{``}complementary\textup{''} degree $s-k$.
\end{thmcb}

\begin{proof}[Proof \rm (in case $X=X_1\cap \dots\cap X_{n-1}$ is nonsingular)]
The proof follows exactly the lines of the proof of the last version, modulo
the following changes: in the setting of the current theorem we have
$\Gamma\sim d_n\cdot H$, so we replace $e$ in the previous proof with $d_n$. 
In the new setting the adjunction formula tells us that
\begin{equation*} 
\begin{split}
K_X&\sim \left(\sum^{n-1}_1 d_i-n-1\right) \cdot H\\
&=(s-d_n)\cdot H,
\end{split}
\end{equation*}
so we replace $d-3$ in the previous proof by $s-d_n$.
\end{proof}

\subsection*{{\rm 1.3.} The twentieth century}
The next stage in the evolution of the Cayley-Bacharach Theorem may seem in
some sense a purely technical advance: we shall eliminate the hypothesis that
the hypersurfaces $X_i$ intersect transversely and replace it with the weaker
one that they intersect in isolated points; that is, in modern language, we no
longer assume that the scheme $\Gamma=X_1\cap \cdots \cap X_n$ of intersection
of the $X_i$ is reduced, only that it is zero-dimensional.  In fact, though,
this is the generalization that in our view transforms the problem and
ultimately unifies the approaches above.

Unfortunately, the geometric content of the new version is slightly less
intuitive; the intersections must be treated as \emph{schemes}.  Some would
even claim that what we are about to do amounts to nothing other than
forgetting geometry and working directly in algebra.  But we feel that the
geometric language still carries the clearest picture.  
So to read further the reader will have to accept the notion that
\emph{every} ring (or at least every finitely generated algebra over our still
unnamed ground field) corresponds to some geometric object, a scheme, in a way
extending the correspondence of reduced finitely generated algebras over an 
algebraically closed field with algebraic sets.  For a relatively simple
introduction to this notion, the reader may consult Eisenbud-Harris
\cite{EH2}.  In any case, we shall not use any deep property of this notion,
and the algebra that we need we shall introduce from scratch.

Before we can state this version of the Cayley-Bacharach Theorem, we have to
generalize the notion of expressing the intersection $X_1\cap \dots \cap X_n=
\Gamma$ as a disjoint union $\Gamma'\cup \Gamma''$ to the case where $\Gamma$
is an arbitrary zero-dimensional scheme.  In general, we would like to define a
notion of ``residual subscheme'' $\Gamma''$ to a subscheme $\Gamma'$ of a 
zero-dimensional scheme $\Gamma$.  This should play the role of a
``complementary subset'', except that the support of $\Gamma''$ may overlap
with that of $\Gamma'$.  We want this definition to have two basic properties:
the degrees of the two subschemes $\Gamma'$ and $\Gamma''$ should add up to
that of $\Gamma$, and the process should be symmetric; i.e., $\Gamma'$ should
be in turn the subscheme of $\Gamma$ residual to $\Gamma''$.  An obvious
desideratum for the residual subscheme is that the product of a function
vanishing on $\Gamma'$ and one vanishing on $\Gamma''$ should vanish on
$\Gamma$.  We shall define the residual subscheme as the maximal subscheme with
this property.  In other terms:

\begin{dfn}
Let $\Gamma$ be a zero-dimensional scheme with coordinate ring $A(\Gamma)$. 
Let $\Gamma'\subset \Gamma$ be a closed
subscheme and $I_{\Gamma'}\subset A(\Gamma)$ its ideal.  
By the subscheme of $\Gamma$ \emph{residual} to $\Gamma'$ we shall
mean the subscheme defined by the ideal
\[I_{\Gamma''}=\Ann (I_{\Gamma'}/I_\Gamma).\]
\end{dfn}

The problem is simply that in general this definition has neither of the 
properties we desire, as can be seen already in the case of $\Gamma\subset
\mathbb{A}^2$ a ``fat point'', that is, 
\[I_\Gamma=(x^2,xy,y^2)\subset k[x,y].\]
If we take $\Gamma'$ the reduced point---that is, $I_{\Gamma'}=(x,y)$---then by
our definition the residual scheme $\Gamma''$ will again be the reduced point,
which violates the degree condition.  (There are also more complicated examples
in which $\deg(\Gamma')+\deg (\Gamma'')>\deg (\Gamma)$.) At the same time, if
we take $\Gamma'$ a subscheme of degree 2---say, for example,
$I_{\Gamma'}=(x,y^2)$---then the residual scheme $\Gamma''$ to $\Gamma'$ will
be again the reduced point, violating the requirement that the residual to the
residual be the original subscheme.  These examples suggest 
the impossibility of any definition satisfying the 
conditions we have set.

We are saved in the present circumstance because $\Gamma$ is not an arbitrary
zero-dimensional scheme; it is a complete intersection.  In particular, the 
local rings $\mathcal{O}_{\Gamma,p}$ (which are the localizations of
$A(\Gamma)$ at its maximal ideals) are \emph{Gorenstein} rings, which is
exactly the property we need for the definition of residuation to work.

Although Gorenstein rings are quite commonplace in commutative algebra, their
usual rather abstract definition has given them the reputation in some
geometric circles of being arcane and only for experts.  In the present
circumstance, however, we do not need the general definition: since $\Gamma$ is
zero-dimensional and the rings $\mathcal{O}_{\Gamma,p}$ 
are correspondingly finite-dimensional vector spaces over the ground field, 
we can say in concrete terms what it means for the local 
ring $A=\mathcal{O}_{\Gamma,p}$ at a point 
$p\in \supp(\Gamma)$ to be Gorenstein.  To start with the
simplest definition:

\begin{dfn} 
Let $A$ be a local Artinian ring, $\mathfrak{m}\subset A$ its maximal ideal.  We
say that $A$ is \emph{Gorenstein} if the annihilator of $\mathfrak{m}$ has 
dimension one as a vector space over $K=A/\mathfrak{m}$.
\end{dfn}

%(The Cohen Structure Theorem guarantees that any ring $A$ as in the definition
%contains a copy of its residue class field $K$, and we shall regard $A$ as a
%$K$-algebra.  The naive reader might simply want to assume that $A$ is a 
%$K$-algebra from the outset; this will be the case in the geometric situation
%coming from Cayley-Bacharach theorems for projective space over an
%algebraically closed field.)

In general, for any local Artinian ring $A$ with maximal ideal $\mathfrak{m}$, 
the annihilator of $\mathfrak{m}$ as an $A$-module is called the \emph{socle}
of $A$.  To see what the condition that the socle of $A$ have length 1 means,
observe first of all that we know $\mathfrak{m}^k=0$ for $k$ large, and let $
\mathfrak{m}$ be the largest integer $k$ such that $\mathfrak{m} ^k\ne 0$. 
Certainly everything in $\mathfrak{m}^m$ annihilates $\mathfrak{m}$, so that
the Gorenstein condition above at least implies that $\mathfrak{m}^m$ has
length one, i.e.,
\begin{equation}\label{eq:ast}
\mathfrak{m}^{\MM}\cong K. \tag {$*$}
\end{equation}
In fact, it says more: given any nonzero $f\in A$, we claim there exists a
$g\in A$ such that the product is nonzero, but $fg\in \mathfrak{m}^{\MM}$.  
This is easy: if $l$ is the largest integer such that $f\cdot \mathfrak{m}^l\ne
0$, then any element of $f\cdot \mathfrak{m}^l$ annihilates $\mathfrak{m}$ and
so must lie in $\mathfrak{m}^{\MM}$.  

Now suppose that $A$ contains a field.  It follows from the Cohen Structure 
Theorem (\cite[Theorem 7.7]{E}) that $A$ contains a copy of its residue field
$K$.  (The reader might simply assume that $K$ is the ground field, the usual
geometric situation.)  
The condition that the annihilator of
$\mathfrak{m}$ is one-dimensional thus also implies that
\begin{equation} \label{eq:astast}
\begin{split}
&\text{Given any $K$-linear map $\rho A\to \mathfrak{m}^{\MM}$ that restricts to}\\
&\text {the identity on $\mathfrak{m}^{\MM}$, the pairing }\\
&\text{$A\times A\to A\rho \mathfrak{m}^{\MM}\cong K$ 
given by multiplication}\\
&\text{is 
a nondegenerate pairing on $A$ as  a vector
space over $K$.}
\end{split} \tag {$**$}
\end{equation}

In fact, it is the case conversely that the two assertions \eqref{eq:ast}
and \eqref{eq:astast} together are equivalent to the Gorenstein condition.  To
see this, observe that \eqref{eq:ast} and \eqref{eq:astast} together in turn
imply that there exists a $K$-linear map $A\to K$ such that the composition
\[Q:A\times A\to A\to K\]
is a nondegenerate pairing on the $K$-vector space $A$. On the other hand, we
see that this implies the Gorenstein condition: Since the maximal ideal 
$\mathfrak{m} \subset A$ has codimension one as a $K$-vector space, the 
orthogonal complement $\mathfrak{m}^\perp$ of $\mathfrak{m}$ with respect to
the pairing $Q$ on $A$ will have dimension one, and this certainly contains the
annihilator of $\mathfrak{m}$.  We thus have an alternative characterization of
Gorenstein Artinian rings:

\begin{prop}\label{prop:seven}
Let $A$ be an Artinian ring with residue field $K$.  The ring $A$ is Gorenstein
iff there exists a $K$-linear map $A\to K$ such that the composition
\[Q:A\times A\to A\to K,\]
where the first map is multiplication in $A$, is a nondegenerate pairing on the
$K$-vector space $A$.
\end{prop}

It may possibly be easier to visualize the Gorenstein condition in case $A$ is
a graded ring.  The Gorenstein condition then says that 

The top graded component $A_m$ of $A$ has length $1$,

\noindent
and

for every pair $k,l$ of nonnegative integers with $k+l=m$, the pairing
\[A_k\times A_l\to A_m=K\]

given by multiplication is a nondegenerate pairing of $K$-vector spaces.

Having described the Gorenstein condition for Artinian rings, it is
possible to state the condition for general local rings.  One bit of
terminology: if $S$ is any local ring with maximal ideal $\mathfrak{m}$, we say
that a sequence $(F_0,\dots, F_k)$ of elements of $\mathfrak{m}$ 
is \emph{regular} if for
each $i=0,\dots,k$, $F_i$ is a nonzerodivisor in $S/(F_0,\dots,F_{i-1})$; we
say that $(F_0,\dots,F_k)$ is a maximal regular sequence if it cannot be
extended to a regular sequence $(F_0,\dots,F_{k+1})$, 
i.e., if every element of the maximal ideal of $S/(F_0,\dots,F_k)$ is a
zerodivisor.  We then make the 

\begin{dfn}
A local ring $S$ is Gorenstein if for every maximal regular sequence
$(F_0,\dots,F_k)$ of elements of $S$ the quotient $A=S/(F_0,\dots,F_k)$ is a
Gorenstein Artinian ring, in the sense above.
\end{dfn}

For the reader's interest we mention that the ring $S$ is called Cohen-Macaulay
if, in the situation above, the quotient $A$ is simply assumed Artinian; thus
the Gorenstein property includes Cohen-Macaulay.

It follows from Proposition \ref{prop:nine} below that the Gorenstein condition
on the quotient 
$S/(F_0,\dots,F_k)$ is independent of the maximal regular sequence 
$(F_0,\dots,F_k)$, so that we could just as well replace the ``every'' in 
the preceding definition with ``some''.  We shall put off until section
1.4 a proof that the local
rings of a zero-dimensional complete intersection scheme $\Gamma$ are 
Gorenstein. (For a general
discussion of the definitions and properties of Gorenstein rings, see Eisenbud
\cite[Ch.\ 21]{E}; for a different discussion of the connection with
Cayley-Bacharach, see Davis-Geramita-Orecchia \cite{DGO}.)


The point is that if the local rings of $\Gamma$ are Gorenstein, then our  
definition of the residual subscheme to a subscheme of $\Gamma$ does 
satisfy the two conditions we laid down: we simply observe that the ideal
of the residual subscheme $\Gamma''$ to a given subscheme $\Gamma'\subset
\Gamma$ is the orthogonal complement, with respect to the pairing $Q$, of the
ideal $I_{\Gamma'}$ of $\Gamma'$.  This tells us immediately that $I_{\Gamma'}$
and $I_{\Gamma''}$ have complementary dimensions as vector subspaces of $A$, so
that $\deg(\Gamma')+\deg(\Gamma'')=\deg(\Gamma)$; and of course since the 
orthogonal complement of the orthogonal complement of $I_{\Gamma'}$ will again
be $I_{\Gamma'}$, the residual subscheme of the residual subscheme of $\Gamma'$
will again be  $\Gamma'$.  With this said, we may state the next version of
Cayley-Bacharach.

\begin{thmcb} \label{thmcb7}
Let $X_1,\dots,X_n$ be hypersurfaces in $\mathbb{P}^n$ of degrees
$d_1,\dots,d_n$, and suppose that the intersection $\Gamma=X_1\cap \dots\cap
X_n$ is zero-dimensional. Let $\Gamma'$ and $\Gamma''$ be subschemes of
$\Gamma$ residual to one another in $\Gamma$, and set $s=\Sigma d_i-n-1$. If
$k\le s$ is a nonnegative integer, then the dimension of the family of curves
of degree $k$ containing $\Gamma'$ \textup{(}modulo 
those containing all of $\Gamma)$ is
equal to the failure of $\Gamma''$ to impose independent conditions of curves
of complementary degree $s-k$.
\end{thmcb}

A proof of this statement could be given along the lines of the preceding proof
by introducing the \emph{dualizing sheaf} $\omega_X$ of the curve (that is, 
one-dimensional scheme) $X=X_1\cap \dots\cap X_{n-1}$ and proving a
Riemann-Roch/duality type theorem relating the global sections of a sheaf $
\mathcal{F}$ on $X$ to the first cohomology of the sheaf $\Hom(\mathcal{F},
\omega_X)$.  But we don't have to do this: Theorem CB\ref{thmcb7} suggests yet
a further (and, for now, final) generalization, from which the last statement
will follow readily and which admits a relatively elementary proof!  We shall
now state this final version of Cayley-Bacharach and then indicate 
how it implies the last version.

Our final generalization of the Cayley-Bacharach Theorem will be expressed in
terms of the homogeneous coordinate ring of $\Gamma$.  We first introduce some
notation.  As before, we let $X_1,\dots,X_n\subset \mathbb{P}^n$ be
hypersurfaces intersecting in a zero-dimensional scheme $\Gamma$.  We let
$S=K[Z_0,\dots,Z_n]$ be the homogeneous coordinate ring of $\mathbb{P}^n$ and
let $F_i(Z)$ be the homogeneous polynomial of degree $d_i$ defining the
hypersurface $X_i$. We write $I(\Gamma)\subset S$ for the ideal of homogeneous 
polynomials vanishing on  $\Gamma$ and set $R=S/I(\Gamma)$, the homogeneous
coordinate ring of $\Gamma$.  In keeping with our algebraic point of view, we
will now write $h_R(k)$ for what we used to call $h_\Gamma(k)$, the number of 
conditions imposed by $\Gamma$ on forms of degree $k$; it is just the vector
space dimension of $R_k$, the $k$th graded component of $R$.

We have had occasion to use Max Noether's $AF+BG$ theorem several times
already, and we shall need a more powerful version of it in the sequel.  This
is the Lasker's Unmixedness Theorem, which extends Noether's Theorem to an
assertion about homogeneous \emph{sequences of parameters}.  We say that a 
sequence of homogeneous forms $F_1,\dots,F_s$ is a sequence of parameters if
the algebraic set defined by the vanishing of $F_1,\dots,F_s$ (either in
$\mathbb{A}^{n+1}$ or in $\mathbb{P}^n$, it comes to the same thing) has 
codimension $s$, the number of forms.

\begin{thm}[Lasker] \label{thm:eight}
If $F_1,\dots,F_s\in S$ is a sequence of parameters and $\Gamma\ne \varnothing
$ is the scheme
in $\mathbb{P}^n$ that is the intersection of the hypersurfaces defined by the
$F_i$, then every polynomial vanishing on $\Gamma$ is a linear combination of
the $F_i$; equivalently \textup{(}in case the ground field is
infinite\textup{)} there exists a linear form $L$ that is a nonzerodivisor
modulo $(F_1,\dots,F_s)$.  
\end{thm}

Noether's Theorem is just the case $n=2$. (The
development of Noether's Theorem does not end here: it was extended by Macaulay
\cite[sections 48--53]{M} to systems of parameters in other rings and became
the basis of the theory of Cohen-Macaulay rings, which may be defined as local
rings for which the Unmixedness Theorem is true; see Nagata \cite{Na}.)

Next, we do something that may strike the geometrically oriented reader as a
bit odd: we let  $H\subset \mathbb{P}^n$ be a general hyperplane (specifically,
one not meeting the support of $\Gamma)$, $L$ the linear form defining $H$,
and introduce the quotient ring
\[A=R/(L)=S/(L,F_1,\dots,F_n).\]
Now, if we think of $R$ simply as the homogeneous coordinate ring of the
projective scheme $\Gamma$, this makes no apparent sense: the ideal
$(L,F_1,\dots,F_n)$ defines the empty set.  It makes much more sense if we
think of $R$ as the affine coordinate ring of the cone $\widetilde \Gamma$ over
$\Gamma$  in affine space $\mathbb{A}^{n+1}$ (Figure 
\ref{fig:four}): then $A$ is the coordinate ring
of the hyperplane section of $\widetilde \Gamma$ corresponding to a hyperplane
meeting $\widetilde \Gamma$ at the origin, and it is not unreasonable to hope
that the scheme structure of this intersection reflects the global geometry of
$\Gamma$.

\begin{figure}
\includegraphics{bull666l-fig-4}
\caption{}
\label{fig:four}
\end{figure}

To see how this works, observe that $R$ and $A$ are graded rings and that $L$
is a nonzerodivisor in $R$.  It follows that the Hilbert function of $A$
(which is again a graded ring) will be the first difference function of the
Hilbert function of $R$:
\[h_A(m)=h_R(m)-h_R(m-1).\]
Moreover, the same is true for any quotient $R'=R/I'$ and corresponding
$A'=R/(I',L)$, as long as $L$ is a nonzerodivisor modulo $I'$.  Thus for any 
subscheme  $\Gamma'\subset \Gamma$ we let $\overline{I'}$ be the image in 
$A=R/(L)$ of the ideal $I'$ of $\Gamma$ and write
\begin{equation*} 
\begin{split}
&h^0(\mathbb{P}^n,I_{\Gamma'}(k)) -h^0(\mathbb{P}^n,I_\Gamma(k))\\ 
&\qquad=h_R(k)-h_{R'}(k)\\
&\qquad =\sum^k_{j=0} (h_A(j)-h_{A'}(j))\\
&\qquad =\sum^k_{j=0}\dim _{K} (\overline {I'}_j).
\end{split}
\end{equation*}

Now let $i(l)$ denote the failure of the subscheme $\Gamma'$ to impose
independent conditions on hypersurfaces of a given degree $l$.  It is likewise
the case that we can measure $i(l)$ in terms of the ring $R'$: we have
\[i(l)=\deg (\Gamma')-h_{R'}(l)\]
so that the successive differences
\[i(l)-i(l+1)=h_{R'}(l+1)-h_{R'}(l).\]
Thus
\[i(l)=\sum^\infty_{j=l+1}h_{R'/(L)}(l).\]
Of course, the sum need only extend over the finite range of $j\ge l+1$ for
which $\overline{I'}_j\subsetneqq R_j$, that is, $h_{R'/(L)}(l)\ne 0$.  

The crucial result, which we prove in the next section, is that the ring $A$ is
itself Gorenstein, with socle in degree $m$.  This means that we get a
nondegenerate pairing
\[Q:A\times A\to A\to A_m\cong K.\]
Let $\Gamma''\subset \Gamma$ be the subscheme residual to $\Gamma$, and let
$I''\subset R$ be its idea; let $\overline{I''}$ in $A$.  We observe that the
ideal $\overline{I''}\subset R$ is just the annihilator of $\overline{I'}$ with
respect to the pairing $Q$: clearly they annihilate each other, and the sum of
their codimensions (as $K$-vector spaces) in $A$ is equal to the dimension $d$
of $R$.  In particular, the dimension of the $j$th graded piece of
$\overline{I'}$ is the codimension of the $(m-j)$th graded piece of
$\overline{I''}$.

Summing this equality over $j=0,1,\dots,l$ yields the statement: \emph{the
number of hypersurfaces of degree $l$ containing $\Gamma'$} (\emph{modulo those
containing} $\Gamma)$ \emph{is equal to the failure of $\Gamma''$ to impose
independent conditions on hypersurfaces of degree} $m-1-l$.  This, of course,
is exactly the statement of our last version of Cayley-Bacharach! 
Cayley-Bacharach thus turns out to be simply the statement that the ring $A$ is
Gorenstein, in other words, the special case $d_0=1$ of the 

\begin{thmcb} \label{thmcb8}
If $F_0,\dots,F_n\in K[Z_0,\dots,Z_n]$ are homogeneous polynomials of degrees
$d_0,\dots,d_n$ forming a sequence of parameters \textup{(}i.e., having no
common zeros in $\mathbb{P}^n)$, then the ring
\[A=K[Z_0,\dots,Z_n]/(F_0,\dots,F_n)\]
is Gorenstein, with socle in degree $\Sigma d_i-n-1$.
\end{thmcb}

Note that in terms of the definition of the Gorenstein condition this statement
is equivalent (modulo specifying the degree of the socle, which is elementary)
to the shorter

\begin{thmcb} \label{thmcb9}
The polynomial ring $K[Z_0,\dots,Z_n]$ is Gorenstein.
\end{thmcb}

This is our final version of the Cayley-Bacharach Theorem.  Whether it does in
fact convey a deeper understanding of the phenomena expressed in the original 
theorems of Pappus, Pascal, and Chasles, or represents merely the addition of a
layer of technical machinery, we leave to the reader.  Perhaps the best
criterion is the simplicity of the proof, which appears in the following
section.
\subsection*{{\rm 1.4.} Proof of the final Cayley-Bacharach Theorem}
\label{subsec:onefour}
In this section we \nolinebreak 
shall, \linebreak  
as advertised, give a proof ``from scratch'' of the
final version of the Cayley-Bacharach Theorem.  We also owe the reader the
verifications of two statements made along the way: that the local rings of a 
zero-dimensional complete intersection subscheme $\Gamma\subset\mathbb{P}^n$
are Gorenstein (the reader will recognize this as a weaker version of 
the final Cayley-Bacharach Theorem) and that the ideal
$(F_1,\dots,F_k)\subset K[Z_1,\dots,Z_n]$ generated by a sequence of parameters
is saturated.

As for the proofs, these involve properties of the Ext functor.  They are
nonetheless elementary in spirit: in fact, all we shall use is the basic 
definition of the Ext functors as right-derived functors of the functor Hom. 
In particular, we do not actually calculate any Ext groups (other than Hom) or
use any of their geometric or algebraic properties.  Rather, as is often the
case with homological functors, we simply use the exact sequences associated to
them to transport information from one setting to another, in this case, from
one quotient of the polynomial ring to another.

To begin with, one piece of terminology.  Let $S$ be any local ring with
maximum ideal $\mathfrak{m}$ and residue field $K$, and $M$ any finitely
generated graded $S$-module.  We say that a sequence $(F_0,\dots,F_k)$  of 
elements of $\mathfrak{m}$ is an $M$-\emph{regular sequence} (or just 
$M$-\emph{sequence}) if for each $i=0,\dots,k$, $F_i$ is a non-zero-divisor for
the module $M/(F_0,\dots,F_{i-1})M$.  Of course, we say that $(F_0,\dots,F_k)$
is a maximal $M$-sequence if it cannot be extended to an $M$-sequence
$(F_0,\dots,F_{k+1})$ that is, if every element of $\mathfrak{m}$ is a
zerodivisor for $M/(F_0,\dots,F_k)M$.

All the needed results will follow directly from

\begin{prop} \label{prop:nine}
With the above notation, if $(F_0,\dots,F_k)$ is a maximal $M$-sequence, then
\begin{align*}
\Ext^i(K,M)&=0 \qquad  \text{for all }i\le k, \text{ and}\\
\Ext^{k+1}(K,M)&=\Hom (K,M/(F_0,\dots,F_k)M)\ne 0.
\end{align*}
\end{prop}

Before we prove this, let's see how it implies all the assertions made so far
about the algebra of the rings $R$ and $A$ associated to a complete
intersection in $\mathbb{P}^n$.

To begin with, the most immediate implication is simply that \emph{all maximal
$M$-sequences contain the same number $k$ of elements}, $k$ being characterized
as the smallest integer such that $\Ext^k(K,M)\ne 0$.  In particular, in case
$M=S$ we see that all maximal regular sequences in the ring $S$ have the same
number of elements; this number is called the \emph{depth} of $S$. Now, in 
case $S=K[Z_0,\dots,Z_n]$ is the polynomial ring, $(Z_0,\dots,Z_n)$ is visibly
a maximal regular $S$-sequence, and it follows then that \emph{all maximal
regular sequences in $K[Z_0,\dots,Z_n]$ have $n+1$ elements}.  This, finally,
means that any regular sequence $(F_1,\dots,F_n)$ is \emph{not} maximal, i.e.,
that the ring $K[Z_0,\dots,Z_n]/(F_1,\dots,F_n)$ contains nonzerodivisors of
positive degree, i.e., that the ideal $(F_1,\dots,F_n)\subset S$ is saturated.

Next, observe that the socle of a local Artinian ring $A$ with residue field
$K$ is the module $\Hom(K,A)$; thus in particular $A$ will be Gorenstein if and
only if $\Hom(K,A)\cong K$.  Now, applying the Proposition with $M=S$, we see
that if $S$ is any local ring with residue field $K$ and $(F_0,\dots,F_k)$ is a
maximal regular sequence in $S$, then
\[\Hom_A(K,A)=\Hom_S(K,A)=\Ext^{k+1}(K,S)\]
is independent of the particular maximal regular sequence!  Thus one quotient
$S/(F_0,\dots,F_k)$ will be Gorenstein only if any such quotient is.  We have
thus verified the remark made immediately following the definition of
Gorenstein rings.  In particular, since the quotient
$K[Z_0,\dots,Z_n]/(Z_0,\dots,Z_n)\cong K$ is trivially Gorenstein, it follows
that $K[Z_0,\dots,Z_n]/(F_0,\dots,F_n)$ is  Gorenstein for any maximal regular 
sequence $(F_0,\dots,F_n)$, thus completing the proof of Theorem CB\ref{thmcb9}.

\begin{proof}[Proof of Proposition~\ref{prop:nine}]
The argument here will be by induction on $k$, the inductive step being easy
and the hard part being the first case $k=0$.  For the inductive step, we just
observe that since $F_0$ is a nonzerodivisor for the module $M$, we have an
exact sequence of $S$-modules
\[0\to M\to M\to M/F_0M\to 0\]
where the first map is multiplication by $F_0$. Moreover, since $F_0\in
\mathfrak{m}$ annihilates the module $K=S/\mathfrak{m}$, the induced map on
$\Ext^*(K,M)$ is zero.  Thus we have for each $l$ a three-term exact sequence
\[0\to \Ext^l(K,M)\to \Ext^l(K,M/F_0M)\to \Ext^{l+1}(K,M)\to 0\]
which establishes the inductive step: assuming the result for the module
$M/F_0M$, with its maximal regular sequence $F_1,\dots,F_k$, we deduce first
that $\Ext^i(K,M)=0$ for $i\le k$ and then that
\begin{equation*} 
\begin{split}
\Ext^{k+1}(K,M)&\cong  \Ext^k(K,M/F_0M)\\
&\cong \Hom(K,M/(F_0,\dots,F_k)M)\ne 0.
\end{split}
\end{equation*}

The hard part, as we indicated, is the initial step.  We know that if $k\ge
0$---i.e.,  $\mathfrak{m}$ does contain a nonzerodivisor for $M$---then
$\Hom(K,M)=0$; we have to establish that conversely if $k=-1$---that is, if
every element of $\mathfrak{m}$ is a zerodivisor for $M$---then $\Hom(K,M) \ne
0$.  This amounts to the statement of
\renewcommand{\qed}{}
\end{proof}

\begin{prop} \label{prop:ten}
If every element $F\in \mathfrak{m}$ kills some element of $M$, then there
exists an element $\alpha\in M$ killed by all of $\mathfrak{m}$.
\end{prop}

\begin{proof} For each element $\alpha\in M$, let $I_\alpha\subset S$ be the
ideal $\Ann(\alpha)$ of elements of $S$ that kill $\alpha$; our hypothesis is
that $\mathfrak{m}$ is the union of the ideals $I_\alpha$.  Of course, we need
only look at the maximal elements of the set of ideals $\{I_\alpha\}$, and
these are prime ideals: if $I_\alpha$ is maximal, and $FG\in I_\alpha$ and
$G\notin I_\alpha$ (i.e., $G\alpha\ne 0)$, then $I_{G\alpha}\supset I_\alpha$,
and hence $I_{G\alpha}=I_\alpha$ and $F\in I_{G\alpha}=I_\alpha$.  In general,
maximal elements of the set of annihilators of elements of a module $M$ are
called \emph{associated primes} of $M$.

We claim next that since $S$ is Noetherian and $M$ is finitely generated, there
can be only finitely many such primes: if $I_\alpha$ is any such prime, we have
a sequence
\[0\to A\cdot \alpha\cong A/I_\alpha\to M\to M'\to 0.\]
Note that since $A/I_\alpha$ is an integral domain, if $\beta\in A\cdot \alpha$
is any nonzero element, then $I_\beta=I_\alpha$.  Thus, if $I_\gamma$ is any
other such prime, $A\cdot \alpha\cap A\cdot \gamma=(0)$, so that $\gamma$ is an
associated prime of $M'$.  If $M'$ has only finitely many associated primes,
then it follows that $M$ does, and the result follows by Noetherian induction.

We have thus expressed the maximal ideal $\mathfrak{m}$ of $S$ as a finite
union of prime ideals $I_\alpha$. It follows that $\mathfrak{m}=I_\alpha$ for
some $\alpha$, i.e., there exists an element of $M$ killed by all of
$\mathfrak{m}$.  If $S$ contains an infinite field, as in the original
geometric case, this is trivial, as a vector space over an infinite field
cannot be the union of finitely many proper subspaces; otherwise, it is
what is called in commutative algebra the ``prime avoidance lemma''.  See, for
example, Eisenbud \cite[section 3.2]{E}.
\end{proof}

\section*{Part II: The future?}
\subsection*{{\rm 2.1.} Cayley-Bacharach Conjectures}
The statement of Theorem CB\ref{thmcb9} represents our best understanding (to
date) of the phenomena observed in the various formulations of Cayley-Bacharach
preceding it; having expressed the result as a relationship between the Hilbert 
functions of residual subsets of an arithmetically Gorenstein subscheme of
$\mathbb{P}^n$, we do not see at present how to extend it to any larger class
of objects.  What we would like to propose instead is an extension of the 
theorem in a different direction, toward a collection of inequalities on the
Hilbert function of a subscheme of a complete intersection.

To see this most clearly, we restrict our attention to the case of a complete
intersection $\Gamma\subset \mathbb{P}^n$ of $n$ quadrics; we shall give the
general version at the end.  Cayley-Bacharach says that any homogeneous 
polynomial of degree $2n-(n+1)=n-1$ on $\mathbb{P}^n$ containing a subscheme $
\Gamma_0$ of colength 1 of $\Gamma$ (that is, of degree $2^n-1)$ must contain
all of $\Gamma$; we consider now what can be said about hypersurfaces of other
degrees.  Of course, Cayley-Bacharach also says that $\Gamma$ does impose
independent conditions on polynomials of degree $n$, so there will exist 
polynomials of degree $n$ vanishing on a subscheme of $\Gamma$ of colength 1 but
not on $\Gamma$.  On the other hand, by B\'ezout, a hyperplane in $\mathbb{P}^n$
can contain a subscheme of $\Gamma$ of degree at most $2^{n-1}$.

To summarize what we know, define for any given $n$ a function $d$ by saying
that $d(k)$ is the largest degree of a subscheme $\Gamma_0$ of a complete
intersection $\Gamma$ such that there is a polynomial $F(Z)$ of degree $k$
vanishing on $\Gamma_0$ but not on $\Gamma$.  Throwing in the trivial case of
degree 0, we have then the four values of $d$:
\begin{align*}
&\text{If }k=n, \text{ then }d(k)=2^n-1\\
&\text{If }k=n-1, \text{ then }d(k)=2^n-2\\
&\cdot\\
&\cdot\\
&\cdot\\
&\text{If }k=1, \text{ then }d(k)=2^n-2^{n-1}\\
&\text{If }k=0, \text{ then }d(k)=2^n-2^n
\end{align*}
and the pattern is clear.  We may make, accordingly, the 

\begin{concb} \label{concb10}
Let $\Gamma$ be a complete intersection of $n$ quadrics in $\mathbb{P}^n$.  If
$X\subset \mathbb{P}^n$ is any hypersurface of degree $k$ containing a
subscheme $\Gamma_0$ of degree strictly greater than $2^n-2^{n-k}$, then $X$
contains $\Gamma$.
\end{concb}

Note that this conjecture is sharp, if true: for any $k<n$, we can find a
complete intersection $\Gamma\subset \mathbb{P}^n$ containing a complete
intersection $\Omega\subset\mathbb{P}^{n-k}\subset \mathbb{P}^n$; by 
Cayley-Bacharach the residual scheme $\Gamma_0$ to $\Omega$ in $\Gamma$ will
then lie on a hypersurface of degree $k$ not containing $\Gamma$. In fact,
these are the only known examples of equality; we may make the further 
conjecture that if $X$ is a hypersurface of degree $k$ with $\deg(X\cap
\Omega)=2^n-2^{n-k}$, the scheme residual to $X\cap \Omega$ in $\Omega$ is a
complete intersection of quadrics in a subspace $\mathbb{P}^{n-k}$.

It should be said that taking $\Gamma$ a complete intersection of quadrics
specifically is not essential; there is a form of the conjecture applicable to
arbitrary zero-dimensional complete intersection subschemes of $\mathbb{P}^n$. 
As might be expected, though, this form is substantially more complicated, and
the pattern of known results less apparent \cite{EGH2}.

It should also be said that we did not arrive at this conjecture by trying to
extrapolate (or interpolate) from the classical statement of Cayley-Bacharach,
as we do above; rather, as described in Eisenbud-Green-Harris \cite{EGH1}, we
were led to this by extending a series of results and conjectures in
Castelnuovo  theory from  Eisenbud-Harris \cite{EH1} and made the connection to
the  Cayley-Bacharach Theorem after the fact.

In the remainder of this paper we shall be concerned with proving cases of the 
Cayley-Bacharach conjecture.  In order to do this, we start by introducing a
related conjecture.

\begin{concb} \label{concb11}
Let $\Gamma$ be any subscheme of a zero-dimensional complete intersection of
quadrics in any projective space $\mathbb{P}^r$. If $\Gamma$ fails to impose
independent conditions on hypersurfaces of degree $m$, then
\[\deg(\Gamma)\ge 2^{m+1}.\]
\end{concb}

Here again this statement is sharp, if true: a complete intersection in
$\mathbb{P}^{m+1}$ provides examples of equality for each $m$.  As in the case
of Conjecture CB\ref{concb10}, moreover, these are the only examples, and we
may conjecture further that equality holds in Conjecture CB\ref{concb11} if and
only if $\Gamma$ is itself a complete intersection of quadrics in
$\mathbb{P}^{m+1}$.

\begin{thm} \label{thm:eleven}
The following are equivalent\textup{:}

\textup{a.} Conjecture \textup{CB\ref{concb10}} for all $k$ and $n$\textup{;}

\textup{b.} Conjecture \textup{CB\ref{concb11}} for all $m$.
\end{thm}

\begin{proof} 
We first prove that Conjecture CB\ref{concb11} for a given value of $m$ implies 
Conjecture CB\ref{concb10} in case $k=n-m-1$.  We do this simply  by applying
the seventh version of the Cayley-Bacharach Theorem.  To begin with, assume
Conjecture CB\ref{concb11} for a given value  of $m$, and let $\Omega\subset
\mathbb{P}^n$ be a complete intersection of quadrics.  Let $X$ be any
hypersurface of degree $k=n-m-1$ not containing $\Omega$, and let $\Gamma$ be
the subscheme of $\Omega$ residual to the intersection $\Omega\cap X$.  By 
Cayley-Bacharach $\Gamma$ must fail to impose independent conditions on
hypersurfaces of degree $m=n-1-k$.  By assumption $\deg(\Gamma)\ge 2^{n-k}$ and
correspondingly $\deg(X\cap\Omega)\le 2^n-2^{n-k}$.

Now assume Conjecture CB\ref{concb10} for all $n$ and $k$.  Let $\Gamma$ be any
subscheme of a complete intersection of quadrics, and suppose that $\Gamma$
fails to impose independent conditions on hypersurfaces of degree $m$. 
Assuming that $\Gamma$ spans a projective space $\mathbb{P}^n$, take $\Omega$ a
complete intersection of quadrics in $\mathbb{P}^n$ containing $\Gamma$, and
let $\Gamma'\subset \Omega$ be the subscheme of $\Omega$ residual to $\Gamma$. 
By Cayley-Bacharach, $\Gamma'$ lies on a hypersurface of degree $n-1-m$ not
containing $\Omega$; it follows that $\deg(\Gamma')\le 2^n-2^{m+1}$ and hence
that $\deg(\Gamma)\ge 2^{m+1}$.
\end{proof}

Note as well that, by the proof of Theorem~\ref{thm:eleven}, Conjecture 
CB\ref{concb11} for all $m\le m_0$ implies Conjecture CB\ref{concb11} for all
$k$ and $n$ such that
\[n-k\le m_0+1;\]
and since Conjecture CB\ref{concb10} is immediate in case $k=n$ or $n-1$, it
follows that Conjecture CB\ref{concb11} for all $m\le m_0$ implies Conjecture 
CB\ref{concb10} for all $k$ and $n$ with $n\le m_0+3$.

Finally, we formulate a general version of our Conjecture CB\ref{concb11} that
does not require quadrics.

\begin{concb} \label{concb12}
Let $\Gamma$ be any subscheme of a zero-dimensional complete intersection of
hypersurfaces of degrees $d_1\le \dots\le d_n$ in a projective space
$\mathbb{P}^n$.  If $\Gamma$ fails to impose independent conditions on
hypersurfaces of degree $m$, then
\[\deg(\Gamma)\ge e\cdot d_s\cdot d_{s+1} \cdot\, \cdots\, \cdot d_n\]
where $e$ and $s$ are defined by the relations
\[\sum^n_{i=s} (d_i-1)\le m+1<\sum^n_{i=s-1} (d_i-1)\]
and
\[e=m+1-\sum^n_{i=s}(d_i-1).\]
\end{concb}

Note that this is the analogue of our second Cayley-Bacharach conjecture.  It
may in turn be translated, by an argument generalizing
Theorem~\ref{thm:eleven}, into a statement analogous to Conjecture
CB\ref{concb10}.

\subsection*{{\rm 2.2.} A proof of Conjecture CB\ref{concb10} in case  $r\le 7$}
We shall here prove Conjecture CB\ref{concb11} in case $m\le 4$ and
correspondingly Conjecture CB\ref{concb10} when $r\le 7$.

We start with an arbitrary zero-dimensional subscheme $\Gamma\subset
\mathbb{P}^r$.  As before, we introduce the coordinate ring $S=S(\Gamma)$ of $
\Gamma$, and we let $R=S/(L)$ be the quotient of $S$ by a general linear form
$L$.  $R$ is then a graded local Artinian ring over the field $K$, generated by
its first graded piece $R_1$, whose dimension as a vector space over $K$ we
shall call $n$. By the \emph{ideal} of $R$ we shall mean the ideal of relations
among the generators of $R$, that is, the kernel $I$ of the surjection from
$\Sym^*R_1$ onto $R$.  We may then translate our geometric hypotheses on the
scheme $\Gamma\subset \mathbb{P}^r$ into statements about the ring $R$ as
follows:

The degree of $\Gamma$ is the length of $R$ (i.e., the dimension of $R$ as a
vector space over $K)$.

The dimension $n$ of $R_1$ as a vector space over $K$ is the dimension of the
span of $\Gamma$ in $\mathbb{P}^r$.

The statement that $\Gamma$ fails to impose independent conditions on
hypersurfaces of degree $m$ is equivalent to the statement that the $(m+1)$st
graded piece $R_{m+1}$ of $R$ is nonzero.

The statement that $\Gamma$ is a subscheme of a complete intersection of
quadratics amounts to the assertion that the ideal of $R$ contains a regular 
sequence of length $n$ in degree 2.

With these conventions, the result from which our conjecture will follow in
case $r\le 7$ is the 

\begin{prop} \label{prop:twelve}
Let $R$ be a graded local Artinian ring over the field $K$, generated by its
first graded piece $R_1$\textup{;} suppose that the ideal of $R$ contains a
regular sequence of length $n=\dim_{K}(R_1)$ in degree $2$.  For $k\le 5$, if
the $k$th graded piece $R_k$ of $R$ is nonzero, then the length of $R$ is at
least $2^k$.
\end{prop}

\begin{proof} 
Note first that we may as well assume $R$ to be Gorenstein: if the socle of $R$
has vector space dimension strictly bigger than 1, we could simply replace $R$
by the quotient of $R$ by any subspace of the socle not containing $R_k$ to
obtain a ring of smaller length.

Now, we break up the argument into two cases: If we assume that the ideal $I$
of $R$ contains a reducible quadric, then as we shall see we may reduce the
problem to one of rings $R'$ with socles in lower degree and proceed by
induction.  If, on the other hand, $I$ contains no such element, we obtain lower
bounds on the Hilbert function of $R$ that suffice, at least in case $k\le 5$,
to establish the desired inequality on the length of $R$.

Assume first that the ideal $I$ of $R$ contains a reducible quadric, that is,
that we have $xy=0$ for some pair of nonzero elements $x,y\in  R_1$ (we do not
assume $x$ and $y$ are linearly independent).  From the exact sequence of
$R$-modules 
\[0\to (x)\to R\to R/(x)\to 0\]
we have the obvious equality
\[l(R)=l((x))+l(R/(x))\]
where $l$ denotes length.  On the other hand, the ideals $(x)$ and $(y)$ in $R$
have as well the structure of quotient rings of $R$: for example, $(y)$ is just
the image of the map $m_y:R\to R$ of $R$-modules given by multiplication by $y$
and so is isomorphic to the quotient ring $R/\Ann(y)$. Moreover, since $xy=0$
in $R$, $x$ is in the kernel of $m_y$, so that in fact the ring $(y)$ is a
quotient of the ring $R(x)$.  In particular, the length of $(y)$ is less than
or equal to that of $R/(x)$, so that we have
\[l(R)\ge l((x))+l((y)).\]

Now, the rings $(x)$ and $(y)$, being quotients of the ring $R$, are likewise
generated by their first graded pieces; similarly, their ideals contain regular 
sequences of maximal length in degree 2.  Moreover, they are again Gorenstein. 
To see this, observe that every ideal of $R$ contains the socle $R_k$ of $R$,
so that for example $R_k\subset (x)$ and is indeed contained in the socle of
the ring $(x)$.  At the same time, since $(x)$ is a submodule of the $R$-module
$R$, any element of the socle of $(x)$ is killed by any element of $R_1$ and so
is an element of the socle of $R$.  Indeed, we see that $(x)$ and $(y)$ have
socles in degree exactly $k-1$, and by induction we may conclude that each has
length at least $2^{k+1}$.  Thus
\[l(R)\ge l((x))+l((y))\ge 2^k,\]
and we are done in this case.

Suppose now that the ideal of $R$ contains no reducible element in degree 2. 
If as before we let $n$ denote the dimension of the vector space $R_1$, then in
the projective space $\mathbb{P}(\Sym^2(R_1))$ the locus $\Sigma$ of reducible 
elements has dimension $2n-2$.  Inasmuch as the kernel $I_2$ of the map
\[\Sym^2(R_1)\to R_2\]
does not meet  $\Sigma$, we may immediately conclude that the codimension of
$I_2$ in $\Sym^2(R_1)$ is at least $2n-1$.  We have thus
\[\dim_K(R_2)\ge 2n-1.\]

Now, under the hypothesis that the ideal of $R$ contains a regular sequence
$f_1,\dots,f_n$ of maximal length in degree $2$, it is a quotient of the ring
\[\Sym^*(R_1)/ (f_1,\dots,f_n),\]
 which has length $2^n$ and socle in degree $n$. 
Thus we must have $n\ge k$, and the result we are after is immediate if $n=k$,
so we may as well assume that $n\ge k+1$.  The Hilbert function $h_R$ thus
satisfies
\begin{align*}
h_R(0)&=1\\
h_R(1)&=n\ge k+1\\
h_R(2)&\ge 2n-1
\end{align*}
from which we conclude immediately that $k\ge 4$: if $k=1$, we would have
$h_R(2)=0$; if $k=2$, we would have $h_R(2)=1$; and if $k=3$, we would have
$h_R(2)=h_R(1)$.  Moreover, in case $k=4$ or 5 the entire Hilbert function is
determined by the first three values above, and we can just add up the 
inequalities we have to obtain the result.  For example, in case $k=4$ we have
$h_R(3)=h_R(1)=n\ge k+1$ and $h_R(4)=1$.  Adding it all up, we see that the
length of $R$ must be
\begin{equation*} 
\begin{split}
l(R)&=\sum h_R(k)\\
&\ge1+n+2n-1+n+1\\
&=4n+1\\
&\ge 4k+5\\
&=21\\
&>2^k,
\end{split}
\end{equation*}
so we are done.  Similarly, in case $k=5$ we have $h_R(3)=h_R(2)\ge 2n-1,
h_R(4)=h_R(1)=n$, and $h_R(5)=1$, so that
\begin{equation*} 
\begin{split}
l(R)&=\sum h_R(k)\\
&\ge 1+n+2n-1+2n-1+n+1\\
&=6n\\
&\ge 6k+6\\
&=36\\
&>2^k. \qquad  \qed
\end{split}
\end{equation*}
\renewcommand{\qed}{}
\end{proof}

\bibliographystyle{amsalpha}
\begin{thebibliography}{EGH2}

\bibitem[Ba]{Ba}
I. Bacharach, \emph{Uber den Cayley'schen Schnittpunktsatz}, Math.
Ann. \textbf{26} (1886), 275--299.

\bibitem[BM]{BM} C. B. Boyer and U. Merzbach, \emph{A history of mathematics},
second ed., Wiley, New York, 1991. \MR{92a:01003}

\bibitem[BN]{BN} A. Brill and M. Noether, \emph{Uber die algebraischen
Functionen und ihre Anwendung in der Geometrie}, Math. Ann. \textbf{7}
(1874), 269--310.

\bibitem[Co]{Co} H. S. M. Coxeter, \emph{Projective geometry}, second ed.,
Univ. of  Toronto Press, Toronto, 1974. \MR{49:11377}

\bibitem[Ca]{Ca} A. Cayley, \emph{On the intersection of curves}, Cambridge
Math. J. \textbf{3} (1843), 211--213; \emph{Collected math papers} I, vols.
25--27, Cambridge Univ. Press, Cambridge, 1889.

%\bibitem[Ch1]{Ch1} M. Chasles, \emph{Apercu Historique sur l'origine et la
%d\'evelopment des m\'ethodes en g\'eometrie} (1837), third ed.,
%Gauthier-Villars, Paris, 1889.

\bibitem[Ch]{Ch} M. Chasles, %\bysame, 
\emph{Trait\'e des sections coniques},
Gauthier-Villars, Paris, 1885.

\bibitem[DGO]{DGO} E. Davis, A. V. Geramita, and F. Orecchia, \emph{Gorenstein 
algebras and the Cayley-Bacharach Theorem}, Proc. Amer. Math. Soc. 
\textbf{93} (1985), 593--597. \MR{86k:14034}

\bibitem[E]{E} D. Eisenbud, \emph{Commutative algebra.  With a view toward 
algebraic geometry}, Springer-Verlag, New York, 1994. \MR{1:322\ 960}

\bibitem[EH1]{EH1} D. Eisenbud and J. Harris, \emph{Castelnuovo theory}, Curves
in Projective Space, Univ. of Montreal Press, Montreal, 1982. \MR{84g:14024}

\bibitem[EH2]{EH2} \bysame, \emph{Schemes}: \emph{The language of modern 
algebraic geometry}, Wadsworth, Belmont, CA, 1992. (To be republished in
revised form by Springer-Verlag with the title \emph{Why schemes}?)
\MR{83k:14001}

\bibitem[EGH1]{EGH1} D. Eisenbud, M. Green, and J. Harris, \emph{Some conjectures
extending Castelnuovo Theory}, Ast\'erisque \textbf{218} (1993), 187--202.
\MR{95a:14057}

\bibitem[EGH2]{EGH2} \bysame, \emph{Hilbert functions and complete
intersections} (in
preparation).

\bibitem[EP]{EP} P. Ellia and C. Peskine, \emph{Groupes de points de $P^2:$
Caract\'ere et position uniforme}, Algebraic Geometry (L'Aquila 1988), A.
Sommese, A. Biancofiore, E. Livorni, Springer Lect. Notes in Math. 
\textbf{1417} (1990), 111--116.\MR{91f:14025}

\bibitem[GKR]{GKR} A. V. Geramita, M. Kreuzer, and L. Robbiano, \emph{
Cayley-Bacharach schemes and their canonical modules},  Trans. Amer. Math. Soc.
\textbf{339} (1993), 163--189. \MR{93k:14065}

\bibitem[H]{H} R. Hartshorne, \emph{Algebraic geometry}, Springer-Verlag, New
York, 1977. \MR{57:3116}

\bibitem[K]{K} M. Kline, \emph{Mathematical thought from ancient to modern 
times}, Oxford Univ. Press, Oxford, 1972. \MR{57:12010}

\bibitem[M]{M} F. S. Macaulay, \emph{Algebraic theory of modular systems},
Cambridge Tracts in Math., vol. 19, Cambridge Univ. Press, Cambridge, 1916.

\bibitem[Na]{Na} M. Nagata, \emph{The theory of multiplicity in general local
rings}, Proc. Internat.  Sympos. (Tokyo-Nikko, 1955), Sci.  Council of Japan,
Tokyo, 1956, pp. 191--226. \MR{18:637b}

\bibitem[No]{No} M. Noether, \emph{Uber ein Satz aus der Theorie der
algebraischen Funktionen}, Math. Ann. \textbf{6} (1873), 351--359.

\bibitem[S]{S} D. Struik, \emph{A source book in mathematics}, 1200--1800,
Harvard Univ. Press, Cambridge, MA, 1969. \MR{39:11}
\end{thebibliography}


\end{document}
