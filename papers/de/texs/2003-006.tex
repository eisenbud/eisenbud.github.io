%_ **************************************************************************
%_ * The TeX source for AMS journal articles is the publishers TeX code     *
%_ * which may contain special commands defined for the AMS production      *
%_ * environment.  Therefore, it may not be possible to process these files *
%_ * through TeX without errors.  To display a typeset version of a journal *
%_ * article easily, we suggest that you retrieve the article in DVI,       *
%_ * PostScript, or PDF format.                                             *
%_ **************************************************************************
%  Author Package
%% Translation via Omnimark script a2l, February 24, 2003 (all in one day!)
\controldates{10-JUL-2003,10-JUL-2003,10-JUL-2003,10-JUL-2003}
 
\RequirePackage[warning,log]{snapshot}
\documentclass{tran-l}
\issueinfo{355}{11}{November}{2003}
\pagespan{4397}{4426}
\dateposted{July 10, 2003}
\PII{S 0002-9947(03)03291-4}
\copyrightinfo{2003}{American Mathematical Society}
\usepackage[cmtip,all]{xy}


\newcommand{\reg}{\operatorname{regularity}}
\newcommand{\coker}{\operatorname{coker}}
\newcommand{\myim}{\operatorname{im}}
\newcommand{\iso}{\cong}
\newcommand{\tensor}{\otimes}
\newcommand{\myHom}{\operatorname{Hom}}
\newcommand{\Ext}{{\operatorname{Ext}}}
\newcommand{\Tor}{{\operatorname{Tor}}}
\newcommand{\tor}{{\operatorname{Tor}}}
\newcommand{\Sym}{\operatorname{Sym}}
\newcommand{\sym}{\operatorname{sym}}
\newcommand{\lin}{\operatorname{lin}}
\newcommand{\myth}{{\operatorname{th}}}
\newcommand{\A}{\mathcal{A}}
\newcommand{\myB}{\mathcal{B}}
\newcommand{\F}{\mathcal{F}}
\newcommand{\FF}{\mathbf{F}}
\newcommand{\G}{\mathcal{G}}
\newcommand{\GG}{\mathbf{G}}
\newcommand{\K}{\mathcal{K}}
\newcommand{\myH}{\mathbf{H}}
\newcommand{\I}{\mathcal{I}}
\newcommand{\LL}{\mathbf{L}}
\newcommand{\MM}{\mathbf{M}}
\newcommand{\myP}{\mathbf{P}}
\newcommand{\PP}{\mathbf{P}}
\newcommand{\RR}{\mathbf{R}}
\newcommand{\TT}{\mathbf{T}}
\newcommand{\W}{\mathcal{W}}
\newcommand{\Z}{\mathbf{Z}}
\newcommand{\gm}{{\mathfrak{m}}}

\theoremstyle{plain}
\newtheorem*{theorem1}{Reciprocity Theorem}
\newtheorem*{theorem2}{Theorem 3.1}
\newtheorem*{theorem3}{Proposition 2.1}
\newtheorem*{theorem4}{Theorem 2.2} 
\newtheorem*{theorem5}{Proposition 2.3}
\newtheorem*{theorem6}{Corollary 2.4}
\newtheorem*{theorem7}{Corollary 2.5}
\newtheorem*{theorem8}{Theorem 2.6}
\newtheorem*{theorem9}{Corollary 2.7}
\newtheorem*{theorem10}{Theorem 3.1}
\newtheorem*{theorem11}{Corollary 3.2}
\newtheorem*{theorem12}{Theorem 3.4}
\newtheorem*{theorem13}{Lemma 3.5}
\newtheorem*{theorem14}{Corollary 3.6}
\newtheorem*{theorem15}{Theorem 3.7}
\newtheorem*{theorem16}{Theorem 4.1}
\newtheorem*{theorem17}{Corollary 4.2}
\newtheorem*{theorem18}{Theorem 4.3}
\newtheorem*{theorem19}{Theorem 5.1}
\newtheorem*{theorem20}{Corollary 5.2}
\newtheorem*{theorem21}{Corollary 5.3}
\newtheorem*{theorem22}{Corollary 5.4}
\newtheorem*{theorem23}{Proposition 5.5}
\newtheorem*{theorem24}{Proposition 5.6}
\newtheorem*{theorem25}{Theorem 6.1}
\newtheorem*{theorem26}{Corollary 6.2}
\newtheorem*{theorem27}{Corollary 6.3}
\newtheorem*{theorem28}{Lemma 7.4}
\newtheorem*{theorem29}{Theorem 8.1}
\newtheorem*{theorem30}{Proposition 8.6}
\newtheorem*{theorem31}{Proposition 8.7}
\newtheorem*{theorem32}{Corollary 8.8}
\newtheorem*{theorem33}{Corollary 8.9}
\newtheorem*{theorem34}{Theorem 8.10}

\theoremstyle{remark}
\newtheorem*{remark1}{Remark $8.14$}

\theoremstyle{definition}
\newtheorem*{definition1}{Example 3.3}
\newtheorem*{definition2}{Example 3.3, continued}
\newtheorem*{definition3}{Example 7.1}
\newtheorem*{definition4}{Example 7.2. The rational normal curve}
\newtheorem*{definition5}{Example 7.3. The Horrocks-Mumford bundle in $\mathbf{P}^{4}$}
\newtheorem*{definition6}{Example 7.3, continued}
\newtheorem*{definition7}{Example 8.2}
\newtheorem*{definition8}{Definition 8.3. Partition monads}
\newtheorem*{definition9}{Example 8.4. Free resolutions}
\newtheorem*{definition10}{Example 8.5. Linear monads}
\newtheorem*{definition11}{Example 8.11. Beilinson's free monad}
\newtheorem*{definition12}{Example 8.12. Walter's monads}
\newtheorem*{definition13}{Example 8.13}

\begin{document}

\title[Sheaf cohomology and free resolutions]
{Sheaf cohomology and free resolutions\linebreak[1] over exterior algebras}
\author[D. Eisenbud]{David Eisenbud}
\address{Department of Mathematics, University of California Berkeley,
Berkeley, California 94720}
\email{eisenbud@math.berkeley.edu}

\author[G. Fl\o ystad]{Gunnar Fl\o ystad}
\address{Mathematisk Institutt, Johs. Brunsgt. 12,
N-5008 Bergen, Norway}
\email{gunnar@mi.uib.no}

\author[F.-O. Schreyer]{Frank-Olaf Schreyer}
\address{FB Mathematik, Universit\"{a}t Bayreuth
D-95440 Bayreuth, Germany}
\email{schreyer@btm8x5.mat.uni-bayreuth.de}

\date{December 1, 2001}
\thanks{The first and third authors are grateful to the NSF for
partial support during the preparation of this paper.
The third author wishes to thank MSRI for its hospitality.}
\subjclass[2000]{Primary 14F05, 14Q20, 16E05}
\begin{abstract}
We derive an explicit version of the
Bernstein-Gel'fand-Gel'fand (BGG) correspondence between bounded
complexes of coherent sheaves on projective space and minimal doubly
infinite free resolutions over its ``Koszul dual'' exterior algebra.
Among the facts about the BGG correspondence that we derive is that
taking homology of a complex of sheaves corresponds to taking the
``linear part'' of a resolution over the exterior algebra.

We explore the structure of
free resolutions over an exterior algebra. For example,
we show that such resolutions are eventually dominated by their
``linear parts" in the sense that erasing all terms of degree $>1$ in
the complex yields a new complex which is eventually exact.

As applications we give a construction of the Beilinson monad which
expresses a sheaf on projective space in terms of its cohomology by
using sheaves of differential forms. The explicitness of our version
allows us to prove two conjectures about the morphisms in the monad,
and we get an efficient method for machine computation of the
cohomology of sheaves. We also construct all the monads for a sheaf
that can be built from sums of line bundles, and show that they are often
characterized by numerical data.
\end{abstract}
\maketitle

\section*{Introduction}Let $V$ be a finite-dimensional vector space over a field $K$, and let
$W=V^{*}$ be the dual space. In this paper we will study complexes and
resolutions over the exterior algebra $E=\bigwedge V$ and their relation
to modules over $S=\sym \, W$ and sheaves on projective space $\myP (W)$.

In this paper we study the 
Bernstein-Gel'fand-Gel'fand (BGG) correspondence \cite{BGG}, usually
stated as an equivalence between the derived category of bounded
complexes of coherent sheaves on $\myP (W)$ and the stable category of
finitely generated graded modules over $E$.
Its
essential content is a functor $\RR $ from complexes of graded $S$-modules
to complexes of graded $E$-modules, and its adjoint $\LL $.
For example, if $M=\bigoplus _{i} M_{i}$ is a graded $S$-module (regarded
as a complex with just one term), then as a
bigraded
$E$-module $\RR (M)=\myHom _{K}(E,M)$, with differential
$\myHom _{K}(E,M_{i})\to \myHom _{K}(E,M_{i+1})$ defined from the multiplication map
on $M$. Similarly, for a graded $E$-module $P$, we have
$\LL (P)=S\otimes _{K}P$. In fact (Proposition 2.1) $\RR $ is an
equivalence from the category of graded
$S$-modules
to the category of linear complexes of free $E$-modules; here {\em linear\/} means essentially that the maps are represented by matrices
of linear forms. A similar statement holds for $\LL $.

We show that finitely generated modules $M$ go to left-bounded complexes
that are exact far to the right, and characterize the point at
which  exactness begins as the Castelnuovo-Mumford regularity of $M$.
A strong form of this is Theorem 3.7, of which the following is
a part:

\begin{theorem1}
If $M$ is a graded $S$-module and $P$ is a
graded
$E$-module, then $\RR (M)$ is an injective resolution of $P$ if and only if
$\LL (P)$ is a free resolution of $M$.
\end{theorem1}


Let  $\F $ be a coherent sheaf on projective space and  take
$M=\bigoplus _{d} \myH ^{0}(\F (d))$. The results above show that the complex
$\RR (M_{\geq r})
$
associated to the truncation of $M$
is acyclic for $r>>0$. If we take a minimal free resolution of the kernel
of the first term in this complex, we obtain a doubly infinite
exact free complex, independent of $r$, which we call
the {\em Tate resolution\/}  $\TT (\F )$:
\begin{equation*}\TT (\F ): \cdots \to T^{r-1}\to T^{r}=\myHom _{K}(E,M_{r})\to \myHom _{K}(E,
M_{r+1})\to \cdots. \end{equation*}
It was
first studied in Gel'fand \cite{Gel}. 
Our first main theorem  (Theorem 4.1) is that
the $e^{\myth }$ term of the Tate resolution is
$T^{e}(\F )=\bigoplus _{j}\myHom _{K}(E, \myH ^{j}(\F (e-j)))$; that is, 
it is made
from the cohomology of the twists of $\F $.
This leads to a new algorithm for computing sheaf
cohomology. We have  programmed
this method in the computer algebra system Macaulay2 of Grayson
and Stillman  [{\tt http://www.math.uiuc.edu/}\linebreak {\tt Macaulay2/}]. In
some cases it gives the fastest known computation of the
cohomology. 

We apply the Tate resolution to study
a result of Beilinson \cite{Bei}, which
gives, for each sheaf $\F $ on projective space, a complex
\begin{equation*}\dots \to \bigoplus _{j=0}^{n}\myH ^{j}(\F (e-j))\otimes _{K}
\Omega _{\myP ^{n}}^{j-e}(j-e)
\to \cdots .\end{equation*}
called the {\em Beilinson monad\/}, whose homology is precisely
$\F $ and whose terms depend only on the cohomology of a few twists of
$\F $.

Our second main result is a constructive version of
Beilinson's theorem \cite{Bei}, which clarifies its connection with the
BGG-correspondence (Theorem 6.1). See Decker and
Eisenbud \cite{DE} for details and for an implementation of the BGG
correspondence and the computation of the Beilinson monad. (That paper
also contains an introduction to the uses of the Beilinson monad.)

Beilinson's original paper
sketched a proof that leads easily to a weak form of the result, the
``Beilinson spectral sequence'', which determines the sheaf $\F $ only
up to filtration. That version is explained in the book of Okonek,
Schneider, and Spindler \cite{OSS}. Kapranov \cite{Kap1} and Ancona and
Ottaviani \cite{AO} have given full proofs. However, their use of the
derived category makes it difficult to compute the Beilinson monad
effectively, and also makes it hard to obtain information about the
maps in the monad.

Our construction of the Beilinson monad leads to new results about its
structure.  There are natural candidates for the linear components of
the maps in the monad for a sheaf $\F $; and given such a monad, there
are natural candidates for most of the maps in the monad of
$\F (1)$. Our techniques allow us to prove that these natural
candidates really do occur (Corollary 6.2 and Corollary 6.3).

A remarkable feature of the theory of resolutions
over the exterior algebra, not visible for
the corresponding theory over a polynomial ring, is that the linear
terms of any  resolution eventually predominate.
To state this precisely, we introduce the
{\em linear part\/} of a free complex $\FF $ over $S$ or $E$. The linear
part
is the complex obtained from $\FF $ by taking a minimal free
complex $\GG $ homotopic to $\FF $, and then erasing all terms of
absolute degree $>1$ from the matrices representing
the differentials of $\GG $. In fact, taking the linear
part is functorial in a suitable sense:
under the BGG correspondence it corresponds to the  homology
functor (Theorem 3.4).
Just as the homology of a complex is simpler than the complex,
one can often compute the linear part of a complex even when the
complex itself is mysterious.

Of course free resolutions may have maps with no linear
terms at all, that is, with linear part equal
to zero. And they can have infinitely many maps with nonlinear
terms unavoidably present (this is even the case for periodic
resolutions). But the linear
terms eventually predominate in the following
sense:

\begin{theorem2} If\/ $\FF $ is the free
resolution of  a finitely generated  module
over the exterior algebra $E$,
then the linear part of\/ $\FF $ is eventually exact.
\end{theorem2}


This predominance can take arbitrarily long to assert itself: the
resolution of the dual of the millionth syzygy of the residue field of
$E$ has a million linear maps followed by a map with
linear part 0, and linear dominance happens only at the
million and first term. In the case of a resolution
of a monomial ideal, however, Herzog and R\"{o}mer \cite{HR} have
shown that the linear part becomes exact after at most
$\dim _{k} V$ steps. It would be interesting to know more results of
this sort.

Beilinson \cite{Bei} also proved the existence of a different monad for a
sheaf $\F $, using the sheaves $\mathcal{O}_{\myP }(i)$ 
for $0\leq i\leq n = \dim \PP (W)$ in place of the $\Omega ^{i}(i)$. Bernstein-Gel'fand-Gel'fand
also introduced a ``linear'' monad using sums of line bundles and only
having maps given by matrices of linear forms.  In the last section we
show that such a monad ``partitions'' the cohomology of the sheaf into
a ``positive'' part that appears as the homology of the corresponding
complex of free $S$-modules and a ``negative'' part that appears as
the cohomology of the dual complex.  We explain how these and other
free monads of a sheaf $\F $ arise from the Tate resolution
$\TT (\F )$. We show that many such monads are characterized by simple
numerical data.

Basic references for the BGG correspondence are
Gel'fand \cite{Gel},
and Gel'fand-Manin \cite{GM}).
Much of the elementary material of this paper could be done for an
arbitrary pair of homogeneous Koszul algebras (in the sense of Priddy
\cite{Pri}) in place of the pair of algebras $S, E$. We use a tiny bit of
this for the pair $(E,S)$.  See Buchweitz \cite{Buc1} for a sketch of the
general case and a statement of general conditions under which the BGG
correspondence holds. Buchweitz has also written
a general treatment of the BGG correspondence over
Gorenstein rings \cite{Buc2}.  Versions of
Beilinson's theorem have been established for some other varieties
through work of Swan \cite{Swa}, Kapranov \cite{Kap1}, \cite{Kap2}, 
and Orlov \cite{Orl}.  Yet
other derived category equivalences have been pursued under the rubric of
``tilting" (see Happel \cite{Hap}). Fl\o ystad \cite{Flo1} gives a general
theory for Koszul pairs, and also studies how far the equivalences
can be extended to unbounded complexes.

The material of our paper grew from two independent preprints of
Eisenbud and Schreyer \cite{ES1} and Fl\o ystad \cite{Flo2}. Since there was
considerable overlap we wrote a more complete joint paper,
which also includes new joint results.  The original preprint by the
second author has also been altered so that the notation and
terminology are more aligned with the present paper.

The material in this paper has been applied to study the
cohomology of hyperplane arrangements (Eisenbud, Popescu, and
Yuzvinsky \cite{EPY}) and to constructing counterexamples to the minimal
free resolution conjecture for points in projective space (Eisenbud,
Popescu, Schreyer, and Walter \cite{EPSW}).  The technique developed here
for the Beilinson monad has been used by Eisenbud and Schreyer to
construct complexes on various Grassmannians that can be used to
compute and study Chow forms \cite{ES2}.  In a direction related to
Green's proof of the linear syzygy conjecture \cite{EW}, Eisenbud and
Weyman have found a general analogue for the Fitting lemma over {\bf Z/2}-graded algebras, including the exterior algebra.

This paper owes much to the experiments we were able to make
using the computer algebra system Macaulay2 of Grayson and Stillman,
and we would like to thank them for their help and patience with
this project. We are also grateful to Luchezar Avramov for
getting us interested in resolutions over exterior algebras.

\section*{1. Notation and background}

Throughout this paper we write $K$ for a fixed field, and $V, W$ for
dual vector spaces of finite dimension $v$ over $K$.  We give the
elements of $W$ degree 1, so that the elements of $V$ have degree
$-1$.  We write $E=\bigwedge V$ and $S=\Sym (W)$ for the exterior and
symmetric algebras; these algebras are graded by their {\em internal
degrees\/}, whereby $\Sym _{i}(W)$ has degree $i$ and $\bigwedge ^{j}V$ has
degree $-j$.  We think of $E$ as $\Ext _{S}^{\bullet }(K,K)$ and $S$ as
$\Ext _{E}^{\bullet }(K,K)$.

We will always write the index indicating the degree of
a homogeneous component of
a graded module as subscripts. Thus if $M=\bigoplus M_{i}$ is a graded module over
$E$ or $S$, then $M_{i}$ denotes the component of degree $i$.
We let $M(a)$ be the shifted module, so that
$M(a)_{b}=M_{a+b}$.
We write complexes cohomologically, with
upper indices and differentials of degree $+1$. Thus if
\begin{equation*}\FF : \cdots \to F^{i}\to F^{i+1}\to \cdots
\end{equation*}
is a complex,
then $F^{i}$ denotes the term of cohomological degree $i$.
We write $\FF [a]$ for the complex whose
term of cohomological degree $j$ is $F^{a+j}$.

We will write
$\omega _{S}=S\otimes _{K}\bigwedge ^{v}W$
for the module associated to the canonical bundle of $\myP (W)$;
note that $\bigwedge ^{v}W$ is a vector space concentrated in
degree $v$, so that $\omega _{S}$ is noncanonically isomorphic
to $S(-v)$. Similarly, we set
$\omega _{E}:=\myHom _{K}(E,K)=E\otimes _{K}\bigwedge ^{v}W$, which is
noncanonically isomorphic to $E(-v)$.
It is easy to check that for any graded vector space $D$ we have
$\myHom _{K}(E,D)\cong \omega _{E}\otimes _{K} D$
as left $E$-modules.
For any $E$-module $P$, we set $P^{*}:=\myHom _{K}(P,K)$.

We often use the fact that the exterior algebra is Gorenstein and
finite dimensional over $K$,
which follows from the fact that $\myHom _{K}(E,K)\iso E$ as above.
As a consequence, the dual of any exact sequence is exact and the notions of
free module, injective module, and projective module coincide.

We also use the notion of Castelnuovo-Mumford regularity.
The most convenient
definition for our purposes is that the Castelnuovo-Mumford
regularity of a graded $S$-module $M=\bigoplus _{i}M_{i}$ is the smallest integer
$r$ such that the truncation $M_{\geq r}=\bigoplus _{i\geq r}M_{i}$
is generated by $M_{r}$ and has
a {\em linear free resolution}---that is, all the maps in
its free resolution are represented by matrices of linear forms.
See for example
Eisenbud-Goto \cite{EG} or Eisenbud \cite{Eis}
for a discussion. The regularity of a sheaf $\F $ on projective space
(equal to the regularity of $\bigoplus _{d} \myH ^{0}(\F (d))$ if this module
is finitely generated) can also be expressed as the minimal
$r$ for which $\myH ^{i}(\F (r-i))=0$ for all $i>0$.

A free complex over $E$ or a graded free complex over $S$ is called {\em minimal\/} if all its maps can be represented by matrices with entries in
the appropriate maximal ideal. For example, any linear complex is
minimal.

\section*{2. The Bernstein-Gel'fand-Gel'fand correspondence}

In this section we give a brief exposition of the main
idea of Bernstein-Gel'fand-Gel'fand \cite{BGG}: a construction of a
pair of adjoint functors between the categories of complexes over $E$
and over $S$. However, we avoid a peculiar convention, used in the
original, according to which the differentials of complexes over $E$
were not homomorphisms of $E$-modules.

Let $e_{i}$ and $x_{i}$ be dual bases of $V$ and $W$ respectively, so that
$\sum _{i} x_{i}\otimes e_{i}\in W\otimes _{K} V$ corresponds to the identity
element under the isomorphism $W\otimes _{K} V = \myHom _{K}(W,W)$.
Let $A$ and $B$ be vector spaces. Giving a map
$A\otimes _{K} W\to ^{\alpha }B$
is the same as giving a map $A\to ^{\alpha '}B\otimes _{K} V$ (where the
tensor products are taken over $K$). For example, given $\alpha $ we
set $\alpha '(a) = \sum _{i} e_{i}\otimes \alpha (a\otimes x_{i})$.

We begin with a special case that will play a central role. We
regard a graded $S$-module $M=\bigoplus M_{d}$
as a complex with only one term, in cohomological degree 0, and
define $\RR (M)$ to be the complex
\begin{align*}
\cdots \overset {\phi }{\longrightarrow }\myHom _{K}(E, M_{d})&\overset {\phi
}{\longrightarrow }\myHom _{K}(E, M_{d+1})\overset {\phi }{\longrightarrow
}\cdots, \\
\phi :\ \alpha &\mapsto \bigl [ e\mapsto \sum _{i}
x_{i}\alpha (e_{i}e)\bigr ]. 
\end{align*}
Here $\myHom _{K}(E,M_{d})$ has cohomological index $d$, and a map
$\alpha \in \myHom _{K}(E, M_{d})$ has degree $t$ if it factors through the
projection from $E$ onto $E_{d-t}$. Note that the complex
$\RR (M)$ is {\em linear\/} in a strong sense:
the $d^{\myth }$ free module $\myHom _{K}(E, M_{d})\iso \omega _{E}\otimes M_{d}$
has socle in degree $d$;
in particular, all the maps are represented by matrices of linear
forms.

\begin{theorem3}
The functor $\RR $
is an equivalence between the category of graded left $S$-modules
and the category of linear free complexes over $E$ (those
for which the $d^{\myth }$ free module has socle in degree $d$).
\end{theorem3}
\begin{proof}  A collection of maps $\mu _{d}: W\otimes _{K} M_{d}\to M_{d+1}$
defines a module structure on the graded
vector space $\bigoplus M_{d}$ if and only if it satisfies
a commutativity and associativity condition
expressed by saying that,
for each $d$, the composition of the multiplication maps
\begin{equation*}W\otimes _{K} (W\otimes _{K} M_{d})\to W\otimes _{K} M_{d+1}
\to M_{d+2}
\end{equation*}
factors through $\Sym _{2}W \otimes _{K} M_{d}$. Since $\bigwedge ^{2} W$
is the kernel of $W\otimes _{K} W\to \Sym _{2} W$, this is the same
as saying that the induced map $\bigwedge ^{2} W\otimes _{K} M_{d}\to M_{d+2}$
is zero, or again that the map
$\phi ^{2}:\ \myHom _{K}(E_{v},M_{d})\to \bigwedge ^{2} V\otimes _{K} \myHom _{K}(E_{v},M_{d+2})$
is zero.
This last is equivalent to $\RR (M)$ being a complex.
As the whole construction is reversible, we are done.
\end{proof}


As a first step in extending $\RR $ to all complexes,
we consider the case of a module regarded as a complex with a single
term, but in arbitrary cohomological degree. Let $M$ be
an $S$-module, regarded as a complex concentrated in cohomological
degree 0. Then $M[ a]$ is a complex concentrated in
cohomological degree $-a$, and we set
\begin{equation*}\RR (M[ a])=\RR (M)[ a].
\end{equation*}
Now consider the general case of
a complex of graded $S$-modules
\begin{equation*}\MM : \cdots \to M^{i}\to M^{i+1}\to \cdots .
\end{equation*}
Applying $\RR $ to each $M^{i}$, regarded as a complex concentrated
in cohomological degree $i$, we get a double complex,
and we define $\RR (\MM )$ to be the total complex of this double complex.
Thus $\RR (\MM )$ is the total complex of
\begin{equation*}\xy \xymatrix {
&&&\\
\dotsb \ar [r]&\operatorname{Hom}_{K}(E,(M^{i+1})_{j})\ar [u]\ar [r]&
\operatorname{Hom}_{K}(E,(M^{i+1})_{j+1})\ar [u]\ar [r]&\dotsb \\
\dotsb \ar [r]&\operatorname{Hom}_{K}(E,(M^{i})_{j})\ar [u]\ar [r]&
\operatorname{Hom}_{K}(E,(M^{i})_{j+1})\ar [r]\ar [u]&\dotsb \\
{}&{}\ar [u]&{}\ar [u]&\\
}
\endxy \end{equation*}
where the vertical maps are induced by the differential of $\MM $ and
the horizontal complexes are the complexes $\RR (M^{i})$ defined above.
As $E$-modules we have
\begin{equation*}(\RR \MM )^{k}=\sum _{i+j=k} \myHom _{K}(E,(M^{i})_{j}),
\end{equation*}
where $(M^{i})_{j}$ is regarded as a vector space concentrated
in degree $j$. Thus, as a bigraded $E$-module, $\RR (\MM )=\myHom _{K}(E,\MM )$, and
the formula for the graded components is
\begin{equation*}\RR (\MM )^{i}_{j}=\sum _{m} \myHom _{K}(E_{m-j} ,(M^{i-m})_{m}).
\end{equation*}

The functor $\RR $ has a left adjoint $\LL $ defined in an
analogous way by tensoring with $S$: on a graded $E$-module $P=\sum P_{j}$
the functor $\LL $ takes the value
\begin{equation*}\LL (P): \cdots \to S\otimes _{K} P_{j}\to S \otimes _{K}
P_{j-1}\to \cdots ,
\end{equation*}
where the map takes $s\otimes p$ to $\sum _{i}x_{i}s\otimes e_{i}p$
and the term $S\otimes _{K} P_{j}$ has cohomological degree $-j$.
If $\PP $ is a complex of graded $E$-modules, then we can apply
$\LL $ to each term to get a double complex, and we define
$\LL (\PP )$ to be the total complex of this double complex,
so that
\begin{equation*}\LL (\PP )^{k}=\sum _{i-j=k}S\otimes _{K}(P^{i})_{j}
\quad \hbox {and}\quad \LL (\PP )^{i}_{j}=\sum _{m}S_{j-m}\otimes _{K}(P^{i+m})_{m}.
\end{equation*}

To see that $\LL $ is the left adjoint of $\RR $, we proceed
as follows. First, if $M$ and $P$ are
left modules over $S$ and $E$, respectively, then
\begin{equation*}\myHom _{S}(S\otimes _{K} P, M)=\myHom _{K}(P,M)=\myHom _{E}(P, \myHom _{K}(E,M)).
\end{equation*}
If now $\MM $ and $\PP $ are complexes of graded modules over $S$ and
$E$, we must prove that $\myHom _{S}(\LL (\PP ),\MM )\iso \myHom _{E}(\PP ,\RR (\MM ))$, where on each side we take the maps of modules
that preserve the internal and cohomological degrees and commute with
the differentials.  As a bigraded module, $\LL (P)=S\otimes _{K}P$, and
similarly for $\RR $.  Direct computation shows that these maps of
complexes correspond to the maps of bigraded $K$-modules
\begin{equation*}\phi =(\phi ^{i}_{j})\in \myHom _{\text{bigraded\ vector \ spaces}}(\PP ,\MM )
\end{equation*}
such that $\phi ^{i}_{j}:\ P^{i}_{j}\to M^{i-j}_{j}$ and
\begin{equation*}\phi d-d\phi =(\sum _{s} x_{s}\otimes e_{s})\phi ,
\end{equation*}
where $(\sum _{s} x_{s}\otimes e_{s})\phi $ takes an element $p\in P^{i}_{j}$
to
$(-1)^{i}\sum _{s} x_{s}\phi (e_{s}p)$.
We have proved:
\begin{theorem4}[Bernstein-Gel'fand-Gel'fand \cite{BGG}]
The functor $\LL $,
from the category of complexes of graded $E$-modules to the category
of complexes of graded $S$-modules, is a left adjoint to the functor
$\RR $.
\end{theorem4}


It is not hard to compute the homology of the complexes
produced by $\LL $ and $\RR $:

\begin{theorem5}
If $M$ is a graded $S$-module and
$P$ is a graded $E$-module, then

{$a)$} $\myH ^{i}(\RR (M))_{j}=\tor ^{S}_{j-i}(K,M)_{j}$, and

{$b)$} $\myH ^{i}(\LL (P))_{j}=\Ext _{E}^{j-i}(K,P)_{j}$.
\end{theorem5}


\begin{proof}  The $j-i^{\myth }$ free module in the free resolution
of $K$ over $E$ is $(\Sym _{j-i}(W))^{*}\otimes _{K} E$, which is generated
by the vector space $(\Sym _{j-i}(W))^{*}$ of degree $i-j$. We can
use this to compute the right-hand side of the equality in $b)$: the
$j^{\myth }$ graded component of the module of homomorphisms of this into $P$
may be identified with $\Sym _{j-i}(W)\otimes _{K} P_{i}$. The differential
is the same as that of $\LL (P)$, and part $b)$ follows. Part $a)$ is
similar (and even more familiar, from Koszul cohomology).
\end{proof}


It follows that the exactness of $\RR (M)$ or $\LL (P)$ is a familiar
condition. First the case of a module over the symmetric algebra:
\begin{theorem6}
$a)$ If $M$ is a finitely generated graded $S$-module,
then
the truncated complex
\begin{equation*}\RR (M)_{\geq d}: \myHom _{K}(E,M_{d})\to \myHom
_{K}(E,M_{d+1})\to \cdots \end{equation*}
is acyclic (that is, has homology only at $\myHom _{K}(E,M_{d})$) if
and only if $M$ is $d$-regular.
\end{theorem6}
\begin{proof}  By Proposition 2.3 applied to $M_{\geq d}$, the
given sequence is acyclic if and only if $M_{\geq d}$ has
linear free resolution.\end{proof}


Since any linear complex is of the form $\LL (P)$ for
a unique graded $E$-module $P$, it is
perhaps most interesting to interpret part $b)$ of
Proposition 2.3 as a statement about linear complexes
over $S$. The result below is implicitly used in Green's \cite{Gre} proof
of the linear syzygy conjecture.

We call a right bounded linear complex
\begin{equation*}\GG : \cdots \to G^{-2}\to G^{-1}\to ^{\phi } G^{0}
\end{equation*}
{\em irredundant\/} if it is a subcomplex of the
minimal free resolution of $\coker (\phi )$ (or equivalently of
any module whose presentation has linear part equal to $\phi $).
(Eisenbud and Popescu in \cite{EPo} called this property {\em linear exactness\/},
but to follow this usage would risk overusing the adjective
``linear''.)
\begin{theorem7}
Let $\GG $ be a minimal linear complex of
free $S$-modules ending on the right with $G_{0}$ as above,  and let
$P^{*}$ be the
$E$-module such that
$\LL (P^{*})=\GG $. The complex $\GG $ is
irredundant if and only the module $P$ is generated by $P_{0}$.
The complex $\GG $ is the linear part of a minimal free resolution
if and only if the module $P$ is linearly presented.
\end{theorem7}
\begin{proof} 
Let $\phi : G^{-1}\to G^{0}$ be the differential of $\GG =\LL (P^{*})$, let
\begin{equation*}\FF : \cdots \to F^{-2}\to F^{-1}\overset {\phi }{\longrightarrow }G^{0}
\end{equation*}
be the minimal free resolution of $\coker (\phi )$,
and let $\kappa :\GG \to \FF $ be a comparison map lifting the
identity on $G^{0}$.
(This comparison map is unique,
because $\FF $ is minimal and $\GG $ is linear.) By induction
one sees that the comparison map is an
injection if and only if $\myH ^{i}\GG _{-i}=0$ for all $i<0$,
and it is an isomorphism onto the linear part of $\FF $ if and
only if in addition $\myH ^{i}\GG _{1-i}=0$ for all $i<0$.
Proposition 2.3 shows that the first condition is
satisfied if and only if $P^{*}$ injects into a direct sum
of copies of $E$, while both conditions are true if and only
if the minimal injective resolution begins with
\begin{equation*}0\to P^{*}\to \omega _{E}^{a}\to \omega _{E}(-1)^{b}
\end{equation*}
for some numbers $a,b$. Dualizing, we get the desired linear presentation
\begin{equation*}E(1)^{b} \to E^{a} \to P \to 0 \end{equation*}
of $P$.
\end{proof}


We now return to the BGG-correspondence.
Both the functors $\LL $ and $\RR $ preserve
mapping cones and homotopies of maps of complexes. For mapping cones
this is immediate. For the second, note that two maps
$f,g:\FF \to \GG $
of complexes are homotopic if and only if the induced map
from $\GG $ to the mapping cone of
$f-g$ is split. This condition is preserved by any additive functor
that preserves mapping cones.

Recall that a free resolution of a right bounded complex
\begin{equation*}\MM : \cdots \to M^{i-1}\to M^{i}\to M^{i+1}\to \cdots \end{equation*}
of graded $S$-modules is a graded free complex $\FF $ with a
morphism $\FF \to \MM $, homogeneous of degree 0, which
induces an isomorphism on homology.
We say that $\FF $
is {\em minimal\/} if $K\otimes _{S}\FF $
has trivial differential. Every right bounded complex
$\MM $
of finitely generated modules has a minimal free resolution,
unique up to isomorphism. It is the minimal part of any free resolution.

The functors $\LL $ and $\RR $ give a general construction of resolutions.
\begin{theorem8}
For any complex of graded $S$-modules $\MM $, the complex
$\LL \RR (\MM )
$
is a free resolution of $\MM $ which surjects onto $\MM $; and
for any complex of graded $E$-modules $\GG $,
the complex
$\RR \LL (\GG )
$
is an injective resolution of $\GG $ into which $\GG $ injects.
\end{theorem8}


In fact we shall see that every free complex whose homology is $M$ up to finite length
comes as $\LL $ of a complex that agrees with $\RR (M)$ in high degrees.

An immediate consequence is:
\begin{theorem9}
The functors $\RR $ and $\LL $
define an equivalence 
\[D^{b}(S\textup{-mod})\iso D^{b}(E\textup{-mod}).\]
\end{theorem9}
\begin{proof}[Proof of Corollary $2.7$] The derived category $D^{b}(S\hbox {-mod})$ of bounded complexes
of finitely generated $S$-modules is equivalent to the derived
category of complexes of finitely generated $S$-modules with
bounded cohomology (that is, having just finitely many
cohomology modules; see for example Hartshorne \cite{Har}, III, Lemma 12.3),
and similarly for $E$. The functors $\LL $ and $\RR $ carry
bounded complexes into complexes with bounded cohomology. This is clear for 
$\LL $, and for $\RR $ it
follows from Corollary 2.4. Thus $\LL $ and $\RR $ are well defined, and by
Theorem 2.6
$\LL \RR$ and $\RR \LL $ are both equivalent to the identity.
\end{proof}
\begin{proof}[Proof of Theorem $2.6$]
The proofs of the two statements are similar,
so we treat only the first. (A slight simplification
is possible in the second case, since
finitely generated modules over $E$ have
finite composition series.)

Because $\LL $ is the left adjoint functor of $\RR $, there is a
natural map $\LL \RR (\MM )\to \MM $ adjoint to the identity map
$\RR (\MM )\to \RR (\MM )$. We claim that this is a surjective
quasi-isomorphism.

To see that it is a surjection,
consider a map $\phi :\; \MM \to \MM '$ such that the composite
$\LL \RR (\MM )\to \MM \to \MM '$ is zero. It follows that
the adjoint composition $\RR (\MM )\to \RR (\MM )\to \RR (\MM ')$
is also zero, and since the first map is the identity,
we get $\RR (\phi )=0$. Since $\RR $ is a faithful functor,
$\phi =0$, proving surjectivity.

The functor $\LL $ preserves direct limits because it is a left
adjoint, while the functor $\RR $ preserves direct limits because
$E$ is a finite-dimensional vector space. Thus it suffices
to prove our claim in the case where $\MM $ is a bounded complex
of finitely generated $S$-modules.

If $\MM $ has the form
\begin{equation*}\MM : \cdots \to M^{d}\to 0\to \cdots, \end{equation*}
then $\MM $ admits $M^{d}[ -d]$
(that is, the module $M^{d}$ considered as
a complex concentrated in cohomological degree $d$) as a subcomplex, and
the quotient is a complex of smaller length. Using the ``five lemma'',
the naturality of the map $\LL \RR (\MM ) \to \MM $, and the exactness
of the functor $\LL \RR $, the claims
will follow, by induction on the length of the complex,
from the case
where $\MM $ has the form $M[ -d]$ for some finitely generated
graded $S$-module $M$ and integer $d$. This reduces immediately
to the case $d=0$.

It thus suffices to see that
$\LL \RR (M)\to M$ is a quasi-isomorphism when $M$ is a finitely
generated graded $S$-module. Now $\RR (M)$ is the linear complex
$\myHom _{K}(E,M_{0})\linebreak\to \myHom _{K}(E,M_{1})\to \cdots $, so
$\LL \RR (M)$ is the total complex
of the following double complex:
\begin{equation*}\xy \xymatrix {
&\\
\dotsb \ar [r]&S\otimes _{K}\operatorname{Hom}_{K}(K,M_{1})\ar [u]\ar [r]&0\\
\dotsb \ar [r]&S\otimes _{K}\operatorname{Hom}_{K}(V,M_{0})\ar [r]\ar [u]
&S\otimes _{K}\operatorname{Hom}_{K}(K,M_{0})\ar [r]\ar [u]&0.
}
\endxy \end{equation*}
In this picture the terms below what is shown are all zero.
The terms of cohomological degree 0 in the total complex
are those along the diagonal going northwest from
$S\otimes _{K}\myHom _{K}(K,M_{0})$. The generators of
$S\otimes _{K}\myHom _{K}(K,M_{0})$ have internal degree 0, while those
of $S\otimes _{K}\myHom _{K}(K,M_{1})$ have internal degree 1, etc.

The $d^{\myth }$ row of this double complex is $S\otimes _{K}\myHom _{K}(E,M_{d})$,
which is equal to the complex obtained by tensoring the
Koszul complex
\begin{equation*}\dots \to S\otimes _{K}{\textstyle{\bigwedge ^{2}}} W \to S\otimes _{K} W \to S \to 0
\end{equation*}
with $M_{d}$. It is thus acyclic, its one cohomology module being
$M_{d}$, in cohomological degree $0$. The spectral sequence
starting with the horizontal cohomology of the double complex
thus degenerates, and we see that the cohomology of the total
complex $\LL \RR (M)$ is a graded module with
component of internal degree equal to $M_{d}$,
concentrated in cohomological degree 0.
Thus
$\LL \RR (M)$ is acyclic, and the Hilbert
function of $\myH ^{0}(\LL \RR (M))$ is the same as that of $M$.
As $\LL \RR (M)$ has no terms in
positive cohomological degree, and $M$ is in cohomological degree 0,
the surjection $\LL \RR (M)\to M$ induces a surjection
$\myH ^{0}(\LL \RR (M))\to M$, and we are done. (One can show
that $\LL \RR (M)$ is the tensor product, over $K$,
of the Koszul complex and $M$, the action of $S$ being
the diagonal action, but the isomorphism is complicated
to write down.) \end{proof}


Though the statement of Theorem 2.6 has an attractive simplicity,
it is not very useful in this form because the resolutions that are
produced are highly nonminimal (for example, the free resolutions
produced over $S$ are nearly always infinite).
Theorem 3.7 shows that a modification of this construction gives at
least an important part of the minimal free resolution.

\section*{3. The linear part of a complex}

If $A$ is a matrix over $E$, then we define the {\em linear part\/},
written $\lin (A)$, to be the matrix obtained by erasing all the terms
of entries of $A$ that are of degree $>1$.  For example, if
$a,b,c,d$ are linear forms of $E$, then
the linear part of
\begin{equation*}\begin{pmatrix}a&0\\
bc&d
\end{pmatrix}
\qquad \hbox {is}\qquad \begin{pmatrix}a& 0\\
0&d\end{pmatrix}
.
\end{equation*}
Taking the linear part is a functorial operation
on maps (see Theorem 3.4 below), but
taking the linear part of a matrix does not
always commute with change of basis.
For example, if $a,b,c$ are linear
forms,
\begin{equation*}d=\begin{pmatrix}a & 0 \\
0 & b
\end{pmatrix}
,
\qquad \hbox {and}\qquad e=\begin{pmatrix}1 & c \\
               0 & 1
\end{pmatrix}
, \end{equation*}
then
$\lin (de) \neq \lin (d)e. $

Suppose that $e:G\to H$ is a second map of free modules and that the
composition $ed=0$. It {\em need not\/} be the case that
$\lin (e)\lin (d)=0$; but if we assume in addition that $d(F)$ is in the
maximal ideal times $G$ and $e(G)$ is in the maximal ideal times $H$,
then $\lin (e)\lin (d)=0$ does follow. Thus, if $\FF $ is a
minimal free complex over $E$, we can define a new complex $\lin (\FF )$
by replacing each differential $d$ of $\FF $ by its linear part,
$\lin (d)$. Note that $\lin (\FF )$ is the direct sum of complexes
$\FF ^{(i)}$ whose $e^{\myth }$ term is a direct sum of copies
of $E(e+i)$ and whose maps are of degree 1. In general, we define
the linear part of any free complex $\FF $ to be the linear part of
a minimal complex homotopic to $\FF $.
\begin{theorem10}
Let $\FF $  be a free or injective resolution
of a finitely generated module over the exterior algebra $E$. The
linear part of $\FF $ is eventually exact.
\end{theorem10}
\begin{proof}  We treat only the case where $\FF $ is an injective resolution;
by duality, the statement for a free resolution is equivalent.
By Theorem 3.4 the linear part of $\FF $ is
the value of $\RR $ on the $S$-module $\Ext _{E}^{\bullet }(K,P)$.
Since any finitely generated $S$-module has finite regularity
(see Eisenbud and Goto \cite{EG}), it suffices by Corollary 2.4
to show that
$\Ext _{E}^{\bullet }(K,P)$ is a finitely generated
$S$-module. This was done by Aramova, Avramov,
and Herzog \cite{AAH}. For the reader's convenience we repeat the
argument: we prove that $\Ext _{E}^{\bullet }(K,P)$ is a finitely generated
$S$-module by induction on the length
of $P$. If $P=K$,
then $\Ext _{E}^{\bullet }(K,P)$ is free of rank 1 over $S$. If $P'$ is a
proper submodule of $P$, then from the exact sequence
\begin{equation*}0\to P' \to P \to P/P' \to 0
\end{equation*}
we get an exact triangle of $S$-modules:
\begin{equation*}\xy \xymatrix {
\Ext _{E}^{\bullet }(K,P/P') \ar [rr]\ar [dr]&&\Ext _{E}^{\bullet }(K,P')\ar [dl]\\
&\Ext _{E}^{\bullet }(K,P)
}
\endxy \end{equation*}
The two $S$-modules in the top row are finitely generated by
induction, and thus $\Ext _{E}^{\bullet }(K,P)$ is finitely generated too.
\end{proof}


If $P$ is an $E$-module, then we write $\lin (P)$ for the cokernel of
$\lin (d)$, where $d$ is the map in a minimal free presentation of
$P$. We can further define a family of modules connecting $P$ and
$\lin (P)$ as follows: Let $d$ be a minimal free presentation of $P$,
choose a representation of $d$ as a matrix, and let $e_{1},\dots ,e_{v}$ be
a basis of $V$.  Let $d'$ be the result of substituting $te_{i}$ for
$e_{i}$ in the entries of $d$, and then dividing each entry by $t$. The
entries of $d$ have no constant terms because $d$ is minimal, and it
follows that $d'$ is a matrix over $K[t]\otimes _{K} E$. Let $P'$ be
the cokernel of $d'$. It has
fibers $P$ at $t\neq 0$ and $\lin (P)$ at 0.
The module $P'$ may not be flat over $K[t]$,
but the module $P'[t^{-1}]$ is flat over $K[t,t^{-1}]$:
in fact, it is
isomorphic to the module obtained
from the trivial family $K[t,t^{-1}]\otimes _{K} P$ by pulling back
along the automorphism $e_{i}\mapsto te_{i}c$ of $E$.

\begin{theorem11}
If $P$ is a finitely generated $E$-module,
then any sufficiently high syzygy $Q$ of $P$ is a flat deformation
of its linear part $\lin (Q)$.
\end{theorem11}
\begin{proof}  If $Q$ is a sufficiently high syzygy, then by
Theorem 3.1 the linear part of the minimal resolution of $Q$
is the resolution of $\lin (Q)$, so that
(with the notation of the preceding paragraph) this free resolution of
$Q$ lifts to a free resolution of $\lin (Q')$ over $K[t]\otimes _{K} E$.
Thus $Q'$ is flat, and the result follows.
\end{proof}


\begin{definition1}
It is sometimes not so obvious what the linear part of the minimal
version of a complex will be, and in particular it may be hard
to read from the linear terms in a nonminimal version. For example,
suppose that $W$  has dimension 2 and that $x,y\in W$ is
a dual basis to $a,b\in V$. Consider the complex
\begin{equation*}\MM : 0\to S/(x,y^{2})\overset {x}{\longrightarrow
}S/(x^{2},y)(1)\to 0,
\end{equation*}
where the notation means that the class of 1 goes to the class of $x$.

Applying $\RR $ to $\MM $, we get the double complex
\begin{equation*}\xy \xymatrix {
0\ar [r]&E(1)\ar [r]^{a}&E\ar [r]&0\\
&0\ar [r]\ar [u]&E\ar [u]^{1}\ar [r]_{b}&E(-1)\ar [u]\ar [r]&0
}
\endxy \end{equation*}
whose total complex is
\begin{equation*}\RR (\MM )=\FF :\quad 0\to E(1)\oplus E
\overset {(\begin{smallmatrix}a&1\\
0&b
\end{smallmatrix})
}{\longrightarrow }E\oplus E(-1)\to 0.
\end{equation*}
Despite the presence of the linear terms
in the differential of $\FF $, the minimal complex
$\FF '$ homotopic
to $\FF $ is
\begin{equation*}\FF ': 0\to E(1)\overset {ab}{\longrightarrow }E(-1)\to 0,
\end{equation*}
so the differential of $\lin (\FF )$ is 0.

Fortunately, we can construct the linear part of a complex directly and
conceptually, without passing to a minimal complex or to matrices.
First note that if $\GG $ is a minimal free complex over $E$,
then giving its linear part is equivalent, by Proposition 2.1,
to giving the maps
$\phi _{i}: \myHom _{E}(K, G^{i}) \to V\otimes _{K} \myHom _{E}(K, G^{i+1})$
corresponding to the
linear terms in the differential of $\GG $.
If $\FF $ is a (possibly nonminimal) free complex homotopic to $\GG $,
then $\myHom _{E}(K, G^{i})=\myH ^{i} \myHom _{E}(K,\FF )$. We will construct
natural maps
$\psi _{i}:\myH ^{i} \myHom _{E}(K,\FF )\to V\otimes \myH ^{i+1}\myHom _{E}(K,\FF )$,
and prove that $\psi _{i}=\phi _{i}$.

We identify $S$ with $\Ext _{E}(K,K)$ and use the well-known
$\Ext _{E}(K,K)$-module structure on $\myH ^{\bullet }\myHom _{E}(K, \FF )$.
To formulate this explicitly,
we make use of the exact sequence
\begin{equation*}\eta : 0\to V\to E/(V)^{2}\to K\to 0.
\end{equation*}
The extension class
\begin{equation*}\eta \in \Ext ^{1}_{E}(K,V)=\Ext ^{1}_{E}(K,K)\otimes _{K} V=\myHom _{K}(W,\Ext ^{1}_{E}(K,K))
\end{equation*}
corresponds to the inclusion $W=\Sym _{1}W\subset \Sym W$.
Since $\FF $ is a free complex, the sequence
$\myHom (\eta ,\FF )$ is an exact sequence of complexes, and we obtain
the homomorphism $\psi _{i}:\myH ^{i} \myHom _{E}(K,\FF )\to V\otimes \myH ^{i+1}\myHom _{E}(K,\FF )$ from the connecting homomorphism
\begin{equation*}\delta _{i}:
W\otimes _{K} \myH ^{i}\myHom _{E}(K,\FF )= \myH ^{i} \myHom _{E}(V, \FF ) \to \myH ^{i+1} \myHom _{E}(K, \FF ).
\end{equation*}
\end{definition1}
\begin{theorem12}
If $\FF $ is a complex of free
modules over $E$,
then
\begin{equation*}\lin (\FF )=\RR (\myH ^{\bullet }\myHom _{E}(K, \FF )),
\end{equation*}
where the $S$-module structure
on
$\myH ^{\bullet }\myHom _{E}(K,\FF )$ is given by the Yoneda action of
$\Ext _{E}(K,K)$.
\end{theorem12}


\begin{proof}  We use the notation $\phi _{i},\psi _{i},\delta _{i}$ introduced just before the theorem. 
From the definition of
$\psi _{i}:  \myH ^{i}\myHom _{E}(K,\FF ) \to V \otimes _{K}\myH ^{i+1} \myHom _{E}(K, \FF ) $
 we see that it depends only on the homotopy class of $\FF $,
 so we may
assume that $\FF $ is minimal.
By Proposition 2.1 we may assume that
\begin{equation*}F^{i}=\myHom _{K}(E,M_{i})=E \otimes _{K} \myHom
_{K}(E_{v},M_{i}).
\end{equation*}
Let $1 \tensor a \in K \otimes _{K} \myHom _{K}(E_{v},M_{i})$ be a generator which is 
mapped by $\phi _{i}$ to $\sum v_{j} \otimes b_{j} \in V \otimes _{K} \myHom _{K}(E_{v},M_{i+1})$.
Let $s=v_{0} \wedge v_{1} \wedge \cdots \wedge v_{n} \in E_{v}$ be a generator
of the socle of $E$. To prove $\psi _{i}=\phi _{i}$ we have to
show that an element of the form
$w \tensor s \tensor a \in W\otimes _{K} E_{v} \tensor \myHom _{K}(E_{v},M_{i})=W \tensor _{K} \myHom _{E}(K,F^{i})=\myHom _{E}(V,F^{i})$
is mapped to 
$\{1 \mapsto \sum _{j} sw(v_{j})\tensor b_{j}\} \in E_{v} \tensor _{K} \myHom _{K}(E_{v},M_{i+1})=\myHom _{E}(K,F^{i+1})$
under the connecting homomorphism $\delta _{i}$.
   
The element $w\tensor s \tensor a$ corresponds to $\{ v \mapsto sw(v) \tensor a\}$ in $\myHom _{E}(V,F^{i})$.
We must lift $\{v\mapsto sw(v)\tensor a \}$ to an element
of $\myHom _{E}(E/(V)^{2},F^{i})$. The image of $1 \in E/(V)^{2}$ will be an element $c \in F^{i}=E\tensor _{K} \myHom _{K}(E_{v},M_{i})$
which satisfies $v\cdot c = s\tensor w(v) a$ for all $v \in V$,
and any such element defines a lifting.
We can take $c= s \lnot w \tensor a$.
The image of $w\otimes s\tensor a$ under the connecting homomorphism is the
map  $\{1 \mapsto d(c)\} \in \myHom _{E}(K,F^{i+1})$, where $d: F^{i} \to F^{i+1}$ is the 
differential of $\FF $. With
\begin{equation*}d(c)=s\lnot w \wedge d(a)= s\lnot w \wedge (\sum _{j}  v_{j} \tensor b_{j} + \hbox { higher terms}) = \sum _{j} s w(v_{j}) \tensor b_{j} 
\end{equation*}
we arrive at the desired formula.
\end{proof}


To understand the linear parts of complexes obtained from
the functor $\RR $, we will
employ a general result:
if the vertical differential of a suitable double complex splits,
then the associated total complex is homotopic to one built from
the homology of the vertical differential in a simple way.
\begin{theorem13}
Let $\FF $ be a double complex
\begin{equation*}\xy \xymatrix {
&&\\
\dotsb \ar [r]&F^{i+1}_{j}\ar [u]\ar [r]^{d_{hor}}&F^{i+1}_{j+1}\ar [r]\ar [u]&\dotsb \\
\dotsb \ar [r]&F^{i}_{j}\ar [u]^{d_{vert}}\ar [r]_{d_{hor}}&F_{j+1}^{i}\ar [r]
\ar [u]_{d_{vert}}&\dotsb \\
{}&{}\ar [u]&{}\ar [u]&\\
}
\endxy \end{equation*}
in some abelian category such that $F^{i}_{j}=0$ for $i\ll 0$.
Suppose that the vertical differential of $\FF $ splits, so that for
each $i,j$
there is a decomposition
$F^{i}_{j}=G^{i}_{j}\oplus d_{vert}G^{i-1}_{j}\oplus H^{i}_{j}$
such that the kernel of $d_{vert}$ in $F^{i}_{j}$ is
$H^{i}_{j}\oplus d_{vert}G^{i-1}_{j}$,
and such that $d_{vert}$ maps $G^{i-1}_{j}$ isomorphically to
$d_{vert}(G^{i}_{j})$.
If we write $\sigma :\; F^{i}_{j}\to H^{i}_{j}$ for the projection corresponding
to this decomposition, and
$\pi :\; F^{i}_{j}\to d_{vert}G^{i-1}_{j}\to G^{i-1}_{j}$
for the composition of the projection with the inverse of
$d_{vert}$ restricted to $G^{i-1}_{j}$, then the total complex of $\FF $ is
homotopic to the complex
\begin{equation*}\dots \to \bigoplus _{i+j=k}H^{i}_{j}\overset {d}{\longrightarrow }
\bigoplus _{i+j=k+1}H^{i}_{j}
\to \cdots \end{equation*}
with differential
\begin{equation*}d=\sum _{\ell \geq 0}\sigma (d_{hor} \pi )^{\ell }d_{hor}.
\end{equation*}
\end{theorem13}
\begin{proof}  We write $d_{tot}=d_{vert}\pm d_{hor}$ for the differential
of the total complex. Note first that $\sigma (d_{hor} \pi )^{\ell }d_{hor}$ takes
$H^{i}_{j}$ to $H^{i-\ell }_{j+1+\ell }$. Since $F^{i-\ell }_{j+1+\ell }=0$
for $\ell >>0$,
the sum defining $d$ is finite.

Let $F$ denote $\FF $ without the differential, that is, as a bigraded module.
We will first show that
$F$ is the direct sum of the three components
\begin{equation*}G=\bigoplus _{i,j} G^{i}_{j},
\quad d_{tot}G,
\quad \hbox {and   }
H=\bigoplus _{i,j} H^{i}_{j},
\end{equation*}
and that $d_{tot}$ is a monomorphism on $G$.

The same statements, with $d_{tot}$ replaced by
$d_{vert}$, are true by hypothesis.
In particular, any element of $F$ is a sum of elements of the form
$g'+d_{vert}g+h$ with $g'\in G^{i}_{j},\ g\in G^{i-1}_{j}$ and $h\in H^{i}_{j}$
for some $i,j$. Modulo $G+d_{tot}G+H$ this element can be written as
$d_{hor}(g)\in F^{i-1}_{j+1}$. As $F^{s}_{t}=0$ for $s<<0$, we may
do induction on $i$, and
assume that $d_{hor}g \in G+d_{tot}G+H$, so we see
that $F=G+d_{tot}G+H$.

Suppose
\begin{equation*}g'\in G=\bigoplus _{i+j=\ell } G^{i}_{j},
\quad g\in G=\bigoplus _{i+j=\ell -1} G^{i}_{j},
\quad 
h\in H=\bigoplus _{i+j=\ell } H^{i}_{j}
\end{equation*}
and $g'+d_{tot}g+h=0$; we must show that $g=g'=h=0$.
Write $g=\sum _{k=a}^{b}  g^{k-1}_{\ell -k}$ with
$g^{s}_{t}\in G^{s}_{t}$.
If $b-a=-1$, then
$d_{tot}=0$, and the desired result is
a special case of the hypothesis. In any case, there is no
component of $g$ in $G^{b}_{\ell -b-1}$, so
the
component of $d_{tot}g$ in
$G^{b}_{\ell -b}$ is equal to $d_{vert}g^{b-1}_{\ell -b}$. From the
hypothesis we see that $d_{vert}g^{b-1}_{\ell -b}=0$,
so $g^{b-1}_{\ell -b}=0$, and we are done by induction
on $b-a$. This shows that $F=G\oplus d_{tot}G\oplus H$ and that
$d_{tot}$ is an isomorphism from $G$ to $d_{tot}G$.

The modules $G\oplus d_{tot}G$ form a
double complex contained in $\FF $ that we will call $\GG $.
Since $d_{tot}:G\to d_{tot}G$ is
an isomorphism, the total complex $tot(\GG )$ is split exact.
It follows that the total complex $tot(\FF )$ is
homotopic to $tot(\FF )/tot(\GG )$, and the modules of
this last complex are isomorphic to $\bigoplus _{i+j=k}H^{i}_{j}$.
We will complete
the proof by showing that the induced differential on
$tot(\FF )/tot(\GG )$ is the differential $d$ defined above.

Choose $h\in H^{i}_{j}$. The image of $h$ under the induced
differential is the unique element $h'\in H$ such that
$d_{tot}h\equiv h'\ (\hbox {mod }G+dG)$.
Now
\begin{equation*}d_{tot}h
=d_{hor}h
\equiv \sigma d_{hor} h + (d_{vert}\pi ) d_{hor} h\ (\hbox {mod }G).
\end{equation*}
However,
\begin{equation*}d_{vert}\pi \equiv d_{hor}\pi \equiv \sigma (d_{hor}\pi ) + d_{vert}\pi (d_{hor}\pi )
\ (\hbox {mod }G+d_{tot}G).
\end{equation*}
Continuing this way, and using again the fact that $F^{i}_{j}=0$ for
$i<<0$, we obtain
\begin{equation*}d_{tot}h
\equiv \sum _{\ell }\sigma (d_{hor}\pi )^{\ell }d_{hor}h
\ (\hbox {mod }G+d_{tot}G),
\end{equation*}
as required.
\end{proof}


We apply Theorem 3.4 to complexes of the form
$\RR (\MM )$:
\begin{theorem14}
If $\MM $ is a left-bounded
complex of graded $S$-modules,
then
\begin{equation*}\lin (\RR (\MM ))=\bigoplus _{i}\RR (\myH ^{i}(\MM )),
\end{equation*}
where $\myH ^{i}(\MM )$ is regarded as a complex of one term, concentrated
in cohomological degree $i$. A similar statement holds for
the linear part of $\LL (\PP )$ when $\PP $ is a left-bounded
complex of graded $E$-modules.
\end{theorem14}
\begin{proof}  As $\MM $ is a left-bounded complex of finitely generated
modules, the double complex  whose total complex is $\RR (\MM )$
satisfies the conditions of Lemma 3.5. The
bigraded module underlying
$\RR (\myH ^{i}(\MM ))$ is precisely the module $H$ of
Lemma 3.5,
and the differential is the map $\sigma d_{hor}$ restricted to $H$.
This is a linear map. But the other terms in the sum
$d=\sum _{\ell }\sigma (\pi d_{hor})^{\ell }d_{hor}$ all involve two or
more iterations of $d_{hor}$, and are thus represented by matrices
whose entries have degree at least 2.
\end{proof}
\begin{definition2}
Note that the homology of $\MM $ is
$\myH ^{\bullet }(\MM )=K(-1)\oplus K(1)[-1]$.
We may write
$\lin (\FF ')=\RR (K(-1))\oplus \RR (K(1))[-1]$,
as required by Corollary 3.6.
\end{definition2}


Here is the promised information about the minimal resolution of a module:
\begin{theorem15}
$a)$ {\bf Reciprocity}: If $M$ is a finitely
generated graded $S$-module and $P$ is a finitely generated graded
$E$-module, then $\LL (P)$ is a free resolution of $M$
if and only if $\RR (M)$ is an injective resolution of $P$.

$ b)$ More generally, for any minimal
bounded complex of finitely generated
graded $S$-modules $\MM $, the linear part of the minimal
free resolution of $\MM $ is
$\LL (\myH ^{\bullet }(\RR (\MM )));
$
and
for any minimal bounded complex of finitely generated
graded $E$-modules $\PP $,
the linear part of the minimal injective resolution of $\PP $
is
$\RR (\myH ^{\bullet }(\LL (\PP ))).
$
\end{theorem15}
\begin{proof} 
The two parts of $b)$ being
similar, we prove only the first statement. By Theorem 2.6
the complex $\LL \RR (\MM )$ is a free resolution. The complex
$\RR (\MM )$ is left-bounded because $\FF $ is bounded and contains
only finitely generated modules. Thus we may apply
Corollary 3.6, proving the first statement.

For the reciprocity statement $a)$, suppose that $\LL (P)$ is a minimal free
resolution of
$M$. By part $b)$ the linear part of the minimal
injective resolution of $P$ is $\RR (\myH ^{\bullet }(\LL (P)))$. Since
$\LL (P)$ is a resolution of $M$, this is $\RR (M)$. All the
terms of cohomological degree $d$ of this complex have degree $-d$,
so there is no room for nonlinear differentials, and the linear part
of the resolution is the resolution. \end{proof}


\section*{4. Sheaf cohomology and exterior syzygies}

In this section we establish a formula for the free
modules that appear in resolutions over $E$.
Because $E$ is Gorenstein, it is
natural to work with doubly infinite resolutions:

A {\em Tate resolution\/} over $E$ is a doubly infinite free
complex
\begin{equation*}T: \cdots \to T^{d}\to T^{d+1}\to \cdots \end{equation*}
that is everywhere exact.

There is a Tate resolution naturally associated to a
coherent sheaf $\F $
on $\myP (W)$, defined as follows. Let
$M$ be a finitely generated graded
$S$-module representing $\F $,
for example $M=\bigoplus _{\nu \geq 0}\myH ^{0}(\F (\nu ))$.
If $d\geq \reg (M)$, then by Corollary 2.4
the complex $\RR (M_{\geq d})$ is acyclic. Thus,
if $d>\reg (M)$, then, since $\RR (M_{\geq d})$ is minimal,
$\myHom _{K}(E,M_{d})$ minimally covers
the kernel of the map $\myHom _{K}(E,M_{d+1})\to \myHom _{K}(E,M_{d+2})$

Fixing $d>\reg (M)$, we may complete $\RR (M_{\geq d})$ to
a minimal
Tate resolution $\TT (\F )$ by adjoining a free resolution of
\begin{equation*}\ker \bigl [\myHom _{K}(E,M_{d})\to \myHom _{K}(E,M_{d+1})\bigr ].
\end{equation*}
Since any two modules representing $\F $ are equal
in large degree, the Tate resolution
is independent of which $M$ and which large $d$ is chosen,
and depends only on
the coherent sheaf $\F $. It has the form
\begin{equation*}\begin{split}
\TT (\F )&: \cdots \to
T^{d-2}\to T^{d-1}\to \myHom _{K}(E, \myH ^{0}(\F (d)))\\
&\to \myHom _{K}(E, \myH ^{0}(\F (d+1)))
\to \cdots, \end{split}
\end{equation*}
where the $T_{i}$ are graded free $E$-modules.

The main theorem of this section expresses
the linear part of this Tate resolution in terms of the
$S$-modules $\bigoplus _{e} \myH ^{j}(\F (e))$ given by
the (Zariski) cohomology of $\F $. We regard
$\bigoplus _{e}\myH ^{j}(\F (e))$ as a complex of $S$-modules concentrated in
cohomological degree $j$.
\begin{theorem16}
If $\F $ is a coherent sheaf
on $\myP (W)$, then the
linear part of the Tate resolution $\TT (\F )$ is
$\bigoplus _{j} \RR (\bigoplus _{e} \myH ^{j}(\F (e)))$.
In particular,
\begin{equation*}T^{e}=\bigoplus _{j} \myHom _{K}(E,\myH ^{j}(\F (e-j))),
\end{equation*}
where $\myH ^{j}(\F (e-j))$ is regarded as a vector space of
internal degree $e-j$.
\end{theorem16}


A special case of the theorem appears without proof as Remark 3 after
Theorem 2 in Bernstein-Gel'fand-Gel'fand \cite{BGG}.
The proof below could be extended to
cover the case of a bounded complex of coherent sheaves, replacing the
cohomology in the formula with hypercohomology.
We will postpone the proof of Theorem 4.1
until the end of this section.

Rewriting the indices in Theorem 4.1, we emphasize
the fact that we can compute any part of the cohomology of $\F $
from the Tate resolution.
\begin{theorem17}
For
all $j,\ell \in \Z $,
$\myH ^{j}(\F (\ell )) = \myHom _{E}(K,T^{j+\ell })_{-\ell }.
$
\end{theorem17}


Corollary 4.2 provides the basis for
an algorithm
computing the cohomology of $\F $ with
any computer program that can provide free resolutions
of modules over the symmetric and exterior algebras,
such as the program Macaulay2 of Grayson and Stillman
[{\tt http://www.math.uiuc.edu/Macaulay2/}].
For an explanation of the algorithm in practical terms, see
Decker and Eisenbud \cite{DE}.

To prove Theorem 4.1 we will use
the reciprocity result Theorem 3.7. We actually prove
a slightly more general version,
involving local cohomology. We write $\gm $ for the homogeneous maximal
ideal $SW$ of $S$, and for any graded $S$-module $M$ we write
$\myH ^{j}_{\gm }(M)$ for the $j^{\myth }$ local cohomology module of $M$, regarded
as a graded $S$-module.

\begin{theorem18}
Let $M$ be a graded $S$-module
generated in degree $d$, and having
linear free resolution $\LL (P)$. Let
$\FF :\ \cdots \to F^{-1}\to F^{0}$ be the minimal free resolution of $P$.
The linear part
of $\FF $ is
\begin{equation*}\lin (\FF )=\bigoplus _{j}\RR (\myH ^{j}_{\gm }(M)),
\end{equation*}
where $\myH ^{j}_{\gm }(M)$ is
regarded as a complex with one term, concentrated in
cohomological degree $j$. In particular, we have
\begin{equation*}F^{-i}=\bigoplus _{j}\myHom _{K}(E, \myH ^{j}_{\gm }(M)_{-j-i}).
\end{equation*}
\end{theorem18}
\begin{proof}[Proof of Theorem $4.3$] We compute the linear part of the free resolution of $P$ by
taking the dual (into $K$) of the linear part of the injective
resolution of $P^{*}$.
By Theorem 3.7, the linear part of the injective resolution
of $P^{*}$ is $\RR (\myH ^{\bullet }(\LL (P^{*})))$.
It follows at once from
the definitions that $\LL (P^{*})=\myHom _{S}(\LL (P), S)$.
By Theorem 3.7 once more,  $\LL (P)$ is the minimal free resolution
of $M$, so
$\myH ^{\bullet }(\LL (P^{*}))=\Ext _{S}^{\bullet }(M,S)$.
 Thus
the linear part of the free resolution of $P$ is
$[\RR \Ext _{S}^{\bullet }(M,S)]^{*}$, where $\Ext _{S}^{j}(M,S)$ is
thought of as a module concentrated in cohomological degree
$j$.

Because $E^{*}=\omega _{E}=E\otimes \bigwedge ^{v}W$, we have, for
any graded vector space $D$, natural
identifications
\begin{equation*}\begin{split}
(\myHom _{K}(E,D))^{*}
&=(E^{*}\otimes _{K}D)^{*}\\
&=E^{*}\otimes _{K} \textstyle{\bigwedge ^{v}}W^{*} \otimes _{K}D^{*}\\
&=\myHom _{K}(E,D^{*})\otimes _{K}\textstyle{\bigwedge ^{v}}V.\end{split}
\end{equation*}
(Here all the duals of $E$-modules are Hom into $K$.) If $D$
has the structure of a graded $S$-module, then
$D^{*}$ is again a graded $S$-module, and this becomes  an
isomorphism of graded $S$-modules. If we think of
$D$ as a complex with just one term, in cohomological degree
$d$, then
$\RR (D)^{*}=\RR (D^{*}\otimes _{K}\bigwedge ^{v}V)$, where, to make
all the indices come out right, we must think of
$D^{*}\otimes _{K} \bigwedge ^{v}V=(D\otimes _{K}\bigwedge ^{v}W)^{*}$ as a complex of one
term concentrated in cohomological degree $v-d$.

If we take $D=\Ext _{S}^{\ell }(M,S)$, then, by local duality,
\begin{equation*}\begin{split}
D^{*}&=(\Ext _{S}^{\ell }(M,S)\otimes _{K}\textstyle{\bigwedge ^{v}}
W\otimes _{K} \textstyle{\bigwedge ^{v}}V)^{*}\\
&=(\Ext _{S}^{\ell }(M,\omega _{S})\otimes _{K} \textstyle{\bigwedge ^{v}}
V)^{*}\\
&=\myH ^{v-\ell }_{\gm }(M)\otimes \textstyle{\bigwedge ^{v}}W.\end{split}
\end{equation*}
Thus
\begin{equation*}\begin{split}
\RR (\Ext _{S}^{\ell }(M,S))^{*}&=\RR (\myH ^{v-\ell }_{\gm }(M)
                         \otimes _{K}\textstyle{\bigwedge ^{v}}
W)\otimes \textstyle{\bigwedge ^{v}}V\\
&=\RR (\myH ^{v-\ell }_{\gm }(M)),
\end{split}
\end{equation*}
where $\myH ^{j}_{\gm }(M)$
is regarded as a complex with just one term, of
cohomological degree $-j$, as required.\end{proof}
\begin{proof}[Proof of Theorem $4.1$] For each $i=0, \dots , v-1$ we write $\myH ^{i}$ for the
cohomology module $\bigoplus _{d=-\infty }^{\infty }\myH ^{i}(\F (d))$.
If we choose $d\geq \reg (\myH ^{0}_{\geq 0})$  as in the
construction of $\TT (\F )$, the module $M:=\myH ^{0}_{\geq d}$
has a linear free resolution, so we may
apply Theorem 4.3. We deduce that
the linear part of the  free resolution of
$P:=\ker [\myHom _{K}(E,\myH ^{0}(\F (d)))\to \myHom _{K}(\myH ^{0}(\F (d+1)))]$
is
$\lin (\FF )=\bigoplus _{j}\RR (\myH ^{j}_{\gm }(M)).
$
If we insist that $d>\reg (\myH ^{0}_{\geq 0})$,
then $\myH ^{0}_{\gm }(M)=0$. From the exactness of the
sequence
\begin{equation*}0\to \myH ^{0}_{\gm }(M)\to M\to \bigoplus _{d=-\infty }^{\infty }\myH ^{0}(\F (d))\to \myH ^{1}_{\gm }(M)\to 0
\end{equation*}
it follows that
the local cohomology module $\myH ^{1}_{\gm }(M))$
agrees with the global cohomology module
$\myH ^{0}$ in all degrees strictly less than $d$,
and of course we have
$\myH ^{i}=\myH ^{i+1}_{\gm }(M)$. This concludes the proof.
\end{proof}


\section*{5. Powers of the maximal ideal of $E$}

In this section we provide a basic example of the
action of the functors $\LL $ and $\RR $.
Among the most interesting graded $S$-modules are the syzygy modules
that occur in the Koszul complex. We write
\begin{equation*}\Omega ^{i}=\coker \bigl [ S\otimes _{K}
\textstyle{\bigwedge ^{i+2}}W
\to S\otimes _{K}\textstyle{\bigwedge ^{i+1}}W\bigr ],
\end{equation*}
where as usual the elements of $W$ have internal degree 1, so that
the generators of $\Omega ^{i}$ have internal degree $i+1$. For example,
 $\Omega ^{-1}=K$ while $\Omega ^{0}=(W)\subset S$
and $\Omega ^{v-1}=S\otimes \bigwedge ^{v}W$, a free module of rank one
generated in degree $v$.
The sheafifications of these modules are the exterior
powers of the cotangent bundle on projective space
(see Eisenbud \cite{Eis}, Section 17.5, for more details).
In this section we shall show that
under the functors $\LL $ and $\RR $ introduced
in Section 2
the $\Omega ^{i}$
correspond to powers of the maximal ideal $\gm \subset E$. To
make the correspondence completely functorial, we make use of
the $E$-modules $\gm ^{i}\omega _{E}$, where
$\omega _{E}=\myHom _{K}(E,K)$. Recall that $\omega _{E}$ is a
rank one free $E$-module generated in degree $v$; its
generators may be identified with the nonzero elements of $\bigwedge ^{v}W$.
\begin{theorem19}
The minimal $S$-free resolution
of $\Omega ^{i}$ is $\LL (\omega _{E}/\gm ^{v-i}\omega _{E})$;
the minimal $E$-injective
resolution of $\omega _{E}/\gm ^{v-i}\omega _{E}$ is $\RR (\Omega ^{i})$.
\end{theorem19}


Since $\Omega ^{i}$ is generated in degree $i+1$,
the complex $\RR (\Omega ^{i})$
begins in cohomological degree $i+1$, and we regard
$\omega _{E}/\gm ^{v-i}\omega _{E}$ as concentrated in cohomological
degree $i+1$.

\begin{proof}  The complex $\LL (\omega _{E})$ is the Koszul complex
over $S$, so $\LL (\omega _{E}/\gm ^{v-i}\omega _{E})$ is the truncation
\begin{equation*}0\to S\otimes \textstyle{\bigwedge ^{v}}
W \to \cdots \to S\otimes 
\textstyle{\bigwedge ^{i+1}}W,
\end{equation*}
which is the resolution of $\Omega ^{i}$, proving the first statement.
The second statement follows from Theorem 3.7.\end{proof}


Since the $K$-dual of a minimal $E$-injective resolution is a
minimal $E$-free resolution, we may immediately derive the
free resolution of
\begin{equation*}\gm ^{i+1}=\myHom _{E}(\omega _{E}/\gm ^{v-i}\omega _{E}, \omega _{E})
=\myHom _{K}(\omega _{E}/\gm ^{v-i}\omega _{E}, K).
\end{equation*}
\begin{theorem20}
The minimal $E$-free resolution of $\gm ^{j}$ is
\begin{equation*}\myHom _{K}(\RR (\Omega ^{j-1}), K).
\end{equation*}
\end{theorem20}


These resolutions can be made explicit using
the Schur functors $\bigwedge ^{i}_{j}$ associated to ``hook'' diagrams
(see for example
Buchsbaum and Eisenbud \cite{BE} or
Akin, Buchsbaum, Weyman \cite{ABW}).
We may define
$\bigwedge ^{i}_{j}$ (called $L^{i}_{j}$ by Buchsbaum and Eisenbud)
by the formula
\begin{equation*}\textstyle{\bigwedge ^{i}_{j}}(W)=\myim \bigl 
[\textstyle{\bigwedge ^{i}}W\otimes _{K}\sym _{j-1}W\to 
\textstyle{\bigwedge ^{i-1}}W\otimes _{K}\sym _{j}W\bigr ].
\end{equation*}
Note that
\begin{equation*}\textstyle{\bigwedge ^{i}_{j}}(W)=\begin{cases}0 & 
\text{if $i<1$ or $j<1$},\\
                     \bigwedge ^{i}W & \text{if $j=1$},\\
                     \Sym _{j}W & \text{if $i=1$}.\end{cases}
\end{equation*}
Buchsbaum and Eisenbud use these functors to
give (among other things)
a $GL(W)$-equivariant resolution
\begin{equation*}\cdots \to S\otimes _{K}\textstyle{\bigwedge ^{2}_{j}}(W)
\to S\otimes _{K}\textstyle{\bigwedge ^{1}_{j}}(W)
\to (W)^{j}
\to 0
\end{equation*}
of the $j^{\myth }$ power
$(W)^{j}$ of the maximal ideal of $S$.
The $\bigwedge ^{i}_{j}$ also provide the terms in the resolutions above:
\begin{theorem21}
For $i>0$ the minimal free resolution of $\gm ^{i}$ has the form
\begin{equation*}\cdots \to E\otimes (\textstyle{\bigwedge ^{i}_{2}} W)^{*}\to E\otimes
(\textstyle{\bigwedge ^{i}_{1}} W)^{*}\to \gm ^{i}
\to 0.
\end{equation*}
For $i< v$ the minimal injective resolution of
$\omega _{E}/\gm ^{v-i}\omega _{E}$ has the form
\begin{equation*}\myHom _{K}(E, \textstyle{\bigwedge ^{i+1}_{1}}(W))\to \myHom _{K}(E, 
\textstyle{\bigwedge ^{i+1}_{2}}(W))\to \cdots .
\end{equation*}
\end{theorem21}
\begin{proof}  From the exactness of the Koszul complex we see that
$(\Omega ^{i})_{j} = \bigwedge ^{i+1}_{j-i}W,
$
so the second statement follows from Theorem 5.1.
The first statement follows similarly from
Corollary 5.2. \end{proof}


Using the exact sequence
\begin{equation*}0\to \gm ^{v-i}\omega _{E}\to \omega _{E}\to \omega _{E}/\gm
^{v-i}\omega _{E}\to 0,
\end{equation*}
we may paste together the injective and free resolutions considered above
into the Tate resolution $\TT (\Omega ^{i}_{\myP })$.
\begin{theorem22}
There is an exact sequence 
\begin{equation*}\begin{split}\TT (\Omega ^{i}_{\myP })\!:
\cdots &\to \myHom _{K}(E,\textstyle{\bigwedge ^{i+1}_{1}}W)\to \myHom _{K}
(E,\textstyle{\bigwedge ^{i+1}_{2}}W)
\to \myHom _{K}(E,K) \\
&\to \myHom _{K}(E,(\textstyle{\bigwedge ^{v-i}_{1}}W)^{*})\to \myHom _{K}
(E,(\textstyle{\bigwedge ^{v-i}_{2}}W)^{*})
\to \cdots, \end{split}\end{equation*}
where $\myHom _{K}(E,K)=\omega _{E}$ is the term in cohomological degree $i$.
\end{theorem22}


The following well-known result now follows from Corollary 5.4
by inspection.
\begin{theorem23}
In the range $ 0 \le j \le v-1$  or  $1 \le q \le v-2$,
\begin{equation*}\myH ^{q}(\mathcal{O} _{\myP }(-j)\otimes 
\Omega _{\myP }^{p}(p)) =
\begin{cases}K, & \text{if }p=q=j, \\
 0, & \text{otherwise}. 
\end{cases}
\end{equation*}
\end{theorem23}


\begin{proof}  Writing the ranks of the free modules in the Tate resolution
for $\Omega _{\myP }^{p}$ in
Macaulay notation, we find
\begin{equation*}\begin{matrix}(v-p){\binom{v+1 }{v-p+1}} & {\binom{v }{v- p}}  & . & . & .  \\
  . & . & . & . & .  \\
  . & . & 1 & . & .  \\
  . & . & . & . & .  \\
  . & . & . & . & .  \\
  . & . & . & {\binom{v }{p+1}} &  (p+1){\binom{v+1 }{p+2}} 
   \\
\end{matrix}
\end{equation*}
with the rank $1$ module sitting in homological degree $p$ and the ingoing
and outgoing
 map from it given by bases of the forms in $\bigwedge ^{v-p} V$ and
$\bigwedge ^{p+1} V$ respectively.
\end{proof}


If we shift the rank 1 module into homological degree $0$, then we have the Tate resolution of
 $\Omega ^{p}(p)$. Following Beilinson (\cite{Bei}, Lemma 2), we can also compute
$\myHom (\Omega ^{i}(i),\Omega ^{j}(j))$ for any $i,j$, which will play a
major role in Section 6.


\begin{theorem24}
If $\Omega ^{i}(i)$ are the $S$-modules defined in
Section 5 and $0\leq i,j < v$, then
\begin{equation*}\hom _{S}(\Omega ^{i}(i), \Omega ^{j}(j))=
\textstyle{\bigwedge ^{i-j}}V = \hom _{E}(\omega _{E}(i),\omega _{E}(j)),
\end{equation*}
where in each case $\hom $ denotes the (degree $0$) homomorphisms;
for other values of $i,j$ the left-hand side is $0$.
The product of
homomorphisms corresponds to the product in $\bigwedge V$.
\end{theorem24}
\begin{proof}  The modules $\Omega ^{i}(i)$ are 0 for $i<0$ and $i\geq v$.
For $0\leq i<v$  they
have linear resolution, so we may apply Theorem 3.7.
As they are 0 in degrees $<1$ and generated in degree 1,
we have
$\myH ^{1}\RR (\Omega ^{i}(i))=\omega _{E}(i)/\gm ^{v-i}\omega _{E}(i)$ if
$v>i$,
by Theorem 5.1. For $0\leq i,j <v$,  maps
$\omega _{E}(i)/\gm ^{v-i}\omega _{E}(i)\to \omega _{E}(j)/\gm ^{v-j}\omega _{E}(j)$
are in one-to-one correspondence with maps $\omega _{E}(i)\to \omega _{E}(j)$. Since $\omega _{E}$ is
a rank one free $E$-module, these may be identified with elements of
$E_{j-i}=\bigwedge ^{i-j}V$. \end{proof}


\section*{6. Beilinson's monad}

Beilinson's paper \cite{Bei} contains
two main results. The first says that, given
a sheaf $\F $ on a projective space $\myP =\myP (W)$, there is a complex
\begin{equation*}\myB : \cdots \to B^{-1}\to B^{0} \to B^{1}\to \cdots \end{equation*}
with
\begin{equation*}B^{e}=\bigoplus _{j}\myH ^{j}(\F (e-j))\otimes 
\Omega _{\myP }^{j-e}(j-e)
\end{equation*}
such that $\myB $ is exact except at $B^{0}$ and
the homology at $B^{0}$ is $\F $.

We show that
the complex $\myB $ may be obtained by applying a certain functor to the
Tate resolution
$\TT (\F )$ over $E$. Beilinson's second main result gives another monad,
which we will treat in Theorem 8.1.

Given any graded free complex $\TT $ over $E$, we may
write each module of $\TT $ as a direct sum of copies of
$\omega _{E}(i)=\myHom _{K}(E, K(i))$ with varying $i$. We
define
$\Omega (\TT )$ to be the complex of sheaves on $\myP $
obtained by replacing each
summand $\omega _{E}(i)$ by the sheaf $\Omega _{\myP }^{i}(i)$ and using the
isomorphism of Hom in Proposition 5.6 to provide the maps.
\begin{theorem25}
If $\F $ is a
coherent sheaf on $\myP (W)$ with associated Tate resolution
$\TT (\F )$,
then the only homology of  $\Omega (\TT (\F ))$ is
in cohomological degree $0$, and is isomorphic to $\F $.
\end{theorem25}
\begin{proof}  To simplify the notation we set $\TT =\TT (\F )$,
and we let $\overline{\TT }$ be $\TT $
modulo the elements of internal degree
$\geq 0$.
Let $\mathcal{L}$ be the double complex of sheaves that arises by
sheafifying the double complex of $S$-modules used to
construct the complex $\LL (\TT )$; that is, if $T^{e}$ is
the component of $\TT $ of cohomological degree $e$, and
$T^{e}_{j}$ is its component of internal degree $j$, then the
double complex
$\mathcal{L}$ has the form
\begin{equation*}\mathcal{L}\colon \xy \xymatrix {
&&&\\
\dotsb \ar [r]&T_{j+1}^{e}\otimes _{K}\mathcal{O}(j+1)\ar [u]\ar [r]&
T_{j+1}^{e+1}\otimes _{K}\mathcal{O}(j+1)\ar [u]\ar [r]&\dotsb \\
\dotsb \ar [r]&T_{j}^{e}\otimes _{K}\mathcal{O}(j)\ar [r]\ar [u]&T_{j}^{e+1}
\otimes _{K}\mathcal{O}(j)\ar [r]\ar [u]&\dotsb \\
{}&{}\ar [u]&{}\ar [u]&\\
}
\endxy \end{equation*}
Since $\TT $ is exact, the rows are exact; since the columns are
direct sums of sheafified Koszul complexes over $S$, they are
exact as well.

Choose an integer $f>>0$ (greater than the regularity of
$\F $ will be sufficient) and let $\mathcal{L}'$ be the double complex
obtained from $\mathcal{L}$ by taking only those terms $T^{e}_{j}\otimes _{K} \mathcal{O} (j)$
with $e<f$ and $j>0$. If $e<<0$, then $T^{e}$ is generated in negative
degrees, so the double complex $\mathcal{L}'$ is finite, and is exact
except at the right ($e=f-1$) and at $j=1$. An
easy spectral sequence argument shows that the
complex obtained as the vertical homology of $\mathcal{L}'$
has the same homology as the complex obtained as the
horizontal homology of $\mathcal{L}'$.

If we write
$T^{e}$ as a sum of copies of $\omega _{E}(i)$ for various $i$,
then the $e^{\myth }$ column of $\mathcal{L}'$ is correspondingly a sum of
copies of the sheafification of
$\LL (\omega _{E}(i)/\gm ^{v-i}\omega _{E}(i))$. 
As in Theorem 5.1, the
vertical homology of this column is correspondingly a sum of
copies of $\Omega _{\myP }^{i}(i)$; that is, it is $\Omega (T^{e})$. Thus
the complex obtained as the vertical homology of $\mathcal{L}'$ is
$\Omega (\TT )$.

As $e$ goes to infinity, the degrees of the generators of $T^{e}$
become more and more positive; thus for $e$ large
the $e^{\myth }$  column of $\mathcal{L}'$ is the same as that of $\mathcal{L}$, that is,
it is $\LL (T^{e})$.
Since $f>>0$, the horizontal homology of $\mathcal{L}'$ is
the sheafification of $\LL (H)$, where $H$ is
the homology of $\TT _{<f}$.
As $\TT $ is exact, $H$ may also be written as the homology of
$\TT _{\geq f}$. Taking $f>\reg \F $ and using Theorem 3.7,
we see that $\LL (H)$ is a free resolution of the module
$\bigoplus _{e\geq f}\myH ^{0}(\F (e))$, whose sheafification is $\F $,
as required.\end{proof}


\begin{theorem26}
The map in
the complex $\Omega (\TT (\F ))$ corresponding to
\begin{equation*}\myH ^{j}(\F (j-i))\otimes \Omega _{\myP
}^{i-j}(i-j)\longrightarrow \myH ^{j}(\F (j-i+1))\otimes \Omega _{\myP
}^{i-j-1}(i-j-1)
\end{equation*}
corresponds to the multiplication map
$W\otimes _{K} \myH ^{j}(\F (j-i))\longrightarrow \myH ^{j}(\F (j-i+1))$.
\end{theorem26}
\begin{proof}  This follows from Theorem 4.1, since we have identified
not only the modules but the maps in the linear strands of the
resolution.
\end{proof}
\begin{theorem27}
The maps in the complex
$\Omega (\TT (\F ))$ correspond to the maps in the complex
$\Omega (\TT (\F (1)))$ under the natural correspondence
\begin{equation*}\hom _{\myP }(\Omega _{\myP }^{i}(i), 
\Omega _{\myP }^{j}(j))=\textstyle{\bigwedge ^{i-j}}V
=\hom _{\myP }(\Omega _{\myP }^{i+1}(i+1), \Omega _{\myP }^{j+1}(j+1))
\end{equation*}
whenever $0\leq i, i+1, j, j+1 <v$.
\end{theorem27}
\begin{proof}  The Tate resolution $\TT (\F (1))$ is obtained by simply
shifting $\TT (\F )$.\end{proof}


\section*{7. Examples}\begin{definition3}
Let $C$ be an elliptic quartic
curve in $\myP ^{3}$, and consider $\mathcal{O} _{C}$ as
a sheaf on $\myP ^{3}$. Write $\omega _{E}=\bigwedge ^{v}W\otimes E$, as usual.
Computing cohomology,
one sees that $\TT (\mathcal{O} _{C})$ has the form
\begin{equation*}\cdots \longrightarrow \omega _{E}^{8}(2)
\longrightarrow \omega _{E} \oplus \omega _{E}^{4}(1)
\longrightarrow ^{d}
\omega _{E}^{4}(-1)\oplus \omega _{E}
\longrightarrow \omega _{E}^{8}(-2)
\longrightarrow \cdots. \end{equation*}

If $C \subset \myP ^{3}$ is taken to be Heisenberg invariant, say
$C = \{ x_{0}^{2}+x_{2}^{2}
+\lambda x_{1}x_{3}=x_{1}^{2}+x_{3}^{2}+\lambda x_{0}x_{2} = 0 \}$
for some $\lambda \in \mathbf{A} ^{1}_{k}$, then $d$ can be represented
by the matrix
\begin{equation*}\begin{pmatrix}0 &e_{0} &         e_{1}   &     e_{2}   &    e_{3}         \\
 e_{0} &-\lambda e_{1}e_{3}  & e_{2}e_{3}  & 0 & e_{1}e_{2}+{\frac{\lambda ^{2} }{2}}e_{0}e_{3} \\
 e_{1}  &e_{2}e_{3} & \lambda e_{0}e_{2}    &  -e_{0}e_{3}-{\frac{\lambda ^{2} }{2}} e_{1}e_{2} & 0 \\
 e_{2} &0       & -e_{0}e_{3}-{\frac{\lambda ^{2} }{2}} e_{1}e_{2} & \lambda e_{1}e_{3}  &  e_{0}e_{1}
\\
 e_{3} &e_{1}e_{2}+{\frac{\lambda ^{2} }{2}}e_{0}e_{3}& 0 & e_{0}e_{1} & -\lambda e_{0}e_{2} \\
\end{pmatrix}
.\end{equation*}
\end{definition3}
\begin{definition4}
Let $C \subset \myP ^{d}$ be the
curve parame\-trized by $(s:t) \mapsto (s^{d}:s^{d-1}t:\ldots :t^{d})$. We consider
 the line bundles $\mathcal{L}_{k}$ on $C$ associated to
$\bigoplus _{m=0}^{\infty }H^{0}(\myP ^{1},\mathcal{O} (k+md))$ for $k=-1,\ldots ,d-2$. The Tate
resolution
$\TT (\mathcal{L}_{k})$ has Betti numbers
\begin{equation*}\begin{matrix}* \; * & 3d-k-1 & 2d-k-1 & d-k-1 & .     & .      &\ldots \\
         \ldots & .      & .      & k+1 & d+k+1 & 2d+k+1  &*\;* \\
\end{matrix}
\end{equation*}
The $[(k+1)+(d-k-1)] \times (2d-k-1)$ matrix near the middle and the matrices surrounding it
 have in case $d=4$ and $k=1$ the following shapes:
\begin{equation*}
\scriptscriptstyle {\left(\begin{matrix}
0&e_{0}&e_{0}e_{2}&e_{0}e_{1} \\
e_{0}&e_{1}&e_{1}e_{2}+e_{0}e_{3}&e_{0}e_{2} \\
e_{1}&e_{2}&e_{1}e_{3}+e_{0}e_{4}&e_{0}e_{3} \\
e_{2}&e_{3}&e_{1}e_{4}&e_{0}e_{4}\\
e_{3}&e_{4}&0&0\\
e_{4}&0&0&0\end{matrix}\right)}{\textstyle{,}}
\left(\begin{matrix}
0&0& e_{0}e_{4}& e_{1}e_{4}        &   e_{2}e_{4} & e_{3}e_{4} \\
0&0& e_{0}e_{3}& e_{1}e_{3}+e_{0}e_{4} & e_{2}e_{3}+e_{1}e_{4} & e_{2}e_{4} \\
0&e_{0}&e_{1}&e_{2}&e_{3}&e_{4}\\
e_{0}&e_{1}&e_{2}&e_{3}&e_{4}&0\\
\end{matrix}\right)
\end{equation*}
and
\begin{equation*}
\left(\begin{matrix}e_{0}&e_{1}&e_{2}&e_{3}&e_{4}&0&0&0&0&0\\
0&e_{0}&e_{1}&e_{2}&e_{3}&e_{4}&0&0&0&0\\
0&0&e_{0}&e_{1}&e_{2}&e_{3}&e_{4}&0&0&0\\
0&0&0&e_{0}&e_{1}&e_{2}&e_{3}&e_{4}&0&0\\
0&0&0&0&e_{0}&e_{1}&e_{2}&e_{3}&e_{4}&0\\
0&0&0&0&0&e_{0}&e_{1}&e_{2}&e_{3}&e_{4}\\
\end{matrix}\right).
\end{equation*}
All other matrices look similar to the last one.

In case $k=-1$ we obtain a $d \times d$ symmetric
matrix of 2-forms:
\begin{equation*}
\left(\begin{matrix}e_{0}e_{1}&e_{0}e_{2}&e_{0}e_{3}&e_{0}e_{4}\\
e_{0}e_{2}&e_{1}e_{2}+e_{0}e_{3}&e_{1}e_{3}+e_{0}e_{4}& e_{1}e_{4}\\
e_{0}e_{3}&e_{1}e_{3}+e_{0}e_{4}&e_{2}e_{3}+e_{1}e_{4}&e_{2}e_{4} \\
e_{0}e_{4}&e_{1}e_{4}&e_{2}e_{4} & e_{3}e_{4}
\end{matrix}\right)
.
\end{equation*}
If we interpret  2-forms as coordinate functions
\begin{equation*}e_{ij}=e_{i}e_{j}=e_{i}\wedge e_{j} \in \myH ^{0}(\GG
(W,2),\mathcal{O} (1)) \cong \myH ^{0}(\myP (\textstyle{\bigwedge^{2}} V),/\mathcal{O}
(1))\end{equation*}
 on the Grassmannian
of codimension 2 linear subspaces in $\myP (W)$, then the  determinant of the
matrix above defines the Chow divisor
of $C\subset \myP ^{d}$, which is by definition the
hypersurface
 $\{\myP ^{d-2} \in \GG (W,2) | \myP ^{d-2} \cap C \ne \emptyset \}$.
Eisenbud and Schreyer \cite{ES1} give a general computation of Chow forms along these lines.
\end{definition4}
\begin{definition5}
A famous Beilinson monad was discovered by Horrocks and
Mumford \cite{HM}:
Consider for $\myP ^{4}$ the Tate resolution $\TT (\varphi )$ of the matrix
\begin{equation*}
\varphi = \left(\begin{matrix}
e_{1}e_{4} & e_{2}e_{0} &e_{3}e_{1} & e_{4}e_{2} & e_{0}e_{3} \\
e_{2}e_{3} & e_{3}e_{4} &e_{4}e_{0} & e_{0}e_{1} & e_{1}e_{2} 
\end{matrix}\right).
\end{equation*}
By direct computation we find the Betti numbers
\begin{equation*}
\setcounter{MaxMatrixCols}{13}
\begin{matrix}
?         & .& . & . & . & . & . & . & . & . & . & . & \ldots \\
  *\; *
   &100&35 & 4 & 0 & . & . & . & . & . & . & . & \ldots \\
\ldots 
   & 0 & 2 & 10&10 & 5 & 0 & . & . & . & . & . &\ldots \\
\ldots 
   & . &.  & . & . & 0 & 2 & 0 & . & . & . & . &\ldots \\
\ldots 
   & . &.  & . & . & . & 0 & 5 & 10&10 & 2 & 0 &\ldots \\
\ldots 
   & . &.  & . & . & . & . & . & 0 & 4 & 35&100&*\;* \\
\ldots 
   & . &.  & . & . & . & . & . & . & . & .& .& ? 
\end{matrix}.
\end{equation*}
\setcounter{MaxMatrixCols}{10}
\end{definition5}


To deduce that this Tate resolution comes from a sheaf, we use

\begin{theorem28} Let $\TT $ be a Tate resolution over $E$. Suppose
 that  $(K \tensor _{E} T^{0})_{j} = 0$  for all $j < 0$.
Then
$(K \tensor _{E} T^{l})_{m} = 0$  if
$ l>0$  and  $ m <l$, or if
$l<0$ and $m<l-n$.
\end{theorem28}
\begin{proof}  Pictorially the statement says that vanishing in a single $T^{e}$ implies vanishing in the
indicated range:
\begin{equation*}\begin{matrix}? &? & \vdots & ? &  \ldots \\
\ldots & ? & 0_{-n} & ? &  \ldots \\
 & \vdots & \vdots & \vdots & \\
\ldots & ? &  0_{-2} & ? &\ldots \\
\ldots & ? &  0_{-1} & ? &\ldots \\
\ldots & ? & * & ? &\ldots \\
\end{matrix}
\quad \Longrightarrow \quad \begin{matrix}0 &0 & \vdots & 0 &  \ldots \\
\ldots & ? & 0_{-n} & 0 &  \ldots \\
 & \vdots & \vdots & \vdots & \\
\ldots & ? &  0_{-2} & 0 &\ldots \\
\ldots & ? &  0_{-1} & 0 &\ldots \\
\ldots & ? & * & ? &\ldots \end{matrix}
\end{equation*}

The first vanishing follows, because  $\myHom _{E}(\TT ,E)$ is also a minimal
complex. For the second we note for $P=\ker (T^{0} \to T^{1})$
that $P_{j}=0$ holds for all $j<-v$ by our assumption. By 
Corollary 5.3,
$(K \tensor _{E} T^{l})=\Tor ^{E}_{-l-1}(K,P)$, which is a subquotient of 
$(\Sym _{-l-1} W)^{*} \tensor _{K} P$. Thus this group vanishes
in all degrees $m<l-v+1$.
\end{proof}
\begin{definition6}
By applying  Lemma 7.4 to a shift of  $\TT (\varphi )$ and\linebreak
$\myHom (\TT (\varphi ),E)$ we
see that the $\TT (\varphi )$ has  terms only in the indicated
range of rows; in particular, the rows with the question marks 
contain only zeros.  
So $\TT (\varphi )$ is the Tate resolution of some sheaf $\F $.
Moreover $\F $ is a bundle, since the middle cohomology has only finitely many
terms. The $4^{\myth }$ difference function of  $\chi (\F (m))$
has constant value $2$. So
$\F $ has rank $2$. It is the famous  bundle on $\myP ^{4}$
discovered by Horrocks and Mumford \cite{HM}. In Decker and Schreyer \cite{DS1}
 it
is proved that any stable rank $2$ vector bundle on $\myP ^{4}$ with the same
Chern classes equals $\F $ up to projectivities.
\end{definition6}
\section*{8. Free monads}

A {\em free monad} $\mathcal{L}$ for a coherent sheaf $\F $ is a finite complex
\begin{equation*}0 \to \mathcal{L}^{-N} \to \cdots \to \mathcal{L}^{-1} \to 
\mathcal{L}^{0} \to \mathcal{L}^{1} \to \cdots \to \mathcal{L}^{M} \to 0\end{equation*}
on $\PP ^{n}=\PP (W)$ whose components $\mathcal{L}^{i}$ are direct sums of line bundles and whose homology
is $\F $:
\begin{equation*}\myH ^{\bullet }(\mathcal{L})=\myH ^{0}(\mathcal{L}) \cong \F .\end{equation*}
The complex of twisted global sections of $\mathcal{L}$ is a complex $L= \Gamma _{*}(\mathcal{L})$
of free $S$-modules. If $L$ is a minimal complex,  then we speak of a
minimal free  monad. The most familiar free monads are the sheafifications of the
minimal free resolutions of the modules $\bigoplus _{m\geq m_{0}}\myH ^{0}\F (m)$
for various $m_{0}$.

Free monads  have been constructed by Horrocks \cite{Hor},
Barth \cite{Bar}, Bernstein, Gel'fand and Gel'fand \cite{BGG}
and Beilinson \cite{Bei},
mainly for the study of vector bundles on projective spaces.
Rao \cite{Rao} and
Martin-Deschamps and Perrin \cite{MP} used free monads in their studies
of space curves. 
Fl\o ystad \cite{Flo3} gave a complete classification of a certain class of 
linear monads on projective spaces.
The general construction of free monads is the following:
\begin{theorem29}
Let $\F $ be a coherent sheaf on $\myP ^{n}$, and let $\TT '$ be a left-bounded
complex of finite free $E$-modules with $\TT ^{\prime \ge r}=\TT (\F )^{\ge r}$ for some $r$.
Let  $L =\min \ \LL (\TT ')$ be the minimalized complex of the BGG transform
$\LL (\TT ')$. Its sheafication
$\mathcal{L}(\TT ') =\tilde L $
is a free monad for $\F $. Every minimal free monad $\mathcal{L}$ of $\F $
arises as $\mathcal{L}=\mathcal{L}(\TT ')$ in this way
with $\TT '= \min \RR (L)$.
\end{theorem29}
\begin{proof}  Suppose $\TT '$ satisfies the assumption. Since $\TT '$ is 
left-bounded and
acyclic for large degrees, $L =\min \LL (\TT ')$ is a finite complex by the second statement in Corollary 3.6.
The cohomology of the complex $L$ can be
computed  by taking linear parts:
$\bigoplus _{i} \RR (\myH ^{i}(L ))= \lin (\RR (L ))=
\lin (\RR (\LL (\TT ')))=\lin (\TT ')$ by Corollary 3.6.
So $\myH ^{j}(L )$  is of finite length for $j\not =0$, and sheafifying gives
\begin{equation*}\myH ^{\bullet }(\mathcal{L})=\myH ^{0}(\mathcal{L})=  (\Gamma _{\ge r}\F )\widetilde {} = \F .\end{equation*}
Conversely, if $\mathcal{L}$ is a free monad for $\F $
and $L=\Gamma _{*}\mathcal{L}$, then  $\myH ^{j}(L )$
has finite length for $j\not = 0$. Thus
$\TT '=\min \RR (L )$ is a left-bounded complex with
$\TT ^{'\ge r} = \TT (\F )^{\ge r}$ by  Corollary 3.6, and
$ \min \LL (\TT ')=\min {\LL }(\min \RR (L ))=
\min {\LL }\RR (L )=\min L =L $.
\end{proof}
\begin{definition7}
Consider $\F = \mathcal{O} _{p}$, the structure sheaf of a
point in $\PP ^{1}$. Its Tate resolution is periodic:
\begin{equation*}\cdots \overset {e}{\longrightarrow }\omega _{E}(1) \overset {e}{\longrightarrow }\omega _{E}
\overset {e}{\longrightarrow }\omega _{E}(-1) \overset {e}{\longrightarrow }\cdots .
\end{equation*}
If we take $\TT '$ to be the truncation
\begin{equation*}0 \longrightarrow \omega _{E} \overset {e}{\longrightarrow }\omega _{E}(-1)\overset {e}{\longrightarrow }
\cdots, \end{equation*}
then the monad $\mathcal{L}(\TT ')$ is the sheafified free resolution
\begin{equation*}0 \longrightarrow \mathcal{O} (-1) \longrightarrow ^{x} \mathcal{O} \longrightarrow 0.
\end{equation*}
If instead we take $\TT '$ to be the complex
\begin{equation*} 0 \longrightarrow \omega _{E} \longrightarrow ^{ef}
\omega _{E}(-2) \overset {e}{\longrightarrow }\omega _{E}(-3)
\overset {e}{\longrightarrow }\cdots ,
\end{equation*}
then $L(\TT ')$ is the free resolution of
$S/(x^{2},xy)$,
which has sheafification $\mathcal{L}(\TT ')$ of the form
\begin{equation*} 0 \longrightarrow \mathcal{O} (-3) \longrightarrow \mathcal{O} (-2)^{2}
\longrightarrow \mathcal{O} \longrightarrow 0.
\end{equation*}

For the rest of this section we will study a class of free monads we
call {\em partition monads\/} (because they partition the cohomology
of $\F $ into two simple pieces, which occur as $\myH ^{\bullet }(L)$ and  $\myH ^{\bullet }(L^{*}))$. This class includes
the sheafified free resolutions and most of the other free monads
found in the literature.
\end{definition7}
\begin{definition8}
Given a weakly increasing sequence of integers
\begin{equation*}\mu =(m_{0} \leq m_{1}\leq \cdots \leq m_{n}),
\end{equation*}
we define
$\TT _{\mu }(\F )$ to be the subcomplex of $\TT (\F )$ given by
\begin{equation*}T_{ \mu }^{e}(\F ) = \bigoplus _{i: e \ge m_{i}} \myH ^{i}\F (e-i) \tensor _{K}
\omega _{E}(i-e).
\end{equation*}
We shall also make use of the complementary complex
$\TT ^{\mu }$
defined by the exact sequence
\begin{equation*}0\to \TT _{\mu }\to \TT \to \TT ^{\mu }\to 0.
\end{equation*}
We set
$L_{\mu }(\F ) : = \min \LL (\TT _{\mu }(\F ))$,
and write
$\mathcal{L}_{ \mu }(\F )=\tilde L_{\mu }(\F )$ for the monad which is its
sheafification.
\end{definition8}
\begin{definition9}
Let
$m_{0}$ be any integer, and choose
$m_{1},\dots ,m_{n}$ greater than the Castelnuovo-Mumford
regularity of $\F $. The complex
$\TT _{\mu }(\F )$ is $\RR (\bigoplus _{m\geq m_{0}}\myH ^{0}(\F (m)))$.
Thus by Theorem 2.6 the complex $L_{\mu }(\F )$ is the minimal free
resolution of $\bigoplus _{m\geq m_{0}}\myH ^{0}(\F (m))$.
\end{definition9}
\begin{definition10}
Consider the case
$m=m_{0}=m_{1}=\cdots =m_{n}$. In this case $\TT _{\mu }=\TT ^{\ge m}$ is an
injective resolution of
$P_{m} = \ker (T^{m} \to T^{m+1})$,
and $\mathcal{L}_{\mu }= \widetilde \LL (P_{m})$ has only linear maps.
\end{definition10}


Like free resolutions, partition monads enjoy a strong 
homotopy functoriality:
\begin{theorem30}
The partition monad $\mathcal{L}_{\mu }(\F )$ is
functorial in $\F $ up to homotopy of complexes in such a way that if
$\phi :\,\F \to \G $ is a map, then $\phi =\myH ^{0}\mathcal{L}_{\mu }(\phi )$. Moreover, any map
of complexes $\mathcal{L}_{\mu }(\F )\to \mathcal{L}_{\mu }(\G )$ is determined up to homotopy
by the induced map $\F =\myH ^{0}(\mathcal{L}_{\mu }(\F ))\to \myH ^{0}(\mathcal{L}_{\mu }(\G ))=\G $.
\end{theorem30}
\begin{proof}  The first statement follows from the homotopy functoriality
of $\TT $ and $\LL $. For the second statement, it suffices to show
that every map $\mathcal{L}_{\mu }(\F )\to \mathcal{L}_{\mu }(\G )$ is homotopic to a map
of the form $\mathcal{L}_{\mu }(\phi )$. But every map $\TT _{\mu }(\F )\to \TT _{\mu }(\G )$
is homotopic to a map $\TT _{\mu }(\phi )$, and since $\TT _{\mu }(\G )$ is
an injective resolution, it is homotopic to $\RR \LL \TT _{\mu }(\G )$.
Using the adjointness of $\RR $ and $\LL $, we see that, up to 
homotopy, indeed every map is in the image of the the composite homomorphism
\begin{equation*}\begin{split}
\myHom (\F ,\G )&\to \myHom (\TT _{\mu }(\F ),\TT _{\mu }(\G ))\\
&\to \myHom (\TT _{\mu }(\F ),\RR \LL \TT _{\mu }(\G ))\\
&=
\myHom (\LL \TT _{m}u(\F ), \LL \TT _{\mu }(\G ))\\
&\to \myHom (\mathcal{L}_{\mu }(\F ), \mathcal{L}_{\mu }(\G )).
\end{split}\end{equation*}

\end{proof}
\begin{theorem31}
The cohomology of the complexes $L_{\mu }$ and
$L_{\mu }^{*}=\myHom _{S}(L_{\mu }, S)$ are given by
\begin{equation*}\myH ^{i}L_{\mu } = \bigoplus _{d\ge m_{i}-i} \myH ^{i} \F (d); \qquad \myH ^{n-i}L_{\mu }^{*} = 
\bigoplus _{d< m_{i}-i} \myH ^{i} (\F (d))^{*}\otimes \textstyle{\bigwedge ^{v}}V,
\end{equation*}
where $\myH ^{i} \F (d)$ occurs in degree $d$ and 
$\myH ^{i} (\F (d))^{*}\otimes \bigwedge ^{v}V$
occurs in  degree $-n-d-1$. In particular, for $j<0$ we
have $\myH ^{j}L_{\mu }= \myH ^{j}L_{\mu }^{*} = 0$.
\end{theorem31}
\begin{proof}  Let $\myP _{\mu }$ be the complex
\begin{equation*}0\to T_{\mu }^{m_{0}}\to \cdots \to T_{\mu }^{m_{n}}\to \myim (T_{\mu }^{m_{n}}\to T_{\mu }^{m_{n}+1})
\to 0,
\end{equation*}
so that the complex $\TT _{\mu }$ is an injective resolution of $\myP _{\mu }$.
By part $b)$ of Theorem 3.7 the linear part
of the injective
resolution of $\myP _{\mu }$
is the sum of the
linear complexes
$\RR (\myH ^{i}\LL (\myP _{\mu }))$. Thus
$\myH ^{i}L_{\mu }=\myH ^{i}\LL (\myP _{\mu })= \bigoplus _{e\ge m_{i}} \myH
^{i} \F (e-i)$,
by the definition of $\TT _{\mu }$.

For the  proof of the second formula we first observe that
$L_{\mu }^{*}=\myHom _{S}(L_{\mu }, S)=\min \LL (\myHom _{K}(\myP _{\mu },K))$.
Since $\TT (\F )$ is exact, the induced map
$\TT ^{\mu }[-1]\to \TT _{\mu }$ is a quasi-isomorphism.
Moreover, this map factors through $\myP _{\mu }$.
Thus $\TT ^{\mu }[-1]$ is a projective resolution of
$\myP _{\mu }$, and $\myHom _{K}(\TT ^{\mu },K)[1]$ is an
injective resolution of $\myHom _{K}(\myP _{\mu },K)$.
The terms with $\myH ^{i}$ on the right-hand side of the desired equality
correspond to the $(v-1-i)^{\myth }$ linear strand of $\myHom (\TT ^{\mu },K)[1]$.
Again by Theorem 3.7, the second formula follows.\end{proof}
\begin{theorem32}
Any partition monad
$\mathcal{L}_{\mu }(\F )$ satisfies $\mathcal{L}_{\mu }^{i}=0$ for $|i|>n$.
\end{theorem32}
\begin{proof}  If $L_{\mu }^{i}\neq 0$ but $L_{\mu }^{i+1}=0$, then Nakayama's Lemma implies that 
$\myH ^{i}(L_{\mu })\neq 0$, and similarly for the dual. 
Proposition 8.7
completes the argument.\end{proof}


It is easy to give bounds on the line bundles that
can occur in a partition monad.
Given the sequence  $\mu =(m_{0}\leq \cdots \leq m_{n})$,
it will be
convenient to extend the definition of $m_{i}$ to all $i\in \Z $
by the formulas
\begin{equation*}m_{i}=\begin{cases}m_{0} &\text{if $i<0$},\\
          m_{n} & \text{if $i>n$}.\end{cases}
\end{equation*}
\begin{theorem33}
If $\mathcal{O} (-a)$ is a summand of the $i^{\myth }$ term of the partition monad
$\mathcal{L}_{\mu }(\F )$,  then
\begin{equation*}m_{i}  \leq a+i \leq m_{i+n},
\end{equation*}
where the definition of $m_{i}$ is extended to all $i \in \Z $ as
above. 
\end{theorem33}
\begin{proof}  By Corollary 3.6
\begin{equation*}\lin \,L_{\mu }=\lin \, \LL ( \TT _{\mu }(\F ))= \bigoplus _{e} \LL (\myH ^{e} \TT _{\mu }(\F )),
\end{equation*}
so the $i^{\myth }$ term  of $L_{\mu }$ is
$ \bigoplus _{e} \myH ^{e}(\TT _{\mu })_{e-i} \tensor S(i-e).$
For the first inequality we have to show that if
$\myH ^{e}(\TT _{\mu })_{e-i} \not = 0$, then $m_{i}\leq -(i-e)+i=e$. Since
\begin{equation*}
T^{e}_{\mu }= \bigoplus _{j: e \ge m_{j}} \omega _{E}(j-e) \tensor \myH ^{j}\F (e-j)
\end{equation*}
and $\omega _{E}$ is zero in negative degrees, the condition
$(T^{e}_{\mu })_{e-i} \not =0$ implies $j-e+e-i \ge 0$ for some $j$ with $e \ge m_{j}$.
Thus $i \leq j$ and $m_{i}\leq m_{j}\leq e$, as desired.

For the second inequality we argue similarly, using
$\myH ^{e}(\TT _{\mu }) \cong \myH ^{e-1}(\TT ^{\mu }).$
\end{proof}


Note that if $\mathcal{L}$ is a monad for a sheaf $\F $, then so is
$\mathcal{L}\oplus \A $, where
$\A $ is an acyclic complex---for example, the
sheafification of the free resolution of a module of finite length.

The main result of this section is that partition monads are
characterized by the conditions in 
Corollary 8.8
and 
Corollary 8.9 
up to adding
a direct sum of copies of the sheafified free resolution of 
the residue class field of $S$. In most cases, these summands
cannot occur.
\begin{theorem34}
Let $\mathcal{L}$ be a monad for a coherent sheaf $\F $ on $\myP ^{n}$,
and let
$\mu = (m_{0}\leq \cdots \leq m_{n})$.
If 
$\mathcal{L}^{i}=0$ for $|i|>n$ and the
terms $\mathcal{L}^{i} =\bigoplus _{j} \mathcal{O} (-a_{ij})$ satisfy 
$m_{i}\leq a_{ij}+i\leq m_{i+n}$ for
all $i,j$,
then
$\mathcal{L}$ is isomorphic to the direct sum of $\mathcal{L}_{\mu }(\F )$
and a sum
$\A =\bigoplus _{i=1}^{n} \widetilde \LL (\omega _{E}^{r_{i}}(-m_{i})[-i])$
of twisted Koszul complexes.
Moreover, $r_{i}$ can be nonzero only if $m_{i-1}=m_{i}$.
In particular, if the $m_{i}$ are strictly increasing,
or if we assume that no direct summand of $\mathcal{L}$ is a monad for $\F $,
then  $\mathcal{L}\cong \mathcal{L}_{\mu }(\F )$.
\end{theorem34}
\begin{proof}  
Set $L=\Gamma _{*}\mathcal{L}$, 
and let $\K ^{i}$ and $\myB ^{i}$ be the  kernel and the image of the differential
$d^{i}:\mathcal{L}^{i}\to \mathcal{L}^{i+1}$, respectively. We begin by identifying the homology
of $L$. Note that $(L^{i})_{d}=0$ for $d<m_{i}-i$, so 
$(\myH ^{i}L)_{d} = 0$ for $d<m_{i}-i$ too.

Since $\mathcal{L}$ is exact at $\mathcal{L}^{-i}$ for
$i>0$, and $\mathcal{L}^{-n-1}=0$, we
can use the sequences 
$ 
0\to \myB ^{-i-1}\to \mathcal{L}^{-i}\to \myB ^{-i}\to 0
$
to show that $\myH ^{1}\myB ^{-i}(d)=0$ for $i\geq 2$ and
all $d$. Thus $\myH ^{0}\mathcal{L}^{-j}(d)$ surjects onto $\myH ^{0}\myB ^{-j}(d)$ for $j\geq 1$
and all $d$. It follows that $\myH ^{-i}(L)=0$ for $i>0$, while
$\myH ^{0}(L)$ is the cokernel of 
$\bigoplus _{d}\myH ^{0}\myB ^{-1}(d)\to \bigoplus _{d}\myH ^{0}\K ^{0}(d)$. 

For $0\leq i< n$ the space
$\myH ^{i+1}\myB ^{-1}(d)$ injects into $\myH ^{n}\myB ^{i-n}(d)$.
But $\myH ^{n}\myB ^{i-n}$ is the image of $\myH ^{n}\mathcal{L}^{i-n}$; the
hypothesis on the $a_{ij}$ implies that this cohomology vanishes for
$d\geq m_{i}-i$. In particular, 
$\myH ^{1}\myB ^{-1}(d)=0$ for $d\geq m_{0}$, and it follows that 
$\myH ^{0}L=\bigoplus _{d\geq m_{0}}\myH ^{0}\F (d)$.

We next prove that for each $i>0$ there is a short exact sequence
\begin{equation*}
0\to k^{r_{i}}(m_{i}-i)  
\to \myH ^{i}(L) \to \bigoplus _{d\geq m_{i}-i} \myH ^{i} \F (d)\to 0,
\tag*{($*$)}\end{equation*}
where $r_{i}=0$ unless $m_{i-1}=m_{i}$.
In fact we shall identify this sequence with the direct sum,
over $d\geq m_{i}-i$, of the sequences
\begin{equation*}\myH ^{i}\myB ^{-1}(d)\to \myH ^{i}\K ^{0}(d)\to \myH ^{i}\F
(d)\to \myH ^{i+1}\myB ^{-1}(d),
\end{equation*}
which come from the sequence
$0\to \myB ^{-1}\to \K ^{0} \to \F \to 0
$
expressing the fact that $\mathcal{L}$ is a monad for $\F $.

We have already seen that $\myH ^{i+1}\myB ^{-1}(d)=0$ for $d\geq m_{i}-i$.
It follows that the right-hand term of $(*)$ is 0 for
$d\geq m_{i}-i$, and the left-hand term is 0 unless $m_{i}=m_{i-1}$,
in which case it is $k^{r_{i}}$, where
$r_{i}={\mathrm{h}}^{i}\myB ^{-1}(m_{i-1}-(i-1))$.

From the long exact sequences in cohomology---and, in case $i=n=1$, the
hypothesis on the $a_{1,j}$---we see that 
$\myH ^{i} L=\bigoplus _{d\geq m_{i}-i}\myH ^{1}\K ^{i-1}(d)$.
For all $0<i\leq n$ we have 
\begin{equation*}\bigoplus _{d\geq m_{i}-i} \myH ^{1} \K ^{i-1}(d)=\bigoplus _{d\geq m_{i}-i} \myH ^{i} \K ^{0}(d).
\end{equation*}
These identifications and vanishing
identify the two exact sequences as required.

Set $\TT '=\min \RR (L)$ so that $L= \min \LL (\TT ')$
by Theorem 8.1.
Since $\TT '$ is a complex of free $E$-modules which coincides with the
exact complex $\TT (\F )$
in large cohomological degrees, we can construct a map of complexes
$ \TT ' \to \TT (\F ).$

By Corollary 3.6 we have
$\lin \TT ' = \bigoplus _{i} \RR (\myH ^{i} (L)).$
By the first part of the proof, the terms of 
$\lin \TT ' $ can be nonzero only in the
range of (internal and cohomological) degrees where $\lin \TT _{\mu }$
is equal to $\TT $.
Hence $\TT '$ is mapped to $\TT _{\mu }$,
and we obtain a morphism of monads from the composition
$L = \min \LL (\TT ') \to \LL (\TT ') \to \LL (\TT _{\mu })
\to L_{\mu }$.
The morphism of monads induces an isomorphism in homology
$\F =\myH ^{0}(\mathcal{L}) \to \myH ^{0}(\mathcal{L}_{\mu })=\F $,
because by Theorem 8.1 $\TT '$ and $\TT _{\mu }$ coincide in large
cohomological degrees.

The induced map
$\myH ^{i}(L) \to \myH ^{i}(L_{\mu })$ 
is the surjection of the first part of the proof. Hence the
map  $\TT '\to \TT _{\mu }$ is onto. Its kernel has terms
$\omega _{E}^{r_{i}}(-m_{i})[-i]$, and degree considerations show that
it is a trivial complex. Because these terms occur
in degrees where $\TT _{\mu }$ coincides with the acyclic complex $\TT $,
the differential of $\TT '$ carries
the generators of these modules into boundaries of $\TT _{\mu }$.
Thus, after a change of generators in $\TT '$, we see that 
$\TT '$ is the direct sum of $\TT _{\mu }$ and the trivial
complex $\bigoplus _{i=1}^{n}\omega _{E}^{r_{i}}(-m_{i})[-i]$.\end{proof}
\begin{definition11}
Beilinson's free monad $\myB $ for $\F $ with terms
\begin{equation*}\myB ^{i} = \bigoplus _{p} \myH ^{i-p}(\Omega ^{p}(p) \tensor \F
) \tensor \mathcal{O} (-p)\end{equation*}
is the partition monad for
$\mu = (0,1,\ldots , n)$. This follows from 
Theorem 8.10.
\end{definition11}
\begin{definition12}
Let $c$ be an integer,
and let $\F $ be a  sheaf such that $\sum _{e} \myH ^{i}\F (e)$ is finitely
generated for $i \le c$. Choose
\begin{equation*}m_{0}<m_{1}< \cdots < m_{c} << 0 << m_{c+1} < \cdots < m_{n}\end{equation*}
such that $\myH ^{i} \F (m-i)=0$ for $m < m_{i}$ and $i \le c$, and $\myH ^{i}\F (m-i)=0$ for $m\ge m_{i}$ and
$i> c$.
The monad $\W =\W (\F ,c)=\mathcal{L}_{\mu }(\F )$ does not depend on the precise
values of the $m_{i}$, and hence has only  terms
\begin{equation*}0 \to \W ^{c-n+1} \to \cdots \to \W ^{0} \to \cdots \to \W ^{c} \to 0\end{equation*}
by 
Proposition 8.9. 
 By 
Theorem 8.10 $\W (\F ,c)$ is the unique minimal free
monad of $\F $ with nonzero components only from $c-n+1$ up to $c$.
Thus $\W (\F ,c)$ is the monad constructed by 
Walter \cite{Wal} with cohomology
$\myH ^{i}(\Gamma _{*}\W )= \sum _{e} \myH ^{i}\F (e) \hbox { for } i=0,\ldots ,c$
and zero otherwise.
\end{definition12}
\begin{definition13}
Consider a smooth rational surface 
$X \subset \myP ^{4}$ of degree $d=11$ and sectional
genus 10. The existence of three families of rational surfaces
with these invariants is known; see Schreyer \cite{Sch} or Decker and
Schreyer \cite{DS2}. The Tate resolution of the ideal sheaf of these
surfaces has shape
\begin{equation*}\begin{matrix}*\; *  & 1 & . & . & . & . & . & . & \ldots \\
*\; *  &39 & 30& 10& . & . & . & . &\ldots \\
\ldots & . &.  & . & 2 & . & . & . &\ldots \\
\ldots & . &.  & . & 1 & 5 & 5 & . &\ldots \\
\ldots & . &.  & . & . & . & 5 & 32&*\;* \\
\end{matrix}
\end{equation*}
with $h^{0}\I _{X}(6)=32$ as a reference point.
We display four  monads for $\I _{X}$. The first two monads are monads with
only 4 terms:
\begin{equation*}\begin{matrix}\W (I_{X},2):&&
 \mathcal{O} (-4)^{10}  \to & \mathcal{O} (-3)^{20}  \to & \mathcal{O}
(-2)^{11}  \to &\mathcal{O} (-1)^{2},\\
&&&&&\\
\W (I_{X},1): & \mathcal{O} (-6)^{2}  \to & \mathcal{O} (-5)^{10} \to &
\mathcal{O} (-4)^{10} \to & \mathcal{O} (-2).\qquad \
\end{matrix}
\end{equation*}
The first monad is linear because $\W (I_{X},2)=\mathcal{L}(\TT ^{\ge 2})$.
The two following monads are somewhat more complicated and hence 
less convenient:
\begin{equation*}\begin{matrix}\mathcal{L}(\TT ^{\ge 4}): & \mathcal{O} (-7)^{1} 
\to &\mathcal{O} (-6)^{7} \to &\mathcal{O} (-5)^{20} \to &\mathcal{O} (-4)^{20}
\to &\mathcal{O} (-3)^{5},\\
&&&&&\\
&&&&&\\
\W (I_{X},0):&  \begin{matrix}\mathcal{O} (-9)^{5} \to \\\\\\\end{matrix}
&\begin{matrix}\mathcal{O} (-8)^{20}
\to \\\\\\\end{matrix}
&
\begin{matrix}\mathcal{O} (-7)^{26}\to \\\\\\\end{matrix}
& \begin{matrix}\mathcal{O} (-6)^{7}\\\oplus \\\mathcal{O} (-5)^{5}.\\\quad & \\\end{matrix}
\end{matrix}
\end{equation*}

These monads, up to twist,  are  Beilinson monads for $\I _{X}(m)$ for
$m=1,2,3$ and $5$,
respectively.
The construction of such surfaces in \cite{Sch}
was done by a computer search for monads of
 shape $\W (I_{X},1)$.
\end{definition13}
\begin{remark1}
The degree of smooth rational surfaces in $\PP ^{4}$ is bounded, according to
Ellingsrud and Peskine \cite{EPe}. Smooth rational surfaces with sectional genus
 $\pi >0$ (this excludes the cubic scroll and the projected Veronese  surface)
have a linear Walter monad. Indeed, by Severi's theorem $\myH ^{1}(I_{X}(1))=0$ and hence $\W (I_{X},2)=\mathcal{L}(\TT ^{\ge 2})$. The numerical type of these monads is
\begin{equation*}\mathcal{O} (-4)^{\pi }\to \mathcal{O} (-3)^{2\pi +s-2}\to \mathcal{O} (-2)^{\pi +2s-3} \to \mathcal{O} (-1)^{s}\end{equation*}
with $s=h^{1}\mathcal{O} _{X}(1)=\pi -d+3$.

The conjectured bound is $d \le 15$. Perhaps even $d\le 11$ is true.
However, at present the best known bound is $d \le 52$; see Decker and 
Schreyer \cite{DS2}.
\end{remark1}



\bibliographystyle{amsalpha}
\begin{thebibliography}{EPSW}

\bibitem[ABW]{ABW}
K.~Akin, D.~A.~Buchsbaum, and J.~Weyman: Schur functors and Schur
complexes. Adv. in Math.~44 (1982) 207--278.
\MR{84c:20021}

\bibitem[AO]{AO}
V.~Ancona and G.~Ottaviani: An introduction to the derived
categories and the theorem of Beilinson. Atti Accademia
Peloritana dei Pericolanti, Classe I de Scienze Fis. Mat. et Nat.
LXVII (1989) 99--110.
\MR{92g:14013}

\bibitem[AAH]{AAH}
A.~Aramova, L. A.~Avramov, J.~Herzog:
Resolutions of monomial ideals and cohomology over exterior algebras.
Trans. Amer. Math. Soc. 352 (2000) 579--594.
\MR{2000c:13021}

\bibitem[Bar]{Bar}
W.~Barth: Moduli of vector bundles on the projective plane.
Invent. Math. 42 (1977) 63--91.
\MR{57:324}

\bibitem[Bei]{Bei}
A.~Beilinson: Coherent sheaves on $\mathbf{P} ^{n}$ and problems
of linear algebra.
Funct. Anal. and its Appl. 12 (1978) 214--216.
(Trans. from Funkz. Anal. i. Ego Priloz. 12 (1978) 68--69.)
\MR{80c:14010b}

\bibitem[BGG]{BGG}
I.~N.~Bernstein, I.~M.~Gel'fand and S.~I.~Gel'fand:
Algebraic bundles on $\mathbf{P}^{n}$ and problems of linear algebra.
Funct. Anal. and its Appl. 12 (1978) 212--214.
(Trans. from Funkz. Anal. i. Ego Priloz 12 (1978) 66--67.)
\MR{80c:14010a}

\bibitem[BE]{BE}
D.~A.~Buchsbaum and D.~Eisenbud: Generic free resolutions and
a family of generically perfect ideals. Adv.~Math.~18 (1975) 245--301.
\MR{53:391}

\bibitem[Buc1]{Buc1}
R.-O.~Buchweitz: Appendix to
Cohen-Macaulay modules on quadrics, by R.-O. Buchweitz, D. Eisenbud,
and J. Herzog. In {\sl Singularities,
representation of algebras, and vector bundles (Lambrecht, 1985),\/}
Springer-Verlag Lecture Notes in Math. 1273 (1987) 96--116.
\MR{88h:14001}

\bibitem[Buc2]{Buc2}
R.-O.~Buchweitz: Maximal Cohen-Macaulay modules and
Tate-Cohomology over Gorenstein ring
Preprint (1985).


\bibitem[DE]{DE}
W.~Decker and D.~Eisenbud: Sheaf algorithms using the
exterior algebra,
in {\em Computations in Algebraic Geometry with Macaulay2},
ed.~D.~Eisenbud, D.~Grayson, M.~Stillman, and B.~Sturmfels.
Springer-Verlag, New York, 2001.

\bibitem[DS1]{DS1}
W.~Decker and F.-O.~Schreyer: On the uniqueness of the
Horrocks-Mumford bundle. Math. Ann. 273 (1986) 415--443.
\MR{87d:14009}

\bibitem[DS2]{DS2}
W.~Decker and F.-O.~Schreyer: 
Non-general type surfaces in $\mathbf{P}^{4}$: 
Some remarks on bounds and constructions. 
J. Symbolic Comp. 29 (2000), 545--582.
\MR{2002a:14064}

\bibitem[Dec]{Dec}
W.~Decker: Stable rank 2 bundles with Chern-classes
$c_{1}=-1,\ c_{2}=4$. Math. Ann. 275 (1986) 481--500.
\MR{87j:14017}


\bibitem[Eis]{Eis}
D.~Eisenbud: {\sl Commutative Algebra with a View Toward
Algebraic Geometry.} Springer-Verlag, New York, 1995.
\MR{97a:13001}

\bibitem[EG]{EG}
D.~Eisenbud and S.~Goto: Linear free resolutions and minimal
multiplicity.  J. Algebra 88 (1984) 89--133.
\MR{85f:13023}

\bibitem[EPo]{EPo}
D.~Eisenbud and S.~Popescu:
Gale Duality and Free Resolutions of Ideals of Points.
Invent. Math. 136 (1999) 419--449.
\MR{2000i:13014}

\bibitem[EPSW]{EPSW}
D.~Eisenbud, S.~Popescu, F.-O.~Schreyer and C.~Walter:
Exterior algebra methods for the Minimal Resolution Conjecture,
Duke Math. J. 112 (2002), 379--395.


\bibitem[EPY]{EPY}
D.~Eisenbud, S.~Popescu, and S.~Yuzvinsky: Hyperplane
arrangements and resolutions of monomial ideals over an
exterior algebra. Trans. Amer. Math. Soc., this issue.


\bibitem[ES1]{ES1}
D.~Eisenbud and F.-O.~Schreyer: Sheaf cohomology and free resolutions over 
the exterior algebras,
{\tt http://arXiv.org/abs/math.AG/0005055}
Preprint (2000).

\bibitem[ES2]{ES2}
D.~Eisenbud, F.-O.~Schreyer, and Jerzy Weyman: Resultants and 
Chow forms via exterior syzygies.
J. Amer. Math. Soc. 16 (2003) 537--579.

\bibitem[EPe]{EPe}
G.~Ellingsrud and C.~Peskine: Sur le surfaces lisse de $\mathbf{P}^{4}$.
Invent. Math. 95 (1989) 1--12.
\MR{89j:14023}

\bibitem[EW]{EW}
D.~Eisenbud and J.~Weyman:
Fitting's Lemma for ${\mathbb{Z}}/{2}$-graded modules.
Trans. Amer. Math. Soc., this issue.


\bibitem[Flo1]{Flo1}
G.~Fl\o ystad: Koszul duality and equivalences of categories.
{\tt http://arXiv.org/ abs/math.RA/0012264}
Preprint (2000a).

\bibitem[Flo2]{Flo2}
G.~Fl\o ystad: Describing coherent sheaves on projective spaces via Koszul
duality.
{\tt http://arXiv.org/abs/math.RA/0012263}
Preprint (2000b).

\bibitem[Flo3]{Flo3}
G.~Fl\o ystad: Monads on projective spaces. 
Communications in Algebra. 28 (2000c) 5503-5516.
\MR{2002e:14070}

\bibitem[Gel]{Gel}
S.~I.~Gel'fand: Sheaves on $\mathbf{P}^{n}$ and problems in linear algebra.
Appendix to the Russian edition
of [OSS], Mir, Moscow, 1984.
\MR{86i:14005}

\bibitem[GM]{GM}
S.~I.~Gel'fand and Yu.~I.~Manin: {\em Methods of Homological Algebra.\/}
Springer-Verlag, New York, 1996.
\MR{97j:18001}

\bibitem[GS]{GS}
D.~Grayson and M.~Stillman: Macaulay2. {\tt
http://www.math.uiuc.edu/Macaulay2/}.

\bibitem[Gre]{Gre}
M.~Green: The Eisenbud-Koh-Stillman Conjecture on Linear Syzygies.
Invent. Math. 136 (1999) 411--418.
\MR{2000j:13024}

\bibitem[GD]{GD}
A.~Grothendieck and J.~Dieudonn\'{e}:
{\em \'{E}l\'{e}ments de la G\'{e}ometrie Alg\'{e}brique IV:
\'{E}tude locale des sch\'{e}mas et de morphismes de sch\'{e}mas}
(deuxi\`{e}me partie). Inst. Hautes Etudes Sci. Publ. Math. 24 (1965).
\MR{33:7330}

\bibitem[Hap]{Hap}
D.~Happel: {\em Triangulated Categories in the Representation Theory
of Finite-Dimensional Algebras.\/} London
Math.~Soc.~Lecture Notes Ser. 119, 1988.
\MR{89e:16035}

\bibitem[Har]{Har}
R.~Hartshorne: {\em Algebraic geometry \/}, Springer Verlag, 1977.
\MR{57:3116}

\bibitem[HR]{HR}
J.~Herzog and T.~R\"{o}mer: Resolutions of modules over the exterior algebra,
working notes, 1999.

\bibitem[Hor]{Hor}
G.~Horrocks: Vector bundles on the punctured spectrum of a local ring, Proc. 
London Math. Soc.
(3) 14 (1964), 714--718.
\MR{30:121}

\bibitem[HM]{HM}
G.~Horrocks and D.~Mumford: A rank $2$ vector bundle on $\mathbf{P}^{4}$ with 
15,000 symmetries. Topology 12 (1973) 63--81.
\MR{52:3164}

\bibitem[Kap1]{Kap1}
M.~M.~Kapranov:
On the derived categories of coherent sheaves on some homogeneous spaces.
Invent. Math. 92 (1988) 479--508.
\MR{89g:18018}

\bibitem[Kap2]{Kap2}
M.~M.~Kapranov:
On the derived category and $K$-functor
of coherent sheaves on intersections of quadrics. Izv. Akad.
Nauk SSSR Ser. Mat. 52 (1988) 186--199; English transl., Math. USSR Izv. 32
(1989), 191--204.
\MR{89f:14014}

\bibitem[MP]{MP}
M.~Martin-Deschamps and D.~Perrin: {\sl Sur la classification des courbes 
gauches.\/}
Ast\'{e}risque 184--185 (1990).
\MR{91h:14039}

\bibitem[OSS]{OSS}
C.~Okonek, M.~Schneider, and H.~Spindler: {\sl Vector Bundles on
Complex Projective Spaces.\/} Birkh\"{a}user, Boston 1980.
\MR{81b:14001}

\bibitem[Orl]{Orl}
D.~O.~Orlov:
Projective bundles, monoidal transformations,
and derived categories of coherent sheaves. Izv. Ross. Akad. Nauk
Ser. Mat. 56 (1992) 852--862; translation in Russian Acad. Sci. Izv. Math.
41 (1993) 133--141.
\MR{94e:14024}

\bibitem[Pri]{Pri}
S.~B.~Priddy: Koszul resolutions. Trans. Amer. Math. Soc. 152
(1970) 39--60.
\MR{42:346}

\bibitem[Rao]{Rao}
P.~Rao: Liaison equivalence classes, Math. Ann. 258 (1981) 169--173.
\MR{83j:14045}

\bibitem[Sch]{Sch}
F.-O.~Schreyer: Small fields in constructive algebraic geometry.
In S.~Maruyama ed.: {\sl Moduli of Vector Bundles, Sanda 1994}. New York, Dekker 1996,
221--228.
\MR{97h:14050}

\bibitem[Swa]{Swa}
R.~G.~Swan: $K$-theory of quadric hypersurfaces. Ann. of Math.
122 (1985) 113--153.
\MR{87g:14006}

\bibitem[Wal]{Wal}
C.~Walter: Algebraic cohomology methods for the normal bundle of algebraic space curves.
Preprint (1990).

\end{thebibliography}
\end{document}
