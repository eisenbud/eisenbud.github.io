%_ **************************************************************************
%_ * The TeX source for AMS journal articles is the publishers TeX code     *
%_ * which may contain special commands defined for the AMS production      *
%_ * environment.  Therefore, it may not be possible to process these files *
%_ * through TeX without errors.  To display a typeset version of a journal *
%_ * article easily, we suggest that you retrieve the article in DVI,       *
%_ * PostScript, or PDF format.                                             *
%_ **************************************************************************
% Author Package file for use with AMS-LaTeX 1.2

\controldates{25-FEB-2003,25-FEB-2003,25-FEB-2003,25-FEB-2003}
 
\RequirePackage[warning,log]{snapshot}
\documentclass{jams-l}
\issueinfo{16}{3}{July}{2003}
\pagespan{537}{579}
\dateposted{February 27, 2003}
\PII{S 0894-0347(03)00423-5}
\copyrightinfo{2003}{American Mathematical Society}
\usepackage{amscd}
\usepackage[cmtip,all]{xy}


\newtheorem{theorem}{Theorem}[section]
\newtheorem{proposition}[theorem]{Proposition}
\newtheorem{corollary}[theorem]{Corollary}
\newtheorem{lemma}[theorem]{Lemma}
\newtheorem*{problem}{Problem} 

\theoremstyle{definition}
\newtheorem{example}[theorem]{Example}
\newtheorem{furtherwork}[theorem]{Remark}

\theoremstyle{remark}
\newtheorem*{remark}{Remark} 

\newcommand{\thr}{{\mathrm{th}}} 
\newcommand{\st}{{\mathrm{st}}} 

\newcommand{\A}{{\mathcal A}}
\newcommand{\B}{{\mathcal B}}
\newcommand{\CC}{{\mathbf C}}
\newcommand{\C}{{\mathcal C}}
\newcommand{\DD} {{\mathbf D}}
\newcommand{\cE} {{\mathcal E}}
\newcommand{\E} {{\mathcal E}}
\newcommand{\F}{{\mathcal F}}
\newcommand{\cF}{{\mathcal F}}
\newcommand{\FF}{{\mathbf F}}
\newcommand{\G}{{\mathcal G}}
\newcommand{\GG}{{\mathbf G}}
\newcommand{\K}{{\mathcal K}}
\newcommand{\Hrm}{{\mathrm H}}
\newcommand{\cH}{{\mathcal H}}
\newcommand{\cI}{{\mathcal I}}
\newcommand{\KK}{{\mathbf K}}
\newcommand{\cL}{{\mathcal L}}
\newcommand{\Lcal}{{\mathcal L}}
\newcommand{\LL}{{\mathbf L}}
\newcommand{\MM}{{\mathbf M}}
\newcommand{\N}{{\mathcal N}}
\newcommand{\Ocal}{{\mathcal O}}
\newcommand{\cO}{{\mathcal O}}
\newcommand{\PP}{{\mathbf P}}
\newcommand{\cQ}{{\mathcal Q}}
\newcommand{\RR}{{\mathbf R}}
\newcommand{\cT}{{\mathcal T}}
\newcommand{\TT}{{\mathbf T}}
\newcommand{\cU}{{\mathcal U}}
\newcommand{\U}{{\mathcal U}}
\newcommand{\UU}{{\mathbf U}}
\newcommand{\Z}{{\mathbf Z}}
\newcommand{\ZZ}{{\mathbf Z}}
\newcommand{\gm}{{\mathfrak m}}

\newcommand{\reg}{\mathop{\rm regularity}}
\newcommand{\coker}{\mathop{\rm coker}}
\newcommand{\im}{\mathop{\rm im}}
\newcommand{\Chow}{\mathop{\rm Chow}}
\newcommand{\length}{\mathop{\rm length}}
\newcommand{\supp}{\mathop{\rm supp}}
\newcommand{\depth}{\mathop{\rm depth}}
\newcommand{\socle}{\mathop{\rm socle}}
\newcommand{\Pic}{\mathop{\rm Pic}}
\newcommand{\codim}{\mathop{\rm codim}}
\newcommand{\ann}{\mathop{\rm ann}}
\newcommand{\rank}{\mathop{\rm rank}}
\newcommand{\sing}{\mathop{\rm Sing}}
\newcommand{\iso}{\cong}
\newcommand{\tensor}{\otimes}
\newcommand{\dsum}{\oplus}
\newcommand{\intersect}{\cap}
\newcommand{\Hom}{{\mathop{\rm Hom}\nolimits}}
\newcommand{\GL}{{\operatorname{GL}}}
\newcommand{\Ext}{\text{\rm Ext}}
\newcommand{\extt}{\text{\rm Ext}}
\newcommand{\Sym}{{\mathop{\rm Sym}\nolimits}}
\newcommand{\sym}{{\mathop{\rm Sym}\nolimits}}
\newcommand{\D}{{\mathop{\rm Sym}\nolimits}}
\newcommand{\coh}{{\operatorname{Coh}}}
\newcommand{\lin}{{\operatorname{lin}}}
\newcommand{\h}{{\mathrm h}}

\newcommand{\cal}{\mathcal}

\newcommand{\rTo}{\xrightarrow}
\newcommand{\lTo}{\xleftarrow}
\newcommand{\rTox}{\rightarrow}
\newcommand{\lTox}{\leftarrow}


\newcommand{\Fitt}{\mathop{\rm Fitt}}
\newcommand{\Supp}{\mathop{\rm Supp}}

\begin{document}

\title{Resultants and Chow forms via exterior syzygies}

\author[David Eisenbud and Frank-Olaf Schreyer]{David Eisenbud}
\address{Department of Mathematics, University of California, Berkeley,
Berkeley, California 94720}
\email{eisenbud@math.berkeley.edu}

\author[]{Frank-Olaf Schreyer}
\address{Mathematik und Informatik, Geb.\ 27,
Universit\"at des Saarlandes,
D-66123 Saar\-br\"ucken,
Germany}
\email{schreyer@math.uni-sb.de}

\author[]{Appendix by Jerzy Weyman}
\address{Department of Mathematics, Northeastern
University, Boston, Massachusetts 02115}
\email{weyman@neu.edu}

\date{November 16, 2001}

\keywords{Chow form, resultants, Beilinson monad, Ulrich modules.}
\subjclass[2000]{Primary 13P05, 14Q99; Secondary 13D25, 14F05.}


\begin{abstract}
Given a sheaf on a projective space ${\mathbf P}^n$,
we define a sequence of canonical and effectively 
computable {\it Chow
complexes\/} on the Grassmannians of planes in ${\mathbf P}^n$, generalizing
the well-known Beilinson monad on ${\mathbf P}^n$. If the sheaf has
dimension $k$, then the Chow form of the associated $k$-cycle is the
determinant of the Chow complex on the Grassmannian of planes of
codimension $k+1$.  Using the theory of vector bundles and the
canonical nature of the complexes, we are able to give explicit
determinantal and Pfaffian formulas for resultants in some cases where no
polynomial formulas were known. For example, the Horrocks--Mumford
bundle gives rise to a polynomial formula for the resultant of five
homogeneous forms of degree eight in five variables.
\end{abstract}
\maketitle

Let $W$ be a vector space of dimension $n+1$ over a field $K$. The {\it Chow
divisor\/} of a $k$-dimensional variety $X$ in ${\mathbf P}^n={\mathbf P}(W )$
is the
hypersurface, in the Grassmannian $\GG_{k+1}$ of planes of codimension $k+1$ in
${\mathbf P}^n$, 
whose points (over the algebraic closure of $K$) are the planes that
meet $X$. The {\it Chow form} of $X$ is the defining equation of the Chow
divisor. For example, the resultant of $k+1$ forms of degree $e$ in $k+1$
variables is the Chow form of ${\mathbf P}^{k}$ 
embedded by the $e$-th Veronese mapping
in ${\mathbf P}^n$ 
with $n={k+e\choose k} -1$. More generally, the Chow divisor of a
$k$-cycle $\sum_i n_i [V_i]$ on projective space is defined to be $\sum_i n_i
D_i$, where $D_i$ is the Chow divisor of $V_i$. The Chow divisor of a sheaf
$\F$ with $k$-dimensional support is the Chow divisor of the associated
$k$-cycle of $\F$.


In this paper we will give a new expression for the Chow divisor and apply it
to give explicit formulas in many new cases. Starting with a sheaf $\F$ on
${\mathbf P}^n$, 
we use exterior algebra methods to define a canonical and effectively
computable {\it Chow complex\/} of $\F$ on each Grassmannian of planes in
${\mathbf P}^n$. 
If $\F$ has $k$-dimensional support, we show that the Chow form of $\F$
is the determinant of the Chow complex of $\F$ on the Grassmannian of planes
of codimension $k+1$. The Beilinson monad of $\F$ \cite{Beilinson 1978} is 
the Chow
complex of $\F$ on the Grassmannian of 0-planes (that is, on ${\mathbf P}^n$ 
itself.)

In particular, we are able to give explicit determinantal and Pfaffian formulas
for resultants in some cases where no polynomial formulas were known. For
example, the Horrocks-Mumford bundle gives rise to polynomial formulas for the
resultant of five homogeneous forms of degrees 4, 6 or 8 in five variables.
The easiest of our new formulas to write down is for the resultant of 3
quadratic forms in three variables, the Chow form of the Veronese surface in
${\mathbf P}^5$. Using 
the tangent bundle of ${\mathbf P}^2$,
conclude that it can be written
in ``B\'ezout form'' (described below) as the Pfaffian of the matrix

\[
\begin{pmatrix} \scriptscriptstyle  0    
&\scriptscriptstyle   [245] &\scriptscriptstyle [345] 
&\scriptscriptstyle   [135] &\scriptscriptstyle   [045] 
&\scriptscriptstyle    [035]&\scriptscriptstyle    [145] 
&\scriptscriptstyle     [235]   \cr\scriptscriptstyle -[245] 
&  \scriptscriptstyle   0   &\scriptscriptstyle -[235] 
&\scriptscriptstyle   [035] &  \scriptscriptstyle  [025] 
&  \scriptscriptstyle  [015] &  \scriptscriptstyle  [125] 
& \scriptscriptstyle -[125]+[045] \cr
\scriptscriptstyle -[345] &  \scriptscriptstyle [235] 
&  \scriptscriptstyle  0   & \scriptscriptstyle [134] 
&\scriptscriptstyle    [035] &   \scriptscriptstyle [034] 
& \scriptscriptstyle   [135] &     \scriptscriptstyle[234] \cr
\scriptscriptstyle -[135] & \scriptscriptstyle -[035] 
& \scriptscriptstyle -[134] &  \scriptscriptstyle  0   
&  \scriptscriptstyle [023] &   \scriptscriptstyle  [013] 
& \scriptscriptstyle [123]-[034]&\scriptscriptstyle   -[123]   \cr
\scriptscriptstyle -[045] & \scriptscriptstyle -[025] 
& \scriptscriptstyle -[035] & \scriptscriptstyle -[023] 
& \scriptscriptstyle    0   & \scriptscriptstyle [012] 
&  \scriptscriptstyle -[015] & \scriptscriptstyle -[024]+[015] \cr
\scriptscriptstyle -[035] & \scriptscriptstyle -[015] 
& \scriptscriptstyle -[034] &\scriptscriptstyle  -[013] 
& \scriptscriptstyle  -[012] & \scriptscriptstyle    0  
& \scriptscriptstyle [023]-[014] &\scriptscriptstyle  -[023]  \cr
\scriptscriptstyle -[145] & \scriptscriptstyle -[125] & 
\scriptscriptstyle -[135] &\scriptscriptstyle-[123]+[034] 
& \scriptscriptstyle  [015] &\scriptscriptstyle-[023]+[014] 
& \scriptscriptstyle 0 &\scriptscriptstyle -[124]+[035] \cr
\scriptscriptstyle -[235] &\scriptscriptstyle [125]-[045] 
&\scriptscriptstyle -[234] &   \scriptscriptstyle [123] & 
\scriptscriptstyle [024]-[015] & \scriptscriptstyle [023] 
& \scriptscriptstyle [124]-[035] &\scriptscriptstyle 0   
\end{pmatrix}
\]

\noindent where the brackets $[ijk]$ denote the 
corresponding Pl\"ucker coordinates of the space spanned by the three given
quadratic forms with respect to the ordered basis $x^2,xy,xz,y^2,yz,z^2$ for
the space of all quadratic forms. 

There are (at least) two types of formulas for resultants or Chow forms:

\subsection*{B\'ezout formulas for
resultants.} The classic formula of B\'ezout (see, for example, 
\cite[Chapter 12, (1.17) and (1.18 )]{Gelfandetal.1994}) gives 
the resultant of two
homogeneous forms in two variables as a determinant of linear forms in the
Pl\"ucker coordinates of the space generated by the two forms. We will call any
formula for the Chow form in Pl\"ucker coordinates a {\it B\'ezout
expression}. Our simplest new B\'ezout expression is for the resultant of three
quadratic forms in three variables: it is the Pfaffian ($\equiv$ square root of
the determinant) of the alternating matrix of linear forms in the Pl\"ucker
coordinates. Using the theory of rank two vector bundles on ${\mathbf P}^2$, 
we can
construct B\'ezout formulas for forms in three variables of any degree. In
fact, we construct continuous families of such formulas.


\subsection*{Stiefel formulas for resultants}

The Grassmannian is a quotient of an
open set in the variety of $(k+1) \times (n+1)$ matrices over $K$; the
entries of these matrices are called {\it Stiefel coordinates} on the Grassmannian
(or on the Stiefel manifold). Pulling back the Chow divisor, we get a divisor
whose ideal is generated by a polynomial in the Stiefel coordinates. 
For example,
if $X$ is the rational normal curve, this polynomial is the Sylvester determinant.
Even when we cannot express the Chow form of a variety as the determinant or Pfaffian
of a matrix in the Pl\"ucker coordinates, we can sometimes express it as
the determinant or Pfaffian of a map of equivariant vector bundles on the 
Grassmannian. 
Such maps pull back to matrices in the Stiefel coordinates whose determinant
or Pfaffian defines the (closure of the) preimage of the Chow divisor.
We say that such a matrix gives a {\it Stiefel expression\/} for
the Chow form. The classical Sylvester determinant is such
a Stiefel expression. 

Explicit Stiefel expressions
for the resultant of $k+1$ forms of degree $d\geq 2$ in $k+1$ variables
(Chow form of the $d$-uple embedding of $\PP^k$) have been known
for $k\leq 3$ (all $d$) and $k=4$, $d=2,3$ and $k=5$, $d=2$ 
(see, for example,
\cite[Chapter 13, Prop.~1.6]{Gelfandetal.1994},
\cite{D'Andrea and Dickenstein 2001}. 
Using our method and constructions of vector bundles on $\PP^k$, we give new
Stiefel expressions.
In particular,
the Horrocks--Mumford bundle 
gives rise to Pfaffian Stiefel expressions for\linebreak
the resultants of 5 forms of degrees 4, 6, or 8 in 5 variables.
The matrices involved are too large to exhibit here; but
Macaulay2 programs
for producing them\linebreak and other new examples can be found at
\texttt{http://msri.org/people/staff/de/\linebreak ChowM2scripts}.

\section*{What is the Chow divisor of a sheaf?}


Let
\[
\PP^n \lTo{\pi_1} \FF_l  \rTo{\pi_2} \GG_l
\]  
be the incidence correspondence; that is, let 
\[
\FF_l = \{(p,L)\in \PP^n\times \GG_l \mid p\in L\}.
\] 
In the case
$l=k+1$, the Chow divisor of a
$k$-dimensional subvariety
$X\subset \PP^n$ is  by definition $D_X=\pi_2(\pi_1^{-1}X)$;
one can check immediately that this is a divisor in $\GG_l$.

More generally 
if $\F$ is any sheaf on $\PP^n$ whose
support $Y$ has dimension $k$, then the $k$-cycle associated
to $\F$ is 
\[
\sum_{\dim X = k} 
\length_{\Ocal_{\PP^n, X}} (\Ocal_{\PP^n,X}\otimes \F)\cdot [X],
\]
where the sum is taken over all 
$k$-dimensional subvarieties of
$\PP^n$ (or equivalently over the $k$-dimensional components
of $Y$).
The Chow divisor of $\F$ is defined to be the corresponding
sum of Chow divisors
\[
\sum_{\dim X = k} 
\length_{\Ocal_{\PP^n, X}} (\Ocal_{\PP^n,X}\otimes \F)\cdot D_X.
\]
For example, if $X\subset \PP^n$ 
is any subvariety of dimension $k$
and if $\F=\Ocal_X$ (or any line bundle on $X$),
then the Chow divisor of $\F$ is the Chow divisor of $X$;
more generally if $\F$ is a vector bundle of rank $r$
on the $X$, then the Chow divisor
of $\F$ is $r$ times that of $X$.


Since the generic plane of codimension $k+1$
meeting a component $X$ of $Y=\Supp \F$ meets $X$ in just
one general point of $X$, we see (in the case $l=k+1$)
that
$\G=(\pi_2)_*\pi_1^* \F$ is supported precisely on
the set $\pi_2\pi_1^{-1}Y$. The same argument shows
that the generic rank
of $\G$ on $D_X$ is 
$\length_{\Ocal_{\PP^n, X}} \Ocal_{\PP^n,X}\otimes \F.$
Thus the Chow divisor of $\F$ is actually the 
divisor associated to the sheaf $\G$ on $\GG_l$.

\subsection*{Finding the divisor associated to a sheaf}

To make use of this idea,
we need to be able to go from the sheaf $\G$,
supported in codimension 1, to a description of the divisor
that is its support. A divisor is best described as
a line bundle and a global section of that line bundle.
Any line bundle on the Grassmannian
is a power of the hyperplane bundle, so
the divisor can be represented simply as
a polynomial in the Pl\"ucker coordinates, the Chow form.
However, the sheaf $\G$
itself, from which we must start,
 may be very complicated. For example, it may have high projective
dimension and embedded components.

Consider for a moment the general problem of computing
the divisor associated to a sheaf $\G$ with codimension 1 support on
a smooth variety $Z$. 
We suppose that $\G$ is
presented as the cokernel
of a map $\phi: A\to B$ of vector bundles,
and we wish to find---as explicitly as possible---a 
line bundle on $Z$ and
a global section of it whose divisor is the divisor
of $\F$. 

Let
$b$ be the rank of $B$. The $b$-th exterior power
of $\phi$ is a map $\bigwedge^b\phi: \bigwedge^t A \to \bigwedge^b B$,
which gives rise to a map
$\bigwedge^b A \otimes \bigwedge^b B^* \to \Ocal_Z$. The
zero-th Fitting ideal $\Fitt_0(\G)$ of $\G$
 is by definition the 
image of this map. The divisor associated to $\G$ is
the same as the divisor associated to $\Ocal_Z/\Fitt_0(\G)$.
(This may be proved by localizing
at a prime of codimension 1 and then using
\cite[Example A.2.3]{Fulton 1984}.) 
Since $\G$ has codimension 1 support we must have
$\rank A \ge \rank B$. In case $\rank A = \rank B$, the
desired line bundle is  
$\bigwedge^b A^* \otimes \bigwedge^b B
$
and the
dual
of the map 
$\bigwedge^b A \otimes \bigwedge^b B^* \to \Ocal_Z$
sends $1\in \Ocal_Z$ to the desired global section.
This means
that the divisor of $\G$ is defined by the 
determinant of $\phi$. 
One of the central goals of this paper,
in the setting of Chow forms, is to give a
simple characterization of some sheaves
$\F$ for which the corresponding
sheaf $\G$ on the Grassmannian is naturally presented by
a map between vector bundles of the same rank
or
has a presentation by a map represented by
a square matrix of linear forms. These are the
``weakly Ulrich sheaves'' and ``Ulrich
sheaves'' described below.

When $\rank A > \rank B$,
the situation is much more complicated. 
The desired divisor is defined locally by the
greatest common divisor of the $b\times b$ minors
of a matrix representing $\phi$, but this is much
less explicit than the description above.

A better generalization of the case $\rank A = \rank B$ 
was discovered (in a special case) by Arthur Cayley
\cite{Cayley 1848} and greatly generalized by Grothendieck in
an unpublished letter to David Mumford in 1962; the details are
worked out in \cite{Knudsen and Mumford 1976}, where the letter is
described. In brief, the
divisor of $\F$ is the {\it determinant} divisor of any finite
complex $\C$ of vector bundles whose homology differs from
$\G$ only in codimension $\geq 2$. In the local case
the determinant of $\C$ may be represented as 
a rational function, the alternating
product of certain minors in matrices representing the 
differentials of $\C$. A good introduction to part
of the Cayley--Grothendieck theory can be found in
the
Appendix A of \cite{Gelfandetal.1994}.

\subsection*{Chow complexes} 

Let us now return to the setting of the 
Chow form, and take $Z=\GG_l$, with $l=k+1$ and
$\G=(\pi_2)_*\pi_1^* \F$. Cayley studied the case
where $X$ is the $d$-th Veronese embedding of $\PP^k$
and $\F$ is a sufficiently positive line bundle on $\PP^k$.
He produced an explicit free resolution of $\G$ to
play the role of the complex $\C$ above. His constructions were
studied and generalized by
Macaulay, Jouanolou and  other authors
who derived in this way expressions for resultants as rational functions in the
Pl\"ucker or Stiefel coordinates.  
For modern results, see \cite{Weyman and Zelevinsky 1994},
\cite{Jouanolou 1995}. An exposition may be found in \cite{Gelfandetal.1994}.
Of course the Chow form is
a polynomial:  in these rational function expressions the denominator
divides the numerator. However, it is not known how to make the
quotient explicit. Refinements aimed at reducing the degree of the
denominator are an active subject of research; see, for example,
\cite{D'Andrea and Dickenstein 2001}
and \cite{D'Andrea 2002}.

Grothendieck extended Cayley's theory to apply to any sheaf
$\F$. He observed that  there exists a locally free
complex $\C$, well-defined up to quasi-isomorphism, with 
\[
\C \simeq {{\mathbf R}\pi_2}_*(\pi_1^*\F)
\]  
and $\Hrm^0\C\iso \G$ while
$\Hrm^i\C$ is isomorphic to the higher
direct image ${{\mathbf R}\pi_2}_*(\pi_1^*\F)$, which is supported
in codimension $\geq 2$ in $\GG$.
Thus the Chow divisor of $\F$ is the divisor of the determinant
of $\C$ (in Cayley's case all the higher
direct images are 0).
The problem with Grothendieck's idea 
is that for general $\F$
it has not been possible until now to give
an effectively computable 
complex $\C \simeq {{\mathbf R}\pi_2}_*(\pi_1^*\F)$.
However, the determinant is so robust that
it can be computed from the associated graded complex
of a filtered complex representing
${{\mathbf R}\pi_2}_*(\pi_1^*\F)$; such complexes can
sometimes be
computed from spectral sequences (the construction
of the Chow form as the ``determinant of a spectral sequence''
by Weyman and Zelevinsky, described in
\cite[Section 3.4.C]{Gelfandetal.1994}, is such
a computation).

\section*{Main Results}

This paper divides naturally into two parts. In Sections 1--3
we treat the general theory of Chow complexes and (weakly) Ulrich
sheaves. In Sections 4--6 we deal with various families of
examples and with the question of the existence of (weakly)
Ulrich sheaves in these examples.

\subsection*{Describing a Chow complex.}
Our first main result gives a canonical 
Chow complex 
\[
\UU_{k+1}(\F) \simeq {{\mathbf R}\pi_2}_*(\pi_1^*\F)
\]  
for each coherent sheaf $\F$, part of a 
sequence of complexes generalizing the Beilinson monad for $\F$.
The construction is so explicit that it can be made on a computer.
Recall that a plane of any
codimension $l$ in $\PP^n$ corresponds to
an $(n+1-l)$-quotient of $W$, 
and thus to an $l$-dimensional subspace of
$W$. We write $U_l$ for 
the tautological $l$-subbundle on $\GG_l$.

\begin{theorem}\label{first main} For any coherent sheaf $\F$ on $\PP^n$
and any $0\leq l\leq n$ there
is a canonical complex $\UU_l(\F)$ of
vector bundles on $\GG_l$ 
with
\[
\UU_l(\F) \simeq {{\mathbf R}\pi_{2}}_*(\pi_{1}^*\F).
\] 
The $e$-th term of $\UU_l(\F)$ is
$
\sum_j \Hrm^j(\F(e-j))\otimes\bigwedge^{j-e}U_l.
$
\end{theorem}
The complex $\UU_n(\F)$ is the Beilinson monad on $\PP^n$
that is defined in\linebreak \cite{Eisenbudetal.2001}. The sheaf $\F$ can be
recovered from $\UU_n(\F)$ simply by taking homology. The sheaf 
$\F$ can be recovered from some
of the other $\UU_l(\F)$ as well: Just as one
can recover a variety of dimension
$k$ from its Chow divisor in $\GG_{k+1}$, so one can recover any sheaf $\F$
whose support has dimension at most $k$ from the Chow complex $\UU_l(\F)$
as long as $l>k$. All these matters are explained in Section~\ref{chow
complex}.
 
Most significant in our treatment is that we can give
an explicit and canonical description of the maps
in the complex $\UU_l$. Until now, in general,
it has only been possible
to write down the sheaves in such a complex 
(see, for example, \cite[Section 3.4E]{Gelfandetal.1994},
``Weyman's complexes") or to approximate
the maps via a spectral sequence. With
enough vanishing of cohomology, it was possible to write
down the maps; but these cases were often not
the ones of primary interest.
Also, previous authors seem only to have considered 
formulas coming from the case where $\F$ is a line bundle on its support.
Our technique allows us to recover  explicit expressions of the Chow
form in all the previously known cases, and, using vector bundles
as in the examples mentioned above, some new ones.

\subsection*{Finding simple Chow complexes. }
The most useful formulas for the Chow form
occur when the complex $\UU=\UU_{k+1}(\F)$ has just one nontrivial map $\Psi$:
\[
\UU: \cdots\rTox 0\rTox 0\rTox C^{-1}\rTo{\Psi} C^{0}\rTox 0\rTox 0\rTox\cdots .
\]
In this case the 
Chow form of $\F$ is given by the determinant of $\Psi$, and if the bundles 
$C^i$ are direct sums of exterior powers
of the tautological bundles, then one gets a determinantal
expression for the Chow form in Stiefel coordinates.

An even better case occurs when 
$C^{-1}$ and $C^0$ are direct sums of line bundles.  Then  the 
 Chow form of $\F$ is given
directly as a determinant in the Pl\"ucker coordinates---that is, we get a 
B\'ezout expression  for a
power of the Chow form of the support of $\F$. If $\F$ has
rank 1, or if $\F$ has rank 2 and the map $\Psi$ is skew-symmetric,
so that we can extract the square root of the determinant as the
Pfaffian, then we get the Chow form of the support of $\F$ itself.

Such cases are considered in Section~\ref{Ulrich}.
Our second main result describes precisely the
conditions on the sheaf $\F$ that are necessary for the Chow complex
$\UU_{k+1}(\F)$ to degenerate to one of these special forms. For example:

\begin{theorem}\label{second main} The Chow complex $\UU_{k+1}(\F)$
  above degenerates 
to a single map $\Ocal_\GG^d(-1)\to\Ocal_\GG^d$ if and only if the 
module of twisted global sections 
$\bigoplus_m \Hrm^0(\F(m))$ is a Cohen--Macaulay module
with a linear free resolution.
\end{theorem}

Here by a linear free resolution we mean a 
free resolution over the polynomial ring $S=k[x_0,\ldots,x_r]$ of the form
\[
\cdots\rTox S^{r_2}(-2)\rTox S^{r_1}(-1)\rTox S^{r_0}.
\] 
Cohen--Macaulay graded modules with linear free resolutions $M$
were studied by Bernd Ulrich under the name ``maximally generated
maximal Cohen--Macaulay modules" \cite{Ulrich 1984} and have been
studied by others
under the names ``linear maximal Cohen--Macaulay modules" or 
``Ulrich modules"; see \cite{Brennanetal.1987}, 
\cite[1989]{Backelin and Herzog} and 
the references given there. We shall call the
corresponding sheaves {\it Ulrich sheaves\/}. For example, a line
bundle $\F$ on a curve $X$ of genus $g$ embedded in $\PP^n$ is an
Ulrich sheaf if and only if $\F(-1)$ has degree $g-1$ and no global
sections; that is, $\F(-1)$ corresponds to a point in $\Pic^{g-1}(X)$
which lies outside the theta divisor $\Theta \subset \Pic^{g-1}(X)$.

More generally
we say that a sheaf $\F$ whose Chow complex $\UU_{k+1}(\F)$ has only two terms
is {\it weakly Ulrich}, because the necessary cohomological vanishing is
nearly the same as for Ulrich sheaves. For example, a Cohen--Macaulay module
$M=\bigoplus_m \Hrm^0(\F(m))$ with $\Hrm^0(\F(-1))=0$ and $\Hrm^0(\F) \not= 0$
corresponds to a weakly Ulrich sheaf if and only if $M$ is 1-regular. 
 
\subsection*{Duality and Pfaffian expressions}

In Section~\ref{atiyah} we turn to the problem of giving
determinantal and Pfaffian expression for the Chow form of an 
Ulrich sheaf $\F$.
We can express them
directly in terms of the free resolution of the corresponding module $M$
by using a construction developed in \cite{AngeniolandLejeune-Jalabert1989}
to describe Atiyah classes. Suppose that
\[
0\to F_c\rTo{\phi_c}\cdots \rTox F_1\rTo{\phi_1}F_0
\]
is a linear free resolution of $M$ as above.
Regarding the $\phi_i$ as matrices of elements of $W$, we can compose
them as if they were matrices of linear forms in the exterior
algebra: we write $\Psi_\F:=(1/c!)\phi_1\wedge\phi_2\wedge\cdots\wedge \phi_c$
for this product (defined in a slightly different way in
positive characteristic), which is represented by a matrix of 
forms in $\bigwedge^cW$. We may identify $\bigwedge^cW$ with the 
the space of linear forms on $\GG_{k+1}$ and we have:

\begin{theorem}\label{third main} If $\F$ is an Ulrich sheaf, then
$\Psi_\F$ is the (only) nonzero map in the Chow complex $\UU_{k+1}(\F)$.
In particular
the Chow form of $\F$ is $\det \Psi_\F$. If $\F$ is
a vector bundle on a $k$-dimensional variety $X$, and
$\F$ is skew-symmetric in an appropriate sense, then
(in characteristic not $2$) $\Psi_\F$  is skew-symmetric, 
and the square-root of the Chow form of $\F$ is the
Pfaffian of $\Psi_\F$.
\end{theorem}

Theorem~\ref{third main} gives a new method for constructing
resultants and Chow forms:
find Ulrich sheaves (or weakly Ulrich sheaves or Ulrich
sheaves satisfying the skew-symmetry condition\dots) and then
construct the map $\Psi_\F$. Given $\F$, we derive two
ways to do this. Most practical for most explicit computations
is the method illustrated in Section~\ref{curves} in the 
classical case of binary forms as well as other situations,
including doubly periodic functions (it is also the method
used for toric surfaces in \cite{Khetan 2002}): one computes
the multiplication map on global sections, $W\otimes H^0\F(n)\to \F(n+1)$,
forms from it a map of free modules over the exterior algebra, and
takes a certain syzygy matrix of the kernel of this map.
On the other hand, we get somewhat more theoretical control from
the product formula above; for example, we need this to prove
Theorem~\ref{third main}.

\subsection*{Explicit computations }

Sections 4--6 of
this paper treat a number of examples. 
We can be completely explicit in
only a few cases, and these sections 
leave open a multitude of theoretical and practical problems.
Of greatest importance is this:

\begin{problem} Is every variety (or even scheme) $X\subset \PP^n$ 
the support of an Ulrich 
sheaf? If so, what is the smallest possible rank for such a sheaf?
\end{problem}

For example, Brennan, Herzog, and Ulrich showed that when $X$ is
an arithmetically Cohen--Macaulay curve over an infinite field, or a
a complete intersection, or a linear determinantal variety, then
$X$ has an Ulrich sheaf \cite{Brennanetal.1987}
and \cite{Backelin and Herzog}. In case
$X$ is the hypersurface $F=0$, for example,
they construct an Ulrich sheaf whose rank is exponential
in the number of monomials required to express $F$.
For a regular quadric hypersurface
in $\PP^r$ the situation is 
completely 
understood: There  
\cite{BuchweitzEisenbudandHerzog1987}
(see also \cite{Swan 1985}) show that the minimal rank of an Ulrich module
is precisely $2^{\lfloor (r)/2\rfloor-1}$. 
Also, in the case of plane curves over an algebraically closed
field, Ulrich modules of rank 1 always exist
(see below.)

Turning to Veronese embeddings of projective spaces,
Doug Hanes showed
in his thesis under Hochster that
the $d$-uple embeddings of $\PP^k$ have Ulrich sheaves when
$k\leq 2$ or $k=3$ and $d=2^r$ is a power of two \cite{Hanes 2000}. We prove a
number of new existence results, which we now summarize.

\subsection*{Curves. }
Section~\ref{curves} is devoted to the case of curves. 
We complete (and reprove) 
the result of
Brennan, Ulrich, and Herzog by showing that,
if the ground field is infinite,
every curve $X\subset\PP^n$ 
has skew-symmetric rank 2 Ulrich sheaves.
If the field is algebraically closed, there are rank 1 Ulrich sheaves; they are in
one-to-one correspondence with the line bundles of degree $g-1$ on $X$ 
that have
no sections. Thus there are
B\'ezout expressions for the Chow forms of such curves.
This generalizes the case
of binary forms, in which
$X=\PP^1$ and the line bundle is $\Ocal_{\PP^1}(-1)$.
It also generalizes the
well-known result that the equation 
of any plane curve over an algebraically
closed field can be written as the determinant of 
a matrix of linear forms; see for example 
\cite{Vinnikov 1989,Beauville 2000}.

Such Ulrich sheaves give rise, in principle, to continuous 
families of resultant formulas for the
sections of any very ample line bundle on a curve of genus $\geq 1$, 
but it is not easy to make
such formulas explicit. We illustrate with the case of hyperelliptic
curves and provide a resultant formula for functions of the form
$a+b\sqrt{f}$, \; $c+d\sqrt{f}$ where $a,b,c,d$ and $f$ are polynomials
in one variable. We carry out the proof completely only in case the degrees
of the various polynomials are small. In the special case of elliptic curves, we get
a resultant formula for doubly periodic functions written in terms of the
Weierstrass $\wp$-function and its derivative.

\subsection*{Projective spaces: New resultant formulas. }
In Section~\ref{Projective spaces} we turn to the case where $X\subset \PP^n$ is
the $d$-th Veronese ($d$-uple) embedding of $X=\PP^k$. This is the 
case that gives rise to resultant formulas for $k+1$ forms of degree $d$
in $k+1$ variables. We give cohomological criteria for a bundle on
$\PP^k$ to be Ulrich for the $d$-uple embedding.
Following a suggestion of Jerzy Weyman, we use this to 
extend Hanes' results and show that
every Veronese variety has an Ulrich sheaf, obtained
by applying a certain (unique) Schur functor to the tautological
quotient bundle. This gives a way of writing a power
of the resultant as the determinant of a matrix of linear
forms in the Pl\"ucker coordinates. This 
might be useful for computation:
to determine whether or not a set
of polynomials has a common zero, a power of the resultant 
is just as good as the resultant itself.

In Section~\ref{Projective spaces} we also find (many) Ulrich modules of rank 2 for each
Veronese embedding of $\PP^2$. 
We prove a lower bound on the ranks of possible Ulrich modules
and using this and a result of Hartshorne--Hirschowitz on the 
existence of mathematical instanton
bundles, we show that rank 2 Ulrich modules exist on the $d$-uple embedding
of $\PP^3$ if and only if $d$ is not divisible by 3. 
On $\PP^4$ we show that the Horrocks--Mumford bundle is weakly Ulrich
for the 4, 6, and 8-uple embeddings and satisfies
the skew-symmetry condition necessary for us to get a Pfaffian Stiefel formula
for the corresponding resultants.

\subsection*{Surfaces}

Section~\ref{Surfaces} is concerned with the existence of skew-symmetric rank 2 Ulrich sheaves
on various surfaces and thus with Pfaffian resultant formulas generalizing the B\'ezout
formula for $\PP^2$ given at the beginning of this introduction. We use Mukai's
construction of vector bundles on surfaces and describe the necessary data. Our main result 
is the existence of skew-symmetric rank 2 Ulrich 
modules for certain embeddings
of the plane blown up at a set of points.
These modules lead to Pfaffian B\'ezout expressions for the
resultant of 3 ternary forms of degree $d$ with assigned simple base points, valid when
the ideal defining the set of base points is generated in degree  $<d$.

\subsection*{Maps of tautological sheaves. }
Throughout this paper we rely on a certain construction of
homomorphisms between exterior powers of the tautological bundle
on a Grassmannian, explained in Section~\ref{chow complex}. In the Appendix
Jerzy Weyman proves---in all characteristics---that 
in fact every homomorphism arises from this construction.

\section{Chow complexes obtained from the Beilinson monad}
\label{chow complex}

As above we
write $\GG_l$ for the Grassmannian of 
planes of codimension $l$ in $\PP:=\PP^n=\PP(W)$
and   $\FF_l$ for the flag variety
of flags consisting of a point $p \in \PP$ and a
plane $L\in \GG_l$ of codimension $l$ in $\PP$
containing $p$. Throughout this section
we will consider the incidence correspondence
\[
\PP \lTo{\pi_1} \FF_l  \rTo{\pi_2} \GG_l.
\]
Let $0 \rTox U \rTox W \tensor \Ocal_{\GG_l}  \rTox Q \rTox 0$
be the tautological sequence on the Grassmanian $\GG_l$,
so that $U=U_l$ is a bundle of rank $l$. We write $E$ for the
exterior algebra $\bigwedge V$, where $V=W^*$. Any element
$a\in \bigwedge^p V$ gives rise to a ``contraction'' mapping
$a: \bigwedge^qU \to \bigwedge^{q-p}U$ as follows: 
First, $a$ induces a homomorphism
$\bigwedge^pW\to K$ by the usual contraction,
and thus a homomorphism of sheaves
\[
\bigwedge^pU\hookrightarrow 
\bigwedge^pW\tensor \Ocal_{\GG_l} \to 
\Ocal_{\GG_l}.
\]
Using the diagonal map \cite[A2.4, p.~582]{Eisenbud 1995}
$\bigwedge^q U\rTo{\Delta_U}\bigwedge^{q-p}U\otimes\bigwedge^pU$,
we get the desired map
\[
\bigwedge^qU\rTo{(1\tensor a)\Delta_U}\bigwedge^{q-p}U.
\]

We use the well-known part (a) of the following lemma
heavily. Part (b) is needed for Proposition~\ref{determine F}.

\begin{proposition}\label{Hom computation} Let $U=U_l$ be the tautological
subbundle on $\GG_l$.
\begin{enumerate}
\item[(a)] The maps above make
$\bigwedge U$ into a module over $\bigwedge V$. 
\item[(b)]{\rm\ (J. Weyman)} The maps
\[
\bigwedge^pV \to \Hom(\bigwedge^q U, \bigwedge^{q-p}U)
\]
are isomorphisms for all integers 
$p,q$ such that $0\leq q{-}p,q\leq l$.
\end{enumerate}
\end{proposition}

\begin{proof} (a) With notation as above,
the naturality of the diagonal maps shows that the 
diagrams
\[
\begin{CD}
\bigwedge^qU& @>>>& \bigwedge^qW\otimes \Ocal_{\GG_l}\\
@V{(1\tensor a)\Delta_U}VV && @VV{(1\tensor a)\Delta_W}\tensor 1V\\
\bigwedge^{q-p}U& @>>> &
\bigwedge^{q-p}W\tensor\Ocal_{\GG_l}
\end{CD}
\]
commute. Since $\bigwedge W$ is  naturally a module over $E=\bigwedge V$
by this action
(see for example \cite[Appendix A2.4.1]{Eisenbud 1995}), so is $\bigwedge U$.

\medbreak
(b) This is proved in an appendix to this
paper by J.~Weyman. In characteristic 0 the result follows
from Bott's vanishing
theorem \cite{Jantzen 1987}. In arbitrary characteristic it is more delicate.
\end{proof}

We will grade $E$ by the convention that the elements of $V$ have
degree $-1$. As usual we write $E(q)$ for the free graded $E$-module
of rank 1, with generator in degree $-q$. Thus, for example, if $q>p$, then
$\Hom(E(q), E(q-p))=E_{-p}=\bigwedge^pV$. Recall from 
\cite{Eisenbudetal.2001} that a {\it Tate resolution\/}
is a doubly infinite exact complex of finitely generated free
graded $E$-modules that is {\it minimal,\/} in the sense that each free
module maps into $V$ times the next one.
If $\F$ is any
coherent sheaf on $\PP$, then there is a 
{\it Tate resolution\/} $\TT(\F)$ naturally
associated to $\F$, which can be computed,
using free resolutions over an exterior algebra, from the 
module of twisted global sections $\bigoplus_e \Hrm^0\F(e)$. Its
$e$-th term is isomorphic to
\[
T^e(\F) = \bigoplus_j\Hrm^j(\F(e-j))\otimes E(j-e).
\]
For all this see \cite{Eisenbudetal.2001}.

We can define the additive functor $\UU_l$ from graded free 
modules over $E$ to locally free sheaves on $\GG_{l}$
 by sending $E(p)$ to
$\UU_l(E(p)) = \bigwedge^p U$, where $U=U_l$ is the tautological
subbundle, and 
sending a map $\eta: E(q)\to E(q-p)$ to the map 
$\UU_l(\eta): \bigwedge^{q} U \to \bigwedge^{q-p} U$ made from
the element $\bigwedge^pV$ corresponding to $\eta$.
If $\TT$ is any Tate resolution,  then
for $e\gg 0$ or $e\ll 0$ we have $\UU_l(T^e)=0$, so
$\UU_l(\F):= \UU_l(\TT)$
is a bounded complex of locally free sheaves
on $\GG_l.$ 

For example, $\UU_{n}(\F)$ is 
shown in \cite{Eisenbudetal.2001}
to be a {\it Beilinson
monad\/} for the sheaf $\F$ in the sense that it has the terms above,
and its only homology is $\F$, in degree 0 
(the functor $\UU_{n}$ is called $\Omega$ in that paper).
Here is a generalization for all $l$.

\begin{theorem}\label{incidence complex}
If $\F$  is a sheaf on $\PP^n$,  the complex
$\UU_l(\F)$ represents $\RR{\pi_2}_*(\pi_1^* \F)$ in the 
derived category of sheaves on the Grassmannian $\GG_l$.
\end{theorem}

\begin{proof} 
By \cite[Theorem 6.1]{Eisenbudetal.2001},
$\UU_n(\F)$ is a representative of $\F$ in $D^b(\coh(\PP^n))$.
We will show first that $\UU_l(\F)={\pi_2}_*(\pi_1^* \UU_n(\F))$,
and second that $\RR^i{\pi_2}_*(\pi_1^*(\bigwedge^pU_n))=0$ for $i>0$.
It follows 
that 
${\RR\pi_2}_*\pi_1^* \UU_n(\F)\cong {\pi_2}_*(\pi_1^* \UU_n(\F))
=\UU_l(\F)$,
as desired.

On $\FF$ we have inclusions of the universal subbundles
\[
\pi_2^*(U_l)\subset \pi_1^*(U_n)\subset W\otimes \Ocal_\FF.
\]
Pushing the left hand inclusion forward we get a canonical map
$U_l={\pi_2}_*\pi_2^*U_l \to {\pi_2}_*\pi_1^*U_n$, 
and we deduce similar maps on the exterior powers.
To show that these are isomorphisms, we may compute fiber by
fiber.
If $u \in \GG_l$, then we will also write
$u \subset W$ for the corresponding $l$-dimensional linear subspace.

Setting $\PP'=\PP(W/u)\subset \PP(W)$,
we have the decomposition
\[
\bigwedge^pU_n \mid_{\PP'} \cong
\bigoplus_{i=0}^p
\bigwedge^iu
\otimes
\bigwedge^{p-i}U'_{n-l},
\]
where $U'_{n-l}$ denotes the tautological subbundle on $\PP'$.
Thus the map $\bigwedge^pu\to \Hrm^0(\bigwedge^pU_n\mid_{\PP'})$
is an isomorphism, and all other cohomology of
$\bigwedge^pU_n\mid_{\PP'}$ vanishes.

{}By base change
\cite[III.12]{Hartshorne 1977} we see that
$\RR^i{\pi_2}_*(\pi_1^*\bigwedge^pU_n)=0$ for $i>0$ while
${\pi_2}_*(\pi_1^*\bigwedge^pU_n) \cong \bigwedge^p U_l$.
\end{proof}

The sheaf $\F$ is determined from $\UU_n(\F)$, the Beilinson
monad, by the formula $\F=\Hrm^0(\UU_n(\F))$. 
More generally, when $l>k$, we can still recover $\F$ from
$\UU_l(\F)$.

\begin{proposition}\label{determine F} If $\F$ is a coherent sheaf of 
dimension $k$ on $\PP$ and $l>k$, then $\F$ is determined
by the complex $\UU_l(\F)$.
\end{proposition}

\begin{proof}
The Tate resolution $\TT(\F)$ is determined by any differential 
$\phi_i: T^i(\F)\to T^{i+1}(\F)$, because $\TT^{\ge i+1}(\F)$ 
is the minimal 
injective resolution of $\im \phi_i$ and $\TT^{\le i}(\F)$ 
the minimal projective resolution of  $\im \phi_i$. 
Moreover $\TT(\F)$ determines the Beilinson monad and
hence $\F$. Thus it suffices to reconstruct one of the differentials
of $\TT(\F)$ from $\UU_l(\F)$.


The degrees of the generators of the free module in $T^e(\F)$
range (potentially) from $e-k$ to $e$. Thus the degrees of the
generators of $T^{-1}(\F)$ and $T^{0}(\F)$ range at most from 
$-k-1$ to 0. Replacing the summands $\Lambda^p U$ of $\UU_l(\F)$ by $E(p)$ 
and the maps by the corresponding matrices of exterior forms according to
Proposition~\ref{Hom computation} (b), we recover the differential
$T^{-1}(\F) \to T^0(\F)$ of the Tate resolution.
\end{proof}

Now we come to the case needed for the construction of the Chow divisor.
If $\B$ is a finite complex of locally free sheaves
\[\B: 0 \rTox \ldots \rTox \B^j \rTox \B^{j+1} \rTox \ldots \rTox 0,\]
then the {\it determinant bundle\/} of $\B$ is defined to be
\[
\det(\B) = \prod_{j \textup{ even }} \det(\B^j) 
\tensor
\prod_{j \textup{ odd }} \det(\B^j)^*.
\] 
If $\B$ is generically exact, then there is a
Cartier divisor called the determinant divisor of $\B$
which measures the part of the homology of $\B$ supported in
codimension~1; 
see \cite{Knudsen and Mumford 1976} or \cite[Appendix A]{Gelfandetal.1994} 
for the general definition.
If $\F$ is a $k$-dimensional coherent sheaf on $\PP(W)$,
then the Chow form
$\Chow(\F)$ is the equation of the Chow divisor of $\F$.
It is a section of
$\Ocal_{\GG_{k+1}}(\deg\F)$ defined up to multiplication by a scalar.
The following  theorem is a more explicit version the main result of
\cite[Chapter II]{Knudsen and Mumford 1976}. 

\begin{theorem}\label{chow point} Let $\F$ be a coherent sheaf
on $\PP(W)$. If $\dim \F = k$, then the Chow divisor of
$\F$ is the determinant divisor of the complex $\UU_{k+1}(\F)$.
Moreover, in codimension 1 the only homology of this
complex is at the $0$-th term.
\end{theorem}

We give a proof for the
reader's convenience: 

\begin{proof} We may assume that the ground field is algebraically
closed. 
Since $\UU_{k+1}(\F)$ represents ${\RR\pi_2}_*(\pi_1^*\F)$, its divisor
does not pass through any point $u$ of the Grassmannian 
such that $\supp(\F) \cap \PP(W/u)= \emptyset$. For a general point $u$
of a component of the zero locus of  $\Chow(\F)$
the subspace $\PP(W/u)$ meets the support of $\F$ in a single
point which belongs to a unique component $X$ of the support.
Over the residue class field $\kappa(u)$ of $u \in \GG_{k+1}$ we have
\[
\dim_{\kappa(u)} ({\pi_2}_*\pi_1^* \F) \tensor \kappa(u) 
= \dim_{\kappa(u)} \Hrm^0(\F \tensor \Ocal_{\PP(W/U)})
= \length(\F \tensor \Ocal_{\PP(W/u),X})
\]
and the higher direct images vanish.
Thus ${\RR\pi_2}_*(\pi_1^*\F)$ is generically quasi-\linebreak isomorphic
to ${\pi_2}_*(\pi_1^*\F)$, whose associated divisor is the Chow
divisor of $\F$.
\end{proof}

\section{Ulrich Sheaves} \label{Ulrich}

If $\F$ is a $k$-dimensional sheaf on $\PP^n$
and  $\UU_{k+1}(\cF)$ is a two term complex, then the determinant 
section of $\UU_{k+1}(\cF)$ is the
determinant of a morphism between bundles.
This situation corresponds to the case where the Tate resolution
of $\F$ has ``Betti diagram" of the form
\[
\begin{matrix}
h^k\cF(-k-3)  &h^k\cF(-k-2) & h^k\cF(-k-1) & h^k\cF(-k) & 0&0 \\
       0 & 0 &  h^{k-1}\cF(-k) & h^{k-1}\cF(-k+1) &0&0 \\
                  \vdots&\vdots&\vdots&\vdots &\vdots&\vdots \\
0&0&h^1\cF(-2)&h^1\cF(-1)&0&0\\
0&0&h^0\cF(-1)&h^0\cF&h^0\cF(1) &h^0\cF(2)  \\
\end{matrix}
\]
Here, by the Betti diagram of $\TT(\F)$ we mean
the table whose $(i,j)$ entry
is the number of generators
of degree $j-i$ required by the $j$-th free module
$T^i$ in $\TT(\F)$; by \cite{Eisenbudetal.2001} this is the
dimension of $\Hrm^i(\F(j-i))$.
(This is almost the same as the Betti diagram in the programs
Macaulay of Bayer and Stillman or Macaulay2 of Grayson and Stillman,
except that we think of the arrows in the resolution as going
from left to right. This change of convention is convenient because
of the fact that the generators of $E$ have negative degree.)
For reasons that will become clear in a moment, we will call
a sheaf $\F$ with cohomology as above a {\it weakly Ulrich sheaf\/}.

An even better situation occurs when the Tate resolution has 
Betti diagram of the form
\[
\begin{matrix}
\ldots&h^k\cF(-k-3)  &h^k\cF(-k-2) & h^k\cF(-k-1) &0 & 0&0& \\
      & 0 & 0 &  0 & 0 &0&0 \\
                  &\vdots&\vdots&\vdots&\vdots &\vdots&\vdots \\
&0&0&0&0&0&0\\
&0&0&0&h^0\cF&h^0\cF(1) &h^0\cF(2)&\ldots 
\end{matrix}
\]
In this case we see from the previous section that 
the Chow form of $\cF$ is the determinant of the $h^0(\F)
\times h^k\F(-k-1)$ matrix whose entries are
linear forms in the 
Pl\"ucker coordinates on the Grassmannian $\GG_{k+1}$.
(It follows that $h^0\cF = h^k\cF(-k-1)=\deg(\F)$,
which one can easily see in other ways as well.)

Modules whose associated sheaf have this sort of
Tate resolution were first studied in \cite{Ulrich 1984}. 
We will call them {\it Ulrich sheaves}. Thus a $k$-dimensional sheaf $\F$ 
on $\PP$ is
an Ulrich sheaf  if $\F$ has no intermediate cohomology---that 
is, $\Hrm^q(\F(d))=0$ for
$1\leq q\leq k-1$ and all $d$---and
$\Hrm^0(\F(j))= 0$ for $j<0$ while
$\Hrm^k(\F(j))= 0$ for  $j\geq -k$.
Since an Ulrich sheaf has no intermediate cohomology,
its restriction to the nonsingular part of $X$ is automatically
a vector bundle.

We can characterize Ulrich sheaves without referring to all of the cohomology
in several elementary ways. Since every 0-dimensional sheaf
is an Ulrich sheaf, we will henceforward ignore this case.

\begin{proposition}\label{Ulrich char} Let $\F$ be a coherent, 
$k$-dimensional
sheaf on the projective space $\PP=\PP^n$ over $K$ with $k>0$. The following
are equivalent:
\begin{enumerate}
\item[(a)] $\F$ is an Ulrich sheaf.
\item[(b)] $\Hrm^i \F(-i)=0$ for $i>0$ and $\Hrm^i \F(-i-1)=0$ for $i<k$. 
\item[(c)] If the support of $\F$ is a scheme $X$, then for 
some (respectively all) finite linear projections $\pi:X\to \PP^k$
the sheaf $\pi_*\F$ is the trivial sheaf $\Ocal_{\PP^k}^t$ for some $t$.  
\item[(d)] The
module $M:=\Hrm^0_*(\F):=\bigoplus_d \Hrm^0(\F(d))$ of twisted global sections 
is an {\it Ulrich module,\/} in the sense of 
\cite[1989]{Backelin and Herzog}; that is, $M$ is a
Cohen--Macaulay module of dimension $k+1$ over the homogeneous coordinate ring
$S=k[x_0,\cdots,x_n]$ of\/ $\PP$, whose number of generators is
equal to $\deg \F$, or equivalently whose $S$-free
resolution
\[ 
\FF: 0 \rTox F_{n-k} 
\rTo{\varphi_{n-k}} \ldots 
\rTo{\varphi_2}  F_1 
\rTo{\varphi_1}  F_0 
\rTox M \rTox 0
\]
is linear
in the sense that $F_i$ is generated in degree $i$ for every $i$.
\end{enumerate}
\end{proposition}

\begin{proof} (a) $\Rightarrow$ (b) is trivial.

\medbreak
(b) $\Rightarrow$ (c) By the finiteness and linearity of $\pi$
we have $\Hrm^i(\F(j))=\Hrm^i((\pi_*)\F(j))$.
The vanishing of cohomology of (b) gives vanishing for $\pi_* \F$ which
characterizes the trivial vector bundles on $\PP^k$.
 
\medbreak
(c) $\Rightarrow$ (d) By (c)  $M=\Hrm_*^0(\F)$ is a free
module over $K[x_0,\ldots,x_k]=\Hrm_*^0(\Ocal_{\PP^k})$ generated in degree 0.
Thus $M$ is a linear Cohen--Macaulay module, that is, an Ulrich
module.

\medbreak
(d) $\Rightarrow$ (a) The equivalence of the two
characterizations of Ulrich modules given in (d) may be found in
\cite[Prop.~1.5]{Brennanetal.1987}.
The fact that a graded $S$-module
$M$ is 0-regular if and only if 
the free resolution of $M_{\geq 0}$ is linear
is proved in \cite{Eisenbud and Goto 1984} 
(see also \cite[Theorem 20.18]{Eisenbud 1995}).
If $M$ is a $k+1$-dimensional Cohen--Macaulay module with
linear resolution, then the associated sheaf $\F$ is also 0-regular.
The Cohen--Macaulay property of $M$ gives the vanishing of the
intermediate cohomology of $\F$ and (since $\dim M = k+1 >1$)
also shows that
$M=\Hrm^0_*(\F)$. Thus $\Hrm^0(\F(j))=0$ for $j<0$, and $\F$ is Ulrich.
\end{proof}

From the linearity of the resolution $\FF$ of an Ulrich module 
$M$ it follows, for
example, that the rank of
$F_{i}$ is ${n-k\choose i}\cdot\rank F_0$; to see this,
reduce modulo a maximal
regular sequence and observe that $M$ must reduce to a direct
sum of copies of the residue field $K$. In particular,
$\rank F_{n-k}=\rank F_0$, and this rank is
equal to the degree of $\F$. (For more details, see, for example,
\cite{Brennanetal.1987}.) The same kind of argument gives:

\begin{corollary}\label{coho of Ulrich} 
If $\F$ is an Ulrich sheaf of dimension
$k$ on $\PP^r$, then 
$\chi(\F(e))=h^0(\F){e+k\choose k}$.
\qed
\end{corollary}

In Theorem \ref{d-uple} 
we will generalize this to
sheaves on $X$ that are Ulrich sheaves for the $d$-uple embedding
of $\F$.

In our applications we will be particularly interested in the
case where the Ulrich sheaf is a vector bundle on its support
and is
self-dual up to a twist. In this case the criterion above
can be simplified:

\begin{corollary}\label{self-dual Ulrich}
Let $\F$ be a vector bundle on a $k$-dimensional Gorenstein scheme
$X\subset \PP^r$. If $\F\iso \F^*(k+1)\otimes \omega_X$, then
$\F$ is an Ulrich sheaf on $\PP^r$ if and only if $\F$ is $0$-regular.
\end{corollary}

\begin{proof} The 0-regularity implies that $\Hrm^i(\F(j))=0$ for
$j>-i$. The rest of the necessary vanishing follows from Serre duality.
\end{proof}

\subsection*{First examples.}

Brennan, Herzog and Ulrich discovered  that 
linear determinantal varieties have rank one Ulrich modules
(\cite{Brennanetal.1987}),
so we can
give B\'ezout expressions for their Chow forms using the ideas
above. This series of examples includes rational normal scrolls,
Bordiga--White surfaces and many more.
We can give a different description of their
Ulrich modules as follows:

\begin{example}\label{linear determinantal varieties} 
Let $\varphi \colon F \to G$
with $F=\bigoplus_1^f \cO$ and $G=\bigoplus_1^g \cO(1)$, $f\le g$, be a linear 
$f \times g$ matrix on $\PP^n$ which drops rank in expected
codimension $(f-g+1)$. The Eagon--Northcott type complex 
\[
0 \to \Lambda^f F \tensor D_{f-g+1} G^* \to \ldots \to \Lambda^{g} F
\tensor G^* \to \Lambda^{g-1} F \to \cF \to 0
\]
(see \cite[Theorem A2.10]{Eisenbud 1995}) is a linear resolution of a module
annihilated by the maximal minors of $\varphi$ and has
length $f-g+1$. It is thus the resolution
of an Ulrich sheaf
on $X = V(I_g(\varphi))$, and one can check that the sheaf has rank
1 (it is isomorphic, in the generic case, to $I_{g-1}\varphi'$,
the ideal generated by the $g-1\times g-1$ minors of the submatrix
$\varphi'$ obtained from $\varphi$ by omitting one row). Hence the
Chow form
$Chow(X)=Chow(\cF)$ is polynomial of degree $f \choose {g-1}$ in the Pl\"ucker
coordinates, and $\deg X = {f \choose {g-1}}$. 
\end{example}
  
\begin{example}\label{cubic scroll} Consider the rational normal
scroll $S(2,1) \subset \PP^4$ defined by
\[ \varphi=\begin{pmatrix}x_0&x_1&x_3 \\ x_1 & x_2 & x_4 \\\end{pmatrix}.\]
Using the Ulrich sheaf $\cF$ as above and 
Theorem~\ref{chow of Ulrich}, we obtain its Chow form as the determinant
of the matrix
\[\def~{\phantom{+}}  
\begin{pmatrix} 
~[034] & [013]       & ~[023] \\
-[134] & [023]+[014] & -[123]-[024] \\
~[234] & \llap{$-$}[024]   & ~[124] \\ \end{pmatrix}. 
\]

The Chow forms of  rational normal scrolls have further interpretations:
Consider $(r+1)$-dimensional spaces $\alpha$ of sections of bundles
$\bigoplus_{i=1}^r \Ocal_{\PP^1}(d_i)$. The Chow form of the scroll $S(d_1,\ldots,d_r)
\subset \PP^N$ with $N+1=\sum_i (d_i+1)$ describes those $\alpha$, 
where the minors of the 
corresponding morphism
\[ 
\Ocal_{P^1}^{r+1} \rTo{\alpha} \bigoplus_{i=1}^r \Ocal_{\PP^1}(d_i)
\]
have a common zero. Such formulas were also worked out
by Henri Lombardi and J.-P.~Jouanolou (unpublished).
 
In the case of
$S(2,1)$ there is also an interpretation for plane conics with one assigned 
base point: Since $S(2,1)$ is the image of $\PP^2$ 
by the linear system of conics with
a single assigned base point, say $(1:0:0)$, its Chow form describes
those 3-dimensional subspaces of conics which have a further base point.
In Section 5 we will generalize this example to forms of any degree on 
$\PP^2$ with several simple assigned base points.
\end{example}

From the point of view of examples, it is interesting to note that if
two schemes in projective spaces support (weakly) Ulrich sheaves, then
so does their Segre product:

\begin{proposition}\label{Segre} 
Let $\F_1$ be a coherent sheaf on $\PP(W_1)$ and let
$\F_2$ be a coherent sheaf on $\PP(W_2)$. Set
$d= \dim(\F_1).$ Let $\G$ be the
Segre product of $\F_1$ with $\F_2(d)$ on
$\PP=\PP(W_1\otimes W_2)$; that is, 
$\G=(\pi_1^*\F_1)\otimes(\pi_2^*\F_2(d))$ on the Segre variety
$\PP(W_1)\times \PP(W_2)\subset \PP$.
\begin{enumerate}
\item[(a)] If $\F_1, \F_2$ are weakly Ulrich, then $\G$ is weakly Ulrich.
\item[(b)] If $\F_1, \F_2$ are Ulrich, then $\G$ is Ulrich.
\end{enumerate}
\end{proposition}

Of course a similar result holds for the Segre product
of $\F_1(\dim \F_2)$ and $\F_2$.

\begin{proof} Both parts follow easily from the K\"unneth formula
\[
\Hrm^i(\G(m))=\bigoplus_{i=j+k}\Hrm^j(\F_1(m))\otimes \Hrm^k(\F_2(d+m)).
\]
For example, in part (a) we need 
$\Hrm^j(\F_1(-j-k-2))\otimes \Hrm^k(\F_2(d-j-k-2))=0$
when $j+k<d+\dim \F_2$. If $j<d$, then the first factor
vanishes since $\F_1$ is weakly Ulrich, while if $j=d$,
then the second factor vanishes for the same reason.
\end{proof}

\begin{corollary}\label{product formula} 
With notation as in Proposition~\ref{Segre}, 
suppose that 
$\F_1, \F_2$ are Ulrich sheaves of dimensions $d_1, d_2$,
and let $E_{\rm Segre}=\wedge((W_1\otimes W_2)^*)=
\wedge(W_1^*\otimes W_2^*)$.
The map  
\[
\Hrm^{d_1+d_2}(\G(-d_1-d_2-1))\otimes\omega_{E_{\rm Segre}}
\rTox\Hrm^0(\G)\otimes \omega_{E_{\rm Segre}}
\]
in the Tate resolution of $\G$ 
is derived from the tensor product of the
corresponding maps for $\F_1$ and $\F_2$
over $\bigwedge W_1^*$ and $\bigwedge W_2^*$, respectively, via the
canonical injection 
$\bigwedge W_1^* \otimes \bigwedge W_2^*\subset \wedge(W_1^*\otimes W_2^*)$.
\qed
\end{corollary}

It follows that in situations where
we can compute a B\'ezout expression for the Chow forms of
$\F_1$ and $\F_2$, we can also
compute a B\'ezout expression for the Chow form of the Segre
product. Similar remarks and formulas hold in the case of 
weakly Ulrich sheaves and Stiefel expressions of the Chow form.

\section{Chow forms as determinants and Pfaffians}\label{atiyah} 

Throughout this section we will work with a sheaf $\F$ of
dimension $k$ on $\PP^n=\PP(W)$. For simplicity, we will write
$\UU$ for the functor $\UU_{k+1}$ defined in Section~\ref{chow complex}.
We set $c=n-k$, the codimension of $\F$. Let 
\[
\TT(\F): \cdots \rTox T^{-1}\rTo{\varphi_\F} T^0\rTox \cdots
\]
be the Tate resolution of $\F$, with ``middle'' map
 $\varphi_\F$.
We have seen in the previous section that
if $\F$ is weakly Ulrich,
then the complex $\UU(\F)$ is given by a single map
$\Psi_\F: \UU(\varphi_\F)$
of vector bundles on the Grassmannian, and the Chow form
of $\F$ is the determinant of $\Psi_\F$.
 
One way to compute $\varphi_\F$ is as the $m$-th syzygy matrix
over the exterior algebra starting from the multiplication tensor
\[
\Hrm^0(\F(m-1)) \to \Hrm^0(\F(m)) \tensor (\Hrm^0(\cO_{\PP(W)}(1)))^*
\]
for some $m\ge 2$. 
In this section we
describe another method which, in the Ulrich case,
makes $\varphi_\F$ explicit in terms of the minimal free resolution of $\F$. 
The tools we develop
will allow us to show that if $\F$ is skew-symmetrically self-dual
in a natural sense, then the complex $\UU(\F)$ is skew-symmetric, 
and in particular $\Psi_\F$ is skew-symmetric.
When $\F$ is also weakly Ulrich, the square root
of the Chow form of $\F$ is the Pfaffian of $\Psi_\F$.
In particular, when $\F$ is in addition a sheaf of rank
2 supported on a variety $X$, the Chow form of $X$
itself is the Pfaffian of $\Psi_\F$.

We first review the basics of the Bernstein--Gel'fand--Gel'fand 
correspondence, \cite{BernsteinGelfandandGelfand1978}
in the style of \cite{Eisenbudetal.2001}.
Let $V=Hom_K(W,K)$ be the dual vector space and
let $E=\bigwedge V$ be its exterior algebra,
graded with $V$ in degree $-1$ as usual. There is
a functor $\LL$ from graded $E$-modules to linear free
complexes over $S$ defined as follows:
If $P$ is a graded $E$-module, then $\LL(P)$
is the complex
\[
L:\quad \cdots \rTox S\otimes_K P_{i-1} \rTo{\alpha} S\otimes_K P_{i} 
\rTo{\alpha} S\otimes_K P_{i+1} \rTox \cdots
\]
such that $\alpha_i(1\otimes p) = \sum x_i\otimes e_ip$,
where $\{x_i\}$ and $\{e_i\}$ are dual bases of $W$ and $V$.
Every linear free complex $L$ over $S$ can be written as
$L=\LL(P)$ for a unique graded $E$-module $P$. 

It is easy to write down the module structure of $P$ from
the differentials of $L$. Giving a multiplication map
$V\otimes P_i\to P_{i-1}$ is equivalent
to giving the ``adjoint'' map $P_i\to P_{i-1}\otimes W$.
The (linear) differential
$L_i=S\otimes P_i \to L_{i-1}=S\otimes P_{i-1}$ of $L$ is defined
by the desired map $P_i\to W\otimes P_{i-1} = P_{i-1}\otimes W$.
The skew-commutativity and associativity necessary for an $E$-module 
structure follow immediately from the fact that $L$ is a complex.

We will use this construction to write down the middle
map
$\varphi_\F$ of the Tate resolution of an arbitrary sheaf. 
When $\F$ is Ulrich, our result takes the form given in Theorem~\ref{third main},
which allows the computation of the Chow form directly from the
free resolution of the Ulrich module of twisted global sections of $\F$.
But in the general case we must replace the
resolution by a {\it linear free monad\/}, defined
in \cite[Example 8.5 and Proposition 8.6]{Eisenbudetal.2001}.

\begin{theorem}\label{lin free monad} Let $\F$ be a 
\label{chow of Ulrich}
coherent sheaf on $\PP^n$. There
is a unique linear free complex of $S$-modules
\[
L=\LL(\F):\quad \cdots 
\rTo{\alpha_{-2}} L^{-1}\rTo{\alpha_{-1}}
L^0\rTo{\alpha_0} L^1\rTo{\alpha_1}
\cdots ,
\]
called the linear free monad of $\F$, 
such that  $L^{-i}=S(-i)\otimes P_i$
and $L^i=0$ if $|i|>n$,
with the property that the sheafification of the 
homology of $L$ is zero except for $\widetilde{\Hrm^0(L)} = \F$.
The complex $\LL(\F)$ 
is functorial in $\F$ and may be
constructed explicitly by means
of the Bernstein--Gel'fand--Gel'fand correspondence: Setting
$P=\im \varphi_\F: T^{-1}\rTox T_0,$ 
the image of the middle map
of the Tate resolution of $\F$,
we have  
$\LL(\F)=\LL(P)$.
\qed
\end{theorem}

In the case when the module of
twisted global sections
$M=\bigoplus_d\Hrm^0(\F(d))$ of $\F$ has a linear free
resolution---for example, when $\F$ is 
Ulrich---the uniqueness statement shows that this
resolution is the linear free monad $\LL$, and this gives us
an alternate method of constructing $\LL$. It would be
interesting to have such an alternate method in general.

Associated to $L=\LL(\F) = \LL(P)$ are maps
\[
\varphi_{-i,-j}:\quad P_i\to\bigwedge^{i-j}W\otimes P_j
\]
adjoint to the multiplication maps $\bigwedge^{i-j}V\otimes P_i \to P_{i-j}$
that define the $E$-module structure on $P$. These may also
be computed directly from the differentials of $L$, as follows:
Since the differentials of $L$ are given by matrices of linear forms, they
are determined by vector space maps $P_i\to W\otimes P_{i-1}$.
Composing $j$ such maps, we get $P_i\to \bigotimes_1^jW \otimes P_{i-j}$.
The image of this map is actually contained in 
$\bigwedge^jW \otimes P_{i-j}$ because the composition of the original
differentials is zero over the symmetric algebra. 
The map $\varphi_{-i,-j}$ is the induced map 
$P_i\to \bigwedge^jW \otimes P_{i-j}$. In characteristic 0, it could
also be obtained by composing matrices representing the differentials
as if they were matrices over the exterior algebra and dividing by $j!$.

We can now describe the map $\varphi_\F: T^{-1}(\F)\to T^0(\F)$.
Let $(V)$ be the ideal of elements of negative
degree in $E$ (the augmentation ideal) and define graded
vector spaces $A$ and $B$ by
\[
A=P/(V) P,\qquad B=\{p\in P\mid (V) p = 0\}. 
\]
The map $T^{-1}(\F)\to P$ is a
projective cover: that is, a minimal map from a free $E$-module 
onto $P$. It follows from Nakayama's Lemma that 
we may make the identification $T^{-1}(\F)= E\otimes A$,
and the map to $T^{-1}\to P$ is determined by the data of a 
splitting $\eta: A\to P$ (as 
graded vector spaces) of the natural projection map $P\to A$.

Recall that the injective envelope of the residue field $K=E/(V)$
may be described canonically as $\omega_E = E\otimes \bigwedge^{n+1}W$,
whose degree $i$ component is $\bigwedge^{n+1-i}V = \bigwedge^iW$.
The map $P\to T^0(\F)$ is an injective envelope.
Dually to the situation for projective covers,
an injective envelope $P\to G$ is uniquely determined
by a splitting $\pi: P\to B$ of the inclusion $B\subset P$;
we take $T^0(\F)= \omega_E\otimes B$, and the map $P\to T^0(\F)$
is the unique map to $\omega_E\otimes B$ whose composition
with the projection to $(\omega_E)_0\otimes B = B$ is $\pi$.

By \cite[Theorem 4.1]{Eisenbudetal.2001}, the components of $A$ and $B$ are
\[
A_{c+i}=\Hrm^{k-i}(\F(i-k-1)),\qquad
B_j=\Hrm^j(\F(-j)),
\]
where as usual $c$ denotes the codimension $n-k$ of $\F$. 
This identification will be useful later.
\pagebreak

Summarizing:

\begin{theorem}\label{formula for varphi} 
If $\F$ is a coherent sheaf on $\PP^n$
and $L=\LL(\F) =\LL(P)$ is its linear monad, then, with notation
as above, the middle map 
\[
\varphi_\F:E\otimes A= T^{-1}(\F)\to T^0(\F)=\omega_E\otimes B
\]
in the Tate resolution of $\F$ has components
\begin{align*}
&\Hrm^{k-i+c}(\F(i-c-k-1))\\ 
&\qquad =A_i\rTo{\eta_i} P_i 
\rTo{\varphi_{-i,-j}} 
\bigwedge^{i-j}W\otimes P_j
\rTo{\pi_j} \bigwedge^{i-j}W\otimes B_j\\
&\qquad =\bigwedge^{i-j}W\otimes \Hrm^j(\F(-j)).
\end{align*}
\end{theorem}

For example, if $L$ is a free resolution of
an Ulrich sheaf, by 
\cite[Proposition 8.7]{Eisenbudetal.2001}
$A=P_c$ and $B=P_0$, so in that case
$\varphi_\F$ is the map induced by the map $\varphi_{-c,0}$, 
and the map $\Psi_\F$
is the one given in Theorem~\ref{third main}.
No choice of $\eta$ and $\pi$ is involved because 
$A= P_{-c},\; B=P_0$ in that case.

\subsection{The skew-symmetry of ${\mathbf U}({\mathcal F})$}

We now show that appropriate symmetry or skew-symmetry of $\F$
makes $\UU(\F)$ symmetric or skew-symmetric.
The functor
\[
D: \F \mapsto \E xt^c(\F,\omega_{\PP^n})(k+1)
\] 
defines a duality on the category of $k$-dimensional Cohen--Macaulay sheaves
on $\PP^n$ and there is 
a canonical morphism $\iota: \F\to DD(\F)$.
Let $\epsilon=\pm 1$. 
As with any duality, we
say that a morphism   
$\sigma: \F \rTox D(\F)$ is
$\epsilon$-symmetric if 
\begin{equation*}
\xymatrix{
&&DD(\F)\ar[dd]^{D\sigma}\\
\F\ar[urr]^{\iota}\ar[drr]_{\sigma}&{\qquad\epsilon}&\\
&&D(\F)
}
\end{equation*}
commutes up to the sign $\epsilon.$ 
In case $\epsilon=1$, we say that $\F$ is symmetric;
if $\epsilon=-1$, then $\F$ is called skew-symmetric.

\begin{theorem}\label{symmetry}
Suppose that $\F$ is a
Cohen--Macaulay sheaf of dimension $k$
on $\PP^n$. Any
$\epsilon$-symmetric isomorphism $\F\to D(\F)$
induces an $\epsilon$-symmetric isomorphism
\[
\UU(\F)\to \cH om_{\GG_{k+1}}(\UU(\F), \Ocal_{\GG_{k+1}}(-1))[1].
\] 
In particular the map
$\UU T^{-1}(\F) \rTo{\Psi_\F} \UU T^0(\F)$
is $\epsilon$-symmetric, and for $j>1$ the map
$\UU T^{-j}(\F) \rTox \UU T^{-j+1}(\F)$
is dual to 
$\UU T^{j-1}(\F) \rTox \UU T^{j}(\F)$.
\end{theorem}

We postpone the proof to state a corollary and a lemma.

If $\F$ is skew-symmetric, we define the 
Pfaffian of the skew-symmetric complex $\UU(\F)$ 
by taking an appropriate Pfaffian of the middle map 
$\UU T^{-1}(\F) \rTo{\Psi_\F} \UU T^{0}(\F)$
times the alternating product of those terms from the 
determinant of $\UU(\F)$ that are associated with the maps
$\UU T^{-j}(\F) \rTox \UU T^{-j+1}(\F)$
for $j>0$. The determinant of $\UU(\F)$ is then 
the square of the \pagebreak\ Pfaffian of $\UU(\F)$. 

\begin{corollary}\label{det as pfaff} Assume that the
characteristic of the ground field is not $2$.
If $\F$ is a skew-symmetric
Cohen--Macaulay sheaf of rank $2$
on a $k$-dimensional subscheme $X\subset \PP^n$
such that $\bigwedge^2\F \iso \omega_X(k+1)$, then the
Chow form of $X$ is the Pfaffian of the complex $\UU(\F)$.
In particular
if $\F$ is weakly Ulrich, then
the Chow form of $X$ is the Pfaffian of the skew-symmetric
map of vector bundles $\UU(\Psi_\F).$
\end{corollary}

\begin{remark} In order to include the case
of characteristic 2, we would have to add the condition
that the duality $D$ is alternating, not just skew-symmetric,
and then prove the corresponding result for $\Psi_\F$.
We leave this task to the interested reader.
\end{remark} 

\begin{proof}[Proof of Corollary~\ref{det as pfaff}]
The skew-symmetric pairing 
$\F\otimes \F\to \bigwedge^2\F\iso \omega_X(k+1)$
gives rise to a skew-symmetric isomorphism 
\[
\F\to \cH om(\F,\omega_X(k+1))\iso 
\cE xt^c(\F,\omega_{\PP^n})\iso D(\F).
\]
The rest follows from Theorem~\ref{symmetry} and the discussion above.
\end{proof}

To prove Theorem~\ref{symmetry}, we will first analyse the map
on linear monads induced by the (skew) symmetric
isomorphism $\F\to D\F$. From this analysis will
come a certain symmetry property of $\varphi_\F$.
The map $\varphi_\F$ may be represented by a matrix
of elements of $\bigwedge W$.
An element $\alpha\in \bigwedge^{t} W$ induces
for any integer $s$ a map
(which we again call $\alpha$) defined by
\[
\bigwedge^{s}U\otimes \bigwedge^vV \rTo{\alpha} \bigwedge^{s+t-v}U\qquad:\qquad
u\otimes e \mapsto \alpha(e)(u)
\]
where $\alpha(e)\in \bigwedge^{v-t}V$ acts on $U$ as described
in Section~\ref{chow complex}. The map $\Psi_\F$ is constructed
from these pieces, so we will derive a symmetry property
for $\Psi_F$.

The difficulty in proving Theorem~\ref{symmetry} comes
from the delicacy of the signs involved. For example,
consider the case where the map 
$\varphi_\F: E(k+1-i)\to \omega_E(i)$ in 
Theorem~\ref{formula for varphi} is given by a 
$1\times 1$ matrix whose entry is in $\bigwedge^{c+2i}W$.
One might suppose that any $1\times 1$ matrix would
be symmetric and would correspond 
to a symmetric map of vector bundles
$\Psi: (\bigwedge^iU)^*\iso\bigwedge^{k+1-i}U\otimes \bigwedge^vV\to \bigwedge^iU$ on the 
Grassmannian. But actually $\Psi$ is symmetric 
if $i$ is even and skew-symmetric if $i$ is odd.
The general result we need is the following:

\begin{lemma}\label{gr signs}
Set $c=v-k-1$ and let $\alpha\in \bigwedge^{c+i+j}W$.
The dual into $\Ocal_{\GG_{k+1}}(-1)$ of the map
\[
\bigwedge^{k+1-i}U\otimes \bigwedge^vV \rTo{\alpha} \bigwedge^jU
\]
is the map
\[
\bigwedge^{k+1-j}U\otimes \bigwedge^vV 
\rTo{(-1)^{k(i+j)+ij}\ \cdot\ \alpha}
\bigwedge^iU.
\]
\end{lemma}

\begin{proof}[Proof of Lemma~\ref{gr signs}]
We identify 
$\bigwedge^iU$ with $\Hom(\bigwedge^{k+1-i}U\otimes \bigwedge^vV,\Ocal_{\GG_{k+1}}(-1))$
via the map $\tau$ sending $\beta\otimes e\in \bigwedge^{k+1-i}U\otimes \bigwedge^vV$
to the functional
\[
\tau:\ \bigwedge^iU \ni\chi\mapsto (\chi\wedge\beta)(e)\in \Ocal_{\GG_{k+1}}(-1).
\]
We must show that the diagram
\[
\begin{xy}
\xymatrix{
\bigwedge^{k+1-i}U\otimes \bigwedge^vV\ar[r]^{\alpha}\ar[d]_{\tau} 
& \bigwedge^j U\ar[d]^{\tau^*}\\
(\bigwedge^i U)^* \ar[r]_{\alpha^*} &(\bigwedge^{k+1-j}U\otimes 
\bigwedge^vV)^*
}
\end{xy} 
\]
commutes up to a sign of $(-1)^{k(i+j)+ij}$. Although this
is a diagram of vector bundles, we may reduce the
problem to one of vector spaces by working fiberwise. 
For each $p\in \GG_{k+1}$ the
fiber $U_p$ of $U$ is a subspace of $W$, and
the action of $V$ on $U_p$ is induced by
its action on $W$. Thus the annihilator of $U_p$ in $V$ acts
as zero on $U_p$, and we may therefore replace $W$ by $U_p$
and $V$ by $U_p^*$ and assume that  $U=W$
so that $v=k+1$ and $c=0$.

From the definitions we see that 
\begin{align*}
\alpha^*\tau(\beta\otimes e):\quad &\gamma\otimes e \mapsto 
\bigl[\bigl((\alpha(e))(\gamma)\bigr)\wedge \beta\bigr](e), \\
\tau^*\alpha(\beta\otimes e):\quad &\gamma\otimes e \mapsto 
\bigl[\bigl((\alpha(e))(\beta)\bigr)\wedge\gamma\bigr](e).
\end{align*}
Since these expressions are multilinear in $\alpha, \beta, \gamma$,
it suffices to check the case where $\alpha, \beta, \gamma$ are
products of elements in some fixed basis 
$\{x_1,\dots,x_v\}$ of $W$.
Set
$a=\alpha(e)\in\bigwedge^{k+1-i-j}V$.
The expressions are both zero unless
$a(\beta)\wedge \gamma$ is a scalar times the product of 
all the basis elements $\{x_1,\dots,x_v\}$. Under this 
assumption, what we are trying to prove is equivalent
to the statement that
\[
a(\gamma)\wedge \beta = (-1)^{k(i+j)+ij} a(\beta)\wedge \gamma.
\]

Let $\overline \alpha$ be the
element of $\bigwedge^{k+1-i-j}W$ such that
$\alpha(e)(\overline \alpha) = a(\overline \alpha)=1$. 
Our assumptions imply that we
can factorize $\gamma$ and $\beta$ 
as
$\gamma=\gamma'{\overline\alpha}$ and
$\beta=\overline\alpha \beta'$.
With this notation
\begin{align*}
a(\gamma)\wedge\beta&=
a(\gamma'\wedge\overline \alpha)\wedge \beta =
(-1)^{\gamma'a}\gamma'\wedge \beta =
(-1)^{\gamma'a}\gamma'\wedge \overline \alpha \wedge \beta',\\
a(\beta)\wedge\gamma&=
a(\overline \alpha\wedge\beta')\wedge \gamma =
\beta'\wedge \gamma =
\beta'\wedge \gamma'\wedge \overline \alpha  =
(-1)^{(\gamma'+a)\beta'} \gamma'\wedge \overline \alpha \wedge\beta',
\end{align*}
where we also write $\gamma', a$, and $\beta'$ for the
degrees of $\gamma', a$, and $\beta'$.
Thus the diagram commutes up to the sign
$
(-1)^{(\gamma'+a)\beta'+\gamma'a}.
$
But
$
(\gamma'+a)\beta'+\gamma'a=
\gamma(\beta-a)+(\gamma-a)a=
\gamma\beta-a^2=(k+1-i)(k+1-j)+(k+1-i-j)^2
$
and this is congruent modulo 2 to
$k(i+j)+ij$
as required.
\end{proof}

\begin{proof}[Proof of Theorem~\ref{symmetry}]
We will show
that the
``middle'' differential
\[\UU T^{-1}(\F) \rTo{\Psi_\F} \UU T^0(\F)\]
is $\epsilon$ symmetric. This condition depends
on an identification of 
$\UU T^0(\F)$ with the dual of $\UU T^{-1}(\F)$. 
Changing this identification is the same as 
multiplying $\Psi_\F$ by an automorphism of
its source or target, so it suffices to show
that $\Psi_\F$ times such an isomorphism is
$\epsilon$ symmetric.

Once we know that the middle differential is
$\epsilon$ symmetric,
we can take the injective resolution
of $P$, from which the positively indexed maps of
$\UU(\F)$ are made, to be dual to the free resolution
of $\PP$ from which the negatively indexed maps of 
$\UU(\F)$ are made. 

To analyze $\Psi_\F$ we will make use of the analysis
of $\varphi_\F$ in Theorem~\ref{formula for varphi}.
We have decompositions
\begin{align*}
T^{-1}(\F)&=\sum_i A_{c+i}\otimes E(-c-i)\ \quad\text{\rm and}\\
T^{0}(\F)&=\sum_j B_{-j}\otimes \omega_E(j).
\end{align*}
In terms of this decomposition, the $(i,j)$ component
of $\varphi_\F$ is $\pi_j\varphi_{-c-i,j}\eta_{-c-i}$. 
Applying the functor $\UU$,
we see that $\Psi_\F$ decomposes into maps
\begin{align*}
\UU(A_{c+i}\otimes E(-c-i)) = 
\UU(A_{c+i}\otimes \bigwedge^vV \otimes \omega_E(k+1-i))  
&=A_{c+i}\otimes \bigwedge^vV \otimes \bigwedge^{k+1-i}U\\
\rTo{(\Psi_\F)_{i,j}}\quad
\UU(B_{-j}\otimes \omega_E(j))
&=B_{-j}\otimes \bigwedge^jU
\end{align*}
where $U$ denotes the tautological subbundle on the Grassmanian.
With this indexing, we will show that the maps $(\Psi_\F)_{i,j}$
and $(\Psi_\F)_{j,i}$ are dual up to a certain sign. 

As already noted, we may identify $A_{c+i}$ with
$\Hrm^{k-i}(\F(i-k-1))$
and $B_j$ with $\Hrm^j(\F(-j))$. As we have assumed that
$\F\iso \E xt^c(\F, \omega_{\PP^n}(k+1))$, we have
\[
B_j^*=\Hrm^j(\F(-j))^*
=\Hrm^{j}(\E xt^c(\F, \omega_{\PP^n}(k+1))(-j))^*
=\Hrm^{n-j}(\F(k+1-j))
=A_j
\]
by Serre duality. With this identification it suffices
to check the 
signs in the maps $\varphi_\F$ rather than in the maps
$\varphi_{\bullet,\bullet}$.

The linear complex
$\cH om(L,\omega_{\PP^n})(k+1)[c]$, is 
a linear free monad for the dual sheaf
$D(\F)\iso \F$. By the uniqueness and functoriality
of linear monads (Theorem~\ref{lin free monad}), the isomorphism $\sigma$ induces
an isomorphism
$L\iso \cH om(L,\omega_{\PP^n})(k+1)[c]$.

To simplify notation, we set
$\check L^i = D(L^i)= \cH om(L^i,\omega_{\PP^n})(k+1).$
We follow standard sign conventions (see, for example,
\cite{Iverson 1986})
and define the dual complex 
$\check L = \cH om(L,\omega_{\PP^n})(k+1)$ to 
have differentials $(-1)^i\check\alpha_i$. Shifting the 
complex $c$ steps also introduces the sign $(-1)^c$. 
Thus the isomorphism $L\to \check L[c]$ consists of 
a sequence of isomorphisms $\sigma_j: L^j\to \check L^{-c-j}$
as in the following diagram:
{\smaller[2]{
\[
\begin{CD}
 @>>> & L^{-c-i}& @>{\alpha_{-c-i}}>> & \ldots & 
 @>{\alpha_{-c-1}}>> & L^{-c}& @>>> & \ldots &
 @>{\alpha_{j-1}}>> & L^{j}& @>>>\\
@.& @V{\sigma_{-c-i}}VV @.&& @. @V{\sigma_{-c}}VV @. & &@. @V{\sigma_{j}}VV&  \\
 @>>> & \check L^{i}& @>{\mskip-14mu
(-1)^{c+i-1}\check \alpha_{i-1}}>> & \ldots & 
 @>{(-1)^c \check \alpha_{0}}>> & \check L^{0}& @>>> & \ldots &
 @>{(-1)^{-j} \check \alpha_{-c-j}}>> & \check L^{-c-j}& @>>>  
\end{CD}
\]}}
From the diagram we see that
\begin{equation}
\begin{split}
\varphi_{-c-i,j} &= \sigma_j \alpha_{j-1} \tensor \ldots \tensor 
\alpha_{-c-i} \\
&=(-1)^s \check \alpha_{-c-j} \tensor \ldots 
\tensor \check \alpha_{i-1}\sigma_{-c-i}
\end{split}\tag{$*$}
\end{equation}
with $s=c(c+i+j)+{c+j+1 \choose 2}+{i \choose 2}$,
where the $c(c+i+j)$ comes from the shift, and the rest
is the contribution of the signs in the duality, separating
the parts with positive and negative indices.

We next prove that the map $\sigma_{-c-i}$ is, up to a sign
we shall identify,
the dual of $\sigma_i$. By the uniqueness and functoriality
of linear monads and the $\epsilon$ symmetry
of $\sigma$, the induced map of complexes $\sigma': L\to \check L[c]$
factors as the composite $\sigma' = \epsilon D(\sigma') \iota'$
where $\iota'$ is 
the canonical morphism of complexes
\[
\iota': L \rTox \cH om(\cH om(L,\omega_{\PP^n})[c],\omega_{\PP^n})[c].
\]
The components of $\iota'$ are given by
\[
\iota'_\ell: L_\ell \rTo{(-1)^{(c+1)(c+\ell)} \iota } \check {\check L}_\ell,
\]
where $\iota$ denotes the canonical morphism $M \rTox \check {\check M}$
 of sheaves (see also \cite[p.\ 73]{Iverson 1986}). Thus
$\sigma_{-c-i}= 
\epsilon (-1)^{(c+1)i} \check \sigma_i \iota$.

Combining this equation with $(*)$, we get
\begin{align*}
\varphi_{-c-i,j}
& =\epsilon (-1)^{s+(c+1)i} \check \alpha_{-c-j} \tensor \ldots 
\tensor \check \alpha_{i-1} \check \sigma_i \iota \\
& =\epsilon (-1)^{s+t} {\rm transpose}(\sigma_i \alpha_{i-1} \tensor
\ldots \tensor \alpha_{-c-j}) 
\end{align*}
with $t=(c+1)i+{c+i+j \choose 2}$ since in the transpose matrix the
tensor factors occur in the opposite order, and this tensor lies in
$\bigwedge^{c+i+j} W$.

Now
\begin{align*}
s+t&=c(c+i+j)\,+\,{c+j+1 \choose 2}\,+\,{i \choose 2}\,+\,(c+1)i\,+\,{c+i+j \choose 2} \\
&= c^2\,+\,c(i+j)\,+\,\frac{(c+j)^2+(c+j)}{2}\,+\,\frac{i^2-i}{2}+
ci\,+\,i\, \\
&\quad+\frac{(c+i+j)^2-(c+i+j)}{2} \\
&\equiv (c+1)(i+j)+ij \quad\hbox{mod } 2.
\end{align*}
By Lemma~\ref{gr signs}, we see that all the diagonal
blocks $(\Psi_\F)_{i,i}=\UU(\varphi_{-c-i,i})$
will be $\epsilon$ symmetric. 
Because
$(c+1)(i+j)+ij+k(i+j)+ij \equiv v(i+j)$,
we can multiply the block matrix $\Psi_\F$
by the diagonal matrix of signs
$\Delta=\bigoplus_j(-1)^{vj}Id_{A_j}$,
where $Id_{A_j}$ is the identity map
on $A_j$, to get
a map which is $\epsilon$ symmetric; that is,
setting 
$\Psi_\F'=
\Psi_\F\Delta,
$
we will have
$
\epsilon \cH om((\Psi_\F'),\Ocal_\GG(-1))
=\Psi_\F'.
$
\end{proof}

\section{Curves}\label{curves} 

\subsection*{The projective line} 

Binary forms were the starting point for the theory of resultants
(\cite{Leibniz 1693,Bezout1779,Sylvester 1840--1842};
see \cite{Kline 1972} for some historical remarks), and they
correspond to the simplest cases of Chow forms of curves.  We now
explain how they fit into our theory by redoing the most classical
result in our language.

\begin{example}[Binary forms] \label{binary}
Consider the {\it rational normal curve\/} 
$\iota: \PP^1 \hookrightarrow \PP^d, \, (s:t) \mapsto (s^d,s^{d-1}t,\ldots, t^d)$.
We use 
$[i,j]$ for the $ij$-th Pl\"ucker coordinate of 
$\GG_2=\GG(2,\Hrm^0(\PP^d,\cO(1)))$
with respect to the given basis. 
\end{example}

\begin{proposition}\label{P1}
The Chow form of the rational normal curve of degree $d$
is the determinant of the $d \times d$ symmetric matrix
$A=(a_{ij})$ with
\[
a_{ij} = \sum_{p < \min(i,j) \atop p+q=i+j-1} [p,q].
\]
\end{proposition}

Since
the rational normal curve is a linear determinantal variety,
this formula could be deduced from
Theorem~\ref{chow of Ulrich}. But  from our point of view the most direct 
method is the  computation of a map 
in a Tate resolution.

\begin{proof} Let $\Lcal$ be the line bundle of
degree $-1$ on $\PP^1$.
Let $\iota: \PP^1 \to \PP^d$ be
the $d$-uple embedding. 
The Tate
resolution of $\F=(\iota_*\cL)(1)$, the line
bundle associated to the divisor of degree $d-1$,
has Betti diagram
\[
\begin{matrix} 3d & 2d & d & - & - & - \\
          - &  - & - & d & 2d & 3d \\ \end{matrix}
\]
Let $y_0,\ldots,y_d$ denote the dual basis
in $V$ to the monomial basis in $W=\Hrm^0\Ocal_{\PP^1}(d)$.

The map $T^0(\F)=E^d\to E(-1)^{2d}= T^1(\F)$ matrix comes from the
multiplication $\Hrm^0(\PP^1,\cO(d-1))\times \Hrm^0(\PP^1,\cO(d)) \to
\Hrm^0(\PP^1,\cO(2d-1))$, hence is given by the Sylvester type matrix
\[
B=(b_{kl})=(y_{k-l})=\hbox{ transpose }\begin{pmatrix} y_0 & y_1 & \ldots  & y_d & 0 & \ldots & 0 \\
          0 & y_0 & \ldots & y_{d-1} & y_d &\ldots & 0 \\
          \vdots&\ddots&\ddots& &\ddots &\ddots &\vdots \\
          0 & \ldots & 0 & y_0 & \ldots & y_{d-1} & y_d \\ \end{pmatrix}.
\]
To prove the formula, we must show that the kernel of this matrix is the image
of a matrix $A=(a'_{i,j})$ with
$a'_{i,j}=\sum_{p < \min(i,j) \atop p+q=i+j-1} y_p\wedge y_q.$

The equation $B\cdot A=0$ holds since a 
term $y_{k-l} \wedge y_p \wedge y_q $ arising in the product
$b_{kl}a'_{lj}$ is cancelled either by a term $y_p \wedge y_{k-l} \wedge y_q$
in the product $b_{k,k-p}a'_{k-p,j}$ or by a term $y_{q} \wedge y_{p}
\wedge y_{k-l}$ in $b_{k,k-q}a'_{k-q,j}$, in case $k-l < j$ or $j \le k-l$,
respectively. 

Since the $d$ rows of $A$
are linearly independent, and we know that the kernel of $B$ is
 generated by $d$ independent elements of degree 2, we see that
the rows of $A$ generate the kernel of $B$ as required.
\end{proof}

We can obtain the 
Sylvester formula instead of the B\'ezout formula
by applying $\UU_2$ to the first shift of the Tate
resolution. The resulting complex 
\[ 
\bigoplus^d U \to \bigoplus^{2d} \cO, 
\]
written in Stiefel coordinates,
gives the classical Sylvester formula for two polynomials
$ f=f_0s^d+f_1s^{d-1}t+\ldots+f_dt^d$ and 
$g=g_0s^d+g_1s^{d-1}t+\ldots+g_dt^d$ of equal degree.

\subsection*{Arbitrary curves}
 
We will generalize these formulas to arbitrary curves over
an algebraically closed field, and obtain partial results for
more general ground fields.

By a {\it curve} we will mean a purely $1$-dimensional 
scheme $X$, projective over $K$.
The theory of Ulrich sheaves on curves is significantly simpler
than the theory for higher-dimensional varieties because
it is essentially independent of the embedding. To state the
result, we say that a sheaf $\G$ on a curve $X$ has 
{\it no cohomology\/}
if $\Hrm^0(\G)=\Hrm^1(\G)=0$.

\begin{theorem}\label{arbitrary curves} If $X$ is a curve embedded in
$\PP=\PP^{n+1}$ with hyperplane divisor $H$, then a sheaf
$\F$ is an Ulrich sheaf for $X$ in $\PP$ if and only if
$\F=\G(H)$ for some $\G$ with no cohomology.
\end{theorem}

\begin{proof} 
If $\F$ is Ulrich, then $\Hrm^0(\F(-H))=\Hrm^1(\F(-H))=0$. 
Conversely, if $\G=\F(-H)$ has no cohomology, then 
$\F$ is 0-regular because $\Hrm^1(\F(-H))=\Hrm^1(\G)=0$.
Similarly, $\cE xt^{n-1}_\PP(\F,\Ocal_\PP(-n-1))$ is
2-regular because $\Hrm^1(\check \F (1))=\Hrm^0(\F(-1))=\Hrm^0(\G)=0$. 
(One can also see the
desired vanishing
directly from the Tate resolution: For example, the vanishing of
$\Hrm^0(\G)$ implies that the free module
$T^0(\G)$ has no generators in degree 0; and it follows that
for $j<0$ the module $T^{-j}(\G)$ has no generators in
degree $-j$. But by 
\cite[Theorem 7.1]{Eisenbudetal.2001}
the space of
generators of $T^{-j}(\G)$ in
degree $-j$ is $\Hrm^0(\G(-jH)$).
\end{proof}

To find sheaves with no cohomology,
it suffices to look
for sheaves on a single component of the reduced scheme $X_{\rm red}$
or even on its normalization. Thus we are led to ask: Given a
nonsingular irreducible curve $X$ over an arbitrary field $K$,
what are the sheaves $\G$ over $X$ with no cohomology? Such a sheaf
$\G$ can have no torsion, so (since $X$ is nonsingular) 
$\G$ is automatically locally free. From the vanishing of the
cohomology we see that the Euler characteristic of $\G$ is 0,
so by Riemann--Roch the degree of $\G$ is $\rank(\G)\cdot (g-1)$,
where $g={\rm genus}(X)$. Over an algebraically closed field,
there are always line bundles of this type.
The following proposition
 generalizes the fact that the equation of any plane curve can be
written as the determinant of a linear matrix:

\begin{proposition}\label{ineff} A line bundle $L$ on a
nonsingular irreducible curve $X$ has no cohomology
 if and only if $\deg(L)={\rm genus}(X)-1$ and
$L$ has no sections. If $X$ contains infinitely many $K$-rational
points,
then such line bundles exist on $X$, and thus the Chow form
of $X$, in any projective embedding, can be written as a
determinant of linear forms in the Pl\"ucker coordinates.
\end{proposition}

\begin{proof} The first statement is immediate from the Riemann--Roch theorem.
For the second, take $L=\Ocal_X(p_1+\cdots+p_g-q)$, where the
$p_i$ and $q$ are general $K$-rational points.
\end{proof}

\begin{corollary}\label{ulrich on curves} If $X \subset \PP^n$ is a
$1$-dimensional scheme over an arbitrary field, then
X has an Ulrich sheaf.
\end{corollary}

\begin{proof} By Proposition~\ref{ineff} the normalization of a
component of $X_{\rm red}$ has an Ulrich line bundle defined  
over a finite field extension of the 
ground field, which gives an Ulrich sheaf of 
higher rank on $X$ defined over the ground field.
\end{proof}

To arrive at explicit resultant formulas 
we have to compute the appropiate  differentials in the Tate complex.

\begin{example}[Hyperelliptic resultant formulas] 
\label{hyperelliptic resultant}
Consider a fixed polynomial $f=f_0+f_1t+\cdots+f_{2g+2}t^{2g+2}$ with no 
multiple roots.
To write explicit Stiefel and B\'ezout formulas for
the resultant of
two functions
$a(t)+b(t)\sqrt{f(t)}$ and $c(t)+d(t)\sqrt{f(t)}$ with $a,b,c,d  \in K[t]$,
we consider them as functions on the hyperellipic curve $C$ of genus g 
with function
field $K(t,\sqrt f )$.  Let  $k= \max\{\deg a, g+1+\deg b, \deg c,
g+1+\deg d \}$ and consider the embedding of $C$ given by
$t \mapsto (1:t:\ldots:t^k:\sqrt f:t\sqrt f:\ldots: t^{k-g-1}\sqrt f)$.
We want the Chow form of this embedding. By Theorem~\ref{arbitrary curves}
and Theorem~\ref{symmetry}, we can express the Chow form
 as the determinant of a symmetric matrix
by using the Ulrich sheaf $\Lcal(H)$, where $\Lcal$ is 
a line bundle such that $\Lcal\otimes \Lcal=\omega_C$, the
canonical bundle, and $\Lcal$ has no cohomology. 
A nonvanishing theta characteristic in turn corresponds to a 
factorization $f=f^{(1)}f^{(2)}$ of $f$ into two 
polynomials of degree $g+1$. All of our formulas will depend on the
choice of such factorization and we will obtain
 ${\frac12}{2g+2\choose g+1}$  B\'ezout formulas.
\end{example}

Before we come to the  B\'ezout formulas we will derive a Stiefel
formula for the resultant that is highly parallel to the
Sylvester formula for the ordinary resultant. We will then
deduce a B\'ezout formula in a way that is analogous to our
proof of Proposition~\ref{P1}.
Let
\[
syl(k,r) = \hbox{ transpose } 
\begin{pmatrix} r_0 & r_1 & \ldots  & r_k & 0 & \ldots & 0 \\
          0 & r_0 & \ldots & r_{k-1} & r_k &\ldots & 0 \\
          \vdots&\ddots&\ddots& &\ddots &\ddots &\vdots \\
          0 & \ldots & 0 & r_0 & \ldots & r_{k-1} & r_k \\ \end{pmatrix}
\]
be the $2k\times k$  ``Sylvester block'' of a polynomial $r$ of
degree $k$.  

\begin{proposition}\label{Hyperelliptic Sylvester formula}
With notation as above, two functions
$a+b\sqrt f$ and $c+d\sqrt f$ with $a,b,c,d \in K[t]$ 
have a common zero if and only if the determinant of the $4k \times 4k$ matrix
\[ \begin{pmatrix}
        syl(k,a) & syl(k,bf^{(2)}) & syl(k,c) & syl(k,df^{(2)}) \\  
        syl(k,bf^{(1)}) & syl(k,a) & syl(k,df^{(1)}) & syl(k,c) \\ 
\end{pmatrix}
\]
vanishes. 
\end{proposition}

\begin{proof} Let $\pi: C \to \PP^1$ denote the double
cover corresponding to the inclusion $K(t)\subset K(C)=K(t)[\sqrt f]$.
We consider the embedding of $C$ as a curve of degree $2k$ 
in projective space $\PP^{2k+1-g}$
corresponding to the line bundle $\Ocal_C(H)=\pi^*(\Ocal_{\PP^1}(k))$. 
The space of global sections of $\Ocal_C(H)$ has basis corresponding
to the functions
$1,t,\ldots,t^k,\sqrt f,t\sqrt f,\ldots, t^{k-g-1}\sqrt f$,
so the Chow form of $C$ in this embedding is the resultant we seek.
We write $e_0,\ldots,e_k,e_{k+1},\ldots,e_{2k-g}\in V=\Hrm^0(\Ocal_C(H))^*$
for the dual basis.

Every line bundle $\Lcal$ on $C$ can be described as a rank 2 vector bundle
$\B=\pi_* \Lcal$ on $\PP^1$ together with an action $\B \rTo y \B(g+1)$ satisfying
$y^2= f \cdot id_\B$. For example $\pi_* \Ocal_C = \Ocal \oplus \Ocal(-g-1)$ with the action
defined by $y=\bigl({0 \atop 1}{f\atop 0}\bigr)$. The bundle
$\B=\Ocal(-1)\oplus\Ocal(-1)$ with the action of 
$\bigl({0\atop f^{(2)}}{f^{(1)} \atop 0}\bigr)$ 
corresponds to a nonvanishing 
theta characteristic $\F$ on $C$. In particular, $\F$ is a line bundle of
degree $g-1$ with no cohomology.
See \cite{Buchweitz and Schreyer 2002} for a detailed exposition.
The Stiefel
formula above is obtained by applying the functor $\UU$ to the line bundle
$\F(2H)$.

The space of global sections of $\F(H)$ has a basis corresponding to
the functions
\[
\sqrt{f^{(1)}},t\sqrt{f^{(1)}},\ldots,t^{k-1}\sqrt{f^{(1)}},
\sqrt{f^{(2)}},t\sqrt{f^{(2)}},\ldots,t^{k-1}\sqrt{f^{(2)}},
\]
while  $\Hrm^0(\F(2H))$ has a basis corresponding to
\[
\sqrt{f^{(1)}},t\sqrt{f^{(1)}},\ldots,t^{2k-1}\sqrt{f^{(1)}},
\sqrt{f^{(2)}},t\sqrt{f^{(2)}},\ldots,t^{2k-1}\sqrt{f^{(2)}}.
\]
Thus the map
\[
\Hom(E,\Hrm^0(\F(H))) \to \Hom(E,\Hrm^0(\F(2H)))
\]
in the Tate resolution is given 
by the $4k\times 2k$ matrix over the exterior algebra
\[\def\scriptstyle{\let\ldots\cdots \medmuskip1mu}
B=\begin{pmatrix}\scriptstyle syl(k,e_0+e_1t+\ldots+e_kt^k) &
\scriptstyle \hskip-5pt
syl(k,(e_{k+1}+\ldots+ e_{2k-g}t^{k-g-1})f^{(2)}) \\
\scriptstyle syl(k,(e_{k+1}+\ldots+
e_{2k-g}t^{k-g-1})f^{(1)}) 
&\scriptstyle \hskip-5pt syl(k,e_0+e_1t+\ldots+e_kt^k)
\end{pmatrix}.
\]
The desired Sylvester formula follows by interpreting the induced map
\[
\Hrm^0(\F(H)) \tensor U \rTox \Hrm^0(\F(2H)) \tensor \Ocal_\GG
\]
in terms of Stiefel coordinates.
\end{proof}

We now use these constructions as in Proposition~\ref{P1}
to derive {\bf hyperelliptic B\'ezout
formulas}. It suffices to compute the kernel of 
the map $B$ of Proposition~\ref{Hyperelliptic Sylvester formula}. 
By Theorem~\ref{first main} this will be a
$2k \times 2k$ matrix with entries in $\Lambda^2 V$.
Because $\F$ is a theta characteristic,
 Theorem~\ref{symmetry} shows that the kernel will be represented
in suitable bases by a symmetric matrix.

The final formula may be written in terms of the $2\times 2$
minors of the $2\times (2k+1-g)$ matrix 
\[
\begin{pmatrix} a_0 &\ldots & a_k & b_0 &\ldots & b_{k-g-1}  \\
c_0 &\ldots & c_k & d_0 &\ldots & d_{k-g-1} \\ \end{pmatrix}.
\] 
However we will work with the larger $2 \times 3(k+1)$ matrix
\[
\begin{pmatrix} a_0 &\ldots & a_k & (bf^{(1)})_0 &\ldots & (bf^{(1)})_k & 
(bf^{(2)})_0 &\ldots & (bf^{(2)})_k \\
c_0 &\ldots & c_k & (df^{(1)})_0 &\ldots & (df^{(1)})_k & 
(df^{(2)})_0 &\ldots & (df^{(2)})_k \\ \end{pmatrix}
\]
whose minors are linear combinations of those of the matrix above,
with coefficients
that depend on the coefficients of $f^{(1)}$ and $f^{(2)}$.

If $0 \le p,q \le k$, then we denote by $[p,q]$ the
 minor formed by the columns with indices $p$ and $q$. 
We write
$p^{(1)}$ for the column with index $p+(k+1)$ and $q^{(2)}$ for
the column with index $q+2(k+1)$. Thus brackets like
$[p^{(1)},q]$ and $[p^{(1)},q^{(2)}]$ represent $2\times 2$ minors of
the large matrix. 

Consider the $k \times k$ matrices $A^{11},\ldots,A^{22}$ defined by
\begin{align*}
  A^{11}_{i,j}& =
\sum_{0\le p<q \le k \atop {p < \min(i,j) \atop p+q=i+j-1}} [p^{(2)},q]
+[p,q^{(2)}], \\
A^{12}_{i,j}&=
\sum_{0\le p<q \le k \atop {p < \min(i,j) \atop p+q=i+j-1}} [p,q] \quad
+ \quad \sum_{0\le p,q \le k \atop {p < j \atop p+q=i+j-1}} [p^{(1)},q^{(2)}]
,\\
A^{21}_{i,j}&=
\sum_{0\le p<q \le k \atop {p < \min(i,j) \atop p+q=i+j-1}} [p,q] \quad
+ \quad \sum_{0\le p,q \le k \atop {p < j \atop p+q=i+j-1}} [p^{(2)},q^{(1)}] ,
\\
 A^{22}_{i,j}&=
\sum_{0\le p<q \le k \atop {p < \min(i,j) \atop p+q=i+j-1}} [p^{(1)},q]
+[p,q^{(1)}]. 
\end{align*}

The matrix 
\[
A=\begin{pmatrix}A^{11} & A^{12} \\ A^{21} & A^{22} \\ \end{pmatrix}
\]
 is actually symmetric. This becomes visible if we expand the
expressions into brackets of the smaller $2\times (2k+1-g)$ matrix.

\begin{proposition}\label{hyperelliptic bezout}
Suppose $k \le 12$. The functions $a+b\sqrt f$ and $c+d\sqrt f$ have a common zero
if and only if the determinant of the matrix
$A$ vanishes.
\end{proposition}

\begin{proof} As in the proof of Proposition~\ref{P1} it suffices to check that
$B\cdot A = 0$, when we regard $A$ as a matrix over the exterior algebra,
because the linear independence of the columns of $A$ is visible from
the specialization to the case of binary forms  $b=d=0$.
For
each specific value of $g$ and $k$ this can be checked by computer,
and we did this for all cases $1\le g+1 \le k \le 12$. 
\end{proof}

The formula should certainly hold for any $k$; but as noted in the 
proof, we have performed the necessary computations only up to $k=12$.

Notice that in case $b=d=0$ the matrix reduces to twice the 
Bezout matrix 
for binary forms of degree $k$. This fits with the fact that 
two functions on $\PP^1$ with a common zero have two common 
zeroes when  pulled back to $C$. 

As a concrete application of Proposition~\ref{hyperelliptic bezout} we do
the case of an elliptic curve over the complex numbers. 

\begin{example}[Resultant of doubly periodic functions]
\label{resultant of doubly periodic functions}
Consider an elliptic curve $C=\CC/\Gamma$ and the corresponding
Weierstrass $\wp$-function, with functional equation
\[
\wp'(z)= 4\wp^3(z)-g_2\wp(z)-g_3=4(\wp(z)-\rho_1)(\wp(z)-\rho_2)(\wp(z)-\rho_3)
\]
where the ${\rho_j}$ are the values of $\wp$ at the half periods.
\end{example}

\begin{corollary}\label{doubly periodic} 
Two doubly periodic functions
\[
f(z)= a_0+a_1 \wp(z) + a_2 \wp^2(z)+ b_0 \wp'(z)/2
\]
and 
\[
g(z)= c_0+c_1 \wp(z) + c_2 \wp^2(z)+ d_0 \wp'(z)/2\]
have a common zero if and only if the determinant of
\[
\begin{pmatrix}
-\rho_1 \rho_2 [13]-(\rho_1+\rho_2) [03]&
      -\rho_1 \rho_2 [23]+[03]&
      [01]&
      [02]\\
      -\rho_1 \rho_2 [23]+[03]&
      (\rho_1+\rho_2) [23]+[13]&
      [02]&
      [12]\\
      [01]&
      [02]&
      \rho_3 [13]+[03]&
      \rho_3 [23]\\
      [02]&
      [12]&
      \rho_3 [23]&
      -[23]\\
      \end{pmatrix}\]
vanishes, where the bracket
$
[ij]
$ 
denotes the minor made from the $i$-th and $j$-th
columns of the matrix
\[
\begin{pmatrix}
 a_0 & a_1 & a_2 & b_0 \\  
c_0 & c_1 & c_2 & d_0
\end{pmatrix}.
\]
\end{corollary}

\begin{proof}
This formula follows from Proposition~\ref{hyperelliptic bezout},
with one of the roots of $f$ at infinity and with the 
factorization given by $f^{(1)}= (\wp(z)-\rho_1)(\wp(z)-\rho_2)$.
\end{proof}
 
Returning to our general discussion, we may ask whether it
is possible to give a B\'ezout formula for the Chow
form of a curve over a field $K$ even if the curve does not have enough
$K$-rational
points to apply Theorem~\ref{arbitrary curves}. In this case the curve
may have no rank 1
Ulrich sheaf, as happens, for example, for a conic without real points in 
$\PP^2_{\mathbf R}$. However, it may be that there are always rank 2 Ulrich sheaves.
For example, assuming that $X$ has genus 0, 
the  structure sheaf $\Ocal_X$ and the canonical bundle $\omega_X$
 are defined over $K$, and there is the unique
extension 
\[
\eta:\quad 0\to \omega_X\to \cE \to \Ocal_X\to 0
\]
corresponding to a nonzero element $\eta\in \Hrm^1(\omega_X^{-1}) = K$.
Over an algebraic closure of $K$ the bundle $\cE$ splits as
$\Ocal_{\PP^1}(-1)\oplus \Ocal_{\PP^1}(-1)$ (the sequence above
is the Koszul complex) and thus $\cE$ has no cohomology.

The main theorem of \cite{Brennanetal.1987} generalizes
this example and 
says that if $K$ is algebraically closed and 
$X$ is a 1-dimensional arithmetically Cohen--Macaulay subscheme
of $\PP$, then there exists a rank 2 sheaf $\F$ with no cohomology, 
which in addition
satisfies $\F\iso\cH om(\F, \omega_X)$.
(Their statement does not include the separability hypothesis
below; but they apply a result
of \cite{Eisenbud 1988} which is proved only in the
algebraically closed case. We do not see how to extend their
proof beyond the separable case.)
A variation
on their proof allows one to drop the ``arithmetically Cohen--Macaulay"
hypothesis. Here is a geometric version of the argument, 
developed in conversation with Joe Harris. 

\begin{proposition}\label{inf field case} Let $X$ be a projective
curve, separable
over the field $K$.
If $K$ is infinite,
then $X$ has a coherent sheaf $\cE$ with no cohomology that
is a rank $2$ vector bundle over
the normalization of $X_{\rm red}$
and satisfies $\cH om(\cE, \omega_X)=\cE$ skew-symmetrically.
\end{proposition}

\begin{proof} Let $\pi: C\to X_{\rm red}$ be the normalization.
It is enough to find a rank 2 vector bundle without cohomology on $C$
with $\cH om(\cE, \omega_C) = \cE$, because we have
$\cH om_X(\cE, \omega_X)
=\cH om_C(\cE,\cH om(\Ocal_C,\omega_X))=\cH om_C(\cE, \omega_C)$. 
Since we have dealt with the case of $\PP^1$ above, we will
assume that the genus $g$ of $C$ is greater than 0.
Let $L$ be a line bundle on $C$ of strictly positive degree.

Any extension class
$\eta
\in \extt^1(\omega_C\otimes L, L^{-1})$ gives rise to a short exact
sequence
\[
\eta:\quad 0\to L^{-1}\to \cE \to \omega_C\otimes L\to 0
\]
where $\cE$ is a vector bundle. For any such bundle 
$\bigwedge^2\cE=\omega_C$, whence $\cH om(\cE, \omega_C)=\cE$
skew-symmetrically.

By Serre duality $\chi(\cE)=0$, so $\cE$ will be an Ulrich
sheaf as long as $\Hrm^0(\cE)=0$. 
Since $\Hrm^0(\Lcal^{-1})=0$,  this condition is satisfied if and only
if the connecting homomorphism 
\[
\delta_\eta: \Hrm^0(\omega_C\otimes L)\to \Hrm^1(\Lcal^{-1})=\Hrm^0(\omega_C\otimes L)^*
\]
is an isomorphism. But 
\[
\eta\in \extt^1(\omega_C\otimes L, L^{-1})\iso \Hrm^1(L^{-2}\otimes\omega_C^{-1})
\iso \Hrm^0(L^2\otimes\omega_C^2)^*,
\]
and $\delta_\eta$ is induced by the multiplication pairing
\[
\Hrm^0(L\otimes\omega_C)\otimes \Hrm^0(L\otimes\omega_C)\rTo{m}
\Hrm^0(L^2\otimes\omega_C^2)
\]
in the sense that $\eta$ goes to $\delta_\eta$ under the composition
\begin{align*}
\Hrm^0(L^2\otimes\omega_C^2)^*\rTo{m^*}
&\,\,\Hrm^0(L\otimes\omega_C)^*\otimes\Hrm^0(L\otimes\omega_C)^*\\
&\iso \Hrm^0(L\otimes\omega_C)^*\otimes\Hrm^1(L^{-1})\\
&\iso \Hom(\Hrm^0(L\otimes\omega_C),\ \Hrm^1(L^{-1})).
\end{align*}
The ring $R=\bigoplus_d\Hrm^0(L^d\otimes\omega_C^d)$
is an integral domain. By separability, it splits
into a product of integral domains over the algebraic
closure of $K$. It follows that the 
multiplication pairing is a direct sum of 1-generic
pairings in the sense of
\cite{Eisenbud 1988}. The results of that paper show that
$\delta_\eta$ is an isomorphism unless $\eta$ lies in
a certain proper hypersurface in $\Hrm^0(L^2\otimes\omega_C^2)$.
If $K$ is infinite, then this hypersurface cannot contain
all the $K$-rational points of this vector space.
\end{proof}

\begin{example}[Pointless conics] \label{conic with no points}
The conic $C \subset \PP^2$ defined by $x^2+y^2+z^2=0$ has no line bundle
of degree $-1$ defined over $\RR$. However there are rank 2 Ulrich sheaves.
The cokernel 
\[\F=\coker(\cO_{\PP^2}^4(-2) \rTo{M} \cO_{\PP^2}^4(-1))\]
given by the matrix
\[M= \begin{pmatrix}0 & x & y & z \\
              -x & 0 & z & -y \\
              -y & -z & 0 & x \\
              -z & y & -x & 0 \\ \end{pmatrix}
\]
is a rank 2 sheaf on $C$ with no cohomology. An explicit B\'ezout 
formula for the Chow form of $C$ 
can be derived from the Pfaffian B\'ezout formula for the
resultant of three quadratic forms in three variables given in the
introduction, by specializing one of the three quadratic forms
to $x^2+y^2+z^2$ and eliminating unnecessary variables. 
\end{example} 

\section{Resultant formulas and Veronese embeddings}
\label{Projective spaces}

Generalizing the case of binary forms in another direction,
we consider the resultant of $k+1$ forms of degree $d$
in $k+1$ variables. That is, we consider the Chow
form of the $d$-uple embedding 
\[\PP^k \hookrightarrow \PP^N\] with $N = {d+k \choose k}-1$. 

To find
determinantal or Pfaffian formulas for powers of such Chow forms, we
need to look for vector bundles on $\PP^k$ that become Ulrich sheaves
on the $d$-uple embedding. Similarly, Stiefel formulas come from weakly Ulrich
sheaves.  By an argument shown to us by Jerzy Weyman, Ulrich
sheaves always exist, but we shall see that in
some cases all Ulrich sheaves have very high rank.

\enlargethispage{1\baselineskip}
By way of comparison, the classical search for
B\'ezout or Stiefel formulas was essentially a search for line bundles
on $\PP^k$ that become Ulrich or weakly Ulrich on the $d$-uple
embedding. Weakly Ulrich line bundles exist (and were found
classically, e.g.,
\cite[Chap 13, Prop. 1.6]{Gelfandetal.1994}
if and only if $k\leq 3$ or $k=4,\; d\leq 3$ or $k=5,\;d=2$ (Ulrich line
bundles never exist except when $k\leq 2$ or $d=1$). We get a few more
Stiefel formulas for the resultants themselves (and not just powers)
from the Horrocks--Mumford bundle in the case $k=5,\; d=4,6$ or 8.

It turns out that the cohomology of a sheaf that becomes an Ulrich sheaf
on the $d$-uple embedding of $\PP^k$ is
determined by the rank of the sheaf
alone, and the same idea works for the $d$-uple embedding
of any variety:

\begin{theorem}\label{d-uple} Let $\iota: \PP^m\hookrightarrow \PP^n$ 
be the $d$-uple embedding. Suppose $\F$ is a sheaf
of dimension $k$ on $\PP^m$.
The sheaf $\iota_*\F$ is an Ulrich sheaf on $\PP^n$ if and only
if
\[
h^i(\F(e))\neq 0 \Leftrightarrow
\begin{cases}
i=0,& -d<e \\
0<i<k,& -(i+1)d<e<-id\\
i=k,& e<-kd.\end{cases}
\]
In particular, $\F$ then
has natural cohomology as a sheaf on $\PP^m$. Thus all the 
$h^i(\F(e))$ are determined by the formula
\[
\chi(\F(e))=h^0(\F)\binom{\frac{e}{d} +k}{k}.
\]
If $\F$ is a vector bundle of rank $r$ on $\PP^m$, then
we can rewrite this formula as
$
\chi(\F(e))=
\frac{r}{m!}(e+d)\cdots(e+md)=
(\frac{r}{m!}e^m)+\cdots+rd^m.$
\end{theorem}

The vanishing and nonvanishing results in the first part of 
Theorem~\ref{d-uple}
have a very simple interpretation in terms
of the  Betti diagram of the 
Tate resolution of $\F$: They say that the nonzero terms
form a sequence of nonoverlapping strands
and that all of the strands 
representing intermediate 
cohomology have length
precisely $d-1$. The formulas in the second part then give the 
values of the nonzero terms. For example, if $\F$ is 
a rank 2 vector bundle on $\PP^2$ which is an Ulrich sheaf
for the $d$-uple embedding, 
Theorem~\ref{d-uple}
says precisely that the Tate resolution of $\F$,
considered as a sheaf on $\PP^2$, has Betti diagram
\[\tabskip0pt plus 1fil
\halign to\hsize{&\footnotesize\medmuskip1mu$\hfil#\hfil$\cr
\cdots&2(d+2)&1(d+1)&0&0&0&0&0&0&0&\cdots&0&\cdots\cr
\cdots&0&0&1(d-1)&2(d-2)&\cdots&(d-2)2&(d-1)1&0&0&\cdots&0&\cdots\cr%
\cdots&0&0&0&0&0&0&0&1(d+1)&2(d+2)&\cdots&d(2d)&\cdots&\cr}
\]
\noindent where the zero-th term is the one occuring at the far right
(so that, for example, $h^0(\F)=2d^2$). Further examples are
given in the discussion of sheaves on $\PP^3$ below.

To prove that the cohomology vanishes as we claim, we will repeatedly
use the following elementary result, which is an easy case of
\cite[Lemma 7.4]{Eisenbudetal.2001}.
For the reader's convenience
we give a quick proof.

\begin{lemma}\label{generalized Mumford} Suppose $\G$ is a sheaf on $\PP^k$.
\begin{enumerate}
\item[(a)] If \/ $\Hrm^{i+j}(\G(-1-j))=0$ for all $j\geq 0$, then $\Hrm^i(\G)=0$.
\item[(b)] If \/ $\Hrm^{i-j}(\G(1+j))=0$ for all $j\geq 0$, then $\Hrm^i(\G)=0$.
\end{enumerate}
\end{lemma}

Note that the case $i=1$ in part (a)
is Mumford's result showing that a
$(-1)$-\linebreak regular sheaf is 0-regular. 

\begin{proof}[Proof of Lemma~\ref{generalized Mumford}] (a): 
Translating the condition in (a) to a condition on the Tate
resolution $T^{\bullet}(\G)$
over the exterior algebra $E$, we see that the free summand
$\Hrm^i(\G)\otimes \omega_E$ in $T^0(\G)$ maps injectively into
$T^1(\G)$. Since $T^{\bullet}(\G)$ is a minimal complex
and $E$ is Artinian, this is impossible unless $\Hrm^i(\G)=0.$

Part (b) follows by applying the same argument to the dual of
the Tate resolution.
\end{proof}

\begin{proof}[Proof of Theorem~\ref{d-uple}]
We first show, by induction on $i$, 
that for $i<k$, we have $\Hrm^i(\F(e))=0$
if $e\leq -(i+1)d$. Since $\F$ becomes an Ulrich sheaf
under the $d$-uple embedding,
we have $\Hrm^0(\F(-d))=0$, and it follows that 
$\Hrm^i(\F(e))=0$ for $e\leq -d$, which is the case $i=0$.
For $i>0$ we proceed by descending induction on $e$. Again
since $\F$ becomes Ulrich on the $d$-uple embedding, we have
$\Hrm^i(\F(-(i+1)d))=0$, the initial case.
Assuming that 
$\Hrm^i(\F(e))=0$ for some $e<-(i+1)d$,
the induction on $i$ gives
the hypothesis to apply part (b) of
Lemma~\ref{generalized Mumford}, showing
that $\Hrm^i(\F(-e-1))$ vanishes.

Similarly, 
$\Hrm^i(\F(e))=0$
for $i>0$ and $e\geq -i\cdot d$ follows by induction and part (b) 
of Lemma~\ref{generalized Mumford}. The nonvanishing of the remaining
cohomology follows, since otherwise the Tate resolution for $\F$
would contain terms equal to zero. 

We next prove the formulas for $\chi(\F(e))$.
If $\iota_*\F$ is an Ulrich sheaf,
Corollary~\ref{coho of Ulrich}
shows that
$\chi(\F(dt))=h^0(\F){k+t\choose k}$. Since
$\chi(\F(t))$ is a polynomial, it is determined by
this relation, yielding the first formula.

If in addition $\F$ is a bundle of rank $r$ on $\PP^m$,
so that $k=m$,
then part (c) of Proposition~\ref{Ulrich char}
shows that $\h^0(\F)=\deg\iota_*(\F)$,
which is $r$ times the degree of the $d$-uple embedding of
$\PP^m$, that is, $\h^0(\F)=rd^m$. Substituting this in the
first formula, we get the last formulas. (One could also
argue directly from the fact that the last formula must be
a polynomial of degree $m$ which vanishes at $-nd$ for
$n=1,\dots,m$).
\end{proof}

\begin{corollary}\label{divisibility} Suppose there exists a rank $r$ sheaf on
$\PP^k$ which is an Ulrich sheaf for the $d$-uple embedding.
If a prime $p$ divides $d$ and $p^t$ divides $k!$,
then $p^t$ divides $r$. For example, 
any Ulrich sheaf on the $k!$-uple embedding of $\PP^k$
has rank a multiple of $k!$.
\end{corollary}

\begin{proof} In Theorem~\ref{d-uple}, note that $\chi(\F(1))$
is an integer.
\end{proof}

The next result shows that if $X\subset \PP^n$
is the support of an Ulrich (or weakly Ulrich)
sheaf, then the general problem of finding (weakly) Ulrich sheaves for the
Veronese embeddings of $X$ can be reduced to the problem for
\pagebreak\ projective spaces. 

\begin{proposition}\label{general veronese} Let $X\subset \PP^n$ be
a purely $k$-dimensional scheme, and let $\F$ be an
Ulrich sheaf whose support is $X$. Suppose that
$\pi: X\to \PP^k$ is a finite linear projection. If 
$\E$ is a sheaf on projective space that is (weakly)
Ulrich for the $d$-uple embedding of projective space,
then $\F\otimes \pi^*\E$ is (weakly) Ulrich for the 
$d$-uple embedding of $X$.
\end{proposition}

Note that a finite linear projection always
exists if the base field is infinite---this is
``Noether normalization''.

\begin{proof} Since the cohomology of  $\pi_*\F(n)$ is the same as
the cohomology of $\F(n)$, we see from the cohomological 
characterization of Ulrich sheaves that $\pi_*\F$ is a trivial
bundle $\Ocal_{\PP^k}^t$ on $\PP^k$. Since
\[
\Hrm^q(\F\otimes \pi^*\E(d))=\Hrm^q(\pi_*\F\otimes \E(d)),
\]
this group vanishes for exactly the same values of $q,d$ as
does $\Hrm^q(\E(d))$, and this determines the weakly Ulrich and 
Ulrich properties.
\end{proof}

If we apply Proposition~\ref{general veronese} 
in the case where $X\iso \PP^k$, embedded by the $e$-uple
embedding, we get a weak converse to Corollary~\ref{divisibility}.

\begin{corollary}\label{semigroup Ulrich} If $\PP^k$ has Ulrich sheaves of
ranks $a$ and $b$ on its
$d$-uple and $e$-uple embeddings, respectively, then it has an Ulrich
sheaf of rank $ab$ on its $de$-uple embedding.
\qed
\end{corollary}

Proposition~\ref{general veronese} and Corollary~\ref{semigroup Ulrich} were 
inspired by the proof of the existence of rank
4 Ulrich sheaves on the 4-uple embedding of $\PP^3$ given
by Douglas Hanes in his thesis 
\cite{Hanes 2000}.

If our ground field $K$ has characteric zero, then 
up to twists
any indecomposable homogenous bundle
on $\PP^n$ can be obtained by applying a Schur functor $S_\lambda$ to
the universal rank $n$ quotient bundle $Q=\coker(\Ocal_{\PP^n}(-1) \to \Ocal_{\PP^n}^{n+1})$ 
of $\PP^n$ (the tangent bundle tensor $\Ocal_{\PP^n}(-1)$).
Here $\lambda=(\lambda_1,\ldots,\lambda_n)$ is a partition into at most
$n$ parts. Note $(S_\lambda Q)(1)=S_{\lambda+(1,1,\ldots,1)} Q$ 
and $\Hrm^0(S_\lambda Q) = S_\lambda V$ with
$V = H^0(\cO(1))^*$.
Thus up to twist we may assume that $\lambda_n=0$. The theorem below implies that
$S_\lambda Q$ has Castelnuovo--Mumford regularity precisely zero if and only if $\lambda_n=0$.
For our purposes it is convenient to visualize the partition as a Ferrers diagram
whose row lengths are given by the $\lambda_i$, as follows:
\centerline{
\hbox { \vbox { \vskip 5pt
        \hrule width 64pt
        \vskip 0pt
        \hbox {
                \kern -3.5pt
                \hbox to 13pt{\vrule height10pt depth3pt$\ $ \vrule height10pt depth3pt}
                \kern -3.5pt
                \hbox to 13pt{\vrule height10pt depth3pt$\ $ \vrule height10pt depth3pt}
                \kern -3.5pt
                \hbox to 13pt{\vrule height10pt depth3pt$\ $ \vrule height10pt depth3pt}
                \kern -3.5pt
                \hbox to 13pt{\vrule height10pt depth3pt$\ $ \vrule height10pt depth3pt}
                \kern -3.5pt
                \hbox to 13pt{\vrule height10pt depth3pt$\ $ \vrule height10pt depth3pt}  $\quad \lambda_{1}$
                }
        \vskip 0pt
        \hrule width 64pt
        \vskip 0pt
        \hbox { \hskip 26pt \kern -3.5pt
                \vbox{ \hbox{
                        \kern -3.5pt
                        \hbox to 13pt{\vrule height10pt depth3pt$\ $ \vrule height10pt depth3pt}
                        \kern -3.5pt
                        \hbox to 13pt{\vrule height10pt depth3pt$\ $ \vrule height10pt depth3pt}
                        \kern -3.5pt
                        \hbox to 13pt{\vrule height10pt depth3pt$\ $ \vrule height10pt depth3pt}  $\quad \lambda_{2}$
                        }
                \vskip 0pt
                \hrule width 38pt
                \vskip 0pt
                \hbox { \hskip 13pt \kern -3.5pt
                      \vbox { \hbox { \kern -3.5pt 
                                   \hbox to 13pt{\vrule height10pt depth3pt$\ $ \vrule height10pt depth3pt}
                                   \kern -3.5pt
                                   \hbox to 13pt{\vrule height10pt depth3pt$\ $ \vrule height10pt depth3pt} 
                               }
                               \vskip 0pt
                               \hrule width 25pt
                               \vskip 0pt 
                               \hbox { \kern -3.5pt 
                                    \hbox to 13pt{\vrule height10pt depth3pt$\ $ \vrule height10pt depth3pt}
                                    \kern -3.5pt
                                    \hbox to 13pt{\vrule height10pt depth3pt$\ $ \vrule height10pt depth3pt} $\quad \lambda_{n-2}$
                                }
                               \vskip 0pt
                               \hrule width 25pt
                               \vskip 0pt
                               \hbox { \hskip 13pt  \kern -3.5pt
                                     \vbox{ \hbox{ \kern -3.5pt
                                      \hbox to 13pt{\vrule height10pt depth3pt$\ $ \vrule height10pt depth3pt} $\quad \lambda_{n-1}$}
                                      \vskip 0pt
                                      \hrule width 12pt
                                      \vskip 0pt
                                      }}}}}}}}}  
The following result was pointed out to us by J. Weyman.

\begin{theorem}\label{Tate of homogeneous bundles} 
Suppose that $K$ has characteristic zero. 
Let $\lambda=(\lambda_1,\ldots,\lambda_{n-1})$ be a partition and $Q$
the universal rank n quotient bundle on $\PP^n$. 
The Tate resolution of the homogeneous bundle $\F = S_\lambda Q$ 
has nonzero terms only in the degrees marked $*$
in the following Betti diagram, 
in which the Ferrers diagram has
shape $\lambda$ as above:
\vbox{ \vskip 5pt  \hbox { \hskip 80pt \vbox {
\vskip 5pt $ * $ \hskip 4pt $ *$ \hskip 4pt $ *$ \vskip 2pt
\hbox { \hskip 50pt \vbox{
        \hrule width 64pt
        \vskip 0pt
        \hbox {
                \kern -3.5pt
                \hbox to 13pt{\vrule height10pt depth3pt$\ *$ \vrule height10pt depth3pt}
                \kern -3.5pt
                \hbox to 13pt{\vrule height10pt depth3pt$\ *$ \vrule height10pt depth3pt}
                \kern -3.5pt
                \hbox to 13pt{\vrule height10pt depth3pt$\ $ \vrule height10pt depth3pt}
                \kern -3.5pt
                \hbox to 13pt{\vrule height10pt depth3pt$\ $ \vrule height10pt depth3pt}
                \kern -3.5pt
                \hbox to 13pt{\vrule height10pt depth3pt$\ $ \vrule height10pt depth3pt}
                }
        \vskip 0pt
        \hrule width 64pt
        \vskip 0pt
        \hbox { \hskip 26pt \kern -3.5pt
                \vbox{ \hbox{
                        \kern -3.5pt
                        \hbox to 13pt{\vrule height10pt depth3pt$\ *$ \vrule height10pt depth3pt}
                        \kern -3.5pt
                        \hbox to 13pt{\vrule height10pt depth3pt$\ $ \vrule height10pt depth3pt}
                        \kern -3.5pt
                        \hbox to 13pt{\vrule height10pt depth3pt$\ $ \vrule height10pt depth3pt}
                        }
                \vskip 0pt
                \hrule width 38pt
                \vskip 0pt
                \hbox { \hskip 13pt \kern -3.5pt
                      \vbox { \hbox { \kern -3.5pt 
                                   \hbox to 13pt{\vrule height10pt depth3pt$\ $ \vrule height10pt depth3pt}
                                   \kern -3.5pt
                                   \hbox to 13pt{\vrule height10pt depth3pt$\ $ \vrule height10pt depth3pt}
                               }
                               \vskip 0pt
                               \hrule width 25pt
                               \vskip 0pt 
                               \hbox { \kern -3.5pt 
                                    \hbox to 13pt{\vrule height10pt depth3pt$\ *$ \vrule height10pt depth3pt}
                                    \kern -3.5pt
                                    \hbox to 13pt{\vrule height10pt depth3pt$\ $ \vrule height10pt depth3pt}
                                }
                               \vskip 0pt
                               \hrule width 25pt
                               \vskip 0pt
                               \hbox { \hskip 13pt  \kern -3.5pt
                                     \vbox{ \hbox{ \kern -3.5pt
                                      \hbox to 13pt{\vrule height10pt depth3pt$\ *$ \vrule height10pt depth3pt}}
                                      \vskip 0pt
                                      \hrule width 12pt
                                      \vskip 3pt \hbox {\hskip 16pt $*$ \hskip 4pt $*$ \hskip 4pt $*$  \hskip 4pt $*$}
                                      }}}}}}}}}}}         
More precisely, for $1\le i \le n-1$ the cohomology group 
$\Hrm^i((S_\lambda Q)(m))$ is nonzero
 if and only if  
$\lambda_{n-i+1} < -m-i \le \lambda_{n-i}$,
 $\Hrm^0 S_\lambda Q (m)=0$ if and only if $m<0$ and 
$\Hrm^n S_\lambda Q(m)=0$ if and only if $m\ge -n-\lambda_1-1$.
\end{theorem} 
 
\begin{proof} The cohomology of a homogeneous bundle on the homogeneous space  
\[\PP^n = \GL(n+1)/\begin{pmatrix} \GL(n) & * \\ 0 & \GL(1)
  \end{pmatrix}\] 
is determined by  
Bott's formula; see \cite{Jantzen 1987}. In particular 
$\Hrm^iS_\lambda Q(m) \not= 0 \hbox{ for at most one } i$, and 
\[\Hrm^iS_\lambda Q(m) = 0 \hbox{ for all } i \Leftrightarrow 
-m \in \{\lambda_i+n+1-i \mid i=1,\ldots,n \}.\]
Thus the Hilbert polynomial 
$\chi S_\lambda Q (m)$ has precisely $n$ integral zeroes
and the Tate resolution ``steps down'' precisely at these $n$ values   
by Theorem~\ref{d-uple}. 
\end{proof}

\begin{corollary}\label{homogeneous Ulrich} 
Suppose that $K$ has characteristic zero. 
The unique
indecomposable homogeneous bundle on $\PP^n$ that is an Ulrich sheaf 
for the $d$-uple embedding is
$S_\lambda Q$ with $\lambda=((d-1)(n-1),(d-1)(n-2),\ldots,(d-1),0)$. 
It has rank $d^{n \choose 2}$.
\end{corollary}

\begin{proof} The first statement follows easily from the previous theorem. 
The rank of $S_\lambda Q$ is given by the hook formula (see \cite{Stanley 1971}
or \cite[p.~55]{Fulton 1997})
\[\rank S_\lambda Q = \prod_{(i,j) \in \lambda} \frac{n+i-j}{h(i,j)},\]
where $h(i,j)$ denotes the hook length of the $(i,j)$-th box. 
The largest hook length
is $h(1,1)=(d-1)(n-1)+(n-2)=d(n-1)-1$. The denominators of the first row 
contribute
\begin{align*}
\textstyle\prod_j h(1,j)&=
  (d(n-1)-1)(d(n-1)-2)\\
&\quad\cdot\ldots\cdot(d(n-1)-d+1)(d(n-2)-1) 
\cdot\ldots  \cdot 1\\
&=\frac{[d(n-1)]!}{d^{n-1} (n-1)!}.
\end{align*}
The numerators give 
\vadjust{\vskip2pt}
{\small$\displaystyle\smash{\frac{[d(n-1)]!}{(n-1)!}}$}. 
Thus the first row contributes 
 $d^{n-1}$ and the
 rank is $d^{{n-1 \choose 2}+n-1}=d^{n \choose 2}$ by induction
on the number of rows in $\lambda$.
\end{proof}

\subsection*{Chow forms from line bundles on projective spaces}

All the classically known formulas (and no new ones) 
for the resultant of
$k+1$ forms of degree $d$ in $k+1$ variables come from
applying these ideas to line bundles on projective spaces.
These are rarely Ulrich bundles, and
we get B\'ezout formulas in this way only
for binary forms of
any degree or linear forms in any number of variables.

On the other hand
$\cL= \cO(j)$ on $\PP^k$ gives rise to a two term complex, and hence a
Stiefel formula for the Chow form of the $d$-uple image, if and only if
\[ \Hrm^0 \cL(-H) =0 \, \quad\hbox{ and } \quad\, \Hrm^k \cL(-(k-2)H) = 0, \]
equivalently, if and only if
\[ d-1 \ge j \ge -k +(k-2)d=(k-2)(d-1)-2.\]
Thus the Chow forms of $\PP^1,\PP^2,\PP^3$ for arbitrary $d$-uple embeddings, on
$\PP^4$ for quadrics and cubics and on $\PP^5$ for quadrics, can be written
as determinants of maps of vector bundles on the Grassmannian, or as
determinantal formulas in the Stiefel coordinates. This is
precisely the list of 
\cite[Chap.~13, Prop.~1.6]{Gelfandetal.1994}.
For instance, in the case of three ternary
quadrics we have:

\begin{example}\label{quadrics on P2}
For the 2-uple embedding (quadrics) of
$\PP^2$ the line bundle $\Ocal_{\PP^2}(1)$ is weakly Ulrich,
and we see that the Chow form is the determinant
of a canonical map on the Grassmannian $\GG$
\[
\Ocal_\GG(-1)^6\to U \oplus \Ocal_\GG^3
\]
with $U$ the universal subbundle on $\GG=\GG_3=\GG(3,W)$ and $W=\Hrm^0(\cO_{\PP^2}(2))$.
The map is easy to calculate, for example,
using the computer algebra system Macaulay2,
 as a second syzygy matrix over $E$ of the ``multiplication'' map \linebreak
$\Hrm^0(\cO_{\PP^2}(3)) \to \Hrm^0(\cO_{\PP^2}(5)) \tensor W^*$.
In Stiefel coordinates we obtain the matrix 
\[\hbox{transpose }\begin{pmatrix}
a_0 & b_0 & c_0 & {\scriptstyle[0,1,5]}           &0                  &{\scriptstyle [0,1,2] }\\
a_1 & b_1 & c_1 &{\scriptstyle [0,3,5]}           &{\scriptstyle [0,3,4]
}         & {\scriptstyle [0,1,4]-[0,2,3] }\\
a_2 & b_2 & c_2 &{\scriptstyle [0,4,5]-[1,2,5]} &{\scriptstyle [0,3,5] }         & {\scriptstyle  [0,1,5] }\\          
a_3 & b_3 & c_3 & 0                   &{\scriptstyle [1,3,4]}          &{\scriptstyle [0,3,4] }\\       
a_4 & b_4 & c_4 &{\scriptstyle [2,3,5]}           & {\scriptstyle[2,3,4]+[1,3,5]} &{\scriptstyle [0,3,5]} \\        
a_5 & b_5 & c_5 &{\scriptstyle [2,4,5]}           &{\scriptstyle [2,3,5]}          &0       \\      
\end{pmatrix}.\]
Thus the determinant of this matrix
is the resultant of three quadratic forms
$d=d_0x^2+d_1xy+d_2xz+d_3y^2+d_4yz+d_5z^2$ for $d=a,b,c$
with $(i,j,k)$-th Pl\"ucker coordinate
\[[i,j,k]=\det \begin{pmatrix}
a_i & b_i & c_i \\
a_j & b_j & c_j \\
a_k & b_k & c_k \\ \end{pmatrix}.\]
\end{example}

\subsection*{Ulrich sheaves on ${\mathbf P}^2$}

To get new formulas for resultants, we replace line bundles
with vector bundles of higher rank. The Chow forms of these
bundles are the desired resultants raised to a power equal
to the rank of the bundle. But if the rank of the bundle is 2, then its
natural symplectic structure allows us to
find a polynomial square root by taking a Pfaffian in place
of a determinant, so we get formulas for the resultant itself.

\begin{proposition}\label{rank 2 on P2} If $\alpha$ is
a $(d+1)\times (d-1)$ matrix of linear forms on $\PP^2$ 
whose minors of order $d-1$ generate an ideal of codimension 3
(the generic value), then
\[
\coker \bigl(\Ocal_{\PP^2}(d-2)^{d-1}\rTo{\alpha} \Ocal_{\PP^2}(d-1)^{d+1}\bigr)
\]
is an Ulrich sheaf on the $d$-uple embedding of $\PP^2$.
\end{proposition}

For example, we may take
\[
\alpha=\begin{pmatrix}
x_0 & x_1 & x_2 & 0 & \ldots & 0 \\
0 & x_0 & x_1 & x_2 & & \vdots\\
\vdots && \ddots & \ddots & \ddots  \\
0 & \ldots & & x_0 & x_1 & x_2 \\ \end{pmatrix}.
\]

\begin{proof}[Proof of Proposition~\ref{rank 2 on P2}]
Setting
$\F=\coker \bigl(\Ocal_{\PP^2}(d-2)^{d-1}\rTo{\alpha} \Ocal_{\PP^2}(d-1)^{d+1}
\bigr)$,
we see that 
$\bigwedge^2 \F \iso \Ocal_{\PP^2}(3d-3)= \Ocal_{\PP^2}(3d)\otimes \omega_{\PP^2}$.
Since $\F$ is a rank 2 vector bundle, 
\[
\F=\F^*\otimes \bigwedge^2\F 
= \F^*\otimes \Ocal_{\PP^2}(3d)\otimes \omega_{\PP^2},
\]
so, {\it as a sheaf on the ambient space of the $d$-uple
embedding of $\PP^2$,\/} $\F$ satisfies the duality hypothesis
of Corollary~\ref{self-dual Ulrich}. Furthermore, the given
presentation of $\F$ shows that $\F$ is $(d-1)$-regular
as a sheaf on $\PP^2$,
and thus it is 0-regular on the ambient space
of the $e$-uple embedding for any $e\geq d-1$.
Thus Corollary~\ref{self-dual Ulrich} shows that $\F$ is
an Ulrich sheaf on the $d$-uple embedding.
\end{proof}
  
The Betti diagram of the Tate resolution of a rank
2 sheaf $\F$ satisfying the hypothesis of 
Proposition~\ref{rank 2 on P2} 
is given just after 
Theorem~\ref{d-uple}. 
Instead of specifying $\alpha$, we could define
$\F$ by giving the $(d-1) \times 2(d-2)$ 
matrix $\beta$ of linear forms over $E$
that occurs at the end of the middle strand
of the Tate resolution. For the choice of $\alpha$ above
we get 
\[\beta=\begin{pmatrix}
e_0 & e_1 & 0 & 0 && \ldots && 0 \\
e_1 & e_2 & e_0 & e_1 & &  && \\
0 & 0 & e_1 & e_2 & &&& \vdots\\
\vdots &&&&\ddots  \ddots &&& 0 \\
&&&&& &e_0 & e_1 \\
0&&&\ldots&&0& e_1 & e_2 \\ \end{pmatrix},\]
and the vector bundle $\cE$ has a conic of maximal order jumping
lines. One can show by semi-continuity that $\beta$ can be
taken to be any sufficiently general 
$(d-1) \times 2(d-2)$ matrix of linear forms over $E$, but 
unlike for the matrix $\alpha$, we do not know how to recognize
when $\beta$ is sufficiently general to give rise to 
a Tate resolution of the right form.

\subsection*{Bundles on ${\mathbf P}^3$}

\begin{proposition}\label{Instanton bundles} Suppose $d\geq 2$.
There exist rank 2 Ulrich sheaves
for the $d$-uple embedding of $\PP^3$ if and only if 
$d \not\equiv 0\  ({\rm mod}\ 3)$.
\end{proposition}

\begin{proof}
By \cite{Hartshorne and Hirschowitz 1982}  for any given $c_2$ there exist
rank 2 vector bundles $\F$ with $c_1=0$ 
and natural cohomology on $\PP^3$, i.e., for each twist $t$ the cohomology
groups  $\Hrm^i(\F(t))\not=0 $ for at most one $i$. The Hilbert
\pagebreak \ polynomial $\chi(\F(t))=\frac{1}{6}(t+2)(t^2+4t+3-3c_2)$ has three
integral roots if and only if $1+3c_2$ is a square. Thus for
$d \not\equiv 0\ ({\rm mod}\ 3)$  and
$c_2=(d^2-1)/3$ the sheaf $\F(d-2)$ is Ulrich for the d-uple embedding.
The converse follows from Corollary~\ref{divisibility}.
\end{proof}

\begin{remark} The bundles $\F$ in the proof of the proposition
 are  called {\it instanton bundles\/} \cite{Tikhomirov 1997}
because they satisfy the instanton conditions
\[ 
\F \hbox{ is stable of rank } 2,\, c_1(\F)=0 \hbox{ and } \Hrm^1(\F(-2))=0.
\] 
Equivalently their linear monad $\LL(\F)$ has shape
\[ 0 \rTox \Ocal(-1)^{c_2} \rTox \Ocal^{2c_2+2} \rTox \Ocal(1)^{c_2} \rTox 0.\]
Except for the 2-uple embedding ($c_2=1)$,
it is an open problem  to find an explicit expression for
rank 2 Ulrich sheaves of this type. 
\end{remark}

For the 2-uple embedding the rank 2 Ulrich sheaf is essentially unique:

\begin{proposition}\label{2-uple on P3} 
If $\E$ is the null correlation bundle on $\PP^3$,
then the pushforward of $\F:=\E(-2)$ is,
up to automorphisms of $\PP^3$, the unique
rank 2 Ulrich sheaf on the 2-uple embedding on $\PP^3$.
\end{proposition}

\begin{proof} By Theorem~\ref{d-uple}, $\F$ is a rank 2 Ulrich sheaf if and only if
the Betti diagram of the Tate resolution of $\F$ has the form
\[
\tabskip0pt plus 1fil
\halign to\hsize{&\small\medmuskip1mu$\hfil#\hfil$\cr
***&64&35 & 16 & 5 &. &.&.&.&.&.&\dots\cr
\dots&.&.&.&.&1&.&.&.&.&.&\dots\cr
\dots&.&.&.&.&.&1&.&.&.&.&\dots\cr
\dots&.&.&.&.&.&.&5&16&35&64&*** \cr}
\]  
with Hilbert polynomial $\chi \cF(t) = {\frac13}(t+2)(t+4)(t+6)$.
(Here and henceforward, 
we replace each zero in a Betti diagram by a ``.'' to improve legibility.)
By \cite[II.3.2, Example 6]{Okonek et al. 1980},
the null correlation bundle is determined (up to twist)
by its intermediate cohomology $\cF$
and the choice of a nondegenerate 2-form (the 2-form
is visible here as 
the map in the middle of the Tate resolution).
Thus $\F$ must be a twist of the null correlation bundle;
the twist is determined by a comparison of Hilbert polynomials.
\end{proof}

By Corollary~\ref{divisibility} there is
no rank 2 bundle on $\PP^3$ that is an Ulrich sheaf for the 
3-uple embedding. Corollary \ref{homogeneous Ulrich} gives
a homogeneous bundle of rank 9.
The following example gives a whole family
of rank 3 Ulrich bundles for this case.
These bundles give
determinantal B\'ezout formulas for the cube of the resultant
of 4 forms of degree 3 in 4 variables.

\begin{example}\label{3-uple on P3}[A family of rank 3 
vector bundles on $\PP^3$ which
are Ulrich sheaves for the 3-uple embedding]

By Theorem~\ref{d-uple} $\F$ is an Ulrich sheaf
for the 3-uple embedding if and only if  the Betti diagram of its
Tate resolution 
\pagebreak\ has the  form
\[
\tabskip0pt plus 1fil
\halign to\hsize{&\small\medmuskip1mu$\hfil#\hfil$\cr
\ldots&81&40&14 &  & . &. &.&.&.&.&.\cr
.&.&.&.&5&4&.&.&.&.&.&.\cr
.&.&.&.&.&.&4&5&.&.&.&.\cr
.&.&.&.&.&.&.&.&14&40&81&\ldots& \cr}
\]
Calculation shows that if we take a sufficiently general
$5\times 4$ matrix over the exterior algebra in 4 variables,
then its Tate resolution has this form.
\qed
\end{example}

It follows at once from the definitions that
a sheaf on $\PP^k$ becomes weakly Ulrich on the $d$-uple embedding
if and only if 
\begin{align*}
&h^0\F(-2d)=0; \\
&h^i\F((-i-2)d)=0=h^i\F((-i+1)d),\quad 0<i<k-1;\ \text{\rm and} \\
&h^k\F((-k+1)d)=0.
\end{align*}
From the form of the
cohomology diagram of the ``null correlation bundle'' 
on $\PP^3$ given in the proof
of Proposition \ref{2-uple on P3}, we see that a twist of this bundle 
becomes weakly Ulrich on each $d$-uple embedding and thus gives
a Pfaffian Stiefel formula for the 
 resultant of 4 forms in 4 variables of any degree.
For any $d \ge 2$ the corresponding  2-term complex on 
$\GG(4,\Hrm^0\cO_{\PP^3}(d))$ has the form
\[ 
0 \to \cO(-1)^b \oplus U^a \to 
\cO^b \oplus (\Lambda^3 U)^a \to 0
\]
with $a=d(d^2-4)/3$ and $b=2d(4d^2-4)/3$.

\subsection*{Bundles on ${\mathbf P}^4$}

\begin{example}\label{Horrocks--Mumford} 
The Horrocks--Mumford bundle on $\PP^4$
has rank 2 and  Tate resolution
\[
\tabskip0pt plus 1fil
\halign to\hsize{&\small\medmuskip1mu$\hfil#\hfil$\cr
\ldots & 100 & 35 & 4 & . & . & . &.&.&.&.&.&. \cr
      . & . & 2 & 10 & 10 & 5 & . &.&.&.&.&.&. \cr
                  .&.&.&.&.&.&2 &.&.&.&.&.&. \cr
. &.&.&.&.&.&.&5&10&10&2&.&.\cr
. &.&.&.&.&.&.&.&.&4 &35 & 100 & \ldots \cr}\]
It gives rise to Pfaffian Stiefel formulas for $d=4,6,8$.
\end{example}

\begin{example}\label{d=2 Ulrich sheaf on P4} Suppose again that $k=4$,
and take $d=2$.
By Corollary~\ref{divisibility}
any Ulrich sheaf
on the $2$-uple embedding of $\PP^4$ has rank divisible by 8.
Consider a general map $E^3 \to E^5(-2)$. Its Tate resolution is
\[
\tabskip0pt plus 1fil
\halign to\hsize{&\small\medmuskip1mu$\hfil#\hfil$\cr
\ldots  &128 & 35 & . & . &.&.&.&. \cr
       . & . & . & 5 & . &.&.&.&. \cr
                  .&.&.&.&3 &.&.&.&. \cr
.&.&.&.&.&5&.&.&.\cr
.&.&.&.&.&.&35&128 &\ldots  \cr}\]
This gives a rank 8 Ulrich sheaf.
\end{example}

\section{Surfaces} \label{Surfaces}

Throughout this section, $X$ denotes a nonsingular projective
surface over $K$, and we assume that $K$ has characteristic 0.
We study Ulrich sheaves on $X$.
We write $H$ for a hyperplane divisor and $K_X$ for a canonical
divisor on $X$.

In general it is rare to find an Ulrich line bundle on 
a surface; for example, it is easy to see that there is none on
the $d$-uple embedding of $\PP^2$ when $d>1$. 
Thus we turn to rank 2 bundles.
If $\F$ is a rank 2 vector bundle on $X$ such that
$c_1 \F = 3H+K_X$ and $\F$ is 0-regular, then $\F$ is
Ulrich by Corollary~\ref{self-dual Ulrich}. We will call such a
rank 2 bundle a {\it special rank 2 Ulrich bundle\/};
it gives rise to a B\'ezout expression for the Chow form of $X$
as a Pfaffian.

Many surfaces have no rank 2 Ulrich bundles. 
For example, one can see by considering the
dimensions of the families that
the general surface $X$ of degree $d\ge 16$ in $\PP^3$ 
is not defined by the Pfaffian
of a $2d \times 2d$ skew-symmetric linear matrix
\cite{Beauville 2000}.
Thus such a surface has no special rank 2 Ulrich bundle,
and because Pic $X=\ZZ$ for a general surface, every
rank 2 Ulrich sheaf would be special.

We are particularly interested in the case when $X$ is
a blow-up of $\PP^2$ at a set of points
$E=\{p_1,\dots,p_e\}$. Write $L$ for the pull-back to 
$X$ of the class of a line on $\PP^2$ and $E_i$ for the
preimage of $p_i$. If we take an embedding
of $X$ corresponding to a linear series $|dL-\sum_1^eE_i|$, 
then the vanishing of the Chow form is the condition
for 3 forms of degree $d$ that vanish 
on $E$ to vanish at a further common point. (This has also
been called the ``residual resultant'', studied in the case of
complete intersections in \cite{Buseetal.2001}.)
We are able to find rank 2 Ulrich sheaves, corresponding to
Pfaffian B\'ezout formulas for the resultant, if the ideal
of the set of base points $E$ is generated in
degree $<d$.

\begin{furtherwork}
Stieffel resultant formulas for toric surfaces were obtained by 
A. Khetan, who identifies a class of 
weakly Ulrich line bundles on such surfaces
and finds explicit formulas by constructing syzygies over
the exterior algebra, using the method illustrated in 
Proposition~\ref{P1}.  See \cite{Khetan 2002}.
\end{furtherwork}

\begin{proposition}\label{Ulrich on a surface}
\begin{enumerate}
\item[(a)] Let $C$ be a smooth curve on $X$ of class $3H+K_X$ and let $\cL$ be a line
bundle on $C$ with
\[
\deg \cL = {1\over2} H\cdot(5H+3K_X) + 2\chi\cO_X.
\]
If $\sigma_0,\sigma_1\in \Hrm^0(\cL)$ define a base point free pencil
and  $\Hrm^1 \cL(H+K_X)=0$,
then the bundle $\F$ defined by the ``Mukai exact sequence''
\[
0 \rTox \F^* \rTox \cO^2_X \rTo{(\sigma_0,\sigma_1)} \cL \rTox 0
\] 
is a special rank 2 Ulrich bundle. 
\item[(b)]Every special rank 2 Ulrich bundle on $X$
can be obtained from a Mukai sequence as in part \textup{(a)}.
\end{enumerate}
\end{proposition}

\begin{proof} (a) We begin by proving that, under the hypotheses of part (a),
the map
\[
(\Hrm^0 \cO(H+K_X))^2 \rTo{(\sigma_0,\sigma_1)} \Hrm^0 \cL(H+K_X)\leqno{(*)}
\]
is an isomorphism. Using Riemann--Roch on $X$ and on $C$
and the given degree of $\Lcal$, we immediately compute
$\chi(\Lcal(H+K_X))=2\chi(\Ocal_X(H+K_X))$ and
$\chi(\Lcal(2H+K_X))=2\chi(\Ocal_X(2H+K_X))$. Our hypothesis that
$\Lcal(H+K_X)$ is nonspecial implies that $\Lcal(2H+K_X)$ is also
nonspecial. With this and the
Kodaira vanishing theorem on $X$, we
see that  $\chi$ is equal to $\h^0$ for
all four of these bundles. Thus it suffices to show that
the map $(*)$ is injective.

Since $C \sim 3H+K_X$,
there is an exact sequence
\[
0\rTox \Ocal_X(-2H)\rTox \Ocal_X(H+K_X)\rTox O_C(H+K_X)\rTox 0,
\]
from which we see that the restriction map
$\Hrm^0\cO_X(H+K_X)\cong \Hrm^0\cO_C(H+K_X)$ is an injection.
By the base point free pencil trick there is a left exact sequence
\[
0\rTox \Hrm^0 \cL^*(H+K_X) \rTox (\Hrm^0 \cO_C(H+K_X))^2
\rTo{(\sigma_0,\sigma_1)} \Hrm^0\Lcal(H+K_X).
\]
By the adjunction formula $K_C=(3H+2K_X)\mid_C$, so
our hypothesis and Serre duality
give $0=\h^1\Lcal(H+K_X)=\h^0 \cL^*(2H+K_X)$,
whence $\h^0 \cL^*(H+K_X)=0$ as well. Thus
$(*)$ is an injection.

We can now prove that $\F$ is Ulrich. (The conclusion
depends only on the Mukai sequence (with the class of $C$) and
the fact that $(*)$ is
an isomorphism).
The Mukai sequence implies that $\bigwedge^2\F = \Ocal_X(3H+K_X)$,
so by
Corollary~\ref{self-dual Ulrich} it suffices to show that
$\F$ is 0-regular.
Twisting the Mukai sequence by $H+K_X$ and using 
the fact that $(*)$ is an isomorphism, together with Kodaira vanishing,
we see that $\Hrm^1(\F^*(H+K_X))=\Hrm^2(\F^*(H+K_X)) = 0$.
Serre duality now gives $\Hrm^1(\F(-H))\linebreak =0$ and $\Hrm^0(\F(-H))=0$.
Since $\bigwedge^2\F = \Ocal_X(3H+K_X)$,
we have $\F(-H)=\linebreak \F^*(2H+K_X)$. By Serre duality
$\h^2(\F(-2H))=\h^0(\F^*(2H+K_X))=\h^0(\F(-H))=0$, 
and $\F$ is Ulrich as claimed.

\medbreak
(b) Conversely, if $\F$ is a 
special Ulrich bundle of rank 2, then
two general sections $\tau_0,\tau_1$ of $\F$ become dependent
on a smooth curve $C$ of class $3H+K_X$. The cokernel of the 
induced map
$ 0 \to \F^* \to \bigoplus_1^2 \cO$
is a line bundle $\cL$ on $C$, generated by
global sections, so we obtain the Mukai sequence
\[ 
0 \to \F^* \to \bigoplus_1^2 \cO\rTo{(\sigma_0,\sigma_1)} \Lcal\to 0.
\]
By Serre duality, $\chi(\F^*(H+K_X))=\chi(\F(-H))$, which is
0 since $\F$ is Ulrich. Thus $\chi(\Lcal(H+K_X))=2\chi(O_X(H+K_X))$.
Applying the Riemann--Roch theorems on $X$ and $C$
again, we obtain the
desired formula for the degree of $\Lcal$.
The long exact sequence coming from the Mukai sequence
together with Kodaira vanishing and the 0-regularity
of $\F$ show that $\Lcal$ is nonspecial.
\end{proof}

\begin{corollary} 
Let $C$ be a smooth curve of class $3H+K_X$ and let $\cL$ be a
\linebreak line
bundle on $C$ such that $\mid \cL \mid$ is a base point free pencil
of degree 
$
\deg \cL \linebreak = {1\over2} H\cdot(5H+3K_X) + 2\chi\cO_X.
$
The conditions of Proposition~\ref{Ulrich on a surface} are satisfied
if and only if $\mid \cL \mid$ does not arise as a projection
from $\mid \cO_C(2H+K_X) \mid$.
\end{corollary}

\begin{proof} To say that $\mid \cL \mid$ arises as a projection
from $\mid \cO_C(2H+K_X) \mid$ means that 
$\Hrm^0(\cL^*(2H+K_X))\neq 0$. 
This space is Serre dual to $\Hrm^1(\cL(H+K_X))$.
\end{proof}

\begin{furtherwork} Pencils which arise as projections correspond to
codimension 2 planes that are ${\frac72}H\cdot(H+K_X) + K_X^2 -2\chi\cO_X-$secant 
to $C \subset \PP H^0 \cO(2H+K_X)$. Every component of the variety
of such secants has dimension at least
\[ \tfrac12 H\cdot(H-K_X) + 4\chi \cO_X -4 -K_X^2, \]
and we might expect equality. On the other hand the variety of pencils
$\mid \cL \mid$ has dimension at least
\[
\rho(\cL) = 2 \deg \cL - g_C - 2 =\tfrac12 H\cdot(H-3K_X) + 4\chi \cO_X -1
-K_X^2.
\]
Thus we would
expect the existence of an $\cL$ which is not a projection,
and thus of a special rank 2 Ulrich bundle, in case 
$ 
H\cdot K_X < 3.
$  
\end{furtherwork}

\subsection*{Resultants of ternary forms with base points} 

Consider $X=\PP^2(p_1,\ldots,p_e)$ the
blow up of the plane in $e$ distinct points and a very ample
divisor class $H=dL-\sum_{i=1}^e E_i$. Here $L$ denotes the class of a
line and the $E_i$ the exceptional divisors.
We first treat forms of degree $d=3$---that is, Del Pezzo surfaces.
In this case the results are immediate from what we have already
done.

\begin{corollary}\label{del Pezzo} Suppose that the base field $K$ is
algebraically closed. If $X\subset \PP^n$ is a del Pezzo surface, then
$X$ has a special rank 2 Ulrich bundle. Thus there is a Pfaffian
B\'ezout formula for the resultant of 3 ternary cubics with
$d$ base points in general position.
\end{corollary}

\begin{proof}  In this case $K_X=-H$ and $C\sim 3H+K_X$ is a canonical curve
of genus $g=H^2+1=n+1$.
Any general line bundle of
degree 
\[
\deg \cL = \tfrac12 H\cdot(5H+3K_X) + 2\chi\cO_X=g+1
\] 
defines a nonspecial pencil. Thus we can apply
Proposition~\ref{Ulrich on a surface} to get a special rank 2 Ulrich bundle on $X$.

The space of ternary cubics with $d$ general base points
has dimension $10-d$, so
it suffices to treat the case
of seven or fewer points. The linear series of
cubics with 7 assigned base points maps the plane two-to-one
onto itself, and the condition that three such cubics meet
in an extra point is the condition that three lines in the plane
meet in a point---a determinantal condition. 

For six or fewer assigned base points the resultant is
exactly the Chow form of the corresponding del Pezzo surface.
\end{proof}

In the case of higher degree forms, our results are less
complete:

\begin{theorem}\label{rational surfaces} Let the ground field be infinite. 
Let $E = \{p_1,\ldots,p_e\}$ be a collection of 
$e$ distinct points in $\PP^2$ and let $X = \PP^2(p_1,\ldots,p_e)$ 
be the blow up of $\PP^2$
in these points, embedded by the linear system $|dL-\sum E_i|$. If the 
homogenous ideal
$I_E$ of the points is generated in degree $d-1$, then $X$ has a special 
rank 2 Ulrich sheaf.
\end{theorem}

\begin{proof} 
Let $\eta : X \to \PP^2$ be the blow up. By Proposition~\ref{Ulrich on a surface}
we have to construct a pencil $|\cL|$ of degree $\frac{(d-1)(5d-4)}{2} - e$ 
on a smooth curve of class $(3d-3)L-2\sum_iE_i$ on $X$ which 
satisfies $\Hrm^1 \cL(H+K_X)=\Hrm^1\cL((d-3)L)=0$.
Let $C'=\eta(C) \subset \PP^2$ be the plane model, so
that $C'$ is a curve of degree $3d-3$ passing doubly through the 
points $\{p_1,\dots,p_e\}$. Every pencil on $C$ 
can be written as a pencil of adjoint curves of degree $a$, say, with assigned
base points $F=q_1+\ldots+q_f$ on $C'$, that is,
a pencil $\{\lambda A_0+\mu A_1\} \subset \Hrm^0(\PP^2 ,\cI_{E\cup F}(a))$.
The pencil of plane curves might have additional base points
$G=r_1+\ldots+r_g$ away from $C'$. We have
\[ a^2= e + f +g. \]
In order that $|\cL|$ is not a projection from $|2H+K_X|$ we need $a>2d-3$.
We choose $a=2d-2$ so that we can deal with the fewest number of 
additional points $F$ and $G$.

To make the construction, we will choose $G$
and then the pencil $\langle A_0,A_1\rangle$.
This will determine $F$. Finally,
we will choose $C'$ and $\Lcal'=\eta_* \Lcal$.

Take  $G=r_1+\ldots +r_g$ to be a set of $g= { d \choose 2}$
general points in the plane disjoint from $E$. 
Because $G$ is general, the ideal $I_G$ contains
$d$ independent forms of degree $d-1$ and no forms
of degree $d-2$. It follows from the 
Hilbert--Burch Theorem 
(\cite[20.4]{Eisenbud 1995}) $I_G$ is 
generated by the 
$d-1$ minors of a $d \times (d-1)$ matrix 
$\varphi_1: \cO_{\PP^2}(-1)^{d-1} \to \cO_{\PP^2}^d$ 
with linear entries. 

Since $I_E$ and $I_G$ are generated by forms of
degree $d-1$, and $E\cap G=\emptyset$,
 the sheaf $\cI_{E\cup G}(2d-2)$ is globally generated.
By Bertini's Theorem we may
choose a pencil
\[
A_0,A_1 \in \Hrm^0(\PP^2,\cI_{E\cup G} (2d-2))
\]
of forms of degree $a=2d-2$
vanishing simply at $E\cup G$ and at a set $F$ of $f$ points outside
$E\cup G$.

The set of points $E\cup F$ is thus geometrically
linked to $G$ by the complete intersection $(A_1,A_2)$
in the sense of \cite{Peskine and Szpiro 1974}. It was observed by
Apery and Gaeta
(see, for example, \cite[Proposition 21.24]{Eisenbud 1995}) that
the ideal $I_{E\cup F}=(A_0,A_1):I_G$ of $E\cup F$ 
is generated by
the $d\times d$ minors of the $ d \times (d+1)$ matrix 
$\varphi_2: \cO_{\PP^2}(-1)^{d-1}\oplus \cO_{\PP^2}(-d+1)^2 \to \cO^d$
obtained by bordering the matrix $\varphi_1$ 
with two columns containing the coefficients necessary
to express $A_0$ and $A_1$ as linear 
combinations of the 
generators of $I_G$.

Let $C'$ be the curve defined by a general form of degree $3d-3$
vanishing doubly along $E$ and simply along $F$. Since
this form lies in $I_{E\cup F}$, it can be expressed as a linear
combination of the minors of $\varphi_2$. Thus it is the determinant
of a matrix
$\varphi_3 : \cO_{\PP^2}(-1)^{d-1}\oplus \cO_{\PP^2}(-d+1)^2 \to \cO_{\PP^2}^{d+1}$ 
obtained from $\varphi_2$
by adding a column. Since we can think of the entries of $\varphi_3$ as
general linear forms and general elements of $(I_E)_{d-1}$,
we see that $C'$ is nonsingular away from $E$ and has only 
ordinary double points in $E$. 

We define $\Lcal'$ to be the cokernel of the
transpose of $\varphi_3$, twisted by $\Ocal_{\PP^2}(-d+1)$:
\[
0 \to \cO_{\PP^2}(-d+1)^{d+1} \rTo{\varphi_3^t(-d+1)} \cO_{\PP^2}(-d+2)^{d-1}\oplus 
\cO_{\PP^2}^2 \rTox  \cL' \to 0.
\]
Locally around a point $p_i$ of $E$ the sheaf
$\Lcal'$ is minimally
generated by two elements, since $A_0$ and $A_1$ intersect 
transversally at $p_i$. Thus $\Lcal'$ is locally isomorphic
to $\eta_*\Ocal_C$, and we must have
$\cL'=\eta_* \cL$ for some line bundle $\cL$ on $C$.

Since $\Hrm^0\Lcal=\Hrm^0\Lcal'$, the two global sections of $\Lcal'$ that are the images of
the two global sections of the middle term in the sequence above
come from global sections $\sigma_0, \sigma_1$ of $\Lcal$.
We claim that the kernel $\F$ of the map
$\Ocal_X^2\to \Lcal$ defined by these sections is an Ulrich sheaf on $X$.

As in the proof of Proposition~\ref{Ulrich on a surface} it suffices to show
that the map $(\sigma_0,\sigma_1): (\Hrm^0\Ocal_X(H+K_X))^2\to \Hrm^0\Lcal(H+K_X)$
is an isomorphism. Pushing forward and using 
\[\eta_*\Ocal_X(H+K_X)= \Ocal_{\PP^2}(d-3),\] 
we must show that the induced map 
$(\Hrm^0\Ocal_{\PP^2}(d-3))^2\to \Hrm^0\Lcal'(d-3)$
is an isomorphism.
Since the additional generators of $\sum_m \Hrm^0(\Lcal'(m))$ 
are of degree $d-2$,
this follows from the sequence defining $\Lcal'$.
\end{proof}

\begin{corollary} There exists a Pfaffian B\'ezout formula for ternary forms 
of degree $d$ with $e$ assigned base points if the ideal of the points
is generated in degree $d-1$.
\qed
\end{corollary}

\begin{furtherwork}\label{larger range} Our computations suggest that
the construction of the rank 2
Ulrich sheaf above, and hence the construction of a B\'ezout formula
for forms with base points,
works for a set of points $E$ even under the weaker hypothesis
that
 $I_E$ is generated in degree $d$. For example, if $E$ consists
of $e\le {d+2 \choose 2} -6$ general points, there 
should be plenty of room to arrive at a nodal $C'$ in the construction.
\end{furtherwork}
 
\section*{Appendix: Homomorphisms and extensions between the bundles
$\bigwedge^pU$ on the Grassmannian}
\section*{By Jerzy Weyman}

In this appendix we will prove part (b) of Proposition~\ref{Hom computation}
and also prove a complementary statement about the higher cohomology.

\begin{theorem}\label{all char hom} 
Let $\GG_l$ be the Grassmannian of codimension
$l$ planes in a vector space $W$ with dual $V=W^*$ over a field $K$
of arbitrary characteristic, and let $U$ be the
tautological
$l$-sub-bundle of $W\times \GG_l$. For $0\leq p,q,\leq l$ we have
\[
\Hom(\bigwedge^qU, \bigwedge^pU)=
\begin{cases}0,&\text{if }
p > q\\\bigwedge^{q-p}V, &\text{otherwise.}\end{cases}
\]
Moreover $\Ext^i(\bigwedge^qU, \bigwedge^pU)=0$ for $i>0$ and all $p,q$.
\end{theorem}

In 
characteristic 0 these statements follow from Borel-Weil-Bott theory
\linebreak \cite{Jantzen 1987}.

Let $\GL=\GL_K(W)$ be the
general linear group. We write $Q$ for the tautological quotient
bundle $Q=W/U$ on $\GG_l$.
If $\lambda=(\lambda_1,\dots,\lambda_v)$ is a nonincreasing 
sequence of positive
integers (a highest weight for $\GL$), then we write $L_\lambda W$ for the Schur
module corresponding to the highest weight $\lambda$. 
We may extend this notation to any nonincreasing sequence of
integers $\lambda$ (dominant integral weight)
using the formula
$L_\lambda W=L_{\mu^\prime} W \otimes (\bigwedge^v W)^{\otimes \lambda_v}$
where 
$\mu^\prime$ is the partition conjugate to $\mu = (\lambda_1
-\lambda_v ,\ldots ,\lambda_{v-1}-\lambda_v ,0)$.
The proof of Theorem~\ref{all char hom} rests on the following
facts:

\begin{lemma}\label{A1} The tensor product  
$\bigwedge^p {U}\otimes\bigwedge^q {U}^* $ has a filtration with the
associated graded object
\[\bigoplus_{a+b=p-q\atop {0\le a\le p,\,\,0\le b\le q,\,\, a+b\le l}} 
L_{(1^a , 0^{l-a-b}, (-1)^b )}{U}.\]
\end{lemma}

\begin{lemma}\label {A2} 
\begin{enumerate}
\item[(a)] If  $a>0$, then all cohomology groups of the vector bundles 
$L_{(1^a , 0^{l-a-b},(-1)^b )}{U}$ 
are zero.
\item[(b)] All higher cohomology groups of the bundle 
$L_{( 0^{l-b}, (-1)^b)}{U}$ 
are zero and
\[
H^0 (\GG_l ,L_{( 0^{l-b}, (-1)^b )}{U} )=\bigwedge^b W^*.
\]
\end{enumerate}
\end{lemma}

\begin{proof}[Proof of Lemma~\ref{A1}] 
It is a standard fact on good
filtrations \cite{Donkin 1985} that the tensor product of Schur
modules has a filtration with associated graded module being a direct sum of
Schur modules. The multiplicities of the Schur modules
occurring are the same as in characteristic zero, and we can get the result
by the Littlewood--Richardson rule, using the
isomorphism $\bigwedge^q {U}^* =\bigwedge^{l-q}{U}\otimes
\bigwedge^l {U}^*$.
\end{proof}

\begin{proof}[Proof of Lemma~\ref{A2}] 
Let $\lambda = (\lambda_1 ,\ldots
,\lambda_v )$ be an $v$-tuple of integers. Consider the full
flag variety and the tautological subbundles ${U}_i$ of rank $i$ on it.
We denote by ${\cal L}(\lambda )=
\bigotimes_{1\le i\le v} ({U}_i /{U}_{i-1})^{-\lambda_i}$ 
the line bundle on the full
flag variety ${\GL}/{B}$, where $B$ is the Borel subgroup. Then: 

\begin{lemma}\label {A3} 
\begin{enumerate}
\item[(a)] If $\lambda$ is a dominant integral weight, then the higher
cohomology groups of ${\cal L}(\lambda )$ vanish
and
\[
H^0 ({\GL}/{B} ,{\cal L}(\lambda ))= L_\lambda W.
\]
\item[(b)] Let us assume that for some $i$ we have $\lambda_i=\lambda_{i-1}
+1$. Then all cohomology groups of ${\cal
L}(\lambda )$ vanish.
\end{enumerate}
\end{lemma}

To prove part (a) of Lemma
\ref{A2}, we consider the natural projection 
$\eta: {\GL}/{B}\rightarrow \GG_l$. 
We observe that by Kempf's Vanishing Theorem in the relative
setting \linebreak \cite{Jantzen 1987}, we have
$L_{(1^a , 0^{l-a-b}, (-1)^b )}{U} =\eta_* ({\cal L}(0^{v-l},1^a ,
0^{l-a-b}, (-1)^b )) $  with higher direct images
\[{\cal R}^i\eta_* ({\cal L}(0^{v-l},1^a , 0^{l-a-b}, (-1)^b ))\] being zero
for $i>0$. Since by Lemma \ref{A3} (b) we know that all
cohomology groups of ${\cal L}(0^{v-l},1^a , 0^{l-a-b}, (-1)^b )$ are zero,
by the spectral sequence of the composition we are
done. Now part (b) of Lemma \ref{A2} follows from part (a) of Lemma \ref{A3}.
\end{proof}

\begin{proof}[Proof of Lemma~\ref{A3}] 
Part (a) is Kempf's Vanishing
Theorem  (see \cite{Kempf 1976} or \cite{Haboush 1980}). 
Part (b) follows from the following
consideration. Let $P(i)$ be a parabolic subgroup such that the
corresponding homogeneous space is a flag variety of
flags of dimensions $(1,2,\ldots ,i-1,i+1,\ldots ,v-1,v)$. The projection
$\rho :{\GL}/{ B}\rightarrow {\GL}/P(i)$
allows us to identify ${\GL}/{B}$ with the projectivization 
$\PP({U}_{i+1}/{U}_{i-1})$. The bundle 
${\cal L}(\lambda )$ is of the form 
$\rho^* ({\cal M})\otimes {\cal O}_{\PP({U}_{i+1}/{U}_{i-1})}(-1)$
because all the factors in the definition of ${\cal L}(\lambda_1 ,\ldots
,\lambda_v )$ except of the $i$-th and $i+1$-st are
induced from ${\GL}/P(i)$. Therefore by Serre's Theorem (in the
relative setting) and by the projection formula we see that
all higher direct images ${\cal R}^i\rho_* ({\cal L}(\lambda_1 ,\ldots
,\lambda_v ))$ are zero. This implies part (b).
\end{proof}

\section*{Acknowledgements}
The authors are grateful to
Hans-Christian v.~Bothmer,
Wolfram Decker,\linebreak
Joe Harris,
J\"urgen Herzog,
Michael Kapranov,
Bernd Sturmfels,
and Jerzy \linebreak Weyman
for discussions of various parts of this  material. Finally,
this paper owes \linebreak much to experiments made with the computer
algebra system Macaulay2\linebreak \cite{Grayson and Stillman 1993-- }; our
thanks to Dan Grayson and Mike Stillman
for writing it and for their support in using it for this project.



\bibliographystyle{amsalph}
\begin{thebibliography}{}

\makeatletter
\def\@biblabel#1{}
\itemindent -3.2pt


\newcommand{\bitem}[1]{\bibitem[#1]{#1}}

\bibitem[Ang\'eniol and Lejeune-Jalabert 1989]
{AngeniolandLejeune-Jalabert1989}
B.~Ang\'eniol and M.~Lejeune-Jalabert: {\it Calcul diff\'erentiel et
classes  caract\'eristiques en g\'eom\'etrie alg\'ebrique.\/}
Travaux en Cours 38, Hermann, Paris, 1989. 
\MR{90h:14004}

\bitem{Backelin and Herzog}
J.~Backelin and J.~Herzog: On Ulrich-modules over hypersurface rings.
Commutative algebra (Berkeley, CA, 1987), 63--68, Math. Sci. Res. Inst.
Publ., 15, Springer, New York, 1989. 
\MR{90i:13006}

\bitem{Beauville 2000}
A.~Beauville: Determinantal hypersurfaces. Dedicated
to William Fulton on the occasion of his 60th birthday.
Mich. Math. J. 48 (2000) 39--64.
\MR{2002b:14060}

\bitem{Beilinson 1978}
A.~Beilinson: Coherent sheaves on ${\mathbf P}^n$ and problems
of linear algebra.
Funct. Anal. and its Appl. 12 (1978) 214--216.
(Trans. from Funkz. Anal. i. Ego Priloz 12 (1978) 68--69.)
\MR{80c:14010b}

\bibitem[Bernstein, Gel'fand and Gel'fand 1978]
{BernsteinGelfandandGelfand1978}
I.N.~Bernstein, I.M.~Gel'fand and S.I.~Gel'fand:
Algebraic bundles on ${\mathbf P}^n$ and problems of linear algebra.
Funct. Anal. and its Appl. 12 (1978) 212--214.
(Trans. from Funkz. Anal. i. Ego Priloz 12 (1978) 66--67.)
\MR{80c:14010a}

\bibitem[B\'ezout 1779]
{Bezout1779}
E.~B\'ezout: {\it Th\'eorie g\'en\'erale des \'equation alg\'ebriques},
Pierres, Paris 1779. 

\bibitem[Brennan et al.\ 1987]
{Brennanetal.1987}
J.~Brennan, J.~Herzog, and B.~Ulrich: 
Maximally generated Cohen--Macaulay modules.
Math. Scand.~61 (1987) 181--203.
\MR{89j:13027}

\bitem{Buchweitz and Schreyer 2002}
R.-O.~Buchweitz and F.-O.~Schreyer: Complete intersections of two quadrics,
hyperelliptic curves and their Clifford Algebras, manuscript, 2002.

\bibitem[Buchweitz Eisenbud and Herzog 1987]
{BuchweitzEisenbudandHerzog1987}
R.-O.~Buchweitz, D.~Eisenbud and J.~Herzog:
Cohen--Macaulay modules on quadrics,  In {\it Singularities,
representation of algebras, and vector bundles (Lambrecht, 1985),\/}
Springer- Lecture Notes in Math. 1273 (1987) 96--116.
\MR{89g:13005}

\bibitem[Bus\'e et al.\ 2001]
{Buseetal.2001}
L.~Bus\'e, M.~Elkadi and B.~Mourrain:
Resultant over the residual of a complete intersection. 
J. Pure Appl. Algebra 164, No.1-2, 35-57 (2001). 
\MR{2002h:13042}

\bitem{D'Andrea 2002}
C.~D'Andrea:
Macaulay style formulas for sparse resultants.
Trans. Am. Math. Soc. 354, No.7, 2595-2629 (2002).
\MR{2003a:13032}

\bitem{D'Andrea and Dickenstein 2001}
C.~D'Andrea and A.~Dickenstein:
Explicit formulas for the multivariate resultant.
J. Pure Appl. Algebra 164, No.1-2, 59-86 (2001).
\MR{2002g:13060}

\bitem{Cayley 1848}
A.~Cayley: On the theory of elimination. Cambridge and Dublin
Mathematical J. 3 (1848) 116--120.

\bitem{Donkin 1985}
S.~Donkin: {\it Rational representations of algebraic groups.
Tensor products and filtration.}
Lecture Notes in Math.~1140.
Springer, Berlin, 1985. 
\MR{87b:20054}

\bitem{Eisenbud 1988}
D.~Eisenbud: Linear sections of determinantal varieties.
American J. Math. 110 (1988) 541--575.
\MR{89h:14041}

\bitem{Eisenbud 1995}
D.~Eisenbud: {\it Commutative Algebra with a View Toward 
Algebraic Geometry.} Springer Verlag, 1995.
\MR{97a:13001}

\bitem{Eisenbud and Goto 1984}
D.~Eisenbud and S.~Goto: Linear free resolutions and minimal
multiplicity.  J. Alg. 88 (1984) 89--133.
\MR{85f:13023}

\bibitem[Eisenbud et al.\ 2001]{Eisenbudetal.2001}
D.~Eisenbud, G.~Fl\o ystad and F.-O.~Schreyer: Sheaf cohomology and free
resolutions over the exterior algebra. Preprint (2001), 
arXiv.org/abs/math.AG/0104203.

\bitem{Fulton 1984}
W.~Fulton: Intersection Theory.
Springer, New York, 1984.
\MR{85k:14004}

\bitem{Fulton 1997}
W.~Fulton: Young tableaux : with applications to representation theory and
geometry. 
London Mathematical Society student texts 35,
Cambridge University Press 1997.
\MR{99f:05119}

\bibitem[Gel'fand et al.\ 1994]{Gelfandetal.1994}
I.~M.~Gelfand, M.~Kapranov, and A.~Zelevinsky: 
{\it Discriminants, resultants, and multidimensional determinants}.
Birkh\"auser, Boston, 1994.
\MR{95e:14045}

\bitem{Grayson and Stillman 1993-- }
D.~Grayson and M.~Stillman: Macaulay2.
Available online at
\texttt{http://www.math.uiuc.edu/\linebreak Macaulay2/}.

\bitem{Haboush 1980}
W.~Haboush:
A short proof of the Kempf vanishing theorem. 
Invent. Math. 56, 109-112 (1980).
\MR{81g:14008}

\bitem{Hanes 2000}
D.~Hanes: Special condition on maximal Cohen--Macaulay modules, and 
applications to the theory of multiplicities. Thesis, Michigan, 2000.

\bitem{Hartshorne 1977}
R.~Hartshorne: {\it Algebraic Geometry.} Springer Verlag, 1977.
\MR{57:3116}

\bitem{Hartshorne and Hirschowitz 1982}
R.~Hartshorne and A.~Hirschowitz: Cohomology of the general instanton bundle. 
Ann. scient. \'Ec. Norm. Sup. a s\'erie 15 (1982) 365--390.
\MR{84c:14011}

\bitem{Horrocks 1964}
G.~Horrocks:
Vector bundles on the punctured spectrum of a local ring.
Proc. Lond. Math. Soc., III. Ser. 14, 689--713 (1964). 
\MR{30:120}

\bitem{Horrocks and Mumford 1973}
G.~Horrocks, D.~Mumford: A rank $2$ vector bundle on ${\mathbf P}^4$ with 
$15,000$
symmetries. Topology 12 (1973) 63--81.
\MR{52:3164}

\bitem{Iverson 1986}
B.~Iverson: {\it Cohomology of sheaves.\/}
Springer-Verlag, New York, 1986.

\bitem{Jantzen 1987}
J.~C.~Jantzen:
{\it Representations of algebraic groups.\/}
Pure and Applied Mathematics, 131, 
Academic Press, Boston, MA, 1987. 
\MR{89c:20001}

\bitem{Jouanolou 1995}
J.-P.~Jouanolou: Aspects invariants de l'\'elimination.
Adv. Math. 114 (1995) 1--174.
\MR{96m:14001}

\bitem{Kempf 1976}
G.~R.~Kempf:
Linear systems on homogeneous spaces.
Ann. of Math., II. Ser. 103, 557-591 (1976).
\MR{53:13229}

\bitem{Khetan 2002}
A.~Khetan:
Determinantal Formula for the Chow Form of a Toric Surface, 
In {\it ISSAC Proceedings}, 145--150, ACM 2002.

\bitem{Kline 1972}
M.~Kline: {\it Mathematical thought from ancient to modern times}.
Oxford University Press, New York 1972.
\MR{57:12010}

\bitem{Knudsen and Mumford 1976}
F.~Knudsen and D.~Mumford: The projectivity of the moduli
space of stable curves I: Preliminaries on ``det'' and ``div''.
Math. Scand.~39 (1976) 19--55.
\MR{55:10465}

\bitem{Leibniz 1693}
G.~W.~Leibniz: {\it Mathematische Schriften}, herausgegeben von C.~I.~Gerhardt,
Letter to l'Hospital 28 April 1693, volume II, p. 239; 
Vorwort volume VII, pp. 5.
Georg Olms Verlag Hildesheim, New York 1971.
\MR{25:4979a}, \MR{25:4979g}

\bitem{Okonek et al. 1980}
Ch. Okonek, M. Schneider, and H. Spindler: {\it Vector Bundles on
Complex Projective Spaces.} Birkh\"auser, Boston 1980.
\MR{81b:14001}

\bitem{Peskine and Szpiro 1974}
C.~Peskine and L.~Szpiro: Liaison des vari\'et\'es alg\'ebriques I. 
Invent. Math. 26 (1974) 271--302.
\MR{51:526}

\bitem{Stanley 1971}
R.~Stanley: Theory and application of plane partitions. Studies in
Appl. Math. 1 (1971) 167--187 and 259--279.
\MR{48:3754}

\bitem{Swan 1985}
R.~G.~Swan: $K$-theory of quadric hypersurfaces. Ann. of Math.
122 (1985) 113--153.
\MR{87g:14006}

\bitem{Sylvester 1840--1842}
J.~J.~Sylvester: A method of determining by mere inspection the derivatives from two 
equations of any degree. Philosophical Magazine XVI (1840) 132--135.
Memoir on the dialytic method of elimination. Part I. 
Philsophical Magazine XXI (1842) 534--539. Reprinted in:
{\it The collected Mathematical Papers of James Joseph Sylvester}, Vol.~I.
Chelsea New York 1973, reprint of the Cambridge 1904 edition.

\bitem{Tikhomirov 1997}
A.~A.~Tikhomirov: The main component of the moduli space of 
mathematical instanton vector bundles on ${\mathbf P}^3$.
Journal of the Mathematical Sciences 86 (1997) 3004--3087.
(Translation from the Russian of {\it Itogi Nauki i Tkhniki.
Seriya Sovremennaya Matematika i Ee Prilozheniya. Tematischeskie Obzory},
Vol 25, Algebraic Geometry-2, 1995.)
\MR{99e:14012}

\bitem{Ulrich 1984}
B. Ulrich: Gorenstein rings and modules with high numbers of generators.
Math. Z.~188 (1984) 23--32.
\MR{85m:13021}

\bitem{Vinnikov 1989}
V.~Vinnikov: Complete description of determinantal representations of
smooth irreducible curves. Linear Algebra Appl. 125 (1989), 103--140.
\MR{90m:14028}

\bitem{Weyman and Zelevinsky 1994}
J.~Weyman and A.~Zelevinsky: Determinantal formulas for multigraded resultants.
J.~Alg. Geom.~3 (1994) 569--598.
\MR{95j:14074}

\end{thebibliography}

\end{document}



\ProvidesPackage{url}[1996/01/20]

\edef\atcatcode{\the\catcode`\@}
\catcode`@=11 

\expandafter\ifx\csname urlstyle\endcsname\relax \let\urlstyle\relax \fi
\expandafter\ifx\csname homedir\endcsname\relax 
        \def\homedir{\kern-.1em\lower.9ex\hbox{\char`\~}}\fi
\newif\ifurlragged
\newif\ifurlhyphens

\expandafter\ifx\csname @ifnextchar\endcsname\relax 
\let\@colon\:  
\def\@ifnextchar#1#2#3{\let\@tempe #1\def\@tempa{#2}\def\@tempb{#3}\futurelet
    \@tempc\@ifnch}
\def\@ifnch{\ifx \@tempc \@sptoken \let\@tempd\@xifnch
      \else \ifx \@tempc \@tempe\let\@tempd\@tempa\else\let\@tempd\@tempb\fi
      \fi \@tempd}
\def\:{\let\@sptoken= } \:  
\def\:{\@xifnch} \expandafter\def\: {\futurelet\@tempc\@ifnch}
\let\:\@colon
\fi

\def\@inextremis{\nobreak\hskip\z@ plus .0001fil
        \penalty9999 \hskip\z@ plus -.0001fil }

\def\@lessextreme{\nobreak\hskip\z@ plus .0001fil
        \penalty1000 \hskip\z@ plus -.0001fil }

{\catcode`\/=\active
\gdef/{\@ifnextchar/{\char`\/\kern-.2em}{\char`\/\egroup 
\ifurlragged\@inextremis\else\hskip\z@\fi                
\ifurlhyphens\unhbox0\else\box0\fi      
\setbox0\hbox\bgroup}}}                 

\def\url{\ifurlragged\@lessextreme\fi
\bgroup 
\catcode`\@=12
\catcode`\%=11 \catcode`\_=11 \catcode`\^=11 \catcode`\$=11 \catcode`\#=11 
\catcode`\~=\active \let~\homedir \catcode`\/=\active \processurl}

\def\processurl#1{\urlstyle{\setbox0\hbox{#1}
\ifurlhyphens\unhbox0\else\ifdim\wd0=\z@\else\box0\fi\fi 
\ifurlragged\else\unskip\fi}
\egroup} 

\catcode`@=\atcatcode


\message{<Paul Taylor's commutative diagrams, version 3.88, September 2000>}%







\ifx\diagram\undefined\else\message{WARNING: the \string\diagram\space command
is already defined and will not be loaded again}\expandafter\endinput\fi

\edef\cdrestoreat{
\noexpand\catcode\lq\noexpand\@=\the\catcode\lq\@
\noexpand\catcode\lq\noexpand\#=\the\catcode\lq\#
\noexpand\catcode\lq\noexpand\$=\the\catcode\lq\$
\noexpand\catcode\lq\noexpand\<=\the\catcode\lq\<
\noexpand\catcode\lq\noexpand\>=\the\catcode\lq\>
\noexpand\catcode\lq\noexpand\:=\the\catcode\lq\:
\noexpand\catcode\lq\noexpand\;=\the\catcode\lq\;
\noexpand\catcode\lq\noexpand\!=\the\catcode\lq\!
\noexpand\catcode\lq\noexpand\?=\the\catcode\lq\?
\noexpand\catcode\lq\noexpand\+=\the\catcode\rq53%
}\catcode\lq\@=11 \catcode\lq\#=6 \catcode\lq\<=12 \catcode\lq\>=12 \catcode
\rq53=12 \catcode\lq\:=12 \catcode\lq\;=12 \catcode\lq\!=12 \catcode\lq\?=12

\ifx\diagram@help@messages\undefined\let\diagram@help@messages y\fi

\def\cdps@Rokicki#1{\special{ps:#1}}\let\cdps@dvips\cdps@Rokicki\let
\cdps@RadicalEye\cdps@Rokicki\let\CD@IK\cdps@Rokicki\let\CD@HB\cdps@Rokicki
\def\cdps@Bechtolsheim#1{\special{dvitps: Literal "#1"}}%
\let\cdps@dvitps\cdps@Bechtolsheim\let\cdps@IntegratedComputerSystems
\cdps@Bechtolsheim
\def\cdps@Clark#1{\special{dvitops: inline #1}}
\let\cdps@dvitops\cdps@Clark
\let\cdps@OzTeX\empty\let\cdps@oztex\empty\let\cdps@Trevorrow\empty
\def\cdps@Coombes#1{\special{ps-string #1}}

\count@=\year\multiply\count@12 \advance\count@\month
\ifnum\count@>24068 
\message{***********************************************************}
\message{! YOU HAVE AN OUT OF DATE VERSION OF COMMUTATIVE DIAGRAMS! *}
\message{! it expired in August 2005 and is time-bombed for January *}
\message{! You may get an up to date version of this package from *}
\message{! ftp://ftp.dcs.qmw.ac.uk/pub/tex/contrib/pt/diagrams/ *}
\message{***********************************************************}
\ifnum\count@>24071 
\errhelp{You may press RETURN and carry on for the time being.}\message{! It
is embarrassing to see papers in conference proceedings}\message{! and
journals containing bugs which I had fixed years before.}\message{! It is easy
to obtain and install a new version, which will}\errmessage{! remain
compatible with your files. Please get it NOW.}\fi\fi

\def\CD@DE{\global\let}\def\CD@RH{\outer\def}

{\escapechar\m@ne\xdef\CD@o{\string\{}\xdef\CD@yC{\string\}}
\catcode\lq\&=4 \CD@DE\CD@Q=&\xdef\CD@S{\string\&}
\catcode\lq\$=3 \CD@DE\CD@k=$\CD@DE\CD@ND=$
\xdef\CD@nC{\string\$}\gdef\CD@LG{$$}
\catcode\lq\_=8 \CD@DE\CD@lJ=_
\obeylines\catcode\lq\^=7 \CD@DE\@super=^
\ifnum\newlinechar=10 \gdef\CD@uG{^^J}
\else\ifnum\newlinechar=13 \gdef\CD@uG{^^M}
\else\CD@DE\CD@uG\space\expandafter\message{! input error: \noexpand
\newlinechar\space is ASCII \the\newlinechar, not LF=10 or CR=13.}
\fi\fi}

\mathchardef\lessthan=\rq30474 \mathchardef\greaterthan=\rq30476

\ifx\tenln\undefined
\font\tenln=line10\relax
\fi\ifx\tenlnw\undefined\ifx\tenln\nullfont\let\tenlnw\nullfont\else
\font\tenlnw=linew10\relax
\fi\fi

\ifx\inputlineno\undefined\csname newcount\endcsname\inputlineno\inputlineno
\m@ne\message{***************************************************}\message{!
Obsolete TeX (version 2). You should upgrade to *}\message{! version 3, which
has been available since 1990. *}\message{***********************************%
****************}\fi

\def\cd@shouldnt#1{\CD@KB{* THIS (#1) SHOULD NEVER HAPPEN! *}}

\def\get@round@pair#1(#2,#3){#1{#2}{#3}}
\def\get@square@arg#1[#2]{#1{#2}}
\def\CD@AE#1{\CD@PK\let\CD@DH\CD@@E\CD@@E#1,],}
\def\CD@m{[}\def\CD@RD{]}\def\commdiag#1{{\let\enddiagram\relax\diagram[]#1%
\enddiagram}}

\def\CD@BF{{\ifx\CD@EH[\aftergroup\get@square@arg\aftergroup\CD@YH\else
\aftergroup\CD@JH\fi}}
\def\CD@CF#1#2{\def\CD@YH{#1}\def\CD@JH{#2}\futurelet\CD@EH\CD@BF}

\def\CD@KK{|}

\def\CD@PB{
\tokcase\CD@DD:\CD@y\break@args;\catcase\@super:\upper@label;\catcase\CD@lJ:%
\lower@label;\tokcase{~}:\middle@label;
\tokcase<:\CD@iF;
\tokcase>:\CD@iI;
\tokcase(:\CD@BC;
\tokcase[:\optional@;
\tokcase.:\CD@JJ;
\catcase\space:\eat@space;\catcase\bgroup:\positional@;\default:\CD@@A
\break@args;\endswitch}

\def\switch@arg{
\catcase\@super:\upper@label;\catcase\CD@lJ:\lower@label;\tokcase[:\optional@
;
\tokcase.:\CD@JJ;
\catcase\space:\eat@space;\catcase\bgroup:\positional@;\tokcase{~}:%
\middle@label;
\default:\CD@y\break@args;\endswitch}


\let\CD@tJ\relax\ifx\protect\undefined\let\protect\relax\fi\ifx\AtEndDocument
\undefined\def\CD@PG{\CD@gB}\def\CD@GF#1#2{}\else\def\CD@PG#1{\edef\CD@CH{#1}%
\expandafter\CD@oC\CD@CH\CD@OD}\def\CD@oC#1\CD@OD{\AtEndDocument{\typeout{%
\CD@tA: #1}}}\def\CD@GF#1#2{\gdef#1{#2}\AtEndDocument{#1}}\fi\def\CD@ZA#1#2{%
\def#1{\CD@PG{#2\CD@mD\CD@W}\CD@DE#1\relax}}\def\CD@uF#1\repeat{\def\CD@p{#1}%
\CD@OF}\def\CD@OF{\CD@p\relax\expandafter\CD@OF\fi}\def\CD@sF#1\repeat{\newcommand{\CD@q{#1}\CD@PF}\def\CD@PF{\CD@q\relax\expandafter\CD@PF\fi}\def\CD@tF#1%
\repeat{\def\CD@r{#1}\CD@QF}\def\CD@QF{\CD@r\relax\expandafter\CD@QF\fi}\def
\CD@tG#1#2#3{\def#2{\let#1\iftrue}\def#3{\let#1\iffalse}#3}\if y%
\diagram@help@messages\def\CD@rG#1#2{\csname newtoks\endcsname#1#1=%
\expandafter{\csname#2\endcsname}}\else\csname newtoks\endcsname\no@cd@help
\no@cd@help{See the manual}\def\CD@rG#1#2{\let#1\no@cd@help}\fi\chardef\CD@lF
=1 \chardef\CD@lI=2 \chardef\CD@MH=5 \chardef\CD@tH=6 \chardef\CD@sH=7
\chardef\CD@PC=9 \dimendef\CD@hI=2 \dimendef\CD@hF=3 \dimendef\CD@mF=4
\dimendef\CD@mI=5 \dimendef\CD@wJ=6 \dimendef\CD@tI=8 \dimendef\CD@sI=9
\skipdef\CD@uB=1 \skipdef\CD@NF=2 \skipdef\CD@tB=3 \skipdef\CD@ZE=4 \skipdef
\CD@JK=5 \skipdef\CD@kI=6 \skipdef\CD@kF=7 \skipdef\CD@qI=8 \skipdef\CD@pI=9
\countdef\CD@JC=9 \countdef\CD@gD=8 \countdef\CD@A=7 \def\sdef#1#2{\def#1{#2}%
}\def\CD@L#1{\expandafter\aftergroup\csname#1\endcsname}\def\CD@RC#1{%
\expandafter\def\csname#1\endcsname}\def\CD@sD#1{\expandafter\gdef\csname#1%
\endcsname}\def\CD@vC#1{\expandafter\edef\csname#1\endcsname}\def\CD@nF#1#2{%
\expandafter\let\csname#1\expandafter\endcsname\csname#2\endcsname}\def\CD@EE
#1#2{\expandafter\CD@DE\csname#1\expandafter\endcsname\csname#2\endcsname}%
\def\CD@AK#1{\csname#1\endcsname}\def\CD@XJ#1{\expandafter\show\csname#1%
\endcsname}\def\CD@ZJ#1{\expandafter\showthe\csname#1\endcsname}\def\CD@WJ#1{%
\expandafter\showbox\csname#1\endcsname}\def\CD@tA{Commutative Diagram}\edef
\CD@kH{\string\par}\edef\CD@dC{\string\diagram}\edef\CD@HD{\string\enddiagram
}\edef\CD@EC{\string\\}\def\CD@eF{LaTeX}\ifx\@ignoretrue\undefined
\expandafter\CD@tG\csname if@ignore\endcsname\ignore@true\ignore@false\def
\@ignoretrue{\global\ignore@true}\def\@ignorefalse{\global\ignore@false}\fi

\def\CD@g{{\ifnum0=\lq}\fi}\def\CD@wC{\ifnum0=\lq{\fi}}\def\catcase#1:{\ifcat
\noexpand\CD@EH#1\CD@tJ\expandafter\CD@kC\else\expandafter\CD@dJ\fi}\def
\tokcase#1:{\ifx\CD@EH#1\CD@tJ\expandafter\CD@kC\else\expandafter\CD@dJ\fi}%
\def\CD@kC#1;#2\endswitch{#1}\def\CD@dJ#1;{}\let\endswitch\relax\def\default:%
#1;#2\endswitch{#1}\ifx\at@\undefined\def\at@{@}\fi\edef\CD@P{\CD@o pt\CD@yC}%
\CD@RC{\CD@P>}#1>#2>{\CD@z\rTo\sp{#1}\sb{#2}\CD@z}\CD@RC{\CD@P<}#1<#2<{\CD@z
\lTo\sp{#1}\sb{#2}\CD@z}\CD@RC{\CD@P)}#1)#2){\CD@z\rTo\sp{#1}\sb{#2}\CD@z}%
\CD@RC{\CD@P(}#1(#2({\CD@z\lTo\sp{#1}\sb{#2}\CD@z}
\def\CD@O{\def\endCD{\enddiagram}\CD@RC{\CD@P A}##1A##2A{\uTo<{##1}>{##2}%
\CD@z\CD@z}\CD@RC{\CD@P V}##1V##2V{\dTo<{##1}>{##2}\CD@z\CD@z}\CD@RC{\CD@P=}{%
\CD@z\hEq\CD@z}\CD@RC{\CD@P\CD@KK}{\vEq\CD@z\CD@z}\CD@RC{\CD@P\string\vert}{%
\vEq\CD@z\CD@z}\CD@RC{\CD@P.}{\CD@z\CD@z}\let\CD@z\CD@Q}\def\CD@IE{\let\tmp
\CD@JE\ifcat A\noexpand\CD@CH\else\ifcat=\noexpand\CD@CH\else\ifcat\relax
\noexpand\CD@CH\else\let\tmp\at@\fi\fi\fi\tmp}\def\CD@JE#1{\CD@nF{tmp}{\CD@P
\string#1}\ifx\tmp\relax\def\tmp{\at@#1}\fi\tmp}\def\CD@z{}\begingroup
\aftergroup\def\aftergroup\CD@T\aftergroup{\aftergroup\def\catcode\lq\@%
\active\aftergroup @\endgroup{\futurelet\CD@CH\CD@IE}}\newcount\CD@uA
\newcount\CD@vA\newcount\CD@wA\newcount\CD@xA\newdimen\CD@OA\newdimen\CD@PA
\CD@tG\CD@gE\CD@@A\CD@y\CD@tG\CD@hE\CD@EA\CD@BA\newdimen\CD@RA\newdimen\CD@SA
\newcount\CD@yA\newcount\CD@zA\newdimen\CD@QA\newbox\CD@DA\CD@tG\CD@lE\CD@dA
\CD@bA\newcount\CD@LH\newcount\CD@TC\def\CD@V#1#2{\ifdim#1<#2\relax#1=#2%
\relax\fi}\def\CD@X#1#2{\ifdim#1>#2\relax#1=#2\relax\fi}\newdimen\CD@XH\CD@XH
=1sp \newdimen\CD@zC\CD@zC\z@\def\CD@cJ{\ifdim\CD@zC=1em\else\CD@nJ\fi}\def
\CD@nJ{\CD@zC1em\def\CD@NC{\fontdimen8\textfont3 }\CD@@J\CD@NJ\setbox0=\vbox{%
\CD@t\noindent\CD@k\null\penalty-9993\null\CD@ND\null\endgraf\setbox0=%
\lastbox\unskip\unpenalty\setbox1=\lastbox\global\setbox\CD@IG=\hbox{\unhbox0%
\unskip\unskip\unpenalty\setbox0=\lastbox}\global\setbox\CD@KG=\hbox{\unhbox1%
\unskip\unpenalty\setbox1=\lastbox}}}\newdimen\CD@@I\CD@@I=1true in \divide
\CD@@I300 \def\CD@zH#1{\multiply#1\tw@\advance#1\ifnum#1<\z@-\else+\fi\CD@@I
\divide#1\tw@\divide#1\CD@@I\multiply#1\CD@@I}\def\MapBreadth{%
\afterassignment\CD@gI\CD@LF}\newdimen\CD@LF\newdimen\CD@oI\def\CD@gI{\CD@oI
\CD@LF\CD@V\CD@@I{4\CD@XH}\CD@X\CD@@I\p@\CD@zH\CD@oI\ifdim\CD@LF>\z@\CD@V
\CD@oI\CD@@I\fi\CD@cJ}\def\CD@RJ#1{\CD@zD\count@\CD@@I#1\ifnum\count@>\z@
\divide\CD@@I\count@\fi\CD@gI\CD@NJ}\def\CD@NJ{\dimen@\CD@QC\count@\dimen@
\divide\count@5\divide\count@\CD@@I\edef\CD@OC{\the\count@}}\def\CD@AJ{\CD@QJ
\z@}\def\CD@QJ#1{\CD@tI\axisheight\advance\CD@tI#1\relax\advance\CD@tI-.5%
\CD@oI\CD@zH\CD@tI\CD@sI-\CD@tI\advance\CD@tI\CD@LF}\newdimen\CD@DC\CD@DC\z@
\newdimen\CD@eJ\CD@eJ\z@\def\CD@CJ#1{\CD@sI#1\relax\CD@tI\CD@sI\advance\CD@tI
\CD@LF\relax}\def\horizhtdp{height\CD@tI depth\CD@sI}\def\axisheight{%
\fontdimen22\the\textfont\tw@}\def\script@axisheight{\fontdimen22\the
\scriptfont\tw@}\def\ss@axisheight{\fontdimen22\the\scriptscriptfont\tw@}\def
\CD@NC{0.4pt}\def\CD@VK{\fontdimen3\textfont\z@}\def\CD@UK{\fontdimen3%
\textfont\z@}\newdimen\PileSpacing\newdimen\CD@nA\CD@nA\z@\def\CD@RG{%
\ifincommdiag1.3em\else2em\fi}\newdimen\CD@YB\def\CellSize{\afterassignment
\CD@kB\DiagramCellHeight}\newdimen\DiagramCellHeight\DiagramCellHeight-%
\maxdimen\newdimen\DiagramCellWidth\DiagramCellWidth-\maxdimen\def\CD@kB{%
\DiagramCellWidth\DiagramCellHeight}\def\CD@QC{3em}\newdimen\MapShortFall\def
\MapsAbut{\MapShortFall\z@\objectheight\z@\objectwidth\z@}\newdimen\CD@iA
\CD@iA\z@\def\newarrowhead{\CD@mG h\CD@BG\CD@GG>}\def\newarrowtail{\CD@mG t%
\CD@BG\CD@GG>}\def\newarrowmiddle{\CD@mG m\CD@BG\hbox@maths\empty}\def
\newarrowfiller{\CD@mG f\CD@bE\CD@MK-}\def\CD@mG#1#2#3#4#5#6#7#8#9{\CD@RC{r#1%
:#5}{#2{#6}}\CD@RC{l#1:#5}{#2{#7}}\CD@RC{d#1:#5}{#3{#8}}\CD@RC{u#1:#5}{#3{#9}%
}\CD@vC{-#1:#5}{\expandafter\noexpand\csname-#1:#4\endcsname\noexpand\CD@MC}%
\CD@vC{+#1:#5}{\expandafter\noexpand\csname+#1:#4\endcsname\noexpand\CD@MC}}%
\CD@ZA\CD@MC{\CD@eF\space diagonals are used unless PostScript is set}\def
\defaultarrowhead#1{\edef\CD@sJ{#1}\CD@@J}\def\CD@@J{\CD@IJ\CD@sJ<>ht\CD@IJ
\CD@sJ<>th}\def\CD@IJ#1#2#3#4#5{\CD@HJ{r#4}{#3}{l#5}{#2}{r#4:#1}\CD@HJ{r#5}{#%
2}{l#4}{#3}{l#4:#1}\CD@HJ{d#4}{#3}{u#5}{#2}{d#4:#1}\CD@HJ{d#5}{#2}{u#4}{#3}{u%
#4:#1}}\def\CD@HJ#1#2#3#4#5{\begingroup\aftergroup\CD@GJ\CD@L{#1+:#2}\CD@L{#1%
:#2}\CD@L{#3:#4}\CD@L{#5}\endgroup}\def\CD@GJ#1#2#3#4{\csname newbox%
\endcsname#1\def#2{\copy#1}\def#3{\copy#1}\setbox#1=\box\voidb@x}\def\CD@sJ{}%
\CD@@J\def\CD@GJ#1#2#3#4{\setbox#1=#4}\ifx\tenln\nullfont\def\CD@sJ{vee}\else
\let\CD@sJ\CD@eF\fi\def\CD@xF#1#2#3{\begingroup\aftergroup\CD@wF\CD@L{#1#2:#3%
#3}\CD@L{#1#2:#3}\aftergroup\CD@yF\CD@L{#1#2:#3-#3}\CD@L{#1#2:#3}\endgroup}%
\def\CD@wF#1#2{\def#1{\hbox{\rlap{#2}\kern.4\CD@zC#2}}}\def\CD@yF#1#2{\def#1{%
\hbox{\rlap{#2}\kern.4\CD@zC#2\kern-.4\CD@zC}}}\CD@xF lh>\CD@xF rt>\CD@xF rh<%
\CD@xF rt<\def\CD@yF#1#2{\def#1{\hbox{\kern-.4\CD@zC\rlap{#2}\kern.4\CD@zC#2}%
}}\CD@xF rh>\CD@xF lh<\CD@xF lt>\CD@xF lt<\def\CD@wF#1#2{\def#1{\vbox{\vbox to%
\z@{#2\vss}\nointerlineskip\kern.4\CD@zC#2}}}\def\CD@yF#1#2{\def#1{\vbox{%
\vbox to\z@{#2\vss}\nointerlineskip\kern.4\CD@zC#2\kern-.4\CD@zC}}}\CD@xF uh>%
\CD@xF dt>\CD@xF dh<\CD@xF dt<\def\CD@yF#1#2{\def#1{\vbox{\kern-.4\CD@zC\vbox
to\z@{#2\vss}\nointerlineskip\kern.4\CD@zC#2}}}\CD@xF dh>\CD@xF ut>\CD@xF uh<%
\CD@xF ut<\def\CD@BG#1{\hbox{\mathsurround\z@\offinterlineskip\CD@k\mkern-1.5%
mu{#1}\mkern-1.5mu\CD@ND}}\def\hbox@maths#1{\hbox{\CD@k#1\CD@ND}}\def\CD@GG#1%
}\hbox to\CD@LF{\setbox0=\hbox{\offinterlineskip\mathsurround\z@\CD@k{#1}%
\CD@ND}\dimen0.5\wd0\advance\dimen0-.5\CD@oI\CD@zH{\dimen0}\kern-\dimen0%
\unhbox0\hss}}\def\CD@sB#1{\hbox to2\CD@LF{\hss\offinterlineskip\mathsurround
\z@\CD@k{#1}\CD@ND\hss}}\def\CD@vF#1{\hbox{\mathsurround\z@\CD@k{#1}\CD@ND}}%
\def\CD@bE#1{\hbox{\kern-.15\CD@zC\CD@k{#1}\CD@ND\kern-.15\CD@zC}}\def\CD@MK#%
1{\vbox{\offinterlineskip\kern-.2ex\CD@GG{#1}\kern-.2ex}}\def\@fillh{%
\xleaders\vrule\horizhtdp}\def\@fillv{\xleaders\hrule width\CD@LF}\CD@nF{rf:-%
}{@fillh}\CD@nF{lf:-}{@fillh}\CD@nF{df:-}{@fillv}\CD@nF{uf:-}{@fillv}\CD@nF{%
rh:}{null}\CD@nF{rm:}{null}\CD@nF{rt:}{null}\CD@nF{lh:}{null}\CD@nF{lm:}{null%
}\CD@nF{lt:}{null}\CD@nF{dh:}{null}\CD@nF{dm:}{null}\CD@nF{dt:}{null}\CD@nF{%
uh:}{null}\CD@nF{um:}{null}\CD@nF{ut:}{null}\CD@nF{+h:}{null}\CD@nF{+m:}{null%
}\CD@nF{+t:}{null}\CD@nF{-h:}{null}\CD@nF{-m:}{null}\CD@nF{-t:}{null}\def
\CD@@D{\hbox{\vrule height 1pt depth-1pt width 1pt}}\CD@RC{rf:}{\CD@@D}\CD@nF
{lf:}{rf:}\CD@nF{+f:}{rf:}\CD@RC{df:}{\CD@@D}\CD@nF{uf:}{df:}\CD@nF{-f:}{df:}%
\def\CD@BD{\CD@U\null\CD@@D\null\CD@@D\null}\edef\CD@lG{\string\newarrow}\def
\newarrow#1#2#3#4#5#6{\begingroup\edef\@name{#1}\edef\CD@oJ{#2}\edef\CD@iD{#3%
}\edef\CD@QG{#4}\edef\CD@jD{#5}\edef\CD@LE{#6}\let\CD@HE\CD@sG\let\CD@FK
\CD@BH\let\@x\CD@AH\ifx\CD@oJ\CD@iD\let\CD@oJ\empty\fi\ifx\CD@LE\CD@jD\let
\CD@LE\empty\fi\def\CD@LI{r}\def\CD@SF{l}\def\CD@IC{d}\def\CD@yJ{u}\def\CD@gH
{+}\def\@m{-}\ifx\CD@iD\CD@jD\ifx\CD@QG\CD@iD\let\CD@QG\empty\fi\ifx\CD@LE
\empty\ifx\CD@iD\CD@aE\let\@x\CD@yG\else\let\@x\CD@zG\fi\fi\else\edef\CD@a{%
\CD@iD\CD@oJ}\ifx\CD@a\empty\ifx\CD@QG\CD@jD\let\CD@QG\empty\fi\fi\fi\ifmmode
\aftergroup\CD@kG\else\CD@@A\CD@oB rh{head\space\space}\CD@LE\CD@oB rf{filler%
}\CD@iD\CD@oB rm{middle}\CD@QG\ifx\CD@jD\CD@iD\else\CD@oB rf{filler}\CD@jD\fi
\CD@oB rt{tail\space\space}\CD@oJ\CD@gE\CD@HE\CD@FK\@x\CD@nG l-2+2{lu}{nw}%
\NorthWest\CD@nG r+2+2{ru}{ne}\NorthEast\CD@nG l-2-2{ld}{sw}\SouthWest\CD@nG r%
+2-2{rd}{se}\SouthEast\else\aftergroup\CD@b\CD@L{r\@name}\fi\fi\endgroup}\def
\CD@sG{\CD@vG\CD@LI\CD@SF rl\Horizontal@Map}\def\CD@BH{\CD@vG\CD@IC\CD@yJ du%
\Vertical@Map}\def\CD@AH{\CD@vG\CD@gH\@m+-\Vector@Map}\def\CD@yG{\CD@vG\CD@gH
\@m+-\Slant@Map}\def\CD@zG{\CD@vG\CD@gH\@m+-\Slope@Map}\catcode\lq\/=\active
\def\CD@vG#1#2#3#4#5{\CD@jG#1#3#5t:\CD@oJ/f:\CD@iD/m:\CD@QG/f:\CD@jD/h:\CD@LE
//\CD@jG#2#4#5h:\CD@LE/f:\CD@jD/m:\CD@QG/f:\CD@iD/t:\CD@oJ//}\def\CD@jG#1#2#3%
#4//{\edef\CD@fG{#2}\aftergroup\sdef\CD@L{#1\@name}\aftergroup{\aftergroup#3%
\CD@M#4//\aftergroup}}\def\CD@M#1/{\edef\CD@EH{#1}\ifx\CD@EH\empty\else\CD@L{%
\CD@fG#1}\expandafter\CD@M\fi}\catcode\lq\/=12 \def\CD@nG#1#2#3#4#5#6#7#8{%
\aftergroup\sdef\CD@L{#6\@name}\aftergroup{\CD@L{#2\@name}\if#2#4\aftergroup
\CD@CI\else\aftergroup\CD@BI\fi\CD@L{#1\@name}%
\aftergroup(\aftergroup#3\aftergroup,\aftergroup#5\aftergroup)\aftergroup}}%
\def\CD@oB#1#2#3#4{\expandafter\ifx\csname#1#2:#4\endcsname\relax\CD@y\CD@gB{%
arrow#3 "#4" undefined}\fi}\CD@rG\CD@VE{All five components must be defined
before an arrow.}\CD@rG\CD@SE{\CD@lG, unlike \string\HorizontalMap, is a
declaration.}\def\CD@b#1{\CD@YA{Arrows \string#1 etc could not be defined}%
\CD@VE}\def\CD@kG{\CD@YA{misplaced \CD@lG}\CD@SE}\def\newdiagramgrid#1#2#3{%
\CD@RC{cdgh@#1}{#2,],}
\CD@RC{cdgv@#1}{#3,],}}
\CD@tG\ifincommdiag\incommdiagtrue\incommdiagfalse\CD@tG\CD@@F\CD@IF\CD@HF
\newcount\CD@VA\CD@VA=0 \def\CD@yH{\CD@VA6 }\def\CD@OB{\CD@VA1 \global\CD@yA1
\CD@DE\CD@YF\empty}\def\CD@YF{}\def\CD@nB#1{\relax\CD@MD\edef\CD@vJ{#1}%
\begingroup\CD@rE\else\ifcase\CD@VA\ifmmode\else\CD@YG\CD@E0\fi\or\CD@cE5\or
\CD@YG\CD@F5\or\CD@YG\CD@B5\or\CD@YG\CD@B5\or\CD@YG\CD@C5\or\CD@cE7\or\CD@YG
\CD@D7\fi\fi\endgroup\xdef\CD@YF{#1}}\def\CD@pB#1#2#3#4#5{\relax\CD@MD\xdef
\CD@vJ{#4}\begingroup\ifnum\CD@VA<#1 \expandafter\CD@cE\ifcase\CD@VA0\or#2\or
#3\else#2\fi\else\ifnum\CD@VA<6 \CD@tJ\CD@YG\CD@B#2\else\CD@YG\CD@G#2\fi\fi
\endgroup\CD@DE\CD@YF\CD@vJ\ifincommdiag\let\CD@ZD#5\else\let\CD@ZD\CD@LK\fi}%
\def\CD@yI{\global\CD@yA=\ifnum\CD@VA<5 1\else2\fi\relax}\def\CD@OI{\CD@VA
\CD@yA}\def\CD@cE#1{\aftergroup\CD@VA\aftergroup#1\aftergroup\relax}\def
\CD@HH{\def\CD@nB##1{\relax}\let\CD@pB\CD@FH\let\CD@yH\relax\let\CD@OB\relax
\let\CD@yI\relax\let\CD@OI\relax}\def\CD@FH#1#2#3#4#5{\ifincommdiag\let\CD@ZD
#5\else\xdef\CD@vJ{#4}\let\CD@ZD\CD@LK\fi}\def\CD@YG#1{\aftergroup#1%
\aftergroup\relax\CD@cE}\def\CD@B{\CD@YE\CD@S\CD@ME\CD@Q}\def\CD@G{\CD@YE{%
\CD@yC\CD@S}\CD@XE\CD@QD\CD@Q}\def\CD@F{\CD@YE{*\CD@S}\CD@RE\clubsuit\CD@Q}%
\def\CD@C{\CD@YE{\CD@S*\CD@S}\CD@RE\CD@Q\clubsuit\CD@Q}\def\CD@D{\CD@YE\CD@EC
\CD@TE\\}\def\CD@E{\CD@YE\CD@nC\CD@QE\CD@k}\def\CD@LK{\CD@YA{\CD@vJ\space
ignored \CD@dH}\CD@WE}\def\CD@FE{}\def\CD@d{\CD@YA{maps must never be enclosed
in braces}\CD@OE}\def\CD@dH{outside diagram}\def\CD@FC{\string\HonV, \string
\VonH\space and \string\HmeetV}\CD@rG\CD@ME{The way that horizontal and
vertical arrows are terminated implicitly means\CD@uG that they cannot be
mixed with each other or with \CD@FC.}\CD@rG\CD@XE{\string\pile\space is for
parallel horizontal arrows; verticals can just be put together in\CD@uG a cell%
. \CD@FC\space are not meaningful in a \string\pile.}\CD@rG\CD@RE{The
horizontal maps must point to an object, not each other (I've put in\CD@uG one
which you're unlikely to want). Use \string\pile\space if you want them
parallel.}\CD@rG\CD@TE{Parallel horizontal arrows must be in separate layers
of a \string\pile.}\CD@rG\CD@QE{Horizontal arrows may be used \CD@dH s, but
must still be in maths.}\CD@rG\CD@WE{Vertical arrows, \CD@FC\space\CD@dH s don%
't know where\CD@uG where to terminate.}\CD@rG\CD@OE{This prevents them from
stretching correctly.}\def\CD@YE#1{\CD@YA{"#1" inserted \ifx\CD@YF\empty
before \CD@vJ\else between \CD@YF\ifx\CD@YF\CD@vJ s\else\space and \CD@vJ\fi
\fi}}\count@=\year\multiply\count@12 \advance\count@\month\ifnum\count@>24073
\message{because this one expired in August 2005!}\expandafter\endinput\fi
\def\Horizontal@Map{\CD@nB{horizontal map}\CD@LC\CD@TJ\CD@qD}\def\CD@TJ{%
\CD@GB-9999 \let\CD@ZD\CD@XD\ifincommdiag\else\CD@cJ\ifinpile\else\skip2\z@
plus 1.5\CD@VK minus .5\CD@UK\skip4\skip2 \fi\fi\let\CD@kD\@fillh\CD@nF{%
fill@dot}{rf:.}}\def\Vector@Map{\CD@HK4}\def\Slant@Map{\CD@HK{\CD@EF255\else6%
\fi}}\def\Slope@Map{\CD@HK\CD@OC}\def\CD@HK#1#2#3#4#5#6{\CD@LC\def\CD@WK{2}%
\def\CD@aK{2}\def\CD@ZK{1}\def\CD@bK{1}\let\Horizontal@Map\CD@nI\def\CD@OG{#1%
}\def\CD@NI{\CD@U#2#3#4#5#6}}\def\CD@nI{\CD@TJ\CD@JB\let\CD@ZD\CD@TD\CD@qD}%
\CD@tG\CD@pE\CD@rA\CD@qA\CD@rA\def\cds@missives{\CD@rA}\def\CD@TD{\CD@vE\let
\CD@OG\CD@OC\CD@x\CD@zE\CD@WF\fi\setbox0\hbox{\incommdiagfalse\CD@HI}\CD@pE
\CD@aD\else\global\CD@YC\CD@bD\fi\ifvoid6 \ifvoid7 \CD@eE\fi\fi\CD@zE\else
\CD@BD\global\CD@YC\let\CD@CG\CD@IH\CD@YD\fi\else\CD@NI\CD@MI\global\CD@YC
\CD@YD\fi}\def\CD@LC{\begingroup\dimen1=\MapShortFall\dimen2=\CD@RG\dimen5=%
\MapShortFall\setbox3=\box\voidb@x\setbox6=\box\voidb@x\setbox7=\box\voidb@x
\CD@pD\mathsurround\z@\skip2\z@ plus1fill minus 1000pt\skip4\skip2 \CD@TB}%
\CD@tG\CD@tE\CD@UB\CD@TB\def\CD@U#1#2#3#4#5{\let\CD@oJ#1\let\CD@iD#2\let
\CD@QG#3\let\CD@jD#4\let\CD@LE#5\CD@TB\ifx\CD@iD\CD@jD\CD@UB\fi}\def\CD@qD#1#%
2#3#4#5{\CD@U#1#2#3#4#5\CD@tD}\def\Vertical@Map{\CD@pB433{vertical map}\CD@cD
\CD@LC\CD@GB-9995 \let\CD@kD\@fillv\CD@nF{fill@dot}{df:.}\CD@qD}\def
\break@args{\def\CD@tD{\CD@ZD}\CD@ZD\endgroup\aftergroup\CD@FE}\def\CD@MJ{%
\setbox1=\CD@oJ\setbox5=\CD@LE\ifvoid3 \ifx\CD@QG\null\else\setbox3=\CD@QG\fi
\fi\CD@@G2\CD@iD\CD@@G4\CD@jD}\def\CD@pF#1{\ifvoid1\else\CD@oF1#1\fi\ifvoid2%
\else\CD@oF2#1\fi\ifvoid3\else\CD@oF3#1\fi\ifvoid4\else\CD@oF4#1\fi\ifvoid5%
\else\CD@oF5#1\fi} \def\CD@oF#1#2{\setbox#1\vbox{\offinterlineskip\box#1%
\dimen@\prevdepth\advance\dimen@-#2\relax\setbox0\null\dp0\dimen@\ht0-\dimen@
\box0}}\def\CD@@G#1#2{\ifx#2\CD@kD\setbox#1=\box\voidb@x\else\setbox#1=#2\def
#2{\xleaders\box#1}\fi}\CD@ZA\CD@BK{\string\HorizontalMap, \string
\VerticalMap\space and \string\DiagonalMap\CD@uG are obsolete - use \CD@lG
\space to pre-define maps}\def\HorizontalMap#1#2#3#4#5{\CD@BK\CD@nB{old
horizontal map}\CD@LC\CD@TJ\def\CD@oJ{\CD@UH{#1}}\CD@SH\CD@iD{#2}\def\CD@QG{%
\CD@UH{#3}}\CD@SH\CD@jD{#4}\def\CD@LE{\CD@UH{#5}}\CD@tD}\def\VerticalMap#1#2#%
3#4#5{\CD@BK\CD@pB433{vertical map}\CD@cD\CD@LC\CD@GB-9995 \let\CD@kD\@fillv
\def\CD@oJ{\CD@GG{#1}}\CD@VH\CD@iD{#2}\def\CD@QG{\CD@GG{#3}}\CD@VH\CD@jD{#4}%
\def\CD@LE{\CD@GG{#5}}\CD@tD}\def\DiagonalMap#1#2#3#4#5{\CD@BK\CD@LC\def
\CD@OG{4}\let\CD@kD\undefined\let\CD@ZD\CD@YD\def\CD@WK{2}\def\CD@aK{2}\def
\CD@ZK{1}\def\CD@bK{1}\def\CD@QG{\CD@vF{#3}}\ifPositiveGradient\let\mv\raise
\def\CD@oJ{\CD@vF{#5}}\def\CD@iD{\CD@vF{#4}}\def\CD@jD{\CD@vF{#2}}\def\CD@LE{%
\CD@vF{#1}}\else\let\mv\lower\def\CD@oJ{\CD@vF{#1}}\def\CD@iD{\CD@vF{#2}}\def
\CD@jD{\CD@vF{#4}}\def\CD@LE{\CD@vF{#5}}\fi\CD@tD}\def\CD@aE{-}\def\CD@AD{%
\empty}\def\CD@SH{\CD@EG\CD@bE\CD@aE\@fillh}\def\CD@VH{\CD@EG\CD@MK\CD@KK
\@fillv}\def\CD@EG#1#2#3#4#5{\def\CD@CH{#5}\ifx\CD@CH#2\let#4#3\else\let#4%
\null\ifx\CD@CH\empty\else\ifx\CD@CH\CD@AD\else\let#4\CD@CH\fi\fi\fi}\def
\CD@UH#1{\hbox{\mathsurround\z@\offinterlineskip\def\CD@CH{#1}\ifx\CD@CH
\empty\else\ifx\CD@CH\CD@AD\else\CD@k\mkern-1.5mu{\CD@CH}\mkern-1.5mu\CD@ND
\fi\fi}}\def\CD@yD#1#2{\setbox#1=\hbox\bgroup\setbox0=\hbox{\CD@k\labelstyle(%
)\CD@ND}
\setbox1=\null\ht1\ht0\dp1\dp0\box1 \kern.1\CD@zC\CD@k\bgroup\labelstyle
\aftergroup\CD@LD\CD@xD}\def\CD@LD{\CD@ND\kern.1\CD@zC\egroup\CD@tD}\def
\CD@xD{\futurelet\CD@EH\CD@mJ}\def\CD@mJ{
\catcase\bgroup:\CD@v;\catcase\egroup:\missing@label;\catcase\space:\CD@TF;%
\tokcase[:\CD@XF; 
\default:\CD@zJ;\endswitch}\def\CD@v{\let\CD@MD\CD@c\let\CD@CH}\def\CD@zJ#1{%
\let\CD@UF\egroup{\let\actually@braces@missing@around@macro@in@label\CD@ZH
\let\CD@MD\CD@xC\let\CD@UF\CD@VF#1%
\actually@braces@missing@around@macro@in@label}\CD@UF}\def
\actually@braces@missing@around@macro@in@label{\let\CD@CH=}\def\missing@label
{\egroup\CD@YA{missing label}\CD@PE}\def\CD@xC{\egroup\missing@label}\outer
\def\CD@ZH{}\def\CD@UF{}\def\CD@VF{\CD@wC\CD@UF}\def\CD@MD{}\def\CD@XF{\let
\CD@N\CD@xD\get@square@arg\CD@AE}\CD@rG\CD@PE{The text which has just been
read is not allowed within map labels.}\def\CD@c{\egroup\CD@YA{missing \CD@yC
\space inserted after label}\CD@PE}\def\upper@label{\CD@oD\CD@yD6}\def
\lower@label{\def\positional@{\CD@@A\break@args}\CD@yD7}\def\middle@label{%
\CD@yD3}\CD@tG\CD@yE\CD@pD\CD@oD\def\CD@iF{\ifPositiveGradient\CD@tJ
\expandafter\upper@label\else\expandafter\lower@label\fi}\def\CD@iI{%
\ifPositiveGradient\CD@tJ\expandafter\lower@label\else\expandafter
\upper@label\fi}\CD@ZA\CD@QH{labels as positional arguments are obsolete}\def
\positional@{\CD@QH\CD@yE\CD@tJ\expandafter\upper@label\else\expandafter
\lower@label\fi-}\def\CD@tD{\futurelet\CD@EH\switch@arg}\def\eat@space{%
\afterassignment\CD@tD\let\CD@EH= }\def\CD@TF{\afterassignment\CD@xD\let
\CD@EH= }\def\CD@BC{\get@round@pair\CD@uD}\def\CD@uD#1#2{\def\CD@WK{#1}\def
\CD@aK{#2}\CD@tD}\def\optional@{\let\CD@N\CD@tD\get@square@arg\CD@AE}\def
\CD@JJ.{\CD@sC\CD@tD}\def\CD@sC{\let\CD@iD\fill@dot\let\CD@jD\fill@dot\def
\CD@MI{\let\CD@iD\dfdot\let\CD@jD\dfdot}}\def\CD@MI{}\def\CD@@E#1,{\CD@nH#1,%
\begingroup\ifx\@name\CD@RD\CD@FF\aftergroup\CD@e\fi\aftergroup\CD@jC\else
\expandafter\def\expandafter\CD@RF\expandafter{\csname\@name\endcsname}%
\expandafter\CD@vD\CD@RF\CD@KD\ifx\CD@RF\empty\aftergroup\CD@pC\expandafter
\aftergroup\csname\CD@FB\@name\endcsname\expandafter\aftergroup\csname\CD@FB @%
\@name\endcsname\else\gdef\CD@GE{#1}\CD@gB{\string\relax\space inserted before
`[\CD@GE'}\message{(I was trying to read this as a \CD@tA\ option.)}%
\aftergroup\CD@H\fi\fi\endgroup}\def\CD@vD#1#2\CD@KD{\def\CD@RF{#2}}\def
\CD@jC{\let\CD@CH\CD@N\let\CD@N\relax\CD@CH}\def\CD@H#1],{%
\CD@jC\relax\def\CD@RF{#1}\ifx\CD@RF\empty\def\CD@RF{[\CD@GE]}%
\else\def\CD@RF{[\CD@GE,#1]}
\fi\CD@RF}\def\CD@pC#1#2{\ifx#2\undefined\ifx#1\undefined\CD@gB{option `%
\@name' undefined}\else#1\fi\else\CD@FF\expandafter#2\CD@GK\CD@PK\else\CD@QK
\fi\fi\CD@DH}\CD@tG\CD@FF\CD@QK\CD@PK\def\CD@nH#1,{\CD@FF\ifx\CD@GK\undefined
\CD@e\else\expandafter\CD@oH\CD@GK,#1,(,),(,)[]%
\fi\fi\CD@FF\else\CD@mH#1==,\fi}\def\CD@e{\CD@gB{option `\@name' needs (x,y)
value}\CD@PK\let\@name\empty}\def\CD@mH#1=#2=#3,{\def\@name{#1}\def\CD@GK{#2}%
\def\CD@RF{#3}\ifx\CD@RF\empty\let\CD@GK\undefined\fi}%
\def\CD@oH#1(#2,#3)#4,(#5,#6)#7[]{\def\CD@GK{{#2}{#3}}\def\CD@RF{#1#4#5#6}%
\ifx\CD@RF\empty\def\CD@RF{#7}\ifx\CD@RF\empty\CD@e\fi\else\CD@e\fi}\def
\CD@FB{cds@}\let\CD@N\relax\def\CD@zD#1{\ifx\CD@GK\undefined\CD@gB{option `%
\@name' needs a value}\else#1\CD@GK\relax\fi}\def\CD@BE#1#2{\ifx\CD@GK
\undefined#1#2\relax\else#1\CD@GK\relax\fi}\def\cds@@showpair#1#2{\message{x=%
#1,y=#2}}\def\cds@@diagonalbase#1#2{\edef\CD@ZK{#1}\edef\CD@bK{#2}}\def\CD@DI
#1{\CD@nF{@x}{cdps@#1}\ifx\@x\relax\CD@f{unknown}\else\ifx\@x\empty\CD@f{%
cannot be used}\else\let\CD@IK\@x\fi\fi}\def\CD@f#1{\CD@gB{PostScript
translator `\CD@GK' #1}\CD@ZB\let\cds@PS\empty\let\cds@noPS\empty}\def\CD@PH{%
}\def\CD@PJ{\CD@fA\edef\CD@PH{\noexpand\CD@KB{\@name\space ignored within
maths}}}\def\diagramstyle{\CD@cJ\let\CD@N\relax\CD@CF\CD@AE\CD@AE}\let
\diagramsstyle\diagramstyle\CD@tG\CD@sE\CD@SB\CD@RB\CD@tG\CD@qE\CD@EB\CD@DB
\CD@tG\CD@oE\CD@pA\CD@oA\CD@tG\CD@iE\CD@HA\CD@GA\CD@HA\CD@tG\CD@jE\CD@JA
\CD@IA\CD@tG\CD@kE\CD@LA\CD@KA\CD@tG\CD@vE\CD@aB\CD@ZB\CD@tG\CD@EF\CD@DK
\CD@CK\CD@tG\CD@rE\CD@JB\CD@IB\CD@tG\CD@mE\CD@gA\CD@fA\CD@tG\CD@nE\CD@kA
\CD@jA\CD@tG\CD@AF\CD@iG\CD@hG\CD@RC{cds@ }{}\CD@RC{cds@}{}\CD@RC{cds@1em}{%
\CellSize1\CD@zC}\CD@RC{cds@1.5em}{\CellSize1.5\CD@zC}\CD@RC{cds@2em}{%
\CellSize2\CD@zC}\CD@RC{cds@2.5em}{\CellSize2.5\CD@zC}\CD@RC{cds@3em}{%
\CellSize3\CD@zC}\CD@RC{cds@3.5em}{\CellSize3.5\CD@zC}\CD@RC{cds@4em}{%
\CellSize4\CD@zC}\CD@RC{cds@4.5em}{\CellSize4.5\CD@zC}\CD@RC{cds@5em}{%
\CellSize5\CD@zC}\CD@RC{cds@6em}{\CellSize6\CD@zC}\CD@RC{cds@7em}{\CellSize7%
\CD@zC}\CD@RC{cds@8em}{\CellSize8\CD@zC}\def\cds@abut{\MapsAbut\dimen1\z@
\dimen5\z@}\def\cds@alignlabels{\CD@IA\CD@KA}\def\cds@amstex{\ifincommdiag
\CD@O\else\def\CD{\diagram[amstex]}
\fi\CD@T\catcode\lq\@\active}\def\cds@b{\let\CD@dB\CD@bB}\def\cds@balance{%
\let\CD@hA\CD@AA}\let\cds@bottom\cds@b\def\cds@center{\cds@vcentre
\cds@nobalance}\let\cds@centre\cds@center\def\cds@centerdisplay{\CD@HA\CD@PJ
\cds@balance}\let\cds@centredisplay\cds@centerdisplay\def\cds@crab{\CD@BE
\CD@DC{.5\PileSpacing}}\CD@RC{cds@crab-}{\CD@DC-.5\PileSpacing}\CD@RC{%
cds@crab+}{\CD@DC.5\PileSpacing}\CD@RC{cds@crab++}{\CD@DC1.5\PileSpacing}%
\CD@RC{cds@crab--}{\CD@DC-1.5\PileSpacing}\def\cds@defaultsize{\CD@BE{\let
\CD@QC}{3em}\CD@NJ}\def\cds@displayoneliner{\CD@DB}\let\cds@dotted\CD@sC\def
\cds@dpi{\CD@RJ{1truein}}\def\cds@dpm{\CD@RJ{100truecm}}\let\CD@XA\undefined
\def\cds@eqno{\let\CD@XA\CD@GK\let\CD@EJ\empty}\def\cds@fixed{\CD@qA}\CD@tG
\CD@fE\CD@J\CD@I\def\cds@flushleft{\CD@I\CD@GA\CD@PJ\cds@nobalance\CD@BE
\CD@nA\CD@nA}\def\cds@gap{\CD@AJ\setbox3=\null\ht3=\CD@tI\dp3=\CD@sI\CD@BE{%
\wd3=}\MapShortFall} \def\cds@grid{\ifx\CD@GK\undefined\let\h@grid\relax\let
\v@grid\relax\else\CD@nF{h@grid}{cdgh@\CD@GK}\CD@nF{v@grid}{cdgv@\CD@GK}\ifx
\h@grid\relax\CD@gB{unknown grid `\CD@GK'}\else\CD@WB\fi\fi}\let\h@grid\relax
\let\v@grid\relax\def\cds@gridx{\ifx\CD@GK\undefined\else\cds@grid\fi\let
\CD@CH\h@grid\let\h@grid\v@grid\let\v@grid\CD@CH}\def\cds@h{\CD@zD
\DiagramCellHeight}\def\cds@hcenter{\let\CD@hA\CD@aA}\let\cds@hcentre
\cds@hcenter\def\cds@heads{\CD@BE{\let\CD@sJ}\CD@sJ\CD@@J\CD@vE\else\ifx
\CD@sJ\CD@eF\else\CD@MC\fi\fi}\let\cds@height\cds@h\let\cds@hmiddle
\cds@balance\def\cds@htriangleheight{\CD@BE\DiagramCellHeight
\DiagramCellHeight\DiagramCellWidth1.73205\DiagramCellHeight}\def
\cds@htrianglewidth{\CD@BE\DiagramCellWidth\DiagramCellWidth
\DiagramCellHeight.57735\DiagramCellWidth}\CD@tG\CD@zE\CD@eE\CD@dE\CD@eE\def
\cds@hug{\CD@eE} \def\cds@inline{\CD@gA\let\CD@PH\empty}\def
\cds@inlineoneliner{\CD@EB}\CD@RC{cds@l>}{\CD@zD{\let\CD@RG}\dimen2=\CD@RG}%
\def\cds@labelstyle{\CD@zD{\let\labelstyle}}\def\cds@landscape{\CD@kA}\def
\cds@large{\CellSize5\CD@zC}\let\CD@EJ\empty\def\CD@FJ{\refstepcounter{%
equation}\def\CD@XA{\hbox{\@eqnnum}}}\def\cds@LaTeXeqno{\let\CD@EJ\CD@FJ}\def
\cds@lefteqno{\CD@pA}\def\cds@leftflush{\cds@flushleft\CD@J}\def
\cds@leftshortfall{\CD@zD{\dimen1 }}\def\cds@lowershortfall{%
\ifPositiveGradient\cds@leftshortfall\else\cds@rightshortfall\fi}\def
\cds@loose{\CD@VB}\def\cds@midhshaft{\CD@JA}\def\cds@midshaft{\CD@JA}\def
\cds@midvshaft{\CD@LA}\def\cds@moreoptions{\CD@@A}\let\cds@nobalance
\cds@hcenter\def\cds@nohcheck{\CD@HH}\def\cds@nohug{\CD@dE} \def
\cds@nooptions{\def\CD@aC{\CD@WD}}\let\cds@noorigin\cds@nobalance\def
\cds@nopixel{\CD@@I4\CD@XH\CD@cJ}\def\cds@noPostScript{\global\let\CD@n\empty
\CD@BE\CD@DI\empty\CD@ZB\let\cds@PS\empty\let\cds@noPS\empty}\def\cds@noPS{%
\CD@ZB\global\let\CD@n\empty}\def\cds@notextflow{\CD@RB}\def\cds@noTPIC{%
\CD@CK}\def\cds@objectstyle{\CD@zD{\let\objectstyle}}\def\cds@origin{\let
\CD@hA\CD@iB}\def\cds@p{\CD@zD\PileSpacing}\let\cds@pilespacing\cds@p\def
\cds@pixelsize{\CD@zD\CD@@I\CD@gI}\def\cds@portrait{\CD@jA}\def
\cds@PostScript{\CD@aB\global\let\CD@n\empty\let\cds@PS\CD@aB\let\cds@noPS
\CD@ZB\CD@BE\CD@DI\empty}\def\cds@PS{\CD@aB\global\let\CD@n\empty}\CD@GF\CD@n
{\typeout{\CD@tA: try the PostScript option for better results}}\def
\cds@repositionpullbacks{\let\make@pbk\CD@fH\let\CD@qH\CD@pH}\def
\cds@righteqno{\CD@oA}\def\cds@rightshortfall{\CD@zD{\dimen5 }}\def
\cds@ruleaxis{\CD@zD{\let\axisheight}}\def\cds@cmex{\let\CD@GG\CD@sB\let
\CD@QJ\CD@CJ}\def\cds@s{\cds@height\DiagramCellWidth\DiagramCellHeight}\def
\cds@scriptlabels{\let\labelstyle\scriptstyle}\def\cds@shortfall{\CD@zD
\MapShortFall\dimen1\MapShortFall\dimen5\MapShortFall}\def\cds@showfirstpass{%
\CD@BE{\let\CD@nD}\z@}\def\cds@silent{\def\CD@KB##1{}\def\CD@gB##1{}}\let
\cds@size\cds@s\def\cds@small{\CellSize2\CD@zC}\def\cds@snake{\CD@BE\CD@eJ\z@
}\def\cds@t{\let\CD@dB\CD@fB}\def\cds@textflow{\CD@SB\CD@PJ}\def\cds@thick{%
\let\CD@rF\tenlnw\CD@LF\CD@NC\CD@BE\MapBreadth{2\CD@LF}\CD@@J}\def\cds@thin{%
\let\CD@rF\tenln\CD@BE\MapBreadth{\CD@NC}\CD@@J}\def\cds@tight{\CD@WB}\let
\cds@top\cds@t\def\cds@TPIC{\CD@DK}\def\cds@uppershortfall{%
\ifPositiveGradient\cds@rightshortfall\else\cds@leftshortfall\fi}\def
\cds@vcenter{\let\CD@dB\CD@cB}\let\cds@vcentre\cds@vcenter\def
\cds@vtriangleheight{\CD@BE\DiagramCellHeight\DiagramCellHeight
\DiagramCellWidth.577035\DiagramCellHeight}\def\cds@vtrianglewidth{\CD@BE
\DiagramCellWidth\DiagramCellWidth\DiagramCellHeight1.73205\DiagramCellWidth}%
\def\cds@vmiddle{\let\CD@dB\CD@eB}\def\cds@w{\CD@zD\DiagramCellWidth}\let
\cds@width\cds@w\def\diagram{\relax\protect\CD@bC}\def\enddiagram{\protect
\CD@SG}\def\CD@bC{\CD@g\CD@uI\incommdiagtrue\edef\CD@wI{\the\CD@NB}\global
\CD@NB\z@\boxmaxdepth\maxdimen\everycr{}\CD@aC}\def\CD@aC{\CD@y\let\CD@N
\CD@ZC\CD@CF\CD@AE\CD@WD}\def\CD@ZC{\CD@gE\expandafter\CD@aC\else\expandafter
\CD@WD\fi}\def\CD@WD{\let\CD@EH\relax\CD@nE\CD@vE\else\CD@KB{landscape ignored
without PostScript}\CD@jA\fi\fi\CD@EJ\setbox2=\vbox\bgroup\CD@JF\CD@VD}\def
\CD@cH{\CD@nE\CD@fB\else\CD@dB\fi\CD@hA\nointerlineskip\setbox0=\null\ht0-%
\CD@pI\dp0\CD@pI\wd0\CD@kI\box0 \global\CD@QA\CD@kF\global\CD@yA\CD@XB\ifx
\CD@NK\undefined\global\CD@RA\CD@kF\else\global\CD@RA\CD@NK\fi\egroup\CD@zF
\CD@nE\setbox2=\hbox to\dp2{\dp2=\CD@QA\global\CD@QA\ht2\ht2\wd2\global\CD@iG
\CD@IK{0 1 bturn}\box2\CD@IK{eturn}\hss}\CD@DB\fi\ifnum\CD@yA=1 \else\CD@DB
\fi\global\@ignorefalse\CD@mE\leavevmode\fi\ifvmode\CD@TA\else\ifmmode\CD@PH
\CD@GI\else\CD@qE\CD@gA\fi\ifinner\CD@gA\fi\CD@mE\CD@GI\else\CD@sE\CD@QB\else
\CD@TA\fi\fi\fi\fi\CD@dD}\def\CD@dD{\global\CD@NB\CD@wI\relax\CD@xE\global
\CD@ID\else\aftergroup\CD@mC\fi\if@ignore\aftergroup\ignorespaces\fi\CD@wC
\ignorespaces}\def\CD@fB{\advance\CD@pI\dimen1\relax}\def\CD@eB{\advance
\CD@pI.5\dimen1\relax}\def\CD@bB{}\def\CD@cB{\CD@fB\advance\CD@pI\CD@YB
\divide\CD@pI2 \advance\CD@pI-\axisheight\relax}\def\CD@aA{}\def\CD@iB{\CD@kF
\z@}\def\CD@AA{\ifdim\dimen2>\CD@kF\CD@kF\dimen2 \else\dimen2\CD@kF\CD@kI
\dimen0 \advance\CD@kI\dimen2 \fi}\def\CD@QB{\skip0\z@\relax\loop\skip1%
\lastskip\ifdim\skip1>\z@\unskip\advance\skip0\skip1 \repeat\vadjust{%
\prevdepth\dp\strutbox\penalty\predisplaypenalty\vskip\abovedisplayskip\CD@UA
\penalty\postdisplaypenalty\vskip\belowdisplayskip}\ifdim\skip0=\z@\else
\hskip\skip0 \global\@ignoretrue\fi}\def\CD@TA{\CD@LG\kern-\displayindent
\CD@UA\CD@LG\global\@ignoretrue}\def\CD@UA{\hbox to\hsize{\CD@fE\ifdim\CD@RA=%
\z@\else\advance\CD@QA-\CD@RA\setbox2=\hbox{\kern\CD@RA\box2}\fi\fi\setbox1=%
\hbox{\ifx\CD@XA\undefined\else\CD@k\CD@XA\CD@ND\fi}\CD@oE\CD@iE\else\advance
\CD@QA\wd1 \fi\wd1\z@\box1 \fi\dimen0\wd2 \advance\dimen0\wd1 \advance\dimen0%
-\hsize\ifdim\dimen0>-\CD@nA\CD@HA\fi\advance\dimen0\CD@QA\ifdim\dimen0>\z@
\CD@KB{wider than the page by \the\dimen0 }\CD@HA\fi\CD@iE\hss\else\CD@V
\CD@QA\CD@nA\fi\CD@GI\hss\kern-\wd1\box1 }}\def\CD@GI{\CD@AF\CD@@F\else\CD@SC
\global\CD@hG\fi\fi\kern\CD@QA\box2 }\CD@tG\CD@wE\CD@YC\CD@XC\def\CD@JF{%
\CD@cJ\ifdim\DiagramCellHeight=-\maxdimen\DiagramCellHeight\CD@QC\fi\ifdim
\DiagramCellWidth=-\maxdimen\DiagramCellWidth\CD@QC\fi\global\CD@XC\CD@IF\let
\CD@FE\empty\let\CD@z\CD@Q\let\overprint\CD@eH\let\CD@s\CD@rJ\let\enddiagram
\CD@ED\let\\\CD@cC\let\par\CD@jH\let\CD@MD\empty\let\switch@arg\CD@PB\let
\shift\CD@iA\baselineskip\DiagramCellHeight\lineskip\z@\lineskiplimit\z@
\mathsurround\z@\tabskip\z@\CD@OB}\def\CD@VD{\penalty-123 \begingroup\CD@jA
\aftergroup\CD@K\halign\bgroup\global\advance\CD@NB1 \vadjust{\penalty1}%
\global\CD@FA\z@\CD@OB\CD@j##\CD@DD\CD@Q\CD@Q\CD@OI\CD@j##\CD@DD\cr}\def
\CD@ED{\CD@MD\CD@GD\crcr\egroup\global\CD@JD\endgroup}\def\CD@j{\global
\advance\CD@FA1 \futurelet\CD@EH\CD@i}\def\CD@i{\ifx\CD@EH\CD@DD\CD@tJ\hskip1%
sp plus 1fil \relax\let\CD@DD\relax\CD@vI\else\hfil\CD@k\objectstyle\let
\CD@FE\CD@d\fi}\def\CD@DD{\CD@MD\relax\CD@yI\CD@vI\global\CD@QA\CD@iA\penalty
-9993 \CD@ND\hfil\null\kern-2\CD@QA\null}\def\CD@cC{\cr}\def\across#1{\span
\omit\mscount=#1 \global\advance\CD@FA\mscount\global\advance\CD@FA\m@ne
\CD@sF\ifnum\mscount>2 \CD@fJ\repeat\ignorespaces}\def\CD@fJ{\relax\span\omit
\advance\mscount\m@ne}\def\CD@qJ{\ifincommdiag\ifx\CD@iD\@fillh\ifx\CD@jD
\@fillh\ifdim\dimen3>\z@\else\ifdim\dimen2>93\CD@@I\ifdim\dimen2>18\p@\ifdim
\CD@LF>\z@\count@\CD@bJ\advance\count@\m@ne\ifnum\count@<\z@\count@20\let
\CD@aJ\CD@uJ\fi\xdef\CD@bJ{\the\count@}\fi\fi\fi\fi\fi\fi\fi}\def\CD@cG#1{%
\vrule\horizhtdp width#1\dimen@\kern2\dimen@}\def\CD@uJ{\rlap{\dimen@\CD@@I
\CD@V\dimen@{.182\p@}\CD@zH\dimen@\advance\CD@tI\dimen@\CD@cG0\CD@cG0\CD@cG2%
\CD@cG6\CD@cG6\CD@cG2\CD@cG0\CD@cG0\CD@cG2\CD@cG6\CD@cG0\CD@cG0\CD@cG2\CD@cG2%
\CD@cG6\CD@cG0\CD@cG0\CD@cG2\CD@cG6\CD@cG2\CD@cG2\CD@cG0\CD@cG0}}\def\CD@bJ{%
10}\def\CD@aJ{}\def\CD@XD{\CD@gE\CD@TB\fi\CD@x\CD@WF\CD@HI}\def\CD@x{\CD@QJ
\CD@DC\CD@MJ\ifdim\CD@DC=\z@\else\CD@pF\CD@DC\fi\ifvoid3 \setbox3=\null\ht3%
\CD@tI\dp3\CD@sI\else\CD@V{\ht3}\CD@tI\CD@V{\dp3}\CD@sI\fi\dimen3=.5\wd3
\ifdim\dimen3=\z@\CD@tE\else\dimen3-\CD@XH\fi\else\CD@TB\fi\CD@V{\dimen2}{\wd
7}\CD@V{\dimen2}{\wd6}\CD@qJ\advance\dimen2-2\dimen3 \dimen4.5\dimen2 \dimen2%
\dimen4 \advance\dimen2\CD@eJ\advance\dimen4-\CD@eJ\advance\dimen2-\wd1
\advance\dimen4-\wd5 \ifvoid2 \else\CD@V{\ht3}{\ht2}\CD@V{\dp3}{\dp2}\CD@V{%
\dimen2}{\wd2}\fi\ifvoid4 \else\CD@V{\ht3}{\ht4}\CD@V{\dp3}{\dp4}\CD@V{\dimen
4}{\wd4}\fi\advance\skip2\dimen2 \advance\skip4\dimen4 \CD@tE\advance\skip2%
\skip4 \dimen0\dimen5 \advance\dimen0\wd5 \skip3-\skip4 \advance\skip3-\dimen
0 \let\CD@jD\empty\else\skip3\z@\relax\dimen0\z@\fi}\def\CD@WF{%
\offinterlineskip\lineskip.2\CD@zC\ifvoid6 \else\setbox3=\vbox{\hbox to2%
\dimen3{\hss\box6\hss}\box3}\fi\ifvoid7 \else\setbox3=\vtop{\box3 \hbox to2%
\dimen3{\hss\box7\hss}}\fi}\def\CD@HI{\kern\dimen1 \box1 \CD@aJ\CD@iD\hskip
\skip2 \kern\dimen0 \ifincommdiag\CD@jE\penalty1\fi\kern\dimen3 \penalty
\CD@GB\hskip\skip3 \null\kern-\dimen3 \else\hskip\skip3 \fi\box3 \CD@jD\hskip
\skip4 \box5 \kern\dimen5}\def\CD@MF{\ifnum\CD@LH>\CD@TC\CD@V{\dimen1}%
\objectheight\CD@V{\dimen5}\objectheight\else\CD@V{\dimen1}\objectwidth\CD@V{%
\dimen5}\objectwidth\fi}\def\CD@Y{\begingroup\ifdim\dimen7=\z@\kern\dimen8
\else\ifdim\dimen6=\z@\kern\dimen9 \else\dimen5\dimen6 \dimen6\dimen9 \CD@KJ
\dimen4\dimen2 \CD@dG{\dimen4}\dimen6\dimen5 \dimen7\dimen8 \CD@KJ\CD@iC{%
\dimen2}\ifdim\dimen2<\dimen4 \kern\dimen2 \else\kern\dimen4 \fi\fi\fi
\endgroup}\def\CD@jJ{\CD@JI\setbox\z@\hbox{\lower\axisheight\hbox to\dimen2{%
\CD@DF\ifPositiveGradient\dimen8\ht\CD@MH\dimen9\CD@mI\else\dimen8\dp3 \dimen
9\dimen1 \fi\else\dimen8 \ifPositiveGradient\objectheight\else\z@\fi\dimen9%
\objectwidth\fi\advance\dimen8 \ifPositiveGradient-\fi\axisheight\CD@Y\unhbox
\z@\CD@DF\ifPositiveGradient\dimen8\dp3 \dimen9\dimen0 \else\dimen8\ht\CD@MH
\dimen9\CD@mF\fi\else\dimen8 \ifPositiveGradient\z@\else\objectheight\fi
\dimen9\objectwidth\fi\advance\dimen8 \ifPositiveGradient\else-\fi\axisheight
\CD@Y}}}\def\CD@bD{\dimen6 \CD@aK\DiagramCellHeight\dimen7 \CD@WK
\DiagramCellWidth\CD@jJ\ifPositiveGradient\advance\dimen7-\CD@ZK
\DiagramCellWidth\else\dimen7 \CD@ZK\DiagramCellWidth\dimen6\z@\multiply
\CD@LH\m@ne\fi\advance\dimen6-\CD@bK\DiagramCellHeight\setbox0=\rlap{\global
\CD@iG\kern-\dimen7 \lower\dimen6\hbox{\CD@eD{\the\CD@TC\space\the\CD@LH
\space bturn}\box0 \CD@IK{eturn}}}\ht0\z@\dp0\z@\raise\axisheight\box0 }\def
\CD@vB{\advance\CD@hF-\CD@mI\CD@wJ\CD@hF\advance\CD@wJ\CD@hI\ifvoid\CD@sH
\ifdim\CD@wJ<.1em\ifnum\CD@gD=\@m\else\CD@aG h\CD@wJ<.1em:objects overprint:%
\CD@FA\CD@gD\fi\fi\else\ifhbox\CD@sH\CD@SK\else\CD@TK\fi\advance\CD@wJ\CD@mI
\CD@bH{-\CD@mI}{\box\CD@sH}{\CD@wJ}\z@\fi\CD@hF-\CD@mF\CD@gD\CD@FA\CD@hI\z@}%
\def\CD@SK{\setbox\CD@sH=\hbox{\unhbox\CD@sH\unskip\unpenalty}\setbox\CD@tH=%
\hbox{\unhbox\CD@tH\unskip\unpenalty}\setbox\CD@sH=\hbox to\CD@wJ{\CD@OA\wd
\CD@sH\unhbox\CD@sH\CD@PA\lastkern\unkern\ifdim\CD@PA=\z@\CD@UB\advance\CD@OA
-\wd\CD@tH\else\CD@TB\fi\ifnum\lastpenalty=\z@\else\CD@JA\unpenalty\fi\kern
\CD@PA\ifdim\CD@hF<\CD@OA\CD@JA\fi\ifdim\CD@hI<\wd\CD@tH\CD@JA\fi\CD@jE\CD@hI
\CD@wJ\advance\CD@hI-\CD@OA\advance\CD@hI\wd\CD@tH\ifdim\CD@hI<2\wd\CD@tH
\CD@aG h\CD@hI<2\wd\CD@tH:arrow too short:\CD@FA\CD@gD\fi\divide\CD@hI\tw@
\CD@hF\CD@wJ\advance\CD@hF-\CD@hI\fi\CD@tE\kern-\CD@hI\fi\hbox to\CD@hI{%
\unhbox\CD@tH}\CD@HG}}\CD@tG\ifinpile\inpiletrue\inpilefalse\inpilefalse\def
\pile{\protect\CD@UJ\protect\CD@uH}\def\CD@uH#1{\CD@l#1\CD@QD}\def\CD@UJ{%
\CD@nB{pile}\setbox0=\vtop\bgroup\aftergroup\CD@lD\inpiletrue\let\CD@FE\empty
\let\pile\CD@KF\let\CD@QD\CD@PD\let\CD@GD\CD@FD\CD@yH\baselineskip.5%
\PileSpacing\lineskip.1\CD@zC\relax\lineskiplimit\lineskip\mathsurround\z@
\tabskip\z@\let\\\CD@wH}\def\CD@l{\CD@DE\CD@YF\empty\halign\bgroup\hfil\CD@k
\let\CD@FE\CD@d\let\\\CD@vH##\CD@MD\CD@ND\hfil\CD@Q\CD@R##\cr}\CD@rG\CD@NE{%
pile only allows one column.}\CD@rG\CD@UE{you left it out!}\def\CD@R{\CD@QD
\CD@Q\relax\CD@YA{missing \CD@yC\space inserted after \string\pile}\CD@NE}%
\def\CD@PD{\CD@MD\crcr\egroup\egroup}\def\CD@GD{\CD@MD}\def\CD@FD{\CD@MD
\relax\CD@QD\CD@YA{missing \CD@yC\space inserted between \string\pile\space
and \CD@HD}\CD@UE}\def\CD@QD{\CD@MD}\def\CD@lD{\vbox{\dimen1\dp0 \unvbox0
\setbox0=\lastbox\advance\dimen1\dp0 \nointerlineskip\box0 \nointerlineskip
\setbox0=\null\dp0.5\dimen1\ht0-\dp0 \box0}\ifincommdiag\CD@tJ\penalty-9998
\fi\xdef\CD@YF{pile}}\def\CD@vH{\cr}\def\CD@wH{\noalign{\skip@\prevdepth
\advance\skip@-\baselineskip\prevdepth\skip@}}\def\CD@KF#1{#1}\def\CD@TK{%
\setbox\CD@sH=\vbox{\unvbox\CD@sH\setbox1=\lastbox\setbox0=\box\voidb@x\CD@tF
\setbox\CD@sH=\lastbox\ifhbox\CD@sH\CD@rC\repeat\unvbox0 \global\CD@QA\CD@ZE}%
\CD@ZE\CD@QA}\def\CD@rC{\CD@jE\setbox\CD@sH=\hbox{\unhbox\CD@sH\unskip\setbox
\CD@sH=\lastbox\unskip\unhbox\CD@sH}\ifdim\CD@wJ<\wd\CD@sH\CD@aG h\CD@wJ<\wd
\CD@sH:arrow in pile too short:\CD@FA\CD@gD\else\setbox\CD@sH=\hbox to\CD@wJ{%
\unhbox\CD@sH}\fi\else\CD@gJ\fi\setbox0=\vbox{\box\CD@sH\nointerlineskip
\ifvoid0 \CD@tJ\box1 \else\vskip\skip0 \unvbox0 \fi}\skip0=\lastskip\unskip}%
\def\CD@gJ{\penalty7 \noindent\unhbox\CD@sH\unskip\setbox\CD@sH=\lastbox
\unskip\unhbox\CD@sH\endgraf\setbox\CD@tH=\lastbox\unskip\setbox\CD@tH=\hbox{%
\CD@JG\unhbox\CD@tH\unskip\unskip\unpenalty}\ifcase\prevgraf\cd@shouldnt P\or
\ifdim\CD@wJ<\wd\CD@tH\CD@aG h\CD@wJ<\wd\CD@sH:object in pile too wide:\CD@FA
\CD@gD\setbox\CD@sH=\hbox to\CD@wJ{\hss\unhbox\CD@tH\hss}\else\setbox\CD@sH=%
\hbox to\CD@wJ{\hss\kern\CD@hF\unhbox\CD@tH\kern\CD@hI\hss}\fi\or\setbox
\CD@sH=\lastbox\unskip\CD@SK\else\cd@shouldnt Q\fi\unskip\unpenalty}\def
\CD@cD{\CD@MJ\ifvoid3 \setbox3=\null\ht3\axisheight\dp3-\ht3 \dimen3.5\CD@LF
\else\dimen4\dp3 \dimen3.5\wd3 \setbox3=\CD@GG{\box3}\dp3\dimen4 \ifdim\ht3=-%
\dp3 \else\CD@TB\fi\fi\dimen0\dimen3 \advance\dimen0-.5\CD@LF\setbox0\null\ht
0\ht3\dp0\dp3\wd0\wd3 \ifvoid6\else\setbox6\hbox{\unhbox6\kern\dimen0\kern2pt%
}\dimen0\wd6 \fi\ifvoid7\else\setbox7\hbox{\kern2pt\kern\dimen3\unhbox7}%
\dimen3\wd7 \fi\setbox3\hbox{\ifvoid6\else\kern-\dimen0\unhbox6\fi\unhbox3
\ifvoid7\else\unhbox7\kern-\dimen3\fi}\ht3\ht0\dp3\dp0\wd3\wd0 \CD@tE\dimen4=%
\ht\CD@MH\advance\dimen4\dp5 \advance\dimen4\dimen1 \let\CD@jD\empty\else
\dimen4\ht3 \fi\setbox0\null\ht0\dimen4 \offinterlineskip\setbox8=\vbox spread%
2ex{\kern\dimen5 \box1 \CD@iD\vfill\CD@tE\else\kern\CD@eJ\fi\box0}\ht8=\z@
\setbox9=\vtop spread2ex{\kern-\ht3 \kern-\CD@eJ\box3 \CD@jD\vfill\box5 \kern
\dimen1}\dp9=\z@\hskip\dimen0plus.0001fil \box9 \kern-\CD@LF\box8 \CD@kE
\penalty2 \fi\CD@tE\penalty1 \fi\kern\PileSpacing\kern-\PileSpacing\kern-.5%
\CD@LF\penalty\CD@GB\null\kern\dimen3}\def\CD@cI{\ifhbox\CD@VA\CD@KB{clashing
verticals}\ht\CD@MH.5\dp\CD@VA\dp\CD@MH-\ht5 \CD@yB\ht\CD@MH\z@\dp\CD@MH\z@
\fi\dimen1\dp\CD@VA\CD@xA\prevgraf\unvbox\CD@VA\CD@wA\lastpenalty\unpenalty
\setbox\CD@VA=\null\setbox\CD@lI=\hbox{\CD@JG\unhbox\CD@lI\unskip\unpenalty
\dimen0\lastkern\unkern\unkern\unkern\kern\dimen0 \CD@HG}\setbox\CD@lF=\hbox{%
\unhbox\CD@lF\dimen0\lastkern\unkern\unkern\global\CD@QA\lastkern\unkern\kern
\dimen0 }\CD@tF\ifnum\CD@xA>4 \CD@zI\repeat\unskip\unskip\advance\CD@mF.5\wd
\CD@VA\advance\CD@mF\wd\CD@lF\advance\CD@mI.5\wd\CD@VA\advance\CD@mI\wd\CD@lI
\ifnum\CD@FA=\CD@lA\CD@OA.5\wd\CD@VA\edef\CD@NK{\the\CD@OA}\fi\setbox\CD@VA=%
\hbox{\kern-\CD@mF\box\CD@lF\unhbox\CD@VA\box\CD@lI\kern-\CD@mI\penalty\CD@wA
\penalty\CD@NB}\ht\CD@VA\dimen1 \dp\CD@VA\z@\wd\CD@VA\CD@tB\CD@vB}\def\CD@zI{%
\ifdim\wd\CD@lF<\CD@QA\setbox\CD@lF=\hbox to\CD@QA{\CD@JG\unhbox\CD@lF}\fi
\advance\CD@xA\m@ne\setbox\CD@VA=\hbox{\box\CD@lF\unhbox\CD@VA}\unskip\setbox
\CD@lF=\lastbox\setbox\CD@lF=\hbox{\unhbox\CD@lF\unskip\unpenalty\dimen0%
\lastkern\unkern\unkern\global\CD@QA\lastkern\unkern\kern\dimen0 }}\def\CD@yB
{\dimen1\dp\CD@VA\ifhbox\CD@VA\CD@xB\else\CD@zB\fi\setbox\CD@VA=\vbox{%
\penalty\CD@NB}\dp\CD@VA-\dp\CD@MH\wd\CD@VA\CD@tB}\def\CD@zB{\unvbox\CD@VA
\CD@wA\lastpenalty\unpenalty\ifdim\dimen1<\ht\CD@MH\CD@aG v\dimen1<\ht\CD@MH:%
rows overprint:\CD@NB\CD@wA\fi}\def\CD@xB{\dimen0=\ht\CD@VA\setbox\CD@VA=%
\hbox\bgroup\advance\dimen1-\ht\CD@MH\unhbox\CD@VA\CD@xA\lastpenalty
\unpenalty\CD@wA\lastpenalty\unpenalty\global\CD@RA-\lastkern\unkern\setbox0=%
\lastbox\CD@tF\setbox\CD@VA=\hbox{\box0\unhbox\CD@VA}\setbox0=\lastbox\ifhbox
0 \CD@kJ\repeat\global\CD@SA-\lastkern\unkern\global\CD@QA\CD@JK\unhbox\CD@VA
\egroup\CD@JK\CD@QA\CD@bH{\CD@SA}{\box\CD@VA}{\CD@RA}{\dimen1}}\def\CD@kJ{%
\setbox0=\hbox to\wd0\bgroup\unhbox0 \unskip\unpenalty\dimen7\lastkern\unkern
\ifnum\lastpenalty=1 \unpenalty\CD@UB\else\CD@TB\fi\ifnum\lastpenalty=2
\unpenalty\dimen2.5\dimen0\advance\dimen2-.5\dimen1\advance\dimen2-%
\axisheight\else\dimen2\z@\fi\setbox0=\lastbox\dimen6\lastkern\unkern\setbox1%
=\lastbox\setbox0=\vbox{\unvbox0 \CD@tE\kern-\dimen1 \else\ifdim\dimen2=\z@
\else\kern\dimen2 \fi\fi}\ifdim\dimen0<\ht0 \CD@aG v\dimen0<\ht0:upper part of
vertical too short:{\CD@tE\CD@NB\else\CD@wA\fi}\CD@xA\else\setbox0=\vbox to%
\dimen0{\unvbox0}\fi\setbox1=\vtop{\unvbox1}\ifdim\dimen1<\dp1 \CD@aG v\dimen
1<\dp1:lower part of vertical too short:\CD@NB\CD@wA\else\setbox1=\vtop to%
\dimen1{\ifdim\dimen2=\z@\else\kern-\dimen2 \fi\unvbox1 }\fi\box1 \kern\dimen
6 \box0 \kern\dimen7 \CD@HG\global\CD@QA\CD@JK\egroup\CD@JK\CD@QA\relax}%
\countdef\CD@u=14 \newcount\CD@CA\newcount\CD@XB\newcount\CD@NB\let\CD@LB
\insc@unt\newcount\CD@FA\newcount\CD@lA\let\CD@mA\CD@XB\newcount\CD@MB\CD@tG
\CD@DF\CD@bI\CD@aI\CD@aI\def\CD@nD{-1}\def\CD@K{\ifnum\CD@nD<\z@\else
\begingroup\scrollmode\showboxdepth\CD@nD\showboxbreadth\maxdimen\showlists
\endgroup\fi\CD@bI\CD@zF\CD@CA=\CD@u\advance\CD@CA1 \CD@XB=\CD@CA\ifnum\CD@NB
=1 \CD@JA\fi\advance\CD@XB\CD@NB\dimen1\z@\skip0\z@\count@=\insc@unt\advance
\count@\CD@u\divide\count@2 \ifnum\CD@XB>\count@\CD@KB{The diagram has too
many rows! It can't be reformatted.}\else\CD@NG\CD@WI\fi\CD@cH}\def\CD@NG{%
\CD@NB\CD@CA\CD@uF\ifnum\CD@NB<\CD@XB\setbox\CD@NB\box\voidb@x\advance\CD@NB1%
\relax\repeat\CD@NB\CD@CA\skip\z@\z@\CD@uF\CD@GB\lastpenalty\unpenalty\ifnum
\CD@GB>\z@\CD@KE\repeat\ifnum\CD@GB=-123 \CD@tJ\unpenalty\else\cd@shouldnt D%
\fi\ifx\v@grid\relax\else\CD@NB\CD@XB\advance\CD@NB\m@ne\expandafter\CD@VJ
\v@grid\fi\CD@MB\CD@mA\CD@tB\z@\CD@XG\ifx\h@grid\relax\else\expandafter\CD@LJ
\h@grid\fi\count@\CD@XB\advance\count@\m@ne\CD@YB\ht\count@}\def\CD@KE{%
\ifcase\CD@GB\or\CD@MG\else\CD@uA-\lastpenalty\unpenalty\CD@vA\lastpenalty
\unpenalty\setbox0=\lastbox\CD@WG\fi\CD@wD}\def\CD@wD{\skip1\lastskip\unskip
\advance\skip0\skip1 \ifdim\skip1=\z@\else\expandafter\CD@wD\fi}\def\CD@MG{%
\setbox0=\lastbox\CD@pI\dp0 \advance\CD@pI\skip\z@\skip\z@\z@\advance\CD@NF
\CD@pI\CD@uE\ifnum\CD@NB>\CD@CA\CD@NF\DiagramCellHeight\CD@pI\CD@NF\advance
\CD@pI-\CD@qI\fi\fi\CD@qI\ht0 \CD@NF\CD@qI\setbox\CD@NB\hbox{\unhbox\CD@NB
\unhbox0}\dp\CD@NB\CD@pI\ht\CD@NB\CD@qI\advance\CD@NB1 }\def\CD@WG{\ifnum
\CD@uA<\z@\advance\CD@uA\CD@XB\ifnum\CD@uA<\CD@CA\CD@UG\else\CD@OA\dp\CD@uA
\CD@PA\ht\CD@uA\setbox\CD@uA\hbox{\box\z@\penalty\CD@vA\penalty\CD@GB\unhbox
\CD@uA}\dp\CD@uA\CD@OA\ht\CD@uA\CD@PA\fi\else\CD@UG\fi}\def\CD@UG{\CD@KB{%
diagonal goes outside diagram (lost)}}\def\CD@fI{\advance\CD@uA\CD@XB\ifnum
\CD@uA<\CD@CA\CD@UG\else\ifnum\CD@uA=\CD@NB\CD@VG\else\ifnum\CD@uA>\CD@NB
\cd@shouldnt M\else\CD@OA\dp\CD@uA\CD@PA\ht\CD@uA\setbox\CD@uA\hbox{\box\z@
\penalty\CD@vA\penalty\CD@GB\unhbox\CD@uA}\dp\CD@uA\CD@OA\ht\CD@uA\CD@PA\fi
\fi\fi}\def\CD@WI{\CD@t\CD@AJ\setbox\CD@PC=\hbox{\CD@k A\@super f\CD@lJ f%
\CD@ND}\CD@ZE\z@\CD@JK\z@\CD@kI\z@\CD@kF\z@\CD@NB=\CD@XB\CD@NF\z@\CD@uB\z@
\CD@uF\ifnum\CD@NB>\CD@CA\advance\CD@NB\m@ne\CD@qI\ht\CD@NB\CD@pI\dp\CD@NB
\advance\CD@NF\CD@qI\CD@rI\advance\CD@uB\CD@NF\CD@KC\CD@ZI\CD@w\ht\CD@NB
\CD@qI\dp\CD@NB\CD@pI\nointerlineskip\box\CD@NB\CD@NF\CD@pI\setbox\CD@NB\null
\ht\CD@NB\CD@uB\repeat\CD@wB\nointerlineskip\box\CD@NB\CD@gG\CD@ZE
\DiagramCellWidth{width}\CD@gG\CD@JK\DiagramCellHeight{height}\CD@VA\CD@LB
\advance\CD@VA-\CD@lA\advance\CD@VA\m@ne\advance\CD@VA\CD@mA\dimen0\wd\CD@VA
\CD@tI\axisheight\dimen1\CD@uB\advance\dimen1-\CD@YB\dimen2\CD@kI\advance
\dimen2-\dimen0 \advance\CD@XB-\CD@CA\advance\CD@LB-\CD@lA}\count@\year
\multiply\count@12 \advance\count@\month\ifnum\count@>24074 \loop\iftrue
\message{gone February 2006!}\repeat\fi\def\CD@wB{\CD@qI-\CD@NF\CD@pI\CD@NF
\setbox\CD@MH=\null\dp\CD@MH\CD@NF\ht\CD@MH-\CD@NF\CD@mF\z@\CD@mI\z@\CD@lA
\CD@LB\advance\CD@lA-\CD@MB\advance\CD@lA\CD@mA\CD@FA\CD@LB\CD@VA\CD@MB\CD@sF
\ifnum\CD@FA>\CD@lA\advance\CD@FA\m@ne\advance\CD@VA\m@ne\CD@tB\wd\CD@VA
\setbox\CD@FA=\box\voidb@x\CD@yB\repeat\CD@w\ht\CD@NB\CD@qI\dp\CD@NB\CD@pI}%
\def\CD@gG#1#2#3{\ifdim#1>.01\CD@zC\CD@PA#2\relax\advance\CD@PA#1\relax
\advance\CD@PA.99\CD@zC\count@\CD@PA\divide\count@\CD@zC\CD@KB{increase cell #%
3 to \the\count@ em}\fi}\def\CD@rI{\CD@FA=\CD@LB\penalty4 \noindent\unhbox
\CD@NB\CD@sF\unskip\setbox0=\lastbox\ifhbox0 \advance\CD@FA\m@ne\setbox\CD@FA
\hbox to\wd0{\null\penalty-9990\null\unhbox0}\repeat\CD@lA\CD@FA\advance
\CD@FA\CD@MB\advance\CD@FA-\CD@mA\ifnum\CD@FA<\CD@LB\count@\CD@FA\advance
\count@\m@ne\dimen0=\wd\count@\count@\CD@MB\advance\count@\m@ne\CD@tB\wd
\count@\CD@sF\ifnum\CD@FA<\CD@LB\CD@DJ\CD@XG\dimen0\wd\CD@FA\advance\CD@FA1
\repeat\fi\CD@sF\CD@GB\lastpenalty\unpenalty\ifnum\CD@GB>\z@\CD@vA
\lastpenalty\unpenalty\CD@VG\repeat\endgraf\unskip\ifnum\lastpenalty=4
\unpenalty\else\cd@shouldnt S\fi}\def\CD@VG{\advance\CD@vA\CD@lA\advance
\CD@vA\m@ne\setbox0=\lastbox\ifnum\CD@vA<\CD@LB\setbox\CD@vA\hbox{\box0%
\penalty\CD@GB\unhbox\CD@vA}\else\CD@UG\fi}\def\CD@bG{}\CD@tG\CD@uE\CD@WB
\CD@VB\def\CD@DJ{\advance\dimen0\wd\CD@FA\divide\dimen0\tw@\CD@uE\dimen0%
\DiagramCellWidth\else\CD@V{\dimen0}\DiagramCellWidth\CD@pJ\fi\advance\CD@tB
\dimen0 }\def\CD@XG{\setbox\CD@MB=\vbox{}\dp\CD@MB=\CD@uB\wd\CD@MB\CD@tB
\advance\CD@MB1 }\def\CD@LJ#1,{\def\CD@GK{#1}\ifx\CD@GK\CD@RD\else\advance
\CD@tB\CD@GK\DiagramCellWidth\CD@XG\expandafter\CD@LJ\fi}\def\CD@VJ#1,{\def
\CD@GK{#1}\ifx\CD@GK\CD@RD\else\ifnum\CD@NB>\CD@CA\CD@NF\CD@GK
\DiagramCellHeight\advance\CD@NF-\dp\CD@NB\advance\CD@NB\m@ne\ht\CD@NB\CD@NF
\fi\expandafter\CD@VJ\fi}\def\CD@pJ{\CD@wE\CD@OA\dimen0 \advance\CD@OA-%
\DiagramCellWidth\ifdim\CD@OA>2\MapShortFall\CD@KB{badly drawn diagonals (see
manual)}\let\CD@pJ\empty\fi\else\let\CD@pJ\empty\fi}\def\CD@KC{\CD@VA\CD@mA
\CD@sF\ifnum\CD@VA<\CD@MB\dimen0\dp\CD@VA\advance\dimen0\CD@NF\dp\CD@VA\dimen
0 \advance\CD@VA1 \repeat}\def\CD@bH#1#2#3#4{\ifnum\CD@FA<\CD@LB\CD@OA=#1%
\relax\setbox\CD@FA=\hbox{\setbox0=#2\dimen7=#4\relax\dimen8=#3\relax\ifhbox
\CD@FA\unhbox\CD@FA\advance\CD@OA-\lastkern\unkern\fi\ifdim\CD@OA=\z@\else
\kern-\CD@OA\fi\raise\dimen7\box0 \kern-\dimen8 }\ifnum\CD@FA=\CD@lA\CD@V
\CD@kF\CD@OA\fi\else\cd@shouldnt O\fi}\def\CD@w{\setbox\CD@NB=\hbox{\CD@FA
\CD@lA\CD@VA\CD@mA\CD@PA\z@\relax\CD@sF\ifnum\CD@FA<\CD@LB\CD@tB\wd\CD@VA
\relax\CD@eI\advance\CD@FA1 \advance\CD@VA1 \repeat}\CD@V\CD@kI{\wd\CD@NB}\wd
\CD@NB\z@}\def\CD@eI{\ifhbox\CD@FA\CD@OA\CD@tB\relax\advance\CD@OA-\CD@PA
\relax\ifdim\CD@OA=\z@\else\kern\CD@OA\fi\CD@PA\CD@tB\advance\CD@PA\wd\CD@FA
\relax\unhbox\CD@FA\advance\CD@PA-\lastkern\unkern\fi}\def\CD@ZI{\setbox
\CD@sH=\box\voidb@x\CD@VA=\CD@MB\CD@FA\CD@LB\CD@VA\CD@mA\advance\CD@VA\CD@FA
\advance\CD@VA-\CD@lA\advance\CD@VA\m@ne\CD@tB\wd\CD@VA\count@\CD@LB\advance
\count@\m@ne\CD@hF.5\wd\count@\advance\CD@hF\CD@tB\CD@A\m@ne\CD@gD\@m\CD@sF
\ifnum\CD@FA>\CD@lA\advance\CD@FA\m@ne\advance\CD@hF-\CD@tB\CD@PI\wd\CD@VA
\CD@tB\advance\CD@hF\CD@tB\advance\CD@VA\m@ne\CD@tB\wd\CD@VA\repeat\CD@mF
\CD@kF\CD@mI-\CD@mF\CD@vB}\newcount\CD@GB\def\CD@s{}\def\CD@t{\mathsurround
\z@\hsize\z@\rightskip\z@ plus1fil minus\maxdimen\parfillskip\z@\linepenalty
9000 \looseness0 \hfuzz\maxdimen\hbadness10000 \clubpenalty0 \widowpenalty0
\displaywidowpenalty0 \interlinepenalty0 \predisplaypenalty0
\postdisplaypenalty0 \interdisplaylinepenalty0 \interfootnotelinepenalty0
\floatingpenalty0 \brokenpenalty0 \everypar{}\leftskip\z@\parskip\z@
\parindent\z@\pretolerance10000 \tolerance10000 \hyphenpenalty10000
\exhyphenpenalty10000 \binoppenalty10000 \relpenalty10000 \adjdemerits0
\doublehyphendemerits0 \finalhyphendemerits0 \CD@IA\prevdepth\z@}\newbox
\CD@KG\newbox\CD@IG\def\CD@JG{\unhcopy\CD@KG}\def\CD@HG{\unhcopy\CD@IG}\def
\CD@iJ{\hbox{}\penalty1\nointerlineskip}\def\CD@PI{\penalty5 \noindent\setbox
\CD@MH=\null\CD@mF\z@\CD@mI\z@\ifnum\CD@FA<\CD@LB\ht\CD@MH\ht\CD@FA\dp\CD@MH
\dp\CD@FA\unhbox\CD@FA\skip0=\lastskip\unskip\else\CD@OK\skip0=\z@\fi\endgraf
\ifcase\prevgraf\cd@shouldnt Y \or\cd@shouldnt Z \or\CD@RI\or\CD@XI\else
\CD@QI\fi\unskip\setbox0=\lastbox\unskip\unskip\unpenalty\noindent\unhbox0%
\setbox0\lastbox\unpenalty\unskip\unskip\unpenalty\setbox0\lastbox\CD@tF
\CD@GB\lastpenalty\unpenalty\ifnum\CD@GB>\z@\setbox\z@\lastbox\CD@lB\repeat
\endgraf\unskip\unskip\unpenalty}\def\CD@YJ{\CD@uA\CD@XB\advance\CD@uA-\CD@NB
\CD@vA\CD@FA\advance\CD@vA-\CD@lA\advance\CD@vA1 \expandafter\message{%
prevgraf=\the\prevgraf at (\the\CD@uA,\the\CD@vA)}}\def\CD@XI{\CD@CE\setbox
\CD@lI=\lastbox\setbox\CD@lI=\hbox{\unhbox\CD@lI\unskip\unpenalty}\unskip
\ifdim\ht\CD@lI>\ht\CD@PC\setbox\CD@MH=\copy\CD@lI\else\ifdim\dp\CD@lI>\dp
\CD@PC\setbox\CD@MH=\copy\CD@lI\else\CD@FG\CD@lI\fi\fi\advance\CD@mF.5\wd
\CD@lI\advance\CD@mI.5\wd\CD@lI\setbox\CD@lI=\hbox{\unhbox\CD@lI\CD@HG}\CD@bH
\CD@mF{\box\CD@lI}\CD@mI\z@\CD@yB\CD@vB}\def\CD@CE{\ifnum\CD@A>0 \advance
\dimen0-\CD@tB\CD@iA-.5\dimen0 \CD@A-\CD@A\else\CD@A0 \CD@iA\z@\fi\setbox
\CD@MH=\lastbox\setbox\CD@MH=\hbox{\unhbox\CD@MH\unskip\unskip\unpenalty
\setbox0=\lastbox\global\CD@QA\lastkern\unkern}\advance\CD@iA-.5\CD@QA\unskip
\setbox\CD@MH=\null\CD@mI\CD@iA\CD@mF-\CD@iA}\def\CD@Z{\ht\CD@MH\CD@tI\dp
\CD@MH\CD@sI}\def\CD@FG#1{\setbox\CD@MH=\hbox{\CD@V{\ht\CD@MH}{\ht#1}\CD@V{%
\dp\CD@MH}{\dp#1}\CD@V{\wd\CD@MH}{\wd#1}\vrule height\ht\CD@MH depth\dp\CD@MH
width\wd\CD@MH}}\def\CD@QI{\CD@CE\CD@Z\setbox\CD@lI=\lastbox\unskip\setbox
\CD@lF=\lastbox\unskip\setbox\CD@lF=\hbox{\unhbox\CD@lF\unskip\global\CD@yA
\lastpenalty\unpenalty}\advance\CD@yA9999 \ifcase\CD@yA\CD@VI\or\CD@YI\or
\CD@TI\or\CD@dI\or\CD@cI\or\CD@SI\else\cd@shouldnt9\fi}\def\CD@VI{\CD@FG
\CD@lI\CD@UI\setbox\CD@sH=\box\CD@lF\setbox\CD@tH=\box\CD@lI}\def\CD@YI{%
\CD@FG\CD@lF\setbox\CD@lI\hbox{\penalty8 \unhbox\CD@lI\unskip\unpenalty\ifnum
\lastpenalty=8 \else\CD@xH\fi}\CD@UI\setbox\CD@lF=\hbox{\unhbox\CD@lF\unskip
\unpenalty\global\setbox\CD@DA=\lastbox}\ifdim\wd\CD@lF=\z@\else\CD@xH\fi
\setbox\CD@sH=\box\CD@DA}\def\CD@xH{\CD@KB{extra material in \string\pile
\space cell (lost)}}\def\CD@UI{\CD@yB\ifvoid\CD@sH\else\CD@KB{Clashing
horizontal arrows}\CD@mI.5\CD@hF\CD@mF-\CD@mI\CD@vB\CD@mI\z@\CD@mF\z@\fi
\CD@hI\CD@hF\advance\CD@hI-\CD@mI\CD@hF-\CD@mF\CD@JC\CD@FA}\def\CD@RI{\setbox
0\lastbox\unskip\CD@iA\z@\CD@Z\ifdim\skip0>\z@\CD@tJ\CD@A0 \else\ifnum\CD@A<1
\CD@A0 \dimen0\CD@tB\fi\advance\CD@A1 \fi}\def\VonH{\CD@MA46\VonH{.5\CD@LF}}%
\def\HonV{\CD@MA57\HonV{.5\CD@LF}}\def\HmeetV{\CD@MA44\HmeetV{-\MapShortFall}%
}\def\CD@MA#1#2#3#4{\CD@pB34#1{\string#3}\CD@SD\CD@GB-999#2 \dimen0=#4\CD@tI
\dimen0\advance\CD@tI\axisheight\CD@sI\dimen0\advance\CD@sI-\axisheight\CD@CF
\CD@HC\CD@ZD}\def\CD@HC#1{\setbox0=\hbox{\CD@k#1\CD@ND}\dimen0.5\wd0 \CD@tI
\ht0 \CD@sI\dp0 \CD@ZD}\def\CD@SD{\setbox0=\null\ht0=\CD@tI\dp0=\CD@sI\wd0=%
\dimen0 \copy0\penalty\CD@GB\box0 }\def\CD@TI{\CD@GC\CD@yB}\def\CD@dI{\CD@GC
\CD@vB}\def\CD@SI{\CD@GC\CD@yB\CD@vB}\def\CD@GC{\setbox\CD@lI=\hbox{\unhbox
\CD@lI}\setbox\CD@lF=\hbox{\unhbox\CD@lF\global\setbox\CD@DA=\lastbox}\ht
\CD@MH\ht\CD@DA\dp\CD@MH\dp\CD@DA\advance\CD@mF\wd\CD@DA\advance\CD@mI\wd
\CD@lI}\CD@tG\ifPositiveGradient\CD@CI\CD@BI\CD@CI\CD@tG\ifClimbing\CD@rB
\CD@qB\CD@rB\newcount\DiagonalChoice\DiagonalChoice\m@ne\ifx\tenln\nullfont
\CD@tJ\def\CD@qF{\CD@KH\ifPositiveGradient/\else\CD@k\backslash\CD@ND\fi}%
\else\def\CD@qF{\CD@rF\char\count@}\fi\let\CD@rF\tenln\def\Use@line@char#1{%
\hbox{#1\CD@rF\ifPositiveGradient\else\advance\count@64 \fi\char\count@}}\def
\CD@cF{\Use@line@char{\count@\CD@TC\multiply\count@8\advance\count@-9\advance
\count@\CD@LH}}\def\CD@ZF{\Use@line@char{\ifcase\DiagonalChoice\CD@gF\or
\CD@fF\or\CD@fF\else\CD@gF\fi}}\def\CD@gF{\ifnum\CD@TC=\z@\count@\rq33 \else
\count@\CD@TC\multiply\count@\sixt@@n\advance\count@-9\advance\count@\CD@LH
\advance\count@\CD@LH\fi}\def\CD@fF{\count@\rq\ifcase\CD@LH55\or\ifcase\CD@TC
66\or22\or52\or61\or72\fi\or\ifcase\CD@TC66\or25\or22\or63\or52\fi\or\ifcase
\CD@TC66\or16\or36\or22\or76\fi\or\ifcase\CD@TC66\or27\or25\or67\or22\fi\fi
\relax}\def\CD@uC#1{\hbox{#1\setbox0=\Use@line@char{#1}\ifPositiveGradient
\else\raise.3\ht0\fi\copy0 \kern-.7\wd0 \ifPositiveGradient\raise.3\ht0\fi
\box0}}\def\CD@jF#1{\hbox{\setbox0=#1\kern-.75\wd0 \vbox to.25\ht0{%
\ifPositiveGradient\else\vss\fi\box0 \ifPositiveGradient\vss\fi}}}\def\CD@jI#%
1{\hbox{\setbox0=#1\dimen0=\wd0 \vbox to.25\ht0{\ifPositiveGradient\vss\fi
\box0 \ifPositiveGradient\else\vss\fi}\kern-.75\dimen0 }}\CD@RC{+h:>}{%
\Use@line@char\CD@fF}\CD@RC{-h:>}{\Use@line@char\CD@gF}\CD@nF{+t:<}{-h:>}%
\CD@nF{-t:<}{+h:>}\CD@RC{+t:>}{\CD@jF{\Use@line@char\CD@fF}}\CD@RC{-t:>}{%
\CD@jI{\Use@line@char\CD@gF}}\CD@nF{+h:<}{-t:>}\CD@nF{-h:<}{+t:>}\CD@RC{+h:>>%
}{\CD@uC\CD@fF}\CD@RC{-h:>>}{\CD@uC\CD@gF}\CD@nF{+t:<<}{-h:>>}\CD@nF{-t:<<}{+%
h:>>}\CD@nF{+h:>->}{+h:>>}\CD@nF{-h:>->}{-h:>>}\CD@nF{+t:<-<}{-h:>>}\CD@nF{-t%
:<-<}{+h:>>}\CD@RC{+t:>>}{\CD@jF{\CD@uC\CD@fF}}\CD@RC{-t:>>}{\CD@jI{\CD@uC
\CD@gF}}\CD@nF{+h:<<}{-t:>>}\CD@nF{-h:<<}{+t:>>}\CD@nF{+t:>->}{+t:>>}\CD@nF{-%
t:>->}{-t:>>}\CD@nF{+h:<-<}{-t:>>}\CD@nF{-h:<-<}{+t:>>}\CD@RC{+f:-}{\CD@EF
\null\else\CD@cF\fi}\CD@nF{-f:-}{+f:-}\def\CD@tC#1#2{\vbox to#1{\vss\hbox to#%
2{\hss.\hss}\vss}}\def\hfdot{\CD@tC{2\axisheight}{.5em}}%
\def\vfdot{\CD@tC{1ex}\z@}
\def\CD@bF{\hbox{\dimen0=.3\CD@zC\dimen1\dimen0 \ifnum\CD@LH>\CD@TC\CD@iC{%
\dimen1}\else\CD@dG{\dimen0}\fi\CD@tC{\dimen0}{\dimen1}}}\newarrowfiller{.}%
\hfdot\hfdot\vfdot\vfdot\def\dfdot{\CD@bF\CD@CK}\CD@RC{+f:.}{\dfdot}\CD@RC{-f%
:.}{\dfdot}\def\CD@@K#1{\hbox\bgroup\def\CD@CH{#1\egroup}\afterassignment
\CD@CH
\count@=\rq}\def\lnchar{\CD@@K\CD@qF}\let\laf\lnchar\let\lah\lnchar\def\lad{%
\CD@@K\xlad}\def\xlad{\setbox2=\hbox{\CD@qF}\setbox0=\hbox to.3\wd2{\hss.\hss
}\dimen0=\ht0 \advance\dimen0-\dp0 \dimen1=.3\ht2 \advance\dimen1-\dimen0 \dp
0=.5\dimen1 \dimen1=.3\ht2 \advance\dimen1\dimen0 \ht0=.5\dimen1 \raise\dp0%
\box0}\def\lahh{\CD@@K\xlahh}\def\lat{\CD@@K\xlat}\def\xlat{\setbox0=\hbox{%
\CD@qF}\dimen0=\ht0 \setbox1=\hbox to.25\wd0{\ifcase\DiagonalChoice\box0\hss
\or\hss\box0 \or\hss\box0 \or\box0\hss\fi}\vbox to.25\dimen0{\ifClimbing\box1%
\vss\else\vss\box1\fi\kern\z@}}\def\xlahh{\setbox0=\hbox{\CD@qF}%
\ifPositiveGradient\CD@tJ\copy0 \kern-.7\wd0 \mv.3\ht0\box0 \else\ifClimbing
\CD@tJ\copy0 \kern-.7\wd0 \mv.3\ht0\box0 \else\mv-.3\ht0\copy0 \kern-.7\wd0
\box0 \fi\fi}\def\CD@dF#1{\setbox#1=\hbox{\dimen5\dimen#1 \setbox8=\box#1
\dimen1\wd8 \count@\dimen5 \divide\count@\dimen1 \ifnum\count@=0 \box8 \ifdim
\dimen5<.95\dimen1 \CD@gB{diagonal line too short}\fi\else\dimen3=\dimen5
\advance\dimen3-\dimen1 \divide\dimen3\count@\dimen4\dimen3 \CD@dG{\dimen4}%
\ifPositiveGradient\multiply\dimen4\m@ne\fi\dimen6\dimen1 \advance\dimen6-%
\dimen3 \loop\raise\count@\dimen4\copy8 \ifnum\count@>0 \kern-\dimen6 \advance
\count@\m@ne\repeat\fi}}\def\CD@CG#1{\CD@EF\CD@xJ{#1}\else\CD@dF{#1}\fi}\def
\CD@IH#1{}\newdimen\objectheight\objectheight1.8ex \newdimen\objectwidth
\objectwidth1em \def\CD@YD{\dimen6=\CD@aK\DiagramCellHeight\dimen7=\CD@WK
\DiagramCellWidth\CD@KJ\ifnum\CD@LH>0 \ifnum\CD@TC>0 \CD@aF\else\aftergroup
\CD@VC\fi\else\aftergroup\CD@UC\fi}\def\CD@VC{\CD@YA{diagonal map is nearly
vertical}\CD@NA}\def\CD@UC{\CD@YA{diagonal map is nearly horizontal}\CD@NA}%
\CD@rG\CD@NA{Use an orthogonal map instead}\def\CD@aF{\CD@MJ\dimen3\dimen7%
\dimen7\dimen6\CD@iC{\dimen7}\advance\dimen3-\dimen7 \CD@MF\ifnum\CD@LH>%
\CD@TC\advance\dimen6-\dimen1\advance\dimen6-\dimen5 \CD@iC{\dimen1}\CD@iC{%
\dimen5}\else\dimen0\dimen1\advance\dimen0\dimen5\CD@dG{\dimen0}\advance
\dimen6-\dimen0 \fi\dimen2.5\dimen7\advance\dimen2-\dimen1 \dimen4.5\dimen7%
\advance\dimen4-\dimen5 \ifPositiveGradient\dimen0\dimen5 \advance\dimen1-%
\CD@WK\DiagramCellWidth\advance\dimen1 \CD@ZK\DiagramCellWidth\setbox6=\llap{%
\unhbox6\kern.1\ht2}\setbox7=\rlap{\kern.1\ht2\unhbox7}\else\dimen0\dimen1
\advance\dimen1-\CD@ZK\DiagramCellWidth\setbox7=\llap{\unhbox7\kern.1\ht2}%
\setbox6=\rlap{\kern.1\ht2\unhbox6}\fi\setbox6=\vbox{\box6\kern.1\wd2}\setbox
7=\vtop{\kern.1\wd2\box7}\CD@dG{\dimen0}\advance\dimen0-\axisheight\advance
\dimen0-\CD@bK\DiagramCellHeight\dimen5-\dimen0 \advance\dimen0\dimen6
\advance\dimen1.5\dimen3 \ifdim\wd3>\z@\ifdim\ht3>-\dp3\CD@TB\fi\fi\dimen3%
\dimen2 \dimen7\dimen2\advance\dimen7\dimen4 \ifvoid3 \else\CD@tE\else\dimen8%
\ht3\advance\dimen8-\axisheight\CD@iC{\dimen8}\CD@X{\dimen8}{.5\wd3}\dimen9%
\dp3\advance\dimen9\axisheight\CD@iC{\dimen9}\CD@X{\dimen9}{.5\wd3}%
\ifPositiveGradient\advance\dimen2-\dimen9\advance\dimen4-\dimen8 \else
\advance\dimen4-\dimen9\advance\dimen2-\dimen8 \fi\fi\advance\dimen3-.5\wd3
\fi\dimen9=\CD@aK\DiagramCellHeight\advance\dimen9-2\DiagramCellHeight\CD@tE
\advance\dimen2\dimen4 \CD@CG{2}\dimen2-\dimen0\advance\dimen2\dp2 \else
\CD@CG{2}\CD@CG{4}\ifPositiveGradient\dimen2-\dimen0\advance\dimen2\dp2 \dimen
4\dimen5\advance\dimen4-\ht4 \else\dimen4-\dimen0\advance\dimen4\dp4 \dimen2%
\dimen5\advance\dimen2-\ht2 \fi\fi\setbox0=\hbox to\z@{\kern\dimen1 \ifvoid1
\else\ifPositiveGradient\advance\dimen0-\dp1 \lower\dimen0 \else\advance
\dimen5-\ht1 \raise\dimen5 \fi\rlap{\unhbox1}\fi\raise\dimen2\rlap{\unhbox2}%
\ifvoid3 \else\lower.5\dimen9\rlap{\kern\dimen3\unhbox3}\fi\kern.5\dimen7
\lower.5\dimen9\box6 \lower.5\dimen9\box7 \kern.5\dimen7 \CD@tE\else\raise
\dimen4\llap{\unhbox4}\fi\ifvoid5 \else\ifPositiveGradient\advance\dimen5-\ht
5 \raise\dimen5 \else\advance\dimen0-\dp5 \lower\dimen0 \fi\llap{\unhbox5}\fi
\hss}\ht0=\axisheight\dp0=-\ht0\box0 }\def\NorthWest{\CD@BI\CD@rB
\DiagonalChoice0 }\def\NorthEast{\CD@CI\CD@rB\DiagonalChoice1 }\def\SouthWest
{\CD@CI\CD@qB\DiagonalChoice3 }\def\SouthEast{\CD@BI\CD@qB\DiagonalChoice2 }%
\def\CD@aD{\vadjust{\CD@uA\CD@FA\advance\CD@uA\ifPositiveGradient\else-\fi
\CD@ZK\relax\CD@vA\CD@NB\advance\CD@vA-\CD@bK\relax\hbox{\advance\CD@uA
\ifPositiveGradient-\fi\CD@WK\advance\CD@vA\CD@aK\hbox{\box6 \kern\CD@DC\kern
\CD@eJ\penalty1 \box7 \box\z@}\penalty\CD@uA\penalty\CD@vA}\penalty\CD@uA
\penalty\CD@vA\penalty104}}\def\CD@eH#1{\relax\vadjust{\hbox@maths{#1}%
\penalty\CD@FA\penalty\CD@NB\penalty\tw@}}\def\CD@lB{\ifcase\CD@GB\or\or
\CD@bH{.5\wd0}{\box0}{.5\wd0}\z@\or\unhbox\z@\setbox\z@\lastbox\CD@bH{.5\wd0}%
{\box0}{.5\wd0}\z@\unpenalty\unpenalty\setbox\z@\lastbox\or\CD@TG\else
\advance\CD@GB-100 \ifnum\CD@GB<\z@\cd@shouldnt B\fi\setbox\z@\hbox{\kern
\CD@mF\copy\CD@MH\kern\CD@mI\CD@uA\CD@XB\advance\CD@uA-\CD@NB\penalty\CD@uA
\CD@uA\CD@FA\advance\CD@uA-\CD@lA\penalty\CD@uA\unhbox\z@\global\CD@yA
\lastpenalty\unpenalty\global\CD@zA\lastpenalty\unpenalty}\CD@uA-\CD@yA\CD@vA
\CD@zA\CD@fI\fi}\def\CD@TG{\global\CD@iG\unhbox\z@\setbox\z@\lastbox\CD@uA
\lastpenalty\unpenalty\advance\CD@uA\CD@mA\CD@vA\CD@XB\advance\CD@vA-%
\lastpenalty\unpenalty\dimen1\lastkern\unkern\setbox3\lastbox\dimen0\lastkern
\unkern\setbox0=\hbox to\z@{\unhbox0\setbox0\lastbox\setbox7\lastbox
\unpenalty\CD@eJ\lastkern\unkern\CD@DC\lastkern\unkern\setbox6\lastbox\dimen7%
\CD@tB\advance\dimen7-\wd\CD@uA\ifdim\dimen7<\z@\CD@CI\multiply\dimen7\m@ne
\let\mv\empty\else\CD@BI\def\mv{\raise\ht1}\kern-\dimen7 \fi\ifnum\CD@vA>%
\CD@NB\dimen6\CD@uB\advance\dimen6-\ht\CD@vA\else\dimen6\z@\fi\CD@jJ\setbox0%
\hbox{\CD@eD{\the\CD@TC\space\ifPositiveGradient\else-\fi\the\CD@LH\space
bturn}\box0 \CD@IK{eturn}}\setbox1\null\ht1\dimen6\wd1\dimen7 \dimen7\dimen2
\dimen6\wd1 \CD@KJ\CD@uA\CD@LH\CD@vA\CD@TC\dimen6\ht1 \CD@KJ\setbox2\null
\divide\dimen2\tw@\advance\dimen2\CD@eJ\CD@eG{\dimen2}\wd2\dimen2 \dimen0.5%
\dimen7 \advance\dimen0\ifPositiveGradient\else-\fi\CD@eJ\CD@dG{\dimen0}%
\advance\dimen0-\axisheight\ht2\dimen0 \dimen0\CD@DC\CD@eG{\dimen0}\advance
\dimen0\ht2\ht2\dimen0 \dimen0\ifPositiveGradient-\fi\CD@DC\CD@dG{\dimen0}%
\advance\dimen0\wd2\wd2\dimen0 \setbox4\null\dimen0 .6\CD@zC\CD@eG{\dimen0}%
\ht4\dimen0 \dimen0 .2\CD@zC\CD@dG{\dimen0}\wd4\dimen0 \dimen0\wd2 \ifvoid6%
\else\dimen1\ht4 \advance\dimen1\ht2 \CD@CC6+-\raise\dimen1\rlap{%
\ifPositiveGradient\advance\dimen0-\wd6\advance\dimen0-\wd4 \else\advance
\dimen0\wd4 \fi\kern\dimen0\box6}\fi\dimen0\wd2 \ifvoid7\else\dimen1\ht4
\advance\dimen1-\ht2 \CD@CC7-+\lower\dimen1\rlap{\ifPositiveGradient\advance
\dimen0\wd4 \else\advance\dimen0-\wd7\advance\dimen0-\wd4 \fi\kern\dimen0\box
7}\fi\mv\box0\hss}\ht0\z@\dp0\z@\CD@bH{\z@}{\box\z@}{\z@}{\axisheight}}\def
\CD@CC#1#2#3{\dimen4.5\wd#1 \ifdim\dimen4>.25\dimen7\dimen4=.25\dimen7\fi
\ifdim\dimen4>\CD@zC\dimen4.4\dimen4 \advance\dimen4.6\CD@zC\fi\CD@eG{\dimen4%
}\dimen5\axisheight\CD@dG{\dimen5}\advance\dimen4-\dimen5 \dimen5\dimen4%
\CD@eG{\dimen5}\advance\dimen0\ifPositiveGradient#2\else#3\fi\dimen5 \CD@dG{%
\dimen4}\advance\dimen1\dimen4 } \def\CD@eD#1{\expandafter\CD@IK{#1}}\CD@ZA
\CD@EK{output is PostScript dependent}\def\CD@SC{\CD@IK{/bturn {gsave
currentpoint currentpoint translate 4 2 roll neg exch atan rotate neg exch neg
exch translate } def /eturn {currentpoint grestore moveto} def}}\def\CD@OJ#1{%
\count@#1\relax\multiply\count@7\advance\count@16577\divide\count@33154 }\def
\CD@fD#1{\expandafter\special{#1}} \def\CD@xJ#1{\setbox#1=\hbox{\dimen0\dimen
#1\CD@dG{\dimen0}\CD@OJ{\dimen0}\setbox0=\null\ifPositiveGradient\count@-%
\count@\ht0\dimen0 \else\dp0\dimen0 \fi\box0 \CD@uA\count@\CD@OJ\CD@LF\CD@fD{%
pn \the\count@}\CD@fD{pa 0 0}\CD@OJ{\dimen#1}\CD@fD{pa \the\count@\space\the
\CD@uA}\CD@fD{fp}\kern\dimen#1}}\def\CD@JI{\CD@KJ\begingroup\ifdim\dimen7<%
\dimen6 \dimen2=\dimen6 \dimen6=\dimen7 \dimen7=\dimen2 \count@\CD@LH\CD@LH
\CD@TC\CD@TC\count@\else\dimen2=\dimen7 \fi\ifdim\dimen6>.01\p@\CD@KI\global
\CD@QA\dimen0 \else\global\CD@QA\dimen7 \fi\endgroup\dimen2\CD@QA}\def\CD@KI{%
\CD@hJ\ifdim\dimen7>1.73\dimen6 \divide\dimen2 4 \multiply\CD@TC2 \else\dimen
2=0.353553\dimen2 \advance\CD@LH-\CD@TC\multiply\CD@TC4 \fi\dimen0=4\dimen2
\CD@ZG4\CD@ZG{-2}\CD@ZG2\CD@ZG{-2.5}}\def\CD@AI{\begingroup\count@\dimen0
\dimen2 45pt \divide\count@\dimen2 \ifdim\dimen0<\z@\advance\count@\m@ne\fi
\ifodd\count@\advance\count@1\CD@@A\else\CD@y\fi\advance\dimen0-\count@\dimen
2 \CD@gE\multiply\dimen0\m@ne\fi\ifnum\count@<0 \multiply\count@-7 \fi\dimen3%
\dimen1 \dimen6\dimen0 \dimen7 3754936sp \ifdim\dimen0<6\p@\def\CD@OG{4000}%
\fi\CD@KJ\dimen2\dimen3\CD@dG{\dimen2}\CD@hJ\multiply\CD@TC-6 \dimen0\dimen2
\CD@ZG1\CD@ZG{0.3}\dimen1\dimen0 \dimen2\dimen3 \dimen0\dimen3 \CD@ZG3\CD@ZG{%
1.5}\CD@ZG{0.3}\divide\count@2 \CD@gE\multiply\dimen1\m@ne\fi\ifodd\count@
\dimen2\dimen1\dimen1\dimen0\dimen0-\dimen2 \fi\divide\count@2 \ifodd\count@
\multiply\dimen0\m@ne\multiply\dimen1\m@ne\fi\global\CD@QA\dimen0\global
\CD@RA\dimen1\endgroup\dimen6\CD@QA\dimen7\CD@RA}\def\CD@OC{255}\let\CD@OG
\CD@OC\def\CD@KJ{\begingroup\ifdim\dimen7<\dimen6 \dimen9\dimen7\dimen7\dimen
6\dimen6\dimen9\CD@@A\else\CD@y\fi\dimen2\z@\dimen3\CD@XH\dimen4\CD@XH\dimen0%
\z@\dimen8=\CD@OG\CD@XH\CD@lC\global\CD@yA\dimen\CD@gE0\else3\fi\global\CD@zA
\dimen\CD@gE3\else0\fi\endgroup\CD@LH\CD@yA\CD@TC\CD@zA}\def\CD@lC{\count@
\dimen6 \divide\count@\dimen7 \advance\dimen6-\count@\dimen7 \dimen9\dimen4
\advance\dimen9\count@\dimen0 \ifdim\dimen9>\dimen8 \CD@@C\else\CD@AC\ifdim
\dimen6>\z@\dimen9\dimen6 \dimen6\dimen7 \dimen7\dimen9 \expandafter
\expandafter\expandafter\CD@lC\fi\fi}\def\CD@@C{\ifdim\dimen0=\z@\ifdim\dimen
9<2\dimen8 \dimen0\dimen8 \fi\else\advance\dimen8-\dimen4 \divide\dimen8%
\dimen0 \ifdim\count@\CD@XH<2\dimen8 \count@\dimen8 \dimen9\dimen4 \advance
\dimen9\count@\dimen0 \CD@AC\fi\fi}\def\CD@AC{\dimen4\dimen0 \dimen0\dimen9
\advance\dimen2\count@\dimen3 \dimen9\dimen2 \dimen2\dimen3 \dimen3\dimen9 }%
\def\CD@ZG#1{\CD@dG{\dimen2}\advance\dimen0 #1\dimen2 }\def\CD@dG#1{\divide#1%
\CD@TC\multiply#1\CD@LH}\def\CD@eG#1{\divide#1\CD@vA\multiply#1\CD@uA}\def
\CD@iC#1{\divide#1\CD@LH\multiply#1\CD@TC}\def\CD@hJ{\dimen6\CD@LH\CD@XH
\multiply\dimen6\CD@LH\dimen7\CD@TC\CD@XH\multiply\dimen7\CD@TC\CD@KJ}\ifx
\errorcontextlines\undefined\CD@tJ\let\CD@GH\relax\else\def\CD@GH{%
\errorcontextlines\m@ne}\fi\ifnum\inputlineno<0 \let\CD@CD\empty\let\CD@W
\empty\let\CD@mD\relax\let\CD@uI\relax\let\CD@vI\relax\let\CD@zF\relax
\message{! Why not upgrade to TeX version 3? (available since 1990)}\else\def
\CD@W{ at line \number\inputlineno}\def\CD@mD{ - first occurred}\def\CD@uI{%
\edef\CD@h{\the\inputlineno}\global\let\CD@jB\CD@h}\def\CD@h{9999}\def\CD@vI{%
\xdef\CD@jB{\the\inputlineno}}\def\CD@jB{\CD@h}\def\CD@zF{\ifnum\CD@h<%
\inputlineno\edef\CD@CD{\space at lines \CD@h--\the\inputlineno}\else\edef
\CD@CD{\CD@W}\fi}\fi\let\CD@CD\empty\def\CD@YA#1#2{\CD@GH\errhelp=#2%
\expandafter\errmessage{\CD@tA: #1}}\def\CD@KB#1{\begingroup\expandafter
\message{! \CD@tA: #1\CD@CD}\ifnum\CD@XB>\CD@NB\ifnum\CD@CA>\CD@NB\else\ifnum
\CD@lA>\CD@FA\else\ifnum\CD@LB>\CD@FA\advance\CD@XB-\CD@NB\advance\CD@FA-%
\CD@lA\advance\CD@FA1\relax\expandafter\message{! (error detected at row \the
\CD@XB, column \the\CD@FA, but probably caused elsewhere)}\fi\fi\fi\fi
\endgroup}\def\CD@gB#1{{\expandafter\message{\CD@tA\space Warning: #1\CD@W}}}%
\def\CD@CB#1#2{\CD@gB{#1 \string#2 is obsolete\CD@mD}}\def\CD@AB#1{\CD@CB{%
Dimension}{#1}\CD@DE#1\CD@BB\CD@BB}\def\CD@BB{\CD@OA=}\def\CD@@B#1{\CD@CB{%
Count}{#1}\CD@DE#1\CD@OH\CD@OH}\def\CD@OH{\count@=}\def\HorizontalMapLength{%
\CD@AB\HorizontalMapLength}\def\VerticalMapHeight{\CD@AB\VerticalMapHeight}%
\def\VerticalMapDepth{\CD@AB\VerticalMapDepth}\def\VerticalMapExtraHeight{%
\CD@AB\VerticalMapExtraHeight}\def\VerticalMapExtraDepth{\CD@AB
\VerticalMapExtraDepth}\def\DiagonalLineSegments{\CD@@B\DiagonalLineSegments}%
\ifx\tenln\nullfont\CD@ZA\CD@KH{\CD@eF\space diagonal line and arrow font not
available}\else\let\CD@KH\relax\fi\def\CD@aG#1#2<#3:#4:#5#6{\begingroup\CD@PA
#3\relax\advance\CD@PA-#2\relax\ifdim.1em<\CD@PA\CD@uA#5\relax\CD@vA#6\relax
\ifnum\CD@uA<\CD@vA\count@\CD@vA\advance\count@-\CD@uA\CD@KB{#4 by \the\CD@PA
}\if#1v\let\CD@CH\CD@JK\edef\tmp{\the\CD@uA--\the\CD@vA,\the\CD@FA}\else
\advance\count@\count@\if#1l\advance\count@-\CD@A\else\if#1r\advance\count@
\CD@A\fi\fi\advance\CD@PA\CD@PA\let\CD@CH\CD@ZE\edef\tmp{\the\CD@NB,\the
\CD@uA--\the\CD@vA}\fi\divide\CD@PA\count@\ifdim\CD@CH<\CD@PA\global\CD@CH
\CD@PA\fi\fi\fi\endgroup}\CD@tG\CD@xE\CD@JD\CD@ID\CD@rG\CD@xI{See the message
above.}\CD@rG\CD@lH{Perhaps you've forgotten to end the diagram before
resuming the text, in\CD@uG which case some garbage may be added to the
diagram, but we should be ok now.\CD@uG Alternatively you've left a blank line
in the middle - TeX will now complain\CD@uG that the remaining \CD@S s are
misplaced - so please use comments for layout.}\CD@rG\CD@hD{You have already
closed too many brace pairs or environments; an \CD@HD\CD@uG command was (%
over)due.}\CD@rG\CD@hH{\CD@dC\space and \CD@HD\space commands must match.}%
\def\CD@jH{\ifnum\inputlineno=0 \else\expandafter\CD@iH\fi}\def\CD@iH{\CD@MD
\CD@GD\crcr\CD@YA{missing \CD@HD\space inserted before \CD@kH- type "h"}%
\CD@lH\enddiagram\CD@AG\CD@kH\par}\def\CD@AG#1{\edef\enddiagram{\noexpand
\CD@rD{#1\CD@W}}}\def\CD@rD#1{\CD@YA{\CD@HD\space(anticipated by #1) ignored}%
\CD@xI\let\enddiagram\CD@SG}\def\CD@SG{\CD@YA{misplaced \CD@HD\space ignored}%
\CD@hH}\def\CD@mC{\CD@YA{missing \CD@HD\space inserted.}\CD@hD\CD@AG{closing
group}}\ifx\DeclareOption\undefined\else\ifx\DeclareOption\@notprerr\else
\DeclareOption*{\let\CD@N\relax\let\CD@DH\relax\expandafter\CD@@E
\CurrentOption,}\fi\fi







\catcode\lq\$=3 
\def\vboxtoz{\vbox to\z@}

\def\scriptaxis#1{\@scriptaxis{$\scriptstyle#1$}}
\def\ssaxis#1{\@ssaxis{$\scriptscriptstyle#1$}}
\def\@scriptaxis#1{\dimen0\axisheight\advance\dimen0-\ss@axisheight\raise
\dimen0\hbox{#1}}\def\@ssaxis#1{\dimen0\axisheight\advance\dimen0-%
\ss@axisheight\raise\dimen0\hbox{#1}}

\ifx\boldmath\undefined
\let\boldscriptaxis\scriptaxis
\def\boldscript#1{\hbox{$\scriptstyle#1$}}
\def\boldscriptscript#1{\hbox{$\scriptscriptstyle#1$}}
\else\def\boldscriptaxis#1{\@scriptaxis{\boldmath$\scriptstyle#1$}}
\def\boldscript#1{\hbox{\boldmath$\scriptstyle#1$}}
\def\boldscriptscript#1{\hbox{\boldmath$\scriptscriptstyle#1$}}
\fi

\def\raisehook#1#2#3{\hbox{\setbox3=\hbox{#1$\scriptscriptstyle#3$}%
\dimen0\ss@axisheight
\dimen1\axisheight\advance\dimen1-\dimen0
\dimen2\ht3\advance\dimen2-\dimen0%
\advance\dimen2-0.021em\advance\dimen1 #2\dimen2%
\raise\dimen1\box3}}
\def\shifthook#1#2#3{\setbox1=\hbox{#1$\scriptscriptstyle#3$}\dimen0\wd1%
\divide\dimen0 12\CD@zH{\dimen0}
\dimen1\wd1\advance\dimen1-2\dimen0 \advance\dimen1-2\CD@oI\CD@zH{\dimen1}%
\kern#2\dimen1\box1}

\def\@cmex{\mathchar"03}



\def\make@pbk#1{\setbox\tw@\hbox to\z@{#1}\ht\tw@\z@\dp\tw@\z@\box\tw@}\def
\CD@fH#1{\overprint{\hbox to\z@{#1}}}\def\CD@qH{\kern0.11em}\def\CD@pH{\kern0%
.35em}

\def\dblvert{\def\CD@rH{\kern.5\PileSpacing}}\def\CD@rH{}

\def\SEpbk{\make@pbk{\CD@qH\CD@rH\vrule depth 2.87ex height -2.75ex width 0.%
95em \vrule height -0.66ex depth 2.87ex width 0.05em \hss}}

\def\SWpbk{\make@pbk{\hss\vrule height -0.66ex depth 2.87ex width 0.05em
\vrule depth 2.87ex height -2.75ex width 0.95em \CD@qH\CD@rH}}

\def\NEpbk{\make@pbk{\CD@qH\CD@rH\vrule depth -3.81ex height 4.00ex width 0.%
95em \vrule height 4.00ex depth -1.72ex width 0.05em \hss}}

\def\NWpbk{\make@pbk{\hss\vrule height 4.00ex depth -1.72ex width 0.05em
\vrule depth -3.81ex height 4.00ex width 0.95em \CD@qH\CD@rH}}

\def\puncture{{\setbox0\hbox{A}\vrule height.53\ht0 depth-.47\ht0 width.35\ht
0 \kern.12\ht0 \vrule height\ht0 depth-.65\ht0 width.06\ht0 \kern-.06\ht0
\vrule height.35\ht0 depth0pt width.06\ht0 \kern.12\ht0 \vrule height.53\ht0
depth-.47\ht0 width.35\ht0 }}

\def\NEclck{\overprint{\raise2.5ex\rlap{ \CD@rH$\scriptstyle\searrow$}}}
\def\NEanti{\overprint{\raise2.5ex\rlap{ \CD@rH$\scriptstyle\nwarrow$}}}
\def\NWclck{\overprint{\raise2.5ex\llap{$\scriptstyle\nearrow$ \CD@rH}}}
\def\NWanti{\overprint{\raise2.5ex\llap{$\scriptstyle\swarrow$ \CD@rH}}}
\def\SEclck{\overprint{\lower1ex\rlap{ \CD@rH$\scriptstyle\swarrow$}}}
\def\SEanti{\overprint{\lower1ex\rlap{ \CD@rH$\scriptstyle\nearrow$}}}
\def\SWclck{\overprint{\lower1ex\llap{$\scriptstyle\nwarrow$ \CD@rH}}}
\def\SWanti{\overprint{\lower1ex\llap{$\scriptstyle\searrow$ \CD@rH}}}




\def\rhvee{\mkern-10mu\greaterthan}
\def\lhvee{\lessthan\mkern-10mu}
\def\dhvee{\vboxtoz{\vss\hbox{$\vee$}\kern0pt}}
\def\uhvee{\vboxtoz{\hbox{$\wedge$}\vss}}
\newarrowhead{vee}\rhvee\lhvee\dhvee\uhvee

\def\dhlvee{\vboxtoz{\vss\hbox{$\scriptstyle\vee$}\kern0pt}}
\def\uhlvee{\vboxtoz{\hbox{$\scriptstyle\wedge$}\vss}}
\newarrowhead{littlevee}{\mkern1mu\scriptaxis\rhvee}{\scriptaxis\lhvee}%
\dhlvee\uhlvee\ifx\boldmath\undefined
\newarrowhead{boldlittlevee}{\mkern1mu\scriptaxis\rhvee}{\scriptaxis\lhvee}%
\dhlvee\uhlvee\else
\def\dhblvee{\vboxtoz{\vss\boldscript\vee\kern0pt}}
\def\uhblvee{\vboxtoz{\boldscript\wedge\vss}}
\newarrowhead{boldlittlevee}{\mkern1mu\boldscriptaxis\rhvee}{\boldscriptaxis
\lhvee}\dhblvee\uhblvee
\fi

\def\rhcvee{\mkern-10mu\succ}
\def\lhcvee{\prec\mkern-10mu}
\def\dhcvee{\vboxtoz{\vss\hbox{$\curlyvee$}\kern0pt}}
\def\uhcvee{\vboxtoz{\hbox{$\curlywedge$}\vss}}
\newarrowhead{curlyvee}\rhcvee\lhcvee\dhcvee\uhcvee

\def\rhvvee{\mkern-13mu\gg}
\def\lhvvee{\ll\mkern-13mu}
\def\dhvvee{\vboxtoz{\vss\hbox{$\vee$}\kern-.6ex\hbox{$\vee$}\kern0pt}}
\def\uhvvee{\vboxtoz{\hbox{$\wedge$}\kern-.6ex \hbox{$\wedge$}\vss}}
\newarrowhead{doublevee}\rhvvee\lhvvee\dhvvee\uhvvee

\def\triangleup{{\scriptscriptstyle\bigtriangleup}}
\def\littletriangledown{{\scriptscriptstyle\triangledown}}
\def\rhtriangle{\triangleright\mkern1.2mu}
\def\lhtriangle{\triangleleft\mkern.8mu}
\def\uhtriangle{\vbox{\kern-.2ex \hbox{$\scriptscriptstyle\bigtriangleup$}%
\kern-.25ex}}
\def\dhtriangle{\vbox{\kern-.28ex \hbox{$\scriptscriptstyle\bigtriangledown$}%
\kern-.1ex}}
\def\dhblack{\vbox{\kern-.25ex\nointerlineskip\hbox{$\blacktriangledown$}}}%
\def\uhblack{\vbox{\kern-.25ex\nointerlineskip\hbox{$\blacktriangle$}}}%
\def\dhlblack{\vbox{\kern-.25ex\nointerlineskip\hbox{$\scriptstyle
\blacktriangledown$}}}
\def\uhlblack{\vbox{\kern-.25ex\nointerlineskip\hbox{$\scriptstyle
\blacktriangle$}}}
\newarrowhead{triangle}\rhtriangle\lhtriangle\dhtriangle\uhtriangle
\newarrowhead{blacktriangle}{\mkern-1mu\blacktriangleright\mkern.4mu}{%
\blacktriangleleft}\dhblack\uhblack\newarrowhead{littleblack}{\mkern-1mu%
\scriptaxis\blacktriangleright}{\scriptaxis\blacktriangleleft\mkern-2mu}%
\dhlblack\uhlblack

\def\rhla{\hbox{\setbox0=\lnchar55\dimen0=\wd0\kern-.6\dimen0\ht0\z@\raise
\axisheight\box0\kern.1\dimen0}}
\def\lhla{\hbox{\setbox0=\lnchar33\dimen0=\wd0\kern.05\dimen0\ht0\z@\raise
\axisheight\box0\kern-.5\dimen0}}
\def\dhla{\vboxtoz{\vss\rlap{\lnchar77}}}
\def\uhla{\vboxtoz{\setbox0=\lnchar66 \wd0\z@\kern-.15\ht0\box0\vss}}
\newarrowhead{LaTeX}\rhla\lhla\dhla\uhla

\def\lhlala{\lhla\kern.3em\lhla}
\def\rhlala{\rhla\kern.3em\rhla}
\def\uhlala{\hbox{\uhla\raise-.6ex\uhla}}
\def\dhlala{\hbox{\dhla\lower-.6ex\dhla}}
\newarrowhead{doubleLaTeX}\rhlala\lhlala\dhlala\uhlala

\def\hhO{\scriptaxis\bigcirc\mkern.4mu} \def\hho{{\circ}\mkern1.2mu}%
\newarrowhead{o}\hho\hho\circ\circ
\newarrowhead{O}\hhO\hhO{\scriptstyle\bigcirc}{\scriptstyle\bigcirc}

\def\rhtimes{\mkern-5mu{\times}\mkern-.8mu}\def\lhtimes{\mkern-.8mu{\times}%
\mkern-5mu}\def\uhtimes{\setbox0=\hbox{$\times$}\ht0\axisheight\dp0-\ht0%
\lower\ht0\box0 }\def\dhtimes{\setbox0=\hbox{$\times$}\ht0\axisheight\box0 }%
\newarrowhead{X}\rhtimes\lhtimes\dhtimes\uhtimes\newarrowhead+++++


\newarrowhead{Y}{\mkern-3mu\Yright}{\Yleft\mkern-3mu}\Yup\Ydown


\newarrowhead{->}\rightarrow\leftarrow\downarrow\uparrow

\newarrowhead{=>}\Rightarrow\Leftarrow{\@cmex7F}{\@cmex7E}

\newarrowhead{harpoon}\rightharpoonup\leftharpoonup\downharpoonleft
\upharpoonleft

\def\twoheaddownarrow{\rlap{$\downarrow$}\raise-.5ex\hbox{$\downarrow$}}
\def\twoheaduparrow{\rlap{$\uparrow$}\raise.5ex\hbox{$\uparrow$}}
\newarrowhead{->>}\twoheadrightarrow\twoheadleftarrow\twoheaddownarrow
\twoheaduparrow


\def\rtvee{\greaterthan}
\def\ltvee{\mkern-1mu{\lessthan}\mkern.4mu}
\def\dtvee{\vee}
\def\utvee{\wedge}
\newarrowtail{vee}\greaterthan\ltvee\vee\wedge

\newarrowtail{littlevee}{\scriptaxis\greaterthan}{\mkern-1mu\scriptaxis
\lessthan}{\scriptstyle\vee}{\scriptstyle\wedge}\ifx\boldmath\undefined
\newarrowtail{boldlittlevee}{\scriptaxis\greaterthan}{\mkern-1mu\scriptaxis
\lessthan}{\scriptstyle\vee}{\scriptstyle\wedge}\else\newarrowtail{%
boldlittlevee}{\boldscriptaxis\greaterthan}{\mkern-1mu\boldscriptaxis
\lessthan}{\boldscript\vee}{\boldscript\wedge}\fi

\newarrowtail{curlyvee}\succ{\mkern-1mu{\prec}\mkern.4mu}\curlyvee\curlywedge

\def\rttriangle{\mkern1.2mu\triangleright}
\newarrowtail{triangle}\rttriangle\lhtriangle\dhtriangle\uhtriangle
\newarrowtail{blacktriangle}\blacktriangleright{\mkern-1mu\blacktriangleleft
\mkern.4mu}\dhblack\uhblack\newarrowtail{littleblack}{\scriptaxis
\blacktriangleright\mkern-2mu}{\mkern-1mu\scriptaxis\blacktriangleleft}%
\dhlblack\uhlblack

\def\rtla{\hbox{\setbox0=\lnchar55\dimen0=\wd0\kern-.5\dimen0\ht0\z@\raise
\axisheight\box0\kern-.2\dimen0}}
\def\ltla{\hbox{\setbox0=\lnchar33\dimen0=\wd0\kern-.15\dimen0\ht0\z@\raise
\axisheight\box0\kern-.5\dimen0}}
\def\dtla{\vbox{\setbox0=\rlap{\lnchar77}\dimen0=\ht0\kern-.7\dimen0\box0%
\kern-.1\dimen0}}
\def\utla{\vbox{\setbox0=\rlap{\lnchar66}\dimen0=\ht0\kern-.1\dimen0\box0%
\kern-.6\dimen0}}
\newarrowtail{LaTeX}\rtla\ltla\dtla\utla

\def\rtvvee{\gg\mkern-3mu}
\def\ltvvee{\mkern-3mu\ll}
\def\dtvvee{\vbox{\hbox{$\vee$}\kern-.6ex \hbox{$\vee$}\vss}}
\def\utvvee{\vbox{\vss\hbox{$\wedge$}\kern-.6ex \hbox{$\wedge$}\kern\z@}}
\newarrowtail{doublevee}\rtvvee\ltvvee\dtvvee\utvvee

\def\ltlala{\ltla\kern.3em\ltla}
\def\rtlala{\rtla\kern.3em\rtla}
\def\utlala{\hbox{\utla\raise-.6ex\utla}}
\def\dtlala{\hbox{\dtla\lower-.6ex\dtla}}
\newarrowtail{doubleLaTeX}\rtlala\ltlala\dtlala\utlala

\def\utbar{\vrule height 0.093ex depth0pt width 0.4em}
\let\dtbar\utbar
\def\rtbar{\mkern1.5mu\vrule height 1.1ex depth.06ex width .04em\mkern1.5mu}%
\let\ltbar\rtbar
\newarrowtail{mapsto}\rtbar\ltbar\dtbar\utbar
\newarrowtail{|}\rtbar\ltbar\dtbar\utbar


\def\rthooka{\raisehook{}+\subset\mkern-1mu}
\def\lthooka{\mkern-1mu\raisehook{}+\supset}
\def\rthookb{\raisehook{}-\subset\mkern-2mu}
\def\lthookb{\mkern-1mu\raisehook{}-\supset}

\def\dthooka{\shifthook{}+\cap}
\def\dthookb{\shifthook{}-\cap}
\def\uthooka{\shifthook{}+\cup}
\def\uthookb{\shifthook{}-\cup}

\newarrowtail{hooka}\rthooka\lthooka\dthooka\uthooka\newarrowtail{hookb}%
\rthookb\lthookb\dthookb\uthookb

\ifx\boldmath\undefined\newarrowtail{boldhooka}\rthooka\lthooka\dthooka
\uthooka\newarrowtail{boldhookb}\rthookb\lthookb\dthookb\uthookb\newarrowtail
{boldhook}\rthooka\lthooka\dthookb\uthooka\else\def\rtbhooka{\raisehook
\boldmath+\subset\mkern-1mu}
\def\ltbhooka{\mkern-1mu\raisehook\boldmath+\supset}
\def\rtbhookb{\raisehook\boldmath-\subset\mkern-2mu}
\def\ltbhookb{\mkern-1mu\raisehook\boldmath-\supset}
\def\dtbhooka{\shifthook\boldmath+\cap}
\def\dtbhookb{\shifthook\boldmath-\cap}
\def\utbhooka{\shifthook\boldmath+\cup}
\def\utbhookb{\shifthook\boldmath-\cup}
\newarrowtail{boldhooka}\rtbhooka\ltbhooka\dtbhooka\utbhooka\newarrowtail{%
boldhookb}\rtbhookb\ltbhookb\dtbhookb\utbhookb\newarrowtail{boldhook}%
\rtbhooka\ltbhooka\dtbhooka\utbhooka\fi

\def\dtsqhooka{\shifthook{}+\sqcap}
\def\dtsqhookb{\shifthook{}-\sqcap}
\def\ltsqhooka{\mkern-1mu\raisehook{}+\sqsupset}
\def\ltsqhookb{\mkern-1mu\raisehook{}-\sqsupset}
\def\rtsqhooka{\raisehook{}+\sqsubset\mkern-1mu}
\def\rtsqhookb{\raisehook{}-\sqsubset\mkern-2mu}
\def\utsqhooka{\shifthook{}+\sqcup}
\def\utsqhookb{\shifthook{}-\sqcup}
\newarrowtail{sqhook}\rtsqhooka\ltsqhooka\dtsqhooka\utsqhooka

\newarrowtail{hook}\rthooka\lthookb\dthooka\uthooka\newarrowtail{C}\rthooka
\lthookb\dthooka\uthooka

\newarrowtail{o}\hho\hho\circ\circ
\newarrowtail{O}\hhO\hhO{\scriptstyle\bigcirc}{\scriptstyle\bigcirc}

\newarrowtail{X}\lhtimes\rhtimes\uhtimes\dhtimes\newarrowtail+++++


\newarrowtail{Y}\Yright\Yleft\Yup\Ydown

\newarrowtail{harpoon}\leftharpoondown\rightharpoondown\upharpoonright
\downharpoonright




\newarrowfiller{=}=={\@cmex77}{\@cmex77}
\def\vfthree{\mid\!\!\!\mid\!\!\!\mid}
\newarrowfiller{3}\equiv\equiv\vfthree\vfthree

\def\vfdashstrut{\vrule width0pt height1.3ex depth0.7ex}
\def\vfthedash{\vrule width\CD@LF height0.6ex depth 0pt}
\def\hfthedash{\CD@AJ\vrule\horizhtdp width 0.26em}
\def\hfdash{\mkern5.5mu\hfthedash\mkern5.5mu}
\def\vfdash{\vfdashstrut\vfthedash}
\newarrowfiller{dash}\hfdash\hfdash\vfdash\vfdash


\newarrowmiddle+++++




\iffalse
\newarrow{To}----{vee}
\newarrow{Arr}----{LaTeX}
\newarrow{Dotsto}....{vee}
\newarrow{Dotsarr}....{LaTeX}
\newarrow{Dashto}{}{dash}{}{dash}{vee}
\newarrow{Dasharr}{}{dash}{}{dash}{LaTeX}
\newarrow{Mapsto}{mapsto}---{vee}
\newarrow{Mapsarr}{mapsto}---{LaTeX}
\newarrow{IntoA}{hooka}---{vee}
\newarrow{IntoB}{hookb}---{vee}
\newarrow{Embed}{vee}---{vee}
\newarrow{Emarr}{LaTeX}---{LaTeX}
\newarrow{Onto}----{doublevee}
\newarrow{Dotsonarr}....{doubleLaTeX}
\newarrow{Dotsonto}....{doublevee}
\newarrow{Dotsonarr}....{doubleLaTeX}
\else
\newarrow{To}---->
\newarrow{Arr}---->
\newarrow{Dotsto}....>
\newarrow{Dotsarr}....>
\newarrow{Dashto}{}{dash}{}{dash}>
\newarrow{Dasharr}{}{dash}{}{dash}>
\newarrow{Mapsto}{mapsto}--->
\newarrow{Mapsarr}{mapsto}--->
\newarrow{IntoA}{hooka}--->
\newarrow{IntoB}{hookb}--->
\newarrow{Embed}>--->
\newarrow{Emarr}>--->
\newarrow{Onto}----{>>}
\newarrow{Dotsonarr}....{>>}
\newarrow{Dotsonto}....{>>}
\newarrow{Dotsonarr}....{>>}
\fi

\newarrow{Implies}===={=>}
\newarrow{Project}----{triangle}
\newarrow{Pto}----{harpoon}
\newarrow{Relto}{harpoon}---{harpoon}

\newarrow{Eq}=====
\newarrow{Line}-----
\newarrow{Dots}.....
\newarrow{Dashes}{}{dash}{}{dash}{}

\newarrow{SquareInto}{sqhook}--->

\newarrowhead{cmexbra}{\@cmex7B}{\@cmex7C}{\@cmex3B}{\@cmex38}
\newarrowtail{cmexbra}{\@cmex7A}{\@cmex7D}{\@cmex39}{\@cmex3A}
\newarrowmiddle{cmexbra}{\braceru\bracelu}{\bracerd\braceld}{\vcenter{%
\hbox@maths{\@cmex3D\mkern-2mu}}}
{\vcenter{\hbox@maths{\mkern2mu\@cmex3C}}}
\newarrow{@brace}{cmexbra}-{cmexbra}-{cmexbra}
\newarrow{@parenth}{cmexbra}---{cmexbra}
\def\rightBrace{\d@brace[thick,cmex]}
\def\leftBrace{\u@brace[thick,cmex]}
\def\upperBrace{\r@brace[thick,cmex]}
\def\lowerBrace{\l@brace[thick,cmex]}
\def\rightParenth{\d@parenth[thick,cmex]}
\def\leftParenth{\u@parenth[thick,cmex]}
\def\upperParenth{\r@parenth[thick,cmex]}
\def\lowerParenth{\l@parenth[thick,cmex]}


\let\uFrom\uTo
\let\lFrom\lTo
\let\uDotsfrom\uDotsto
\let\lDotsfrom\lDotsto
\let\uDashfrom\uDashto
\let\lDashfrom\lDashto
\let\uImpliedby\uImplies
\let\lImpliedby\lImplies
\let\uMapsfrom\uMapsto
\let\lMapsfrom\lMapsto
\let\lOnfrom\lOnto
\let\uOnfrom\uOnto
\let\lPfrom\lPto
\let\uPfrom\uPto

\let\uInfromA\uIntoA
\let\uInfromB\uIntoB
\let\lInfromA\lIntoA
\let\lInfromB\lIntoB
\let\rInto\rIntoA
\let\lInto\lIntoA
\let\dInto\dIntoB
\let\uInto\uIntoA
\let\ruInto\ruIntoA
\let\luInto\luIntoA
\let\rdInto\rdIntoA
\let\ldInto\ldIntoA
\let\hEq\rEq
\let\vEq\uEq
\let\hLine\rLine
\let\vLine\uLine
\let\hDots\rDots
\let\vDots\uDots
\let\hDashes\rDashes
\let\vDashes\uDashes

\let\NW\luTo\let\NE\ruTo\let\SW\ldTo\let\SE\rdTo\def\nNW{\luTo(2,3)}\def\nNE{%
\ruTo(2,3)}
\def\sSW{\ldTo(2,3)}\def\sSE{\rdTo(2,3)}
\def\wNW{\luTo(3,2)}\def\eNE{\ruTo(3,2)}
\def\wSW{\ldTo(3,2)}\def\eSE{\rdTo(3,2)}
\def\NNW{\luTo(2,4)}\def\NNE{\ruTo(2,4)}
\def\SSW{\ldTo(2,4)}\def\SSE{\rdTo(2,4)}
\def\WNW{\luTo(4,2)}\def\ENE{\ruTo(4,2)}
\def\WSW{\ldTo(4,2)}\def\ESE{\rdTo(4,2)}
\def\NNNW{\luTo(2,6)}\def\NNNE{\ruTo(2,6)}
\def\SSSW{\ldTo(2,6)}\def\SSSE{\rdTo(2,6)}
\def\WWNW{\luTo(6,2)}\def\EENE{\ruTo(6,2)}
\def\WWSW{\ldTo(6,2)}\def\EESE{\rdTo(6,2)}

\let\NWd\luDotsto\let\NEd\ruDotsto\let\SWd\ldDotsto\let\SEd\rdDotsto\def\nNWd
{\luDotsto(2,3)}\def\nNEd{\ruDotsto(2,3)}
\def\sSWd{\ldDotsto(2,3)}\def\sSEd{\rdDotsto(2,3)}
\def\wNWd{\luDotsto(3,2)}\def\eNEd{\ruDotsto(3,2)}
\def\wSWd{\ldDotsto(3,2)}\def\eSEd{\rdDotsto(3,2)}
\def\NNWd{\luDotsto(2,4)}\def\NNEd{\ruDotsto(2,4)}
\def\SSWd{\ldDotsto(2,4)}\def\SSEd{\rdDotsto(2,4)}
\def\WNWd{\luDotsto(4,2)}\def\ENEd{\ruDotsto(4,2)}
\def\WSWd{\ldDotsto(4,2)}\def\ESEd{\rdDotsto(4,2)}
\def\NNNWd{\luDotsto(2,6)}\def\NNNEd{\ruDotsto(2,6)}
\def\SSSWd{\ldDotsto(2,6)}\def\SSSEd{\rdDotsto(2,6)}
\def\WWNWd{\luDotsto(6,2)}\def\EENEd{\ruDotsto(6,2)}
\def\WWSWd{\ldDotsto(6,2)}\def\EESEd{\rdDotsto(6,2)}

\let\NWl\luLine\let\NEl\ruLine\let\SWl\ldLine\let\SEl\rdLine\def\nNWl{\luLine
(2,3)}\def\nNEl{\ruLine(2,3)}
\def\sSWl{\ldLine(2,3)}\def\sSEl{\rdLine(2,3)}
\def\wNWl{\luLine(3,2)}\def\eNEl{\ruLine(3,2)}
\def\wSWl{\ldLine(3,2)}\def\eSEl{\rdLine(3,2)}
\def\NNWl{\luLine(2,4)}\def\NNEl{\ruLine(2,4)}
\def\SSWl{\ldLine(2,4)}\def\SSEl{\rdLine(2,4)}
\def\WNWl{\luLine(4,2)}\def\ENEl{\ruLine(4,2)}
\def\WSWl{\ldLine(4,2)}\def\ESEl{\rdLine(4,2)}
\def\NNNWl{\luLine(2,6)}\def\NNNEl{\ruLine(2,6)}
\def\SSSWl{\ldLine(2,6)}\def\SSSEl{\rdLine(2,6)}
\def\WWNWl{\luLine(6,2)}\def\EENEl{\ruLine(6,2)}
\def\WWSWl{\ldLine(6,2)}\def\EESEl{\rdLine(6,2)}

\let\NWld\luDots\let\NEld\ruDots\let\SWld\ldDots\let\SEld\rdDots\def\nNWld{%
\luDots(2,3)}\def\nNEld{\ruDots(2,3)}
\def\sSWld{\ldDots(2,3)}\def\sSEld{\rdDots(2,3)}
\def\wNWld{\luDots(3,2)}\def\eNEld{\ruDots(3,2)}
\def\wSWld{\ldDots(3,2)}\def\eSEld{\rdDots(3,2)}
\def\NNWld{\luDots(2,4)}\def\NNEld{\ruDots(2,4)}
\def\SSWld{\ldDots(2,4)}\def\SSEld{\rdDots(2,4)}
\def\WNWld{\luDots(4,2)}\def\ENEld{\ruDots(4,2)}
\def\WSWld{\ldDots(4,2)}\def\ESEld{\rdDots(4,2)}
\def\NNNWld{\luDots(2,6)}\def\NNNEld{\ruDots(2,6)}
\def\SSSWld{\ldDots(2,6)}\def\SSSEld{\rdDots(2,6)}
\def\WWNWld{\luDots(6,2)}\def\EENEld{\ruDots(6,2)}
\def\WWSWld{\ldDots(6,2)}\def\EESEld{\rdDots(6,2)}

\let\NWe\luEmbed\let\NEe\ruEmbed\let\SWe\ldEmbed\let\SEe\rdEmbed\def\nNWe{%
\luEmbed(2,3)}\def\nNEe{\ruEmbed(2,3)}
\def\sSWe{\ldEmbed(2,3)}\def\sSEe{\rdEmbed(2,3)}
\def\wNWe{\luEmbed(3,2)}\def\eNEe{\ruEmbed(3,2)}
\def\wSWe{\ldEmbed(3,2)}\def\eSEe{\rdEmbed(3,2)}
\def\NNWe{\luEmbed(2,4)}\def\NNEe{\ruEmbed(2,4)}
\def\SSWe{\ldEmbed(2,4)}\def\SSEe{\rdEmbed(2,4)}
\def\WNWe{\luEmbed(4,2)}\def\ENEe{\ruEmbed(4,2)}
\def\WSWe{\ldEmbed(4,2)}\def\ESEe{\rdEmbed(4,2)}
\def\NNNWe{\luEmbed(2,6)}\def\NNNEe{\ruEmbed(2,6)}
\def\SSSWe{\ldEmbed(2,6)}\def\SSSEe{\rdEmbed(2,6)}
\def\WWNWe{\luEmbed(6,2)}\def\EENEe{\ruEmbed(6,2)}
\def\WWSWe{\ldEmbed(6,2)}\def\EESEe{\rdEmbed(6,2)}

\let\NWo\luOnto\let\NEo\ruOnto\let\SWo\ldOnto\let\SEo\rdOnto\def\nNWo{\luOnto
(2,3)}\def\nNEo{\ruOnto(2,3)}
\def\sSWo{\ldOnto(2,3)}\def\sSEo{\rdOnto(2,3)}
\def\wNWo{\luOnto(3,2)}\def\eNEo{\ruOnto(3,2)}
\def\wSWo{\ldOnto(3,2)}\def\eSEo{\rdOnto(3,2)}
\def\NNWo{\luOnto(2,4)}\def\NNEo{\ruOnto(2,4)}
\def\SSWo{\ldOnto(2,4)}\def\SSEo{\rdOnto(2,4)}
\def\WNWo{\luOnto(4,2)}\def\ENEo{\ruOnto(4,2)}
\def\WSWo{\ldOnto(4,2)}\def\ESEo{\rdOnto(4,2)}
\def\NNNWo{\luOnto(2,6)}\def\NNNEo{\ruOnto(2,6)}
\def\SSSWo{\ldOnto(2,6)}\def\SSSEo{\rdOnto(2,6)}
\def\WWNWo{\luOnto(6,2)}\def\EENEo{\ruOnto(6,2)}
\def\WWSWo{\ldOnto(6,2)}\def\EESEo{\rdOnto(6,2)}

\let\NWod\luDotsonto\let\NEod\ruDotsonto\let\SWod\ldDotsonto\let\SEod
\rdDotsonto\def\nNWod{\luDotsonto(2,3)}\def\nNEod{\ruDotsonto(2,3)}
\def\sSWod{\ldDotsonto(2,3)}\def\sSEod{\rdDotsonto(2,3)}
\def\wNWod{\luDotsonto(3,2)}\def\eNEod{\ruDotsonto(3,2)}
\def\wSWod{\ldDotsonto(3,2)}\def\eSEod{\rdDotsonto(3,2)}
\def\NNWod{\luDotsonto(2,4)}\def\NNEod{\ruDotsonto(2,4)}
\def\SSWod{\ldDotsonto(2,4)}\def\SSEod{\rdDotsonto(2,4)}
\def\WNWod{\luDotsonto(4,2)}\def\ENEod{\ruDotsonto(4,2)}
\def\WSWod{\ldDotsonto(4,2)}\def\ESEod{\rdDotsonto(4,2)}
\def\NNNWod{\luDotsonto(2,6)}\def\NNNEod{\ruDotsonto(2,6)}
\def\SSSWod{\ldDotsonto(2,6)}\def\SSSEod{\rdDotsonto(2,6)}
\def\WWNWod{\luDotsonto(6,2)}\def\EENEod{\ruDotsonto(6,2)}
\def\WWSWod{\ldDotsonto(6,2)}\def\EESEod{\rdDotsonto(6,2)}


\def\labelstyle{
\ifincommdiag
\textstyle
\else
\scriptstyle
\fi}
\let\objectstyle\displaystyle

\newdiagramgrid{pentagon}{0.618034,0.618034,1,1,1,1,0.618034,0.618034}{1.%
17557,1.17557,1.902113,1.902113}

\newdiagramgrid{perspective}{0.75,0.75,1.1,1.1,0.9,0.9,0.95,0.95,0.75,0.75}{0%
.75,0.75,1.1,1.1,0.9,0.9}

\diagramstyle[
dpi=300,
vmiddle,nobalance,
loose,
thin,
pilespacing=10pt,%
shortfall=4pt,
]

\ifx\ProcessOptions\undefined\else\CD@PK\ProcessOptions\relax\CD@FF\CD@e\fi
\fi

\cdrestoreat

\dimen0 200pt \dimen1 210pt \dimen2 220pt \dimen3 230pt \dimen4 240pt \dimen5
250pt \dimen6 260pt \dimen7 270pt \dimen8 280pt \dimen9 290pt



Let $W$ be a vector space of dimension $n+1$ over a field $K$.  The {\it Chow
divisor\/} of a $k$-dimensional variety $X$ in $\PP^n=\PP(W)$ is the
hypersurface defined over $K$ in the Grassmannian $\GG_{k+1}$ of planes of
codimension $k+1$ in $\PP^n$, whose points over the algebraic closure $\bar K$
are of those planes meeting $X(\bar K)$.  The {\it Chow form} is its defining
equation.  {}For example, the resultant of $k+1$ forms of degree $e$ in $k+1$
variables is the Chow form of $\PP^{k}$ embedded by the $e$-th Veronese mapping
in $\PP^n$ with $n={k+e\choose k}-1$.

In this paper
we will give a new expression for the Chow divisor,
closely related to Beilinson's monad for sheaves on projective space
\cite{Beilinson 1978}
(the term monad
stems from \cite{Horrocks 1964}).
We
derive a number of new polynomial formulas for Chow forms and resultants;
for example, we show that the resultant of 3 quadratic
forms, the Chow form of the Veronese surface in $\PP^5$,
can be written in ``B\'ezout form'' (described below)
as the Pfaffian of the matrix
{\renewcommand{~}{\llap{$-$}}
{\newcommand{\myplus}{\llap{$+$}}
{\scriptsize{
\[
\begin{pmatrix}
\ \ 0&[245]&[345]&[135]&[045]&[035]&[145]&[235]\\[6pt]
\ \ ~[245]&0&~[235]&[035]&[025]&[015]&[125]&~[125]\!+\![045]\\[6pt]
\ \ ~[345]&[235]&0&[134]&[035]&[034]&[135]&[234]\\[6pt]
\ \ ~[135]&~[035]&~[134]&0&[023]&[013]&[123]\!-\![034]&~[123]\\[6pt]
\ \ ~[045]&~[025]&~[035]&~[023]&0&[012]&~[015]&~[024]\!+\![015]\\[6pt]
\ \ ~[035]&~[015]&~[034]&~[013]&~[012]&0&[023]\!-\![014]&~[023]\\[6pt]
\ \
~[145]&~[125]&~[135]&~[123]\!+\![034]&[015]&~[023]\!+\![014]&0&~[124]\!+\![\
035]\\[6pt]
\ \
~[235]&[125]\!-\![045]&~[234]&[123]&[024]\!-\![015]&[023]&[124]\!-\![035]&0\
\\
\end{pmatrix}
\]}}}}

\noindent
Here the monomials in the three variables $x,y,z$ are ordered
$x^2,xy,xz,y^2,yz,z^2$ and the brackets $[ijk]$ denote the
corresponding Pl\"ucker coordinates of the net of quadrics. Using the
theory of rank two vector bundles on $\PP^2$, we can construct many
such formulas for ternary forms of any degree.

\goodbreak
There are several ways in which one can describe a resultant or
Chow form:

\subsection*{B\'ezout formulas for resultants}

The classic formula of B\'ezout (see, for example, \cite[Chapter 12,
(1.17) and (1.18)]{Gelfandetal.1994}) gives the resultant of two
homogeneous forms in two variables as a determinant of linear forms in
the Pl\"ucker coordinates of the space generated by the two forms.  By
analogy we will call any formula for the Chow form in Pl\"ucker
coordinates a {\it B\'ezout expression} of the Chow form. Our simplest
example of a new B\'ezout expression is the one given above.

